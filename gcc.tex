\chapter{GCC}
\section{定义函数别名}
在gcc当中提供了一系列的扩展功能,包含了定义方法的别名。函数的别名可以是强引用,也可以是弱引用。
\begin{code-block}{c}
void __swap_object(void * first, void * last, size_t size)
{
    void * tmp = malloc(size);
    memcpy(tmp, first, size);
    memcpy(first, last, size);
    memcpy(last, tmp, size);
    free(tmp);
}

// 弱引用别名
void swap_weak(void * first, void *last, size_t size)
        __attribute__ ((weak, alias("__swap_object")));
// 强引用别名
void swap_strone(void * first, void *last, size_t size)
        __attribute__ ((alias("__swap_object")));
\end{code-block}

\section{标记函数被废弃}
在gcc也支持标记函数为废弃,来警示相关人员不要使用这样的函数。
\begin{code-block}{c}
void print_hello() __attribute__ ((deprecated));
void print_hello()
{
    printf("hello\n");
}
\end{code-block}

通过以上的方式进行标记时候,在调用printf\_hello这个方法的时候,编译过程中会提示
如下的信息:
\begin{code-block}{bash}
test.c:74:5: 警告:‘print_hello’ is deprecated [-Wdeprecated-declarations]
     print_hello();
     ^~~~~~~~~~~
test.c:46:6: 附注:在此声明
 void print_hello()
\end{code-block}

\section{特殊的gcc扩展}
有的时候,我们可能需要在执行main方法之前执行一些动作,在执行完main方法之后,再
进行一些扫尾的动作。Gcc提供了这样的支持。
\begin{code-block}{c}
// 进入main方法之前就执行该方法
void hello() __attribute__ ((constructor));
void hello()
{
    printf("hello\n");
}

// 退出main方法之后,执行该方法
void bye() __attribute__ ((destructor));
void bye()
{
    printf("bye\n");
}

int main(int argc, char * argv[])
{
    return 0;
}
\end{code-block}

以上代码编译执行之后,其输出为:
\begin{code-block}{bash}
helllo
bye
\end{code-block}

\section{Gcc内置函数}
GCC提供了一系列的内置函数,用于进行优化程序。

\begin{outline}[enumerate]

\1 \_\_builtin\_types\_compatible\_p

该函数为Gcc的扩展函数,用于判断2个数据的类型是否相同。其原型定义如下:
\begin{code-in-enumerate}{c}
  int __builtin_types_compatible_p(type_a, type_b);
\end{code-in-enumerate}

在真正的使用当中,可以用于进行宏定义,来判断传递的参数是否为需要的类型:
\begin{code-in-enumerate}{c}
// 判断传递的参数是否为数组
#define BUILD_BUG_ON_ZERO(e) (sizeof(char[1 - 2 * !!(e)]) - 1)
#define __same_type(a, b) __builtin_types_compatible_p(typeof(a), typeof(b))
#define __must_be_array(a) BUILD_BUG_ON_ZERO(__same_type((a), &(a)[0]))
\end{code-in-enumerate}

关于BUILD\_BUG\_ON\_ZERO在前面的已经仔细讲解过,其主要作用就是用来发现编译时的错误,
\_\_same\_type宏定义用于判断传入的参数是否是相同的数据类型,a表示原始的数据,(a)[0]表示
取a数组的第0个元素,\&(a)[0]表示对a数组的第0个元素取地址,因此,\_\_same\_type((a), \&(a)[0])
的作用就是,通过\&(a)[0]这个操作,来区分数组和其他类型,如果数据a可以执行该操作,可以认为
a是一个数组或指针,反之,就会报错,从而在编译时刻就发现错误。同样的,该宏定义还可以区分数组与指向数组
的指针:typeof(a)返回的是数组类型,而typeof(\&a[0])返回的则是指针类型。

\begin{code-in-enumerate}{c}
int a[] = {1,2,3};
__must_be_array(a);
float b[] = {1,2,3};
__must_be_array(b);
int c = 123;
int *d = &c;
__must_be_array(d); // 编译时提示错误

int *e = a;
__must_be_array(e); // 同样是编译提示错误
\end{code-in-enumerate}

\end{outline}
