\subsection{Job}
Kubernetes支持job类型的任务。
\begin{code-block}{bash}
cat >job.yml<<EOF
apiVersion: batch/v1
kind: Job
metadata:
  name: pi
spec:
  template:
    spec:
      containers:
      - name: pi
        image: perl
        command: ["perl",  "-Mbignum=bpi", "-wle", "print bpi(20000)"]
      restartPolicy: Never
  backoffLimit: 4
EOF
\end{code-block}
Job一般在容器当中的任务运行结束之后,会自行终止。所谓的任务结束,即容器中的主进程退出。
但是,有的时候,容器中的主进程很有可能很长时间都无法结束,因此,有另外一种方式来终止Job的
运行。
\begin{code-block}{bash}
cat >active-job.yml<<EOF
apiVersion: batch/v1
kind: Job
metadata:
  name: pi
spec:
  activeDeadlineSeconds: 100   # 表示该job最多运行100s。如果到了100s,任务还未完成,
                               # 该任务依然会退出。但是需要和pod的对应属性一起使用(bug?)
                               # 并且,pod的对应属性要小于job
  template:
    spec:
      containers:
      - name: pi
        image: perl
        command: ["perl",  "-Mbignum=bpi", "-wle", "print bpi(20000)"]
      restartPolicy: Never
      activeDeadlineSeconds: 99
  backoffLimit: 4
EOF
\end{code-block}