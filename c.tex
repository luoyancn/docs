\chapter{C}

\section{柔性数组}
C语言的数据和python的不一样,是一个定长的,也就是说,需要预先设定好长度。如果需要
使用变长数组,则需要使用指针。通过指针的方式,一个一个的分配。但是,这种方式,不利于
计算数组长度,当需要使用数据长度时,就会出现问题。柔性数组则不一样,可以当成变长
数据使用,同时,还可以确定长度。

柔性数据的定义如下
\begin{code-block}{c}
typedef struct _soft_array * array_ptr;
typedef struct _soft_array{
    size_t lenth;
    int members[1];
}soft_array;
\end{code-block}

柔性数组一般由2部分组成,第一个表示数组长度,第二个表示数据的元素。但是,由于
各个c/c++编译器的不一致,第二个参数,一定要是一个数组,并且,最好这个数组的长度
为1。在gcc当中,这个members的长度可以为0,但在clang/virsual c++当中,则可能报错。
统一设置为1,则不会出现这个问题。

柔性数组的使用
\begin{code-block}{c}
array_ptr init_soft_array(size_t lenth){
    array_ptr arrays = NULL;
    if(NULL == (arrays = malloc(
        offsetof(soft_array, members) + sizeof(int) * lenth))){
        printf("Cannot allocate more memory\n");
        return NULL;
    }
    arrays->lenth = lenth;
    for(size_t index = 0; index < lenth; index++){
        arrays->members[index] = index;
    }
    return arrays;
}
\end{code-block}

\section{指向指针的指针}
指向指针的指针,通常用在需要改变指针的地方。常见的操作,就是使用指向指针的指针
来删除单链表。
\begin{code-block}{c}
void delete_link(nodeptr * header, nodeptr delete_node) {
    nodeptr * current = header;
    nodeptr entry = NULL;
    while(*current) {
        entry = *current;
        if(entry == delete_node) {
            *current = entry -> next;
            free(delete_node);
            delete_node= NULL;
            return;
        } else {
            current = &(entry->next);
        }
    }
}
\end{code-block}
