\section{Linux Kernel}
\subsection{Kernel编译}
\subsubsection{安装必要的软件包}
\begin{code-block}{bash}
yum install ncurses-devel bison flex elfutils-libelf-devel bc openssl-devel -y
\end{code-block}

\subsubsection{设置编译选项}
\begin{code-block}{bash}
make menuconfig
\end{code-block}

\subsubsection{编译内核}
\begin{code-block}{bash}
make
# 如果是在多核服务器上进行编译,可以增加编译参数,提高编译速度
# make -j32 #32表示cpu的核数
\end{code-block}

\subsubsection{安装内核模块}
\begin{code-block}{bash}
make modules_install
\end{code-block}

安装内核模块的操作,会将编译生成的内核模块复制到/lib/modules/\{kernel-version\}/下。

\subsubsection{安装内核}
\begin{code-block}{bash}
make install
\end{code-block}

安装内核的过程主要完成了以下的工作:将编译内核时生成的内核镜像bzImage拷贝到/boot目录下,
并将这个镜像命名为vmlinuz-\{kernel-version\}。如果使用x86的cpu,则该镜像位于arch/x86/boot/目录下;
将目录下的System.map拷贝到/boot/目录下,重新命名为System.map-\{kernel-version\},该文件中存放了内核的符号表。
将目录下的.config拷贝到/boot/目录下,重新命名为config-\{kernel-version\}

\subsubsection{创建initrd.img文件}
\begin{code-block}{bash}
mkinitrd  /boot/initrd.img-{kernel-version} {kernel-version}
\end{code-block}

initrd.img即为初始化的ramdisk文件,它是一个镜像文件。

\subsubsection{修改grub}
\begin{code-block}{bash}
grub2-mkconfig -o /boot/grub2/grub.cfg
\end{code-block}

修改完成之后,重启服务器,即可发现新编译的内核,如下图:
图 \nameref{fig:new-kernel}
\begin{figure}[H]
  \centering
  \includegraphics[width=\linewidth]{new-kernel.png}
  \caption{新编译内核 \protect\footnotemark}
  \label{fig:new-kernel}
\end{figure}

\subsection{编写自己的内核模块}
在编写自己的内核模块的时候,一般需要2个文件:一个c代码文件,包含了自己的内核模块
内在逻辑实现;一个makefile文件,用于编译自己的内核模块。以最简单的hello world为例。
C代码如下:
\begin{code-block}{c}
// hello_kernel.c
#include <linux/init.h>
#include <linux/module.h>
#include <linux/kernel.h>

// 必须,标明模块的许可声明
MODULE_LICENSE("GPL");

// 模块的加载函数,即加载该模块之后,执行的操作
static int hello_init(void)
{
    printk(KERN_ALERT "hello,I am zhangjl\n");
    return 0;
}

// 模块的卸载函数,即该模块卸载之后,应当执行什么操作
static void hello_exit(void)
{
    printk(KERN_ALERT "goodbye,kernel\n");
}

// 注册模块对应的操作
module_init(hello_init);
module_exit(hello_exit);

// 可选,表示该模块的作者和其他信息
MODULE_AUTHOR("zhangjl");
MODULE_DESCRIPTION("This is a simple example!\n");
MODULE_ALIAS("A simplest example");
\end{code-block}

Makefile文件内容如下:
\begin{code-block}{make}
obj-m += hello_kernel.o
#generate the path
CURRENT_PATH:=$(shell pwd)
#the current kernel version number
LINUX_KERNEL:=$(shell uname -r)
#the absolute path
LINUX_KERNEL_PATH:=/usr/src/kernels/$(LINUX_KERNEL)
#complie object
all:
        make -C $(LINUX_KERNEL_PATH) M=$(CURRENT_PATH) modules
#clean
clean:
        make -C $(LINUX_KERNEL_PATH) M=$(CURRENT_PATH) clean
\end{code-block}

然后执行make。执行完毕之后,会在当前目录生成hello\_kernel.ko,这个文件即是我们
所需要的内核模块。执行insmod hello\_kernel.ko,在/var/log/message当中,会发现有hello的输出,执行
rmmod hello\_kernel,在/var/log/message当中,会发现有goodbyd的输出。整个简单的模块
就算完成了。

\subsection{Linux的进程遍历}
一个进程是由进程控制块(PCB),代码段和数据段组成的;并且,OS通常是通过PCB来感知
一个进程的存在。其实PCB就是操作系统对每个进程的代码描述。linux内核中使用task\_struct
结构来描述一个PCB(具体可以在linux/kernel/sched.c查看源码);多个进程则常常使用双链表
等来进行组织。比如可运行状态的进程组成可运行队列,等待状态的进程组成等待队列等。

list\_head为linux内核当中常用的数据结构,用于构造双链表,关于list\_head的具体用法,可以
参见c部分的宏定义高级使用部分。而task\_struct的定义类似于如下的代码:
\begin{code-block}{c}
struct task_struct {
        struct thread_info    thread_info;
        struct list_head      tasks;
};
\end{code-block}

由于该结构体当中存在list\_head的变量,因此,我们可以利用该变量来访问整个task\_strut,
进而获取我们需要的信息。完整代码如下:
\begin{code-block}{c}
#include <linux/init.h>
#include <linux/module.h>
#include <linux/kernel.h>
#include <linux/sched.h>
#include <linux/sem.h>
#include <linux/list.h>

MODULE_LICENSE("GPL");
static int hello_init(void)
{
        printk(KERN_ALERT "hello,I am zhangjl\n");
        return 0;
}

static int traverse_init(void)
{
       struct task_struct *pos;
       struct list_head *current_head;
       int count=0;
       printk("Traversal module is working..\n");
       current_head=&(current->tasks);
       list_for_each_entry(pos,current_head,tasks)
       {
              count++;
              printk("[process %d]: %s\'s pid is %d\n",count,pos->comm,pos->pid);
       }
       printk(KERN_ALERT"The number of process is:%d\n",count);
       return 0;
}

static void hello_exit(void)
{
    printk(KERN_ALERT "goodbye,kernel\n");
    traverse_init();
}

module_init(hello_init);
module_exit(hello_exit);
MODULE_AUTHOR("zhangjl");
MODULE_DESCRIPTION("This is a simple example!\n");
MODULE_ALIAS("A simplest example");

\end{code-block}

其中,current是一个宏,即为系统内正在运行的进程。编译该文件,然后加载该模块,在
系统日志当中,即可发现对应的输出。

\subsection{Linux的进程间通信(IPC)}
Linux常见的进程间通信模式主要如下:
\begin{itemize}
    \item 管道pipe

            管道是一种半双工的通信方式,数据只能单向流动,而且只能在具有亲缘关系的进程间使用。进程的亲缘关系通常是指父子进程关系。
    \item 命名管道FIFO

            有名管道也是半双工的通信方式,但是它允许无亲缘关系进程间的通信。
    \item 消息队列MessageQueue

            消息队列是由消息的链表,存放在内核中并由消息队列标识符标识。消息队列克服了信号传递信息少、管道只能承载无格式字节流以及缓冲区大小受限等缺点。
    \item 共享存储SharedMemory

            共享内存就是映射一段能被其他进程所访问的内存,这段共享内存由一个进程创建,但多个进程都可以访问。共享内存是最快的 IPC 方式,它是针对其他进程间通信方式运行效率低而专门设计的。它往往与其他通信机制,如信号两,配合使用,来实现进程间的同步和通信。
    \item 信号量Semaphore

            信号量是一个计数器,可以用来控制多个进程对共享资源的访问。它常作为一种锁机制,防止某进程正在访问共享资源时,其他进程也访问该资源。因此,主要作为进程间以及同一进程内不同线程之间的同步手段。
    \item 套接字Socket

            套解口也是一种进程间通信机制,与其他通信机制不同的是,它可用于不同及其间的进程通信。
    \item 信号 ( sinal )

            信号是一种比较复杂的通信方式,用于通知接收进程某个事件已经发生。
\end{itemize}

\subsubsection{管道方式}
管道,通常指无名管道,是 UNIX 系统IPC最古老的形式。
\begin{itemize}
    \item 半双工

            数据只能在一个方向上流动,具有固定的读端和写端。
    \item 亲缘关系

            只能用于具有亲缘关系的进程之间的通信(也是父子进程或者兄弟进程之间)。
    \item 特殊文件

            对于它的读写也可以使用普通的read、write 等函数。但是它不是普通的文件,并不属于其他任何文件系统,并且只存在于内存中。
\end{itemize}

当一个管道建立时,它会创建两个文件描述符:fd[0]为读而打开,fd[1]为写而打开。如下图:\nameref{fig:pipe}
\begin{figure}[H]
  \centering
  \includegraphics[width=\linewidth]{pipe.png}
  \caption{管道}
  \label{fig:pipe}
\end{figure}
需要注意的是,fd[0]永远用于读取,不管是子进程还是父进程,都只能从fd[0]读取;fd[1]永远用于写入,子进程和父进程都只能从
fd[1]写入。如果fd的使用搞反,则会导致消息无法正常传递。

单个进程中的管道几乎没有任何用处。所以,通常调用 pipe 的进程接着调用 fork,这样就创建了父进程与子进程之间的 IPC 通道。如下图所示:\nameref{fig:fork_pipe}
\begin{figure}[H]
  \centering
  \includegraphics[width=\linewidth]{fork_pipe.png}
  \caption{fork管道}
  \label{fig:fork_pipe}
\end{figure}

使用管道的具体方式如下:
\begin{code-block}{c}
#include <stdio.h>
#include <unistd.h>

int main(int argc, char * argv[])
{
        int fd[2];
        pid_t pid;
        char buf[20];

        if(0 > pipe(fd))
        {
                printf("Create Pipe Error!\n");
                return -1;
        }
        if(0 > (pid = fork()))
        {
                printf("Fork error\n");
                return -1;
        }
        if(0 == pid)
        {
#if 1
                // 父进程接收
                close(fd[1]);
                read(fd[0], buf, 20);
                printf("%s in pid %d\n", buf, pid);
                close(fd[0]);
#else
                // 父进程输入
                printf("pid: %d\n", pid);
                close(fd[0]);
                write(fd[1], "Hello World\n", 12);
                close(fd[1]);
#endif
        }
        else{
#if 1
                // 子进程输入
                printf("pid: %d\n", pid);
                close(fd[0]);
                write(fd[1], "Hello World\n", 12);
                close(fd[1]);
#else
                // 子进程接收
                close(fd[1]);
                read(fd[0], buf, 20);
                printf("%s in pid %d\n", buf, pid);
                close(fd[0]);
#endif
        }
        return 0;
}
\end{code-block}

\subsubsection{命名管道FIFO}
FIFO,也称为命名管道,它是一种文件类型。FIFO可以在无关的进程之间交换数据,与无名管道不同。FIFO有路径名与之相关联,它以一种特殊设备文件形式存在于文件系统中。
通常使用mkfifo创建一个命名管道。一旦创建了一个 FIFO,就可以用一般的文件I/O函数操作它。FIFO的通信方式类似于在进程中使用文件来传输数据,只不过FIFO类型文件同时具有管道的特性。
在数据读出时,FIFO管道中同时清除数据,并且“先进先出”。
下面的例子演示了使用 FIFO 进行 IPC 的过程。
Server端负责创建fifo,并保持监听。
\begin{code-block}{c}
// server.c
#include <stdio.h>
#include <stdlib.h>
#include <fcntl.h>
#include <errno.h>
#include <sys/stat.h>
#include <unistd.h>
#include <signal.h>

#define FIFO_PATH "/tmp/fifo"
static int fd = -1;
static void ctrl_c(int sig);

static inline void clean()
{
        if(0 < fd)
        {
                close(fd);  // 关闭FIFO文件
        }
        remove(FIFO_PATH);
}

int main(int argc, char * argv[])
{
        int len;
        char buf[1024];

        if (SIG_ERR == signal(SIGINT, ctrl_c))
        {
                printf("\ncan't catch SIGINT\n");
                goto finally;
        }

        if(mkfifo(FIFO_PATH, 0666) < 0 && errno!=EEXIST) // 创建FIFO管道
                perror("Create FIFO Failed");

        if((fd = open(FIFO_PATH, O_RDONLY)) < 0)  // 以读打开FIFO
        {
                perror("Open FIFO Failed");
                exit(1);
        }

        while(1)
        {
                len = read(fd, buf, 1024);
                if (0 < len)
                {
                        printf("Read message: %s", buf);
                }
                else if (0 > len)
                {
                        perror("Unexpected error\n");
                        break;
                }
        }

finally:
        clean();
        return 0;
}

void ctrl_c(int sig)
{
        if (SIGINT == sig)
        {
                printf("Recevied ctrl+c interrupt, try to clean the env\n");
                clean();
                exit(0);
        }
}
\end{code-block}

Client端负责连接fifo,并通过fifo进行通信。
\begin{code-block}{c}
// client.c
#include <stdio.h>
#include <stdlib.h>   // exit
#include <fcntl.h>    // O_WRONLY
#include <sys/stat.h>
#include <time.h>     // time
#include <unistd.h>

#define FIFO_PATH "/tmp/fifo"

int main(int argc, char * argv[])
{
        int fd;
        int n, i;
        char buf[1024];
        time_t tp;
        printf("I am %d process.\n", getpid()); // 说明进程ID

        if((fd = open(FIFO_PATH, O_WRONLY)) < 0) // 以写打开一个FIFO
        {
                perror("Open FIFO Failed");
                exit(1);
        }
        for(;;)
        {
                time(&tp);  // 取系统当前时间
                n=sprintf(buf,"Process %d's time is %s",getpid(),ctime(&tp));
                printf("Send message: %s", buf); // 打印
                if(write(fd, buf, n+1) < 0)  // 写入到FIFO中
                {
                        perror("Write FIFO Failed");
                        close(fd);
                        exit(1);
                }
                sleep(1);  // 休眠1秒
        }
        close(fd);  // 关闭FIFO文件
        return 0;
}
\end{code-block}

稍微特殊的情况是,在server端的代码中,加入了对ctrl+c的中断识别操作,确保server端可以执行对应的扫尾工作。本身对ctrl+c的中断
操作属于信号量和中断的范畴。上述例子可以展成客户进程—服务器进程通信的实例,可以打开多个客户端
向一个服务器发送请求信息,server端实时监控着FIFO的读端,
在之后的内容会有更加详细的讲解。当有数据时,读出并进行处理,但是有一个关键的问题是,
每一个客户端必须预先知道服务器提供的FIFO接口,下图显示了这种安排\nameref{fig:fifo}
\begin{figure}[H]
  \centering
  \includegraphics[width=\linewidth]{fifo.png}
  \caption{FIFO管道}
  \label{fig:fifo}
\end{figure}

\subsubsection{消息队列}
消息队列,是消息的链接表,存放在内核中。一个消息队列由一个标识符(即队列ID)来标识。

消息队列拥有自己的一些特点:
\begin{itemize}
    \item 消息队列是面向记录的,其中的消息具有特定的格式以及特定的优先级。
    \item 消息队列独立于发送与接收进程。进程终止时,消息队列及其内容并不会被删除。
    \item 消息队列可以实现消息的随机查询,消息不一定要以先进先出的次序读取,也可以按消息的类型读取。
\end{itemize}

消息队列的主要原型在如下的头文件当中:
\begin{code-block}{c}
#include <sys/msg.h>
// 创建或打开消息队列:成功返回队列ID,失败返回-1
int msgget(key_t key, int flag);
// 添加消息:成功返回0,失败返回-1
int msgsnd(int msqid, const void *ptr, size_t size, int flag);
// 读取消息:成功返回消息数据的长度,失败返回-1
int msgrcv(int msqid, void *ptr, size_t size, long type,int flag);
// 控制消息队列:成功返回0,失败返回-1
int msgctl(int msqid, int cmd, struct msqid_ds *buf);
\end{code-block}

在以下两种情况下,msgget将创建一个新的消息队列:
\begin{itemize}
    \item 如果没有与键值key相对应的消息队列,并且flag中包含了IPC\_CREAT标志位。
    \item key参数为IPC\_PRIVATE
\end{itemize}

函数msgrcv在读取消息队列时,type参数有下面几种情况:
\begin{itemize}
    \item type == 0,返回队列中的第一个消息
    \item type > 0,返回队列中消息类型为 type 的第一个消息
    \item type < 0,返回队列中消息类型值小于或等于 type 绝对值的消息,如果有多个,则取类型值最小的消息
\end{itemize}

可以看出,type值非0时用于以非先进先出次序读消息。也可以把type看做优先级的权值。
下面的例子使用消息队列进行IPC,服务端程序一直在等待特定类型的消息,当收到该类型的
消息以后,发送另一种特定类型的消息作为反馈,客户端读取该反馈并打印出来。
\begin{code-block}{c}
// server.c
#include <stdio.h>
#include <stdlib.h>
#include <sys/msg.h>
#include <unistd.h>

// 用于创建一个唯一的key
#define MSG_FILE "/etc/passwd"

// 消息结构
struct msg_form {
        long mtype;
        char mtext[256];
};

int main()
{
        int msqid;
        key_t key;
        struct msg_form msg;
        // 获取key值
        // 只有server端和client端获得的key相同,server端和client端才能进行通信
        if((key = ftok(MSG_FILE,'z')) < 0)
        {
                perror("ftok error");
                exit(1);
        }

        // 打印key值
        printf("Message Queue - Server key is: %d.\n", key);

        // 创建消息队列
        if ((msqid = msgget(key, IPC_CREAT|0777)) == -1)
        {
                perror("msgget error");
                exit(1);
        }

        // 打印消息队列ID及进程ID
        printf("My msqid is: %d.\n", msqid);
        printf("My pid is: %d.\n", getpid());

        // 循环读取消息
        for(;;)
        {
                msgrcv(msqid, &msg, 256, 888, 0);// 返回类型为888的第一个消息
                printf("Server: receive msg.mtext is: %s.\n", msg.mtext);
                printf("Server: receive msg.mtype is: %d.\n", msg.mtype);

                msg.mtype = 999; // 客户端接收的消息类型
                sprintf(msg.mtext, "hello, I'm server %d", getpid());
                msgsnd(msqid, &msg, sizeof(msg.mtext), 0);
        }
        return 0;
}
\end{code-block}

而客户端的代码有区别
\begin{code-block}{c}
// client.c
#include <stdio.h>
#include <stdlib.h>
#include <sys/msg.h>
#include <unistd.h>

// 用于创建一个唯一的key
#define MSG_FILE "/etc/passwd"

// 消息结构
struct msg_form {
        long mtype;
        char mtext[256];
};

int main()
{
        int msqid;
        key_t key;
        struct msg_form msg;

        // 获取key值
        if ((key = ftok(MSG_FILE, 'z')) < 0)
        {
                perror("ftok error");
                exit(1);
        }

        // 打印key值
        printf("Message Queue - Client key is: %d.\n", key);

        // 打开消息队列
        if ((msqid = msgget(key, IPC_CREAT|0777)) == -1)
        {
                perror("msgget error");
                exit(1);
        }

        // 打印消息队列ID及进程ID
        printf("My msqid is: %d.\n", msqid);
        printf("My pid is: %d.\n", getpid());

        // 添加消息,类型为888
        msg.mtype = 888;
        sprintf(msg.mtext, "hello, I'm client %d", getpid());
        msgsnd(msqid, &msg, sizeof(msg.mtext), 0);

        // 读取类型为999的消息
        msgrcv(msqid, &msg, 256, 999, 0);
        printf("Client: receive msg.mtext is: %s.\n", msg.mtext);
        printf("Client: receive msg.mtype is: %d.\n", msg.mtype);
        return 0;
}
\end{code-block}

比较有意思的是,内核的消息队列和真正的消息队列服务器行为一致。当消息发出,没有接收者
时,依然会存在消息积压,只不过这些消息是积压在内核空间的。当接收者出现之后,这些
积压在内核空间的消息,还是会被正确投递。

\subsubsection{信号量}
信号量(semaphore)与已经介绍过的IPC结构不同,它是一个计数器。信号量用于实现进程间的互斥与同步,而不是用于存储进程间通信数据。
\begin{itemize}
    \item 信号量用于进程间同步,若要在进程间传递数据需要结合共享内存。
    \item 信号量基于操作系统的PV操作,程序对信号量的操作都是原子操作
    \item 每次对信号量的PV操作不仅限于对信号量值加1或减1,而且可以加减任意正整数。
    \item 支持信号量组。
\end{itemize}

最简单的信号量是只能取0和1的变量,这也是信号量最常见的一种形式,叫做二值信号量(Binary Semaphore)。而可以取多个正整数的信号量被称为通用信号量。
Linux下的信号量函数都是在通用的信号量数组上进行操作,而不是在一个单一的二值信号量上进行操作。实际应用时,我们每次都需要创建一个信号量集,
即使此集合只包含一个信号量。一般我们通过下面函数去创建或者打开一个信号量集。
\begin{code-block}{c}
int semget(key_t key,int nsems,int semflg);
\end{code-block}

当semflg=IPC\_CREATE时,如果当前系统中不存在此信号量集合(key值不存在),
那么semget函数完成一个信号量的创建;否则,semget函数打开这个已存在的信号量集。
当semflg=IPC\_CREATE|IPC\_EXCL时,只会完成创建,如果key值对应的信号量集合以存在,
那么直接返回错误,错误代码为EEXIST。这并不难理解,和open文件的情况类似。此函数
成功执行返回信号量集的标示符,否则为-1。

而常用的信号量函数如下:
\begin{code-block}{c}
#include <sys/sem.h>
// 创建或获取一个信号量组:若成功返回信号量集ID,失败返回-1
int semget(key_t key, int num_sems, int sem_flags);
// 对信号量组进行操作,改变信号量的值:成功返回0,失败返回-1
int semop(int semid, struct sembuf semoparray[], size_t numops);
// 控制信号量的相关信息
int semctl(int semid, int sem_num, int cmd, ...);
\end{code-block}

当semget创建新的信号量集合时,必须指定集合中信号量的个数(即num\_sems),通常为1;
如果是引用一个现有的集合,则将sems\_num指定为 0 。sembuf结构的定义如下:
\begin{code-block}{c}
struct sembuf
{
    short sem_num; // 信号量组中对应的序号,0~sem_nums-1
    short sem_op;  // 信号量值在一次操作中的改变量
    short sem_flg; // IPC_NOWAIT, SEM_UNDO
}
\end{code-block}

通过semid和sem\_num两个字段就可以确定信号量集中的指定信号量。sem\_op取不同的值就会
产生不同的操作。特别的,如果其值为0,则此时sem\_op操作的作用是测试信号量的值是否为0。
sem\_op是一次操作中的信号量的改变量,若sem\_op > 0,表示进程释放相应的资源数,将
sem\_op的值加到信号量的值上。如果有进程正在休眠等待此信号量,则唤醒他们。

若sem\_op < 0,请求sem\_op的绝对值的资源,如果相应的资源数可以满足请求,则将该信号量的值减去sem\_op的绝对值,
函数成功返回。当相应的资源数不能满足请求时,这个操作与sem\_flg有关。sem\_flg 指定IPC\_NOWAIT,则semop函数出错
返回EAGAIN。sem\_flg 没有指定IPC\_NOWAIT,则将该信号量的semncnt值加1,然后进程挂起直到下述情况发生:当相应的
资源数可以满足请求,此信号量的semncnt值减1,该信号量的值减去sem\_op的绝对值。成功返回;此信号量被删除,函数
smeop出错返回EIDRM;进程捕捉到信号,并从信号处理函数返回,此情况下将此信号量的semncnt值减1,函数semop出错
返回EINTR。

若sem\_op==0,进程阻塞直到信号量的相应值为0:当信号量已经为0,函数立即返回。如果信号量的值不为0,则依据
sem\_flg决定函数动作:sem\_flg指定IPC\_NOWAIT,则出错返回EAGAIN。sem\_flg没有指定IPC\_NOWAIT,则将该信号量的
semncnt值加1,然后进程挂起直到下述情况发生:信号量值为0,将信号量的semzcnt的值减1,函数semop成功返回;此
信号量被删除,函数smeop出错返回EIDRM;进程捕捉到信号,并从信号处理函数返回,在此情况将此信号量的semncnt值
减1,函数semop出错返回EINTR。

在semctl函数中的命令有多种,这里就说两个常用的:SETVAL:用于初始化信号量为一个已知的值。所需要的值作为联合
semun的val成员来传递。在信号量第一次使用之前需要设置信号量。IPC\_RMID:删除一个信号量集合。如果不删除信号量,
它将继续在系统中存在,即使程序已经退出,它可能在你下次运行此程序时引发问题,而且信号量是一种有限的资源。

一个简单的例子。如果不使用信号量,父进程会先于子进程输出。但是,使用信号量之后,子进程会先于父进程执行。
\begin{code-block}{c}
#include <stdio.h>
#include <stdlib.h>
#include <sys/sem.h>
#include <unistd.h>

union semun
{
        int              val; /*for SETVAL*/
        struct semid_ds *buf;
        unsigned short  *array;
};

// 初始化信号量
int init_sem(int sem_id, int value)
{
        union semun tmp;
        tmp.val = value;
        if(semctl(sem_id, 0, SETVAL, tmp) == -1)
        {
                perror("Init Semaphore Error");
                return -1;
        }
        return 0;
}

// P操作:
//    若信号量值为1,获取资源并将信号量值-1
//    若信号量值为0,进程挂起等待
int sem_p(int sem_id)
{
        struct sembuf sbuf;
        sbuf.sem_num = 0; /*序号*/
        sbuf.sem_op = -1; /*P操作*/
        sbuf.sem_flg = SEM_UNDO;

        if(semop(sem_id, &sbuf, 1) == -1)
        {
                perror("P operation Error");
                return -1;
        }
        return 0;
}

// V操作:
//    释放资源并将信号量值+1
//    如果有进程正在挂起等待,则唤醒它们
int sem_v(int sem_id)
{
        struct sembuf sbuf;
        sbuf.sem_num = 0; /*序号*/
        sbuf.sem_op = 1;  /*V操作*/
        sbuf.sem_flg = SEM_UNDO;

        if(semop(sem_id, &sbuf, 1) == -1)
        {
                perror("V operation Error");
                return -1;
        }
        return 0;
}

// 删除信号量集
int del_sem(int sem_id)
{
        union semun tmp;
        if(semctl(sem_id, 0, IPC_RMID, tmp) == -1)
        {
                perror("Delete Semaphore Error");
                return -1;
        }
        return 0;
}


int main()
{
        int sem_id;  // 信号量集ID
        key_t key;
        pid_t pid;

        // 获取key值
        if((key = ftok(".", 'z')) < 0)
        {
                perror("ftok error");
                exit(1);
        }

        // 创建信号量集,其中只有一个信号量
        if((sem_id = semget(key, 1, IPC_CREAT|0666)) == -1)
        {
                perror("semget error");
                exit(1);
        }

        // 初始化:初值设为0资源被占用
        init_sem(sem_id, 0);

        if((pid = fork()) == -1)
                perror("Fork Error");
        else if(pid == 0) /*子进程*/
        {
                //sleep(2);
                printf("Process child: pid=%d\n", getpid());
                sem_v(sem_id);  /*释放资源*/
        }
        else  /*父进程*/
        {
                sem_p(sem_id);   /*等待资源*/
                printf("Process father: pid=%d\n", getpid());
                sem_v(sem_id);   /*释放资源*/
                del_sem(sem_id); /*删除信号量集*/
        }
        return 0;
}
\end{code-block}

\section{Linux系统编程}

\subsection{IO函数}
Linux系统当中,通常需要处理IO,而IO的处理,在Linux的函数当中,主要有4个函数:
\begin{itemize}
  \item open //fcntl.h
  \item write //unistd.h
  \item read //unistd.h
  \item close //unistd.h
\end{itemize}

实现简单的touch命令的功能
\begin{code-block}{c}
#include <stdio.h>
#include <unistd.h>
#include <fcntl.h>

int main(int argc, char * argv[])
{
        // 第3个参数可以直接写为0644
        int fd = open(argv[1], O_CREAT|O_WRONLY,
                S_IRUSR|S_IWUSR|S_IRGRP|S_IROTH);
        if (0>fd)
        {
                printf("Cannot create file %s\n", argv[1]);
                return -1;
        }
        printf("Create file %s success\n", argv[1]);
        close(fd);
        return 0;
}
\end{code-block}

但是,由于Linux操作系统本身存在umask(默认为022),因此,如果上述的第3个参数写作0777,
生成的文件的权限与umask进行亦或计算之后,实际上,文件的权限还是755,并不是我们所期待的
777。如果需要保持设置的权限与生成的文件权限完全一致,需要执行如下命令:
\begin{code-block}{bash}
umask 000
# 后续再执行代码,生成文件
\end{code-block}

Open函数只能生成普通文件,如果是管道、字符设备之类的,则无法使用open函数进行创建。
另外,如果只是需要打开文件,并不是创建文件,则open函数的第3个参数不需要。
除此之外,还需要注意一下,文件的打开模式
\begin{itemize}
  \item O\_TRUNC:覆盖文件
  \item O\_EXCL : 与O\_CREAT合用,如果对应文件已经存在,则提示错误
\end{itemize}

Open函数一旦调用,Linux内核会在内核空间打开3个文件描述符,分别是0,1,2。

而对应的,也可以利用write函数向打开的文件句柄当中写入内容
\begin{code-block}{c}
#include <stdio.h>
#include <unistd.h>
#include <fcntl.h>

int main(int argc, char * argv[])
{
        // 第3个参数可以直接写为0644
        int fd = open(argv[1], O_CREAT|O_RDWR,
                S_IRUSR|S_IWUSR|S_IRGRP|S_IROTH);
        if (0>fd)
        {
                printf("Cannot create file %s\n", argv[1]);
                return -1;
        }
        printf("Create file %s success\n", argv[1]);

        char msg[] = "hello world";
        write(fd, msg, sizeof(msg)/sizeof(char)); //会写入一个文件结束符,特殊符号
                                                  // 如果不需要,则将长度-1即可
        close(fd);
        return 0;
}
\end{code-block}

相应的,也可以利用read函数读取打开文件的内容:
\begin{code-block}{c}
#include <stdio.h>
#include <unistd.h>
#include <fcntl.h>
#include <string.h>

int main(int argc, char * argv[])
{
        int fd = open(argv[1], O_RDONLY);
        if (0>fd)
        {
                printf("Cannot open file %s\n", argv[1]);
                return -1;
        }
        printf("Open file %s success\n", argv[1]);

        size_t read_ret = 0;
#if 0
        // 连续多次读取,并非一次性读完
        size_t total = 0;
        char readbuf[128];
        while ((read_ret=read(fd, readbuf, 127))>0) // 每次只能读取max-1,否则末尾存在特殊字符,可能出现溢出
        {
                total += read_ret;
                printf("Read %d chars \n", read_ret);
                printf("The content of file is %s \n", readbuf);
                memset(readbuf, 0, 128);
        }
        printf("The total sizeof file is %d\n", total);
#else
        // 一次性读取
        char readbuf[1024];
        read_ret=read(fd, readbuf, 1024);
        printf("Read %d chars \n", read_ret);
        printf("The content of file is %s \n", readbuf);
#endif
        close(fd);
        return 0;
}
\end{code-block}

高级一点的,我们就可以使用read和write函数来实现一个简单的文件拷贝功能。
\begin{code-block}{c}
#include <stdio.h>
#include <unistd.h>
#include <fcntl.h>
#include <string.h>

int main(int argc, char * argv[])
{
        int readrd = 0, writefd = 0;
        if (0 >= (readrd = open(argv[1], O_RDONLY)))
        {
                printf("Cannot open the source file %s\n", argv[1]);
                return -1;
        }
        if (0 >= (writefd = open(
                argv[2], O_CREAT|O_TRUNC|O_WRONLY, 0644)))
        {
                printf("Cannot create the target file %s\n", argv[2]);
                return -1;
        }

        unsigned char buffer[128];
        memset(buffer, 0, 128);

        size_t readret = 0, writeret = 0;
        while(0 < (readret = read(readrd, buffer, 127)))
        {
                if (0 > (writeret = write(writefd, buffer, readret)))
                {
                        printf("Cannot write content to write file\n");
                        return -1;
                }
                memset(buffer, 0, 128);
        }

        close(readrd);
        close(writefd);
        return 0;
}
\end{code-block}

由于读取使用的是unsigned char,因此,上述文件也可以直接拷贝二进制文件。

\subsection{标准IO函数}
Linux的IO操作包括文件IO和标准IO。所谓的文件IO,即直接调用内核提供的系统调用函数,一般需要使用头文件unistd.h当中的函数;而
标准IO,则是通过调用C的库函数,间接的调用系统调用函数,通常的,使用的头文件stdio.h当中的函数。从功能上看,标准IO与文件IO是
相同的,但是,细节上,他们存在区别。
\begin{code-block}{c}
#include <stdio.h>
#include <unistd.h>

int main(int argc, char * argv[])
{
        char  buffer[] = "hello world";
        printf("stdio %s", buffer);
        write(1, buffer, 11);
        while(1);
        return 0;
}
\end{code-block}

上述代码编译之后,运行,只有hello world能够输出,而printf的stdio hello world则无法输出。问题在于缓存。
Linux程序当中存在几种缓存:
\begin{itemize}
  \item 用户空间缓存:即想从内核读写的数据,即上述代码当中buffer
  \item 内核空间缓存:没打开一个文件,内核会在内核空间开辟一块缓存,这个称之为内核空间的缓存
  \item 库缓存:标准IO的库函数的缓存
\end{itemize}

文件IO中的写,即是将用户空间的缓存写入到内核空间缓存当中;反之,文件IO的读,则是将内核空间的缓存读写到用户空间的缓存当中。
而调用标准IO之后,数据会从用户空间写入到库缓存,当写入的数据包含\textbackslash n时,或者库缓存空间写满时,才会向内核缓存空间提交数据。
因此,如果上述代码修改为
\begin{code-block}{c}
printf("stdio %s\n", buffer); //或者直接将库缓存写满
while(1);
\end{code-block}
则会直接输出。另外,库缓存的大小默认为1024个字节。

常用fgets,gets,printf,sprintf,fprintf,fputs,puts,scanf这些函数在遇到\textbackslash n或者写满缓存时,即
调用系统调用函数,称之为行缓存函数;而fread,fwrite只有在写满缓存之后再调用系统调用函数,这些则称之为全缓存函数;
而只要调用,则会将内容和数据写入到内核当中的函数,称之为无缓存函数,注意,stderr是无缓存的,而stdout则是行缓存的。
fclose函数在关闭文件之前,会刷新缓存当中的数据到文件当中。

需要注意的是fputc是缓存函数,但是,他不是行缓存函数,立即生效的话,需要使用fflush函数进行强制刷新。

除此之外,在标准IO当中,读取文件有可能会出现错误,而fgets函数读取正常时,返回读取到的内容,这个内容与fgets函数的第一个参数的结果一致,
如果读取错误,则会返回一个空指针(char)。但是无法准确判断这个错误是什么类型。判断错误的准确类型,可以使用feof和ferror函数进行判断。
前者表示读取到了文件末尾,而后一个则表示真的文件读取错误,如下代码所示:
\begin{code-block}{c}
FILE *fp = fopen("test.c")
char buffer[128];
char * read_ret = NULL;
read_ret = fgets(buffer, 128, fp);
if (NULL == read_ret)
{
        if(feof(fp))
        {
                printf("Read the end of file\n");
        }
        if(ferror(fp))
        {
                printf("Read error from the stream\n");
        }
}
\end{code-block}

与文件IO相对应的,标准IO使用fopen函数进行文件的创建和读写。但是需要特别注意的是,实际上,fopen函数创建的函数的权限始终是
666,但是由于umask的存在,因此,fopen函数创建的文件的最终权限为644。

全缓存函数fread和fwrite在使用的时候会调用syscall,写入到内核缓存当中,最后写入到硬件当中(文件)。同样的,我们也可以用fread和fwrite实现
Linux的cat命令,简单的如下:
\begin{code-block}{c}
if(NULL == (fp = fopen(argv[1], "rb")))
{
        printf("Cannot open the file %s\n", argv[1]);
        return -1;
}

unsigned char buffer[128];
memset(buffer, 0, 128);
while(0 < fread(buffer, sizeof(char), 128, fp))
{
        fwrite(buffer, sizeof(char), 128, stdout);
        memset(buffer, 0, 128);
        if(feof(fp))
        {
                printf("Read the the of file\n");
                break;
        }
}

fclose(fp); // 调用fflush,直接写入到内核缓存当中
return 0;
\end{code-block}

从执行效率上说,fgetc/fputc<fgets/fputs<fread/fwrite,主要原因在于fread基本都是在内核空间操作,效率有保证。因此,在有高效率要求的情况下,尽可能的使用fread和fwrite
作为IO的操作函数。

\subsection{目录IO}
除了文件IO和标准IO之外,Linux还提供了针对路径(目录)的IO操作函数,具体如图\nameref{fig:dirio}所示
\begin{figure}[H]
  \centering
  \includegraphics[scale=1]{dirio.png}
  \caption{Linux的目录IO函数}
  \label{fig:dirio}
\end{figure}

只是需要注意的是,mkdir函数在sys/stat.h当中,其他的函数大部分在dirent.h当中。目录的创建,可以使用如下的代码:
\begin{code-block}{c}
int ret = mkdir("zhangjl", 0777);
if(0 > ret)
{
        printf("Failed to create dir\n");
        return -1;
}
return 0;
\end{code-block}

而打开目录,则可以如下操作:
\begin{code-block}{c}
#include <dirent.h>

int main(int argc, char * argv[])
{
        DIR *dp = opendir("/root");
        if(NULL ==  dp)
        {
                printf("Failed to open dir\n");
                return -1;
        }

        closedir(dp);
        return 0;
}
\end{code-block}

读取目录内容,则可以使用readdir函数。由于readdir函数在多个头文件当中都有定义,此处应当使用dirent.h当中的函数。
具体的使用如下代码:
\begin{code-block}{c}
#include <stdio.h>
#include <dirent.h>

int main(int argc, char * argv[])
{
        DIR *dp = opendir("/root/cprograms/dirio");
        if(NULL ==  dp)
        {
                printf("Failed to open dir\n");
                return -1;
        }

        struct dirent * dir = NULL;
        while (NULL != (dir = readdir(dp)))
        {
                printf("The inode is %lu, and name is %s\n",
                        dir->d_ino, dir->d_name);
        }

        closedir(dp);
        return 0;
}
\end{code-block}
上述代码需要注意的有几点:
\begin{enumerate}
  \item readdir返回的是一个指针,而这个指针,实际上是一个链表的头指针,因此,通常情况下需要反复调用该函数,读取链表上的所有元素
  \item readdir只能返回一级文件目录当中的内容,子目录以及子目录下的子目录,则无法一次性读取
  \item rewinddir则会将readdir所得到的指针,重新定位到这个链表的头节点,也可以使用seekdir进行指定地址的跳转。
\end{enumerate}

\subsection{Linux进程通信}
首先需要明确的是,在用户空间实现进程间通信是不可能的,需要在Linux内核空间当中进行;但是线程间的通信,在用户空间就可以实现。
最明显的,线程间的通信,通过全局变量即可实现,其原因主要就是多线程之间是共享内存的,如下简单代码:
\begin{code-block}{c}
#include <stdio.h>
#include <pthread.h>
#include <unistd.h>
int main_run = 0;
void *func(void *var)
{
        int i = 0;
        //while(!main_run); //如果需要父进程执行结束之后,再执行子线程,则开启本行注释即可
        for (; i <10; i++)
        {
                usleep(100);
                printf("This is fun i=%d\n", i);
        }
}

int main(int argc, char * argv[])
{
        int i = 0;
        char buf[] = "hello world\n";
        pthread_t tid;
        int ret = 0;
        ret = pthread_create(&tid, NULL, func, (void*)buf);
        if (0 > ret)
        {
                printf("Create thread failure\n");
                return -1;
        }
        for(i = 0; i < 10; i++)
        {
                usleep(100);
                printf("this is main fun i = %d\n", i);
        }
        main_run = 1;
        while(1);
        return 0;
}
\end{code-block}

注意,多线程编译时,需要加入-pthread参数,即
\begin{code-block}{bash}
gcc -pthread -o test test.c
\end{code-block}

但是,与线程不同,进程间的每一种通信方式都是基于文件IO的思想进行设计和实现的。

\subsubsection{管道通信}
管道是一种特殊的文件,由队列来实现,遵循先进先出的顺序。与open函数类似,open函数打开的文件描述符为0,1,2,而管道函数(pipe)
打开的文件描述符则固定为3,4,分别对应fd[0]和fd[1]。
\begin{code-block}{c}
#include <stdio.h>
#include <unistd.h>

int main(int argc, char * argv[])
{
        int fd[2];
        int ret = 0;
        if (0 > (ret=pipe(fd)))
        {
                printf("Cannot create pipe \n");
                return -1;
        }
        printf("%d, %d\n", fd[0], fd[1]);
        return 0;
}
\end{code-block}

由于管道本身是特殊文件,因此,也可以对管道进行读写,但是特别需要注意的是,fd[0]只允许进行读取,而fd[1]则只允许进行写入,如下:
\begin{code-block}{c}
#include <stdio.h>
#include <unistd.h>
#include <string.h>

int main(int argc, char * argv[])
{
        int fd[2];
        int ret = 0;
        if (0 > (ret=pipe(fd)))
        {
                printf("Cannot create pipe \n");
                return -1;
        }
        char buf[] = "hello linux";
        char readbuf[128];
        memset(readbuf, 0, 128);
        size_t writed = write(fd[1], buf, sizeof(buf)/sizeof(char));
        size_t readed = read(fd[0], readbuf, writed);
        printf("Read from pipe: %s\n", readbuf);
        close(fd[0]);
        close(fd[1]);
        return 0;
}
\end{code-block}

\begin{enumerate}
  \item 管道创建在内存当中,进程结束,空间释放,管道就不存在了
  \item 管道当中的数据,一旦读取完毕,就直接从管道当中删除了
  \item 如果管道当中没有内容,则读取操作会一直阻塞;反之,如果没有读取操作,一旦缓冲写满(65536),则写入操作会阻塞
  \item 管道最大为65536字节
  \item 无名管道只能实现父子进程之间的通信
\end{enumerate}

实现父子进程的通信如下:
\begin{code-block}{c}
#include <stdio.h>
#include <unistd.h>
#include <string.h>

int main(int argc, char * argv[])
{
        int fd[2];
        int ret = pipe(fd);
        int inter = 0;
        pid_t pid;
        pid = fork();
        if (0 > ret)
        {
                printf("Cannot create pipe \n");
                return -1;
        }
        if (0 == pid)
        {
                int i = 0;
                read(fd[0], &inter, 1);
                while(!inter);
                for (;i < 5; i++)
                {
                        printf("[%d]In child\n", i);
                }
        }
        if ( 0 < pid)
        {
                int i = 0;
                for(;i < 5; i++)
                {
                        printf("[%d]In parent\n", i);
                }
                inter = 1;
                write(fd[1], &inter, 1);
        }

        close(fd[0]);
        close(fd[1]);
        return 0;
}
\end{code-block}

与无名管道相对应的,则是命名管道,命名管道可以实现无亲缘关系的进程间通信。所谓命名管道,其实也是一个管道,但是,他是存在于文件系统
当中的,并不是仅仅只是在内存当中。命名管道的文件,每个文件节点都含有inode编号,并且其文件为p类型(即管道类型)。管道文件只含有inode
编号,不占用磁盘存储空间,与套接字,字符设备以及块设备一样。管道文件的创建,需要使用mkfifo函数(需要包含<sys/stat.h>头文件),不过,
该函数只是创建了管道文件的inode信息,并没有在内核当中创建管道,只有通过open函数打开这个创建成功的管道文件时,才会在内核空间创建对应
的管道。创建命名管道文件的示例如下:
\begin{code-block}{c}
#include <stdio.h>
#include <sys/stat.h>

int main(int argc, char * argv[])
{
        int ret = 0;
        if (0> (ret = mkfifo("/var/run/zhangjl", 0644)))
        {
                printf("Cannot create fifo file zhangjl\n");
                return -1;
        }
        printf("Create fifo file sucess\n");
        return 0;
}
\end{code-block}

而命名管道的使用,通常就是用于不同进程之间的相互通信。比如下面的例子:
\begin{code-block}{c}
#include <stdio.h>
#include <sys/stat.h>
#include <fcntl.h>
#include <unistd.h>

int main(int argc, char * argv[])
{
        int ret = 0;
        if (0> (ret = mkfifo("/var/run/zhangjl", 0644)))
        {
                printf("Cannot create fifo file zhangjl\n");
                return -1;
        }
        printf("Create fifo file sucess\n");

        int fd = 0;
        if (0 > (fd = open("/var/run/zhangjl", O_WRONLY)))
        {
                printf("Cannot open the named pipe\n");
                return -1;
        }

        for(ret = 0;ret < 5; ret++)
        {
                printf("This is first process [%d]\n", ret);
                usleep(100);
        }

        int completed_signal = 0;
        completed_signal = 100;
        write(fd, &completed_signal, 1);
        while(1);
        close(fd);
        return 0;
}
\end{code-block}

而另外的进程可以直接从该管道当中读取数据,如下:
\begin{code-block}{c}
#include <stdio.h>
#include <sys/stat.h>
#include <fcntl.h>
#include <unistd.h>

int main(int argc, char * argv[])
{
        int fd = 0;
        if (0 > (fd = open("/var/run/zhangjl", O_RDONLY)))
        {
                printf("Cannot open the named pipe\n");
                return -1;
        }
        int completed_signal = 0;
        read(fd, &completed_signal, 1);
        while(!completed_signal);

        int ret = 0;
        for(ret = 0;ret < 5; ret++)
        {
                printf("This is client process [%d]\n", ret);
                usleep(100);
        }

        close(fd);
        return 0;
}
\end{code-block}

\subsubsection{信号通信}
除了使用管道之外,还可以使用信号的方式进行通信。与管道不太一样的是,信号对象存在于内核当中,无需创建,本身已经存在了,并且,无法在用户空间进行信号的发送和接收。
在Linux当中,可以通过kill -l查看总共有多少信号(总共64种),如图\nameref{fig:signal}所示:
\begin{figure}[H]
  \centering
  \includegraphics[scale=0.4]{signal.png}
  \caption{Linux的信号种类}
  \label{fig:signal}
\end{figure}

使用信号,可以简单的实现kill的功能。在实现kill命令功能的时候,需要使用kill函数,具体示例如下:
\begin{code-block}{c}
#include <stdio.h>
#include <stdlib.h>
#include <signal.h>

int main(int argc, char *argv[])
{
        if(argc != 3)
        {
                printf("Usage kill signal pid\n");
                return -1;
        }
        int sig = 0, pid = 0;
        sig = atoi(argv[1]);
        pid = atoi(argv[2]);
        printf("sig is %d and pid in %d\n", sig, pid);
        kill(pid, sig);
        return 0;
}
\end{code-block}

除了使用kill进行信号的发送之外,还可以使用其他的函数进行信号的发送,比如常用的的raise,alarm等;信号的接收,通常采用pause,sleep
以及while(1)等方式;而信号的处理则通常交给signal进行。

Raise函数只会发送信号给自己,基本上等价于kill(getpid(), sig),即希望通过内核给自己发信号,常用于杀掉自身的进程,如下
\begin{code-block}{c}
#include <stdio.h>
#include <signal.h>

int main(int argc, char *argv[])
{
        printf("Raise before\n");
        raise(9);
        printf("Raise after\n");
        return 0;
}
\end{code-block}

上述代码,在编译之后运行,只有before能够输出,raise调用之后,自身进程被直接杀死,因此后续的after无法输出。

而alarm函数只会发送一个定时器信号,当程序接收到定时器信号之后,会终止对应的进程,如下:
\begin{code-block}{c}
#include <stdio.h>
#include <unistd.h>

int main(int argc, char * argv[])
{
        printf("Alarm Before\n");
        alarm(9);
        while(1); //等待9秒之后,该进程自动被终止
        printf("Alarm After\n");
        return 0;
}
\end{code-block}

因此,上述代码当中,after也是无法进行输出的。

而信号的接收,处理方式则有些不同。Pause函数会直接暂停当前的进程,如下:
\begin{code-block}{c}
#include <stdio.h>
#include <unistd.h>

int main(int argc, char * argv[])
{
        printf("Pause Before\n");
        pause();
        printf("Pause After\n");
        return 0;
}
\end{code-block}

Pasue函数一旦调用,则对应的进程会直接变为暂停状态,ps -ajx可以看到状态变为S。退出暂停状态的进程,可以直接使用Ctrl+C进行,而Ctrl+C本身
发送的就是一个终止信号。

上述的信号处理,通用的方式都是终止/暂停对应的进程,很明显并不是所有的场景都需要。因此,如何进行信号处理的自定义呢?我们需要采用signal
函数。Signal函数的定义如下:
\begin{code-block}{c}
void (*signal(int sig, void (*func)(int)))(int);
\end{code-block}

其中,func为一个函数指针,指向自定义的型号处理函数。除了自定义的信号处理函数之外,func这个函数指针还可以的取值为SIG\_IGN(忽略该信号)
和SIG\_DFL(采用系统默认方式处理信号)。简单的signal函数的使用如下:
\begin{code-block}{c}
#include <stdio.h>
#include <unistd.h>
#include <signal.h>

static int quit = 0;
void handler(int signalnum)
{
        printf("Recevied signal %d\n", signalnum);
        quit = 1;
}

int main(int argc, char * argv[])
{
        signal(SIGALRM, handler);
        alarm(9);
        while(!quit);
        printf("Using self defined function to handle signal\n");
        return 0;
}
\end{code-block}

Signal函数在用于子进程的退出处理当中,是比较常用的,比如:
\begin{code-block}{c}
#include <stdio.h>
#include <unistd.h>
#include <signal.h>
#include <stdlib.h>
#include <sys/wait.h>

void handler(int signum)
{
        int i = 0;
        while( i < 5)
        {
                printf("Receved signum %d\n", signum);
                i++;
        }
}

void clean(int signum)
{
        printf("Recevied signum %d, clean up the child process\n", signum);
        wait(NULL); // 需要使用wait函数,回收对应的进程,否则,子进程会成为僵尸进程
}

int main(int argc, char * argv[])
{
        pid_t pid;
        pid = fork();
        signal(SIGUSR1, handler);
        signal(SIGCHLD, clean);
        if (0 < pid)
        {
                int i =0;
                while(1)
                {
                        printf("This is the parent process [%d]\n", i++);
                        sleep(1);
                }
        }
        if(0 == pid)
        {
                sleep(5);
                //kill(getpid(), SIGUSR1);
                raise(SIGUSR1); // 可以直接替代上面的kill函数
                exit(0); // kill(getpid(), SIGCHLD); // 在子进程当中调用exit函数
                                                     // 相当于调用了kill函数,只不过
                                                     // 发送的信号是SIGCHLD,即杀死子进程
        }
        return 0;
}
\end{code-block}

需要注意,无名管道,命名管道以及信号,都是发生在内核空间当中,并没有发生在用户空间。
除了使用上述的方式实现进程间通信之外,在Linux当中,还可以使用IPC实现。而IPC对象包含了3种方式:
\begin{itemize}
  \item 共享内存
  \item 消息队列
  \item 信号量/灯
\end{itemize}

这些IPC对象同样是在内核空间,并没有发生在用户空间,IPC类似于Linux的文件IO操作的相关思想,可以针对文件IO与IPC做一个简单的类比,
如图\nameref{fig:IPC}所示
\begin{figure}[H]
  \centering
  \includegraphics[scale=0.8]{IPC.png}
  \caption{文件IO与IPC的对比}
  \label{fig:IPC}
\end{figure}

\subsubsection{共享内存}
共享内存通常需要使用shmget函数进行创建,而这个函数包含3个参数:
\begin{itemize}
  \item key:IPC\_PRIVATE或者是ftok函数的返回值
  \item size:共享内存的大小,bit
  \item shmflg:共享内存的权限,同open函数
\end{itemize}
共享内存的具体使用示例如下:
\begin{code-block}{c}
#include <stdio.h>
#include <sys/shm.h>

int main(int argc, char * argv[])
{
        int shmid = 0;
        if (0 > (shmid = shmget(IPC_PRIVATE, 128, 0777)))
        {
                printf("Create shared memory failed\n");
                return -1;
        }
        return 0;
}
\end{code-block}

共享内存创建完毕之后,可以直接使用Linux提供的命令进行查看和删除。
\begin{code-block}{c}
# 查看IPC对象,包括共享内存, 或者直接ipcs
ipcs -m -q -s
# 删除IPC对象
ipcrm -m <id>
\end{code-block}
在上述代码当中,创建共享内存使用的是IPC\_PRIVATE这个宏,因此,创建出来的共享内存的
key永远为0。可以改为使用ftok函数,给不同的共享内存分配不同的标识符(key),如下:
\begin{code-block}{c}
#include <stdio.h>
#include <sys/shm.h>
#include <sys/ipc.h>

int main(int argc, char * argv[])
{
        int shmid = 0;
        int key = ftok("sharedmem.c", 's');
        if (0 > key)
        {
                printf("Failed to create shamred memory key\n");
                return -1;
        }
        if (0 > (shmid = shmget(key, 128, IPC_CREAT | 0777)))
        {
                printf("Create shared memory failed\n");
                return -1;
        }
        printf("Shared memory object id is %d\n", shmid);
        return 0;
}
\end{code-block}

IPC\_PRIVATE与ftok创建的共享内存,其关系类似与无名管道和命名管道,也就是说,IPC\_PRIVATE只能用于有亲缘关系的进程间通信,
而ftok的共享内存,则是任意进程间都可以进行通信。共享内存创建完成之后,整个是放在内核空间的,因此,用户空间无法访问,但是,
可以通过映射的方式,将共享内存将这些共享内存映射到用户空间,用户空间可以直接操作这些内存。共享内存的映射,需要使用函数shmat实现。
Shmat函数包含3个参数:id表示共享内存的id号,shmaddr表示映射的地址,NULL表示自动分配,shmflg表示映射内存的权限,0可读可写。
与管道不同,共享内存是可以反复读取的,并且,一直存在与内核当中,直到被删除或者系统关闭。

而共享内存的删除,包含了2部分的操作:1是断开与用户空间的内存映射,这个操作可以使用shmdt函数实现;2是回收内核空间当中的共享内存,
需要使用函数shmctl函数进行操作。Shmctl函数的参数如下:
\begin{itemize}
  \item hmid:表示共享内存的id
  \item cmd:表示针对共享内存的操作,可选的有3个,IPC\_STAT,获取对象属性,IPC\_SET,设置对象属性,以及IPC\_RMID删除共享内存对象
  \item buf:当cmd为IPC\_SET或IPC\_STAT时,需要使用该参数表示对象属性
\end{itemize}

共享内存的整体使用,如下示例:
\begin{code-block}{c}
#include <stdio.h>
#include <sys/shm.h>
#include <sys/ipc.h>

int main(int argc, char * argv[])
{
        int shmid = 0;
        int key = ftok("sharedmem.c", 's');
        if (0 > key)
        {
                printf("Failed to create shamred memory key\n");
                return -1;
        }
        if (0 > (shmid = shmget(key, 128, IPC_CREAT | 0777)))
        {
                printf("Create shared memory failed\n");
                return -1;
        }
        printf("Shared memory object id is %d\n", shmid);
        char * buffer = NULL;
        if (NULL == (buffer = (char *)shmat(shmid, NULL, 0)))
        {
                printf("Cannot mapping shared memory to user namespace\n");
                return -1;
        }

        fgets(buffer, 128, stdin);
        printf("Shared memory data :%s\n", buffer);

        shmdt(buffer); // 删除用户空间的共享内存映射
        buffer = NULL;

        shmctl(shmid, IPC_RMID, NULL); // 删除内核空间的共享内存

        return 0;
}
\end{code-block}

共享内存也常常用于进程间通信,比如父子进程之间的通信,如下所示:
\begin{code-block}{c}
#include <stdio.h>
#include <sys/shm.h>
#include <sys/ipc.h>
#include <unistd.h>
#include <signal.h>

void parent_handler(int signum)
{
}

void child_handler(int signum)
{
}

int main(int argc, char * argv[])
{
        int shmid = 0;
        pid_t pid = 0;
        if (0 > (shmid = shmget(IPC_PRIVATE, 128, 0777)))
        {
                printf("Create shared memory failed\n");
                return -1;
        }
        printf("Shared memory object id is %d\n", shmid);

        pid = fork();
        char * buffer = NULL;
        if (0 < pid)
        {
                signal(SIGUSR2, parent_handler);
                printf("In parent process\n");
                if (NULL == (buffer = (char *)shmat(shmid, NULL, 0)))
                {
                        printf("Cannot mapping shared memory to user namespace in parent process\n");
                        return -1;
                }
                while(1)
                {
                        fgets(buffer, 128, stdin);
                        kill(pid, SIGUSR1); // 发送信号给子进程,唤醒子进程
                        pause(); // 暂停
                }
        }

        if (0 == pid)
        {
                signal(SIGUSR1, child_handler);
                if (NULL == (buffer = (char *)shmat(shmid, NULL, SHM_RDONLY)))
                {
                        printf("Cannot mapping shared memory to user namespace in child process\n");
                        return -1;
                }
                while(1)
                {
                        pause();
                        printf("The shared memory data is %s\n", buffer);
                        kill(getppid(), SIGUSR2); //发送信号给主进程,唤醒主进程
                }
        }

        shmdt(buffer);
        buffer = NULL;

        shmctl(shmid, IPC_RMID, NULL);

        return 0;
}
\end{code-block}

共享内存也可以使用实现没有亲缘关系的进程间的通信,示例如下:
\begin{code-block}{c}
// 服务端的代码
#include <stdio.h>
#include <sys/shm.h>
#include <sys/ipc.h>
#include <signal.h>
#include <stdlib.h>
#include <unistd.h>

typedef struct _buffer{
        int pid;
        char buf[128];
}buffer_t;

void hanlder(int signum){}

int main(int argc, char * argv[])
{
        signal(SIGUSR2, hanlder);
        pid_t pid = 0;
        int shmid = 0;
        buffer_t *buffer = NULL;

        int key = ftok("server.c", 's');
        if (0 > key)
        {
                printf("Failed to create shamred memory key\n");
                return -1;
        }
        if (0 > (shmid = shmget(key, sizeof(buffer_t), IPC_CREAT | 0777)))
        {
                printf("Create shared memory failed\n");
                return -1;
        }
        if (NULL == (buffer = (buffer_t *)shmat(shmid, NULL, 0)))
        {
                printf("Mapping shared memory failed \n");
                return -1;
        }

        buffer->pid = getpid(); // 通过共享内存,向客户端发送自己的pid
        pause(); // 等待客户端的输入,等待信号SIGUSR2唤醒
        pid = buffer->pid; // 获得客户端的pid

        while(1)
        {
                printf("Server process start write share memory\n");
                fgets(buffer->buf, 128, stdin);
                kill(pid, SIGUSR1); // 使用信号SIGUSR1唤醒客户端
                pause();
        }

        shmdt(buffer);
        buffer = NULL;
        shmctl(shmid, IPC_RMID, NULL);

        return 0;
}

// 客户端代码
#include <stdio.h>
#include <sys/shm.h>
#include <sys/ipc.h>
#include <signal.h>
#include <stdlib.h>
#include <unistd.h>

typedef struct _buffer{
        int pid;
        char buf[128];
}buffer_t;

void handler(int signum){}

int main(int argc, char * argv[])
{
        signal(SIGUSR1, handler);
        int shmid = 0;
        buffer_t *buffer = NULL;

        pid_t pid = 0;
        int key = ftok("server.c", 's');
        if (0 > key)
        {
                printf("Failed to create shamred memory key\n");
                return -1;
        }
        if (0 > (shmid = shmget(key, sizeof(buffer_t), IPC_CREAT | 0777)))
        {
                printf("Create shared memory failed\n");
                return -1;
        }
        if (NULL == (buffer = (buffer_t *)shmat(shmid, NULL, 0)))
        {
                printf("Mapping shared memory failed \n");
                return -1;
        }

        pid = buffer->pid; // 通过共享内存获取服务端的pid
        buffer->pid = getpid(); // 输入客户端本身的pid
        kill(pid, SIGUSR2); // 使用信号SIGUSR2唤醒服务端

        while(1)
        {
                pause();
                printf("Client process recevied data from shared memory: %s\n",
                        buffer->buf);
                kill(pid, SIGUSR2);
        }

        shmdt(buffer);
        buffer = NULL;
        shmctl(shmid, IPC_RMID, NULL);

        return 0;
}
\end{code-block}

