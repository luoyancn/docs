\subsection{LifeCycle And Hooks}
由于有的pod有可能依赖于一些额外的pod,但是这些额外的pod又不是必须永久存在的。此时,
可以用initcontainer进行操作。但是需要注意的是,initcontainer操作的是额外的pod,
并不能针对宿主机做任何操作。
\begin{code-block}{bash}
cat >deploy-with-initpod.yml<<EOF
apiVersion: apps/v1
kind: Deployment
metadata:
  name: nginx-cluster
  namespace: zhangjl
spec:
  selector:
    matchLabels:
      run: nginx-cluster
      init: using-busybox-to-init
  replicas: 3
  template:
    metadata:
      labels:
        run: nginx-cluster
        init: using-busybox-to-init
    spec:
      containers:
      - name: nginx-cluster
        image: nginx
        ports:
        - containerPort: 80
      initContainers:
      - name: echo-hello
        image: busybox
        command: ['sh', '-c', 'echo hello']
EOF
\end{code-block}

但是,如同上边所说,initcontainer操作的是额外的pod,并不能真正影响对应的pod。有的时候,
我们是需要对本身的pod做一些操作。这时,需要使用pod的hook进行操作。Pod的hook只有2个,postStart和preStop。
PostStart在容器启动之后就立即执行,preStop则是在容器退出之前执行。
\begin{code-block}{bash}
cat >deploy-with-hook.yml<<EOF
apiVersion: apps/v1
kind: Deployment
metadata:
  name: nginx-cluster
  namespace: zhangjl
spec:
  selector:
    matchLabels:
      run: nginx-cluster
      init: postStart-and-preStop-hook
  replicas: 3
  template:
    metadata:
      labels:
        run: nginx-cluster
        init: postStart-and-preStop-hook
    spec:
      containers:
      - name: nginx-cluster
        image: nginx
        ports:
        - containerPort: 80
        lifecycle:
          postStart:
            exec:
              command: ["/bin/sh", "-c", "echo Hello from the postStart handler >> /opt/message"]
          preStop:
            exec:
              command: ["/bin/sh", "-c", "echo goodye from the preStop handler >> /opt/message"]
        volumeMounts:
          - mountPath: /opt
            name: /host-volume
      volumes:
      - name: host-volume
        hostPath:
          path: /mnt
          type: Directory
EOF
\end{code-block}

为了探测容器内部的状态,kubernetes还提供了一套探针机制来实现该功能,目前可用的探针为livenessProbe和readinessProbe。