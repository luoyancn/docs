\section{超越Unsafe}
Rust屏蔽了一系列的不安全操作来换取应用程序的稳定性和可靠性,但是,可以通过关键字
unsafe,切换到不安全的运行环境当中,并且在unsafe的代码块当中运行。常见的不安全操作
如下:
\begin{enumerate}
  \item 解引用裸指针
  \item 使用不安全的方法/函数
  \item 访问/修改可变的静态变量
  \item 实现不安全的Trait
  \item 访问union的字段
\end{enumerate}
在使用的时候,原则需要明确:保持unsafe块尽可能小,将不安全代码封装进一个安全的
抽象并提供安全API是一种常见的安全操作和手段。

所谓的裸指针,和普通的指针和智能指针相比,存在如下的区别:
\begin{enumerate}
  \item 允许忽略借用规则,可以同时拥有不可变和可变的指针,或多个指向相同位置的可变指针
  \item 不保证指向有效的内存
  \item 允许为空
  \item 不能实现任何自动清理功能
\end{enumerate}
Rust当中存在2个裸指针:分别写作*const T(不可变)和*mut T(可变),其基本的定义方式
如下:
\begin{code-block}{rust}
let mut num = 5;
let r1 = &num as *const i32; // 不可变的裸指针
let r2 = &mut num as *mut i32; // 可变的裸指针
\end{code-block}

裸指针的定义是安全的,但是,它的使用是不安全的,因此裸指针的使用必须在unsafe块
当中:
\begin{code-block}{rust}
fn main() {
    let mut num = 5;
    let r1 = &num as *const i32;
    let r2 = &mut num as *mut i32;
    unsafe {
        *r2 = 10;
        // r1,r2和num都会变更为10
        println!("{},{}", *r1, *r2);
    }
}
\end{code-block}
同样的,unsafe也可以用于定义函数/方法,不过也需要在unsafe块当中使用;但是,unsafe
的方法可以作为安全方法进行导出,在使用时,则不需要使用unsafe进行标记:
\begin{code-block}{rust}
fn main() {
    let mut num = 5;
    // 定义裸指针
    let r1 = &num as *const i32;
    let r2 = &mut num as *mut i32;
    // 使用不安全的函数/方法
    unsafe {
        unsafe_change(r1, r2);
    }
    println!("{}", num);
    safe_change(r1, r2);
    println!("{}", num);
}
// 定义不安全的函数/方法
unsafe fn unsafe_change(r1: *const i32, r2: *mut i32) {
    *r2 = 10;
    println!("{},{}", *r1, *r2);
}
// 将不安全的函数/方法封装进安全的方法当中
fn safe_change(r1: *const i32, r2: *mut i32) {
    unsafe {
        *r2 = 100;
    }
}
\end{code-block}

作为不安全的一部分,某些时候直接在Rust当中调用C语言的类库可以获得更好的性能,此时,
则同样需要在unsafe块当中使用,比如在Rust当中调用标准C的abs(绝对值)函数:
\begin{code-block}{rust}
extern "C" {
    fn abs(input: i32) -> i32;
}
fn main() {
    unsafe {
        println!("The unsafe from C: {}", abs(-200));
    }
}
\end{code-block}
上述代码出现的extern关键字,有助于创建和使用外部函数接口(Foreign Function
Interface,FFI)。外部函数接口是一个编程语言用以定义函数的方式,其允许不同(外部)
编程语言调用这些函数。Extern块中声明的函数在Rust代码中总是不安全的,

特别需要注意的是,Rust当中的可变全局变量(static)同样是不安全的,需要在unsafe
代码块当中使用;而不可变的全局常量(const和static)则不需要在unsafe块当中;另外,
全局变量同样可以是任意数据类型的:
\begin{code-block}{rust}
use std::fmt;
struct Version {
    major: u8,
    minor: u8,
}
impl fmt::Display for Version {
    fn fmt(&self, f: &mut fmt::Formatter) -> fmt::Result {
        write!(
            f,
            "The version of this bin is {}.{}",
            self.major, self.minor
        )
    }
}
// 不可变的全局常量
const __CONST_NUM__: Version = Version { major: 1, minor: 4 };
const __VERSION__: &str = "v1.4.0";
static __NAME__: &str = "lucifer";
// 可变的全局变量
static mut __COUNTER__: u8 = 1;
fn main() {
    println!("{}", __CONST_NUM__);
    println!("{}", __NAME__);
    unsafe {
        println!("{}", __COUNTER__);
    }
}
\end{code-block}

但是,并不是所有情形都适合使用unsafe,Rust本身也无法从编译器层面,保证unsafe的
代码块是完全正确的,不会出现任何错误的。比如,我们在使用裸指针*const T和*mut T
的时候,如果不够仔细,非常容易造成错误的结果:
\begin{code-block}{rust}
let mut y: u32 = 1;
let x = 1_i32;
// 将y转换成u32的裸指针,再转换成i32的裸指针,最后转换成i64的裸指针
let raw_mut = &mut y as *mut u32 as *mut i32 as *mut i64;
unsafe {
    // 对裸指针进行修改,类似于C/C++当中对指针数据的操作
    *raw_mut = -1;
}
info!("The x is {:X} and y is {:X}", x, y);
\end{code-block}
按照我们本来的设想,x会保持不变,始终为1,而y则可能变换成其他的数值,但是,实际
的结果却如下:
\begin{figure}[H]
  \centering
  \includegraphics[width=\linewidth]{rust_raw_pointer.png}
  \caption{具有潜在错误的裸指针示例}
  \label{fig:rust_raw_pointer}
\end{figure}
x变成和y一样的值的原因在于:对指向y的指针类型做了转换,让它以为自己指向的是i64
类型,恰巧x就在y旁边,y被修改的同时,就顺带把x也修改了。因此,使用unsafe必须特别
小心。

在通常的情况下,虽然可以通过引用+mut的方式,可以阻止大部分的内存不安全问题,但是
由于引用+mut的强限制性,也为带来一些比较麻烦和无奈的问题,比如下面的代码:
\begin{code-block}{rust}
#[derive(Debug)]
struct Tuple {
    first: u8,
    second: u8,
    third: u8,
}
fn main() {
    let mut t = Tuple {
        first: 0,
        second: 1,
        third: 2,
    };
    let pa = &mut t.first;
    let pb = &mut t.second;
    let pc = &mut t.third;
    *pc += 10;
    info!("{:?}", t);
}
\end{code-block}
上述代码是正确无误的,可以正常编译和运行,但是,如果我们将结构体变成数组,问题就
出现了:
\begin{code-block}{rust}
fn main() {
    let mut array_x = [1_i32, 2, 8];
    let pa = &mut array_x[0];
    let pb = &mut array_x[1];
    *pb += 10;
    info!("{:?}", t);
}
\end{code-block}
上述代码在Rust 1.50.0版本之前就会出现错误:
\begin{code-block}{bash}
error: cannot borrow `x[..]` as mutable more than once at a time
\end{code-block}
原因在于,在Rust 1.50.0版本之前的结构体当中,pa,pb和pc指向不同的内存区域;
但是在数据当中,Rust编译器会将[\_]识别为一个整体,而\&[0], \&[1]之间都属于重叠,
将pa和pb判断为存在别名关系,即pa和pb实质上相同,违反了借用规则,因此无法通过编译。
采用引用分割才能进行解决:
\begin{code-block}{bash}
let mut array_x = [1_i32, 2, 3];
// 通过split_at_mut将数组切分成2个一定不会重叠的切片
let (first, rest): (&mut [i32], &mut [i32]) = array_x.split_at_mut(1);
let (second, third): (&mut [i32], &mut [i32]) = rest.split_at_mut(1);
first[0] += 100;
second[0] += 200;
third[0] += 300;
info!("{:?}", array_x);
\end{code-block}

由于Rust的目标是系统级的语言,必然需要具备操作硬件,以及裸设备的能力。而这些能力,
在C/C++的表述当中,通常是采用共用体(Union)实现的。为了与之兼容,Rust当中也引入了
Union数据结构,其主要的使用形式如下:
\begin{code-block}{rust}
#[repr(C)]
pub union U {
    pub i: u32,
    pub f: f32,
}

#[repr(C)]
pub struct Value {
    pub tag: u8,
    pub value: U,
}
\end{code-block}
其中,\#[repr(C)]必须使用,因为union的使用场景本身就是为了和C/C++进行对接,表示
该联合体使用和C/C++一样的内存布局。由于在字段当中使用了union,因此,结构体Value
也必须添加repr属性,否则会出现未定义的错误。而在使用的时候,则更加需要注意,只要
是涉及到读取联合体的字段,则必须使用unsafe:
\begin{code-block}{rust}
// 禁用illegal_floating_point_literal_pattern警告
#[allow(illegal_floating_point_literal_pattern)]
pub fn is_zero(v: &Value) -> bool {
    unsafe {
        match &v {
            Value {
                tag: Tag::I,
                value: U { i: 0 },
            } => true,
            Value {
                tag: Tag::F,
                // 会出现#[warn(illegal_floating_point_literal_pattern)]警告
                // 目前rust正在修复该问题
                value: U { f: 0.0 },
            } => true,
            _ => false,
        }
    }
}
\end{code-block}

Rust所有的unsafe实际都来源于性能和C的结合(比如写linux内核模块),因此原生指针
在unsafe当中最为常用。其主要的用途如下:
\begin{itemize}
  \item 在必要的时候跳过Rust安全检查:有的情况下,程序逻辑不会有任何内存安全的问题,原生指针可以跳过安全检查,提升性能
  \item 与C语言进行交互,必须使用原生指针
\end{itemize}

空指针在C语言当中非常常见,Rust当中也可以创建原生的空指针,也可以利用原生指针修改
数据:
\begin{code-block}{rust}
// 创建一个指向unsigned char的原生null指针
let pointer: *const u8 = std::ptr::null();
// 判断指针是否为空
assert!(pointer.is_null());

let mut s = [1, 2, 3];
// 创建一个可变的指针,该指针指向一个unsigned int的数组
let pointer: *mut u32 = s.as_mut_ptr();
assert!(!pointer.is_null());

unsafe {
    // 访问s[1]
    info!("The offset 1 is {}", *pointer.offset(1));
    // 访问s[2]
    info!("The offset 2 is {}", *pointer.offset(2));
    // 修改s[2]
    *pointer.offset(2) = 4;
    info!("The offset 2 is {}", *pointer.offset(2));
    // 将s[2]先转换成u8,然后再转换成char
    info!("The offset 2 is {}", *pointer.offset(2) as u8 as char);
}

info!("The final result of s is {:?}", s);
\end{code-block}

\section{常见错误处理方法}
由于很多代码都是第三方的,而Rust本身也在不断的发展,有可能出现版本不兼容或者特性
不兼容的情况,此时,则需要进行相关的修改。比如下面的一种错误:
\begin{figure}[H]
  \centering
  \includegraphics[width=\linewidth]{rust_feature_error.png}
  \caption{缺少特性支持编译失败}
  \label{fig:rust_feature_error}
\end{figure}
遇到这种错误,则需要直接修改对应的类库的源代码。以上述错误为例,编译的help表示
需要添加代码\codeinline{bash}{add `#![feature(array_value_iter_slice)]` to the crate attributes to enable},
则我们应当在对应的crate的lib.rs的头部当中,添加内容如下:
\begin{figure}[H]
  \centering
  \includegraphics[width=\linewidth]{rust_feature_add.png}
  \caption{增加特性支持}
  \label{fig:rust_feature_add}
\end{figure}

\section{死灵书与实践}

\subsection{随机数实践}
Rust的随机数模块并不包含在标准库当中,需要使用rand这个crate,其基本的使用如下:
\begin{code-block}{rust}
use rand::distributions::{Distribution, Uniform};
use rand::seq::IteratorRandom;
use rand::Rng;
fn main() {
    let mut rng = rand::thread_rng();
    // 生成随机数
    info!("The float64 rand number is {}", rng.gen::<f64>());
    info!("The u32 rand number is {}", rng.gen::<u32>());
    info!("The i32 rand number is {}", rng.gen::<i32>());
    info!("The u8 rand number is {}", rng.gen::<u8>());
    // 从指定区间生成随机数
    info!("The range rand number is {}", rng.gen_range(0..100));
    info!(
        "The range rand float number is {}",
        rng.gen_range(10.0..50.0)
    );
    // 从[0, 5]生成随机数
    info!("The range rand number is {}", rng.gen_range(0..=5));
    // 定义[1, 7)的均匀分布
    let die = Uniform::from(1..7);
    // 从该分布当中生成采样
    let throw = die.sample(&mut rng);
    info!("The sample of uniform is {:>width$}", throw, width = 5);
    // 生成多个随机数
    let tuple: (u8, u8, u8) = rng.gen();
    info!("The tuple of random is {:?}", tuple);
    // 生成随机数组
    let array: [u8; 6] = rng.gen();
    info!("The array of random is {:?}", array);
    let mut exsit_array: [u8; 5] = [1, 2, 34, 5, 6];
    // 使用随机数填充已存在的数组
    rng.fill(&mut exsit_array);
    info!("The array of random is {:?}", exsit_array);
    // 从均匀分布当中随机采样3个数据
    // 得到的结果可能出现重复的情况
    let samples: Vec<u8> = (&mut rng).sample_iter(die).take(3).collect();
    info!("The samples of sample range 1..7 is {:?}", samples);
    let v = vec![1, 2, 3, 4, 5];
    // 从vec当中采样4个数据,得到的结果不会重复
    let sample = v.iter().choose_multiple(&mut rng, 4);
    info!("The samples of sample range 1..5 is {:?}", sample);
    let sample: Vec<u8> = (1..=10).choose_multiple(&mut rng, 4);
    info!("The samples of sample range 1..10 is {:?}", sample);
}
\end{code-block}

Rust的rand crate不仅可以生成随机数,也可以生成自定义的随机数据,比如:
\begin{code-block}{rust}
use rand::distributions::{Distribution, Standard, Uniform};
use rand::seq::IteratorRandom;
use rand::Rng;
struct Point {
    x: u8,
    y: u8,
}
impl fmt::Display for Point {
    fn fmt(&self, f: &mut fmt::Formatter) -> fmt::Result {
        write!(f, "x: {}, y: {}", self.x, self.y)
    }
}
// 在 Point 类型之上,对Standard实现Distribution trait,使得Point可以被gen函数随机生成
impl Distribution<Point> for Standard {
    // 默认的实现方法
    fn sample<R: Rng + ?Sized>(&self, rng: &mut R) -> Point {
        let (rand_x, rand_y) = rng.gen();
        Point {
            x: rand_x,
            y: rand_y,
        }
    }
}
fn main() {
    let mut rng = rand::thread_rng();
    let rand_point = rng.gen::<Point>();
    info!("The rand_point is {}", rand_point);
}
\end{code-block}

同样的,可以生成随机的字符串:
\begin{code-block}{rust}
use rand::distributions::{Alphanumeric, Distribution, Standard, Uniform};
use rand::seq::IteratorRandom;
use rand::Rng;
fn main() {
    let mut rng = rand::thread_rng();
    let rand_string: String = (&mut rng)
        // 从a-z,A-Z以及0-9当中进行选择
        .sample_iter(&Alphanumeric)
        // 获取其中的10个元素
        .take(10)
        // 默认的结果是char类型,需要继续转换成String
        .map(char::from)
        .collect();
    info!("The rand_string is {}", rand_string);
}
\end{code-block}

如果默认的字符集不满足要求,还可以自定义字符集,比如下面的示例:
\begin{code-block}{rust}
use rand::distributions::{Alphanumeric, Distribution, Standard, Uniform};
use rand::seq::IteratorRandom;
use rand::Rng;
const CHARSET: &[u8] = b"ABCDEFGHIJKLMNOPQRSTUVWXYZ\
    abcdefghijklmnopqrstuvwxyz\
    0123456789)(*&^%$#@!~";
const PASSWORD_LEN: usize = 10;
fn main() {
    let mut rng = rand::thread_rng();
    let password: String = (0..PASSWORD_LEN)
        .map(|_| {
            let idx = rng.gen_range(0..CHARSET.len());
            CHARSET[idx] as char
        })
        .collect();
    info!("The password is {}", password);
    // 也可以更换成之前的采样函数,看起来更为精炼
    let passwd: String = CHARSET
        .choose_multiple(&mut rng, 10)
        .map(|r| *r as char)
        .collect();
    info!("The password is {}", passwd);
}
\end{code-block}

同样的,针对自定义的数据类型,同样可以采用采样方法,进行随机数据的提取:
\begin{code-block}{rust}
use rand::distributions::{Alphanumeric, Distribution, Standard, Uniform};
use rand::seq::IteratorRandom;
use rand::Rng;
#[derive(Debug)]
struct Person {
    name: String,
    age: u8,
}
fn main() {
    let mut rng = rand::thread_rng();
    let persons = vec![
        Person {
            name: "lucifer".to_string(),
            age: 18,
        },
        Person {
            name: "titans".to_string(),
            age: 19,
        },
        Person {
            name: "garuda".to_string(),
            age: 36,
        },
    ];
    // 从person的vec当中,随机抽取2个元素
    let rand_person: Vec<_> = persons.choose_multiple(&mut rng, 2).collect();
    info!("The rand person is {:?}", rand_person);
}
\end{code-block}

\subsection{类型再论}
Rust的类型比较多,char,字符串,整数,浮点数等等。这些基础类型和其他语言比较类似,
但是也包含了自己的特点:比如,char类型占据4个字节,可以存放任何一个unicode字符;
对于ASCII字符,只需要一个字节即可,而一个字节的数据,则可以放在u8类型的数据当中,
因此,对于ASCII类型的字符串/字符数组,可以使用u8类型(即单字节)的数组进行存放,
这样,占用的资源空间会比char的数组小:
\begin{code-block}{rust}
fn main() {
    // 字符串前面的b,表示将对应的字面量存放在u8类型当中
    let s: &[u8] = b"hello";
    info!("{:?}", s);
}
\end{code-block}
同时,Rust支持的整数类型比较广泛,包括8bit,16bit,32bit,64bit,最大可以支持到
128bit;而特殊的isize和usize,则是和平台相关。如果平台是32位的,则isize和usize为
32位,如果是64位,则其数据宽度为64位。

整个Rust的类型当中,只有空类型占据的空间是最小的,都是0。Rust的空类型包括单元类型
(unit,即空元组)以及空结构体:
\begin{code-block}{rust}
// empty是空元组类型
let empty : () = ();
// 空结构体
struct Empty();
\end{code-block}
为了查看类型所占用的空间,可以使用size\_of函数进行查看:
\begin{code-block}{rust}
use std::mem;
struct Empty();
fn main() {
    info!("The Empty struct size is {}", mem::size_of::<Empty>());
    // 查看空元组所占据的内存大小
    info!("The none tuple size is {}", mem::size_of::<()>());
}
\end{code-block}

在Rust当中,浮点类型是非常特殊的数据类型。浮点类型当中,存在一个特殊的值:NaN,
即非法的浮点数值,因为该数据的存在,浮点数不具备全序关系(total order)。所谓的
全序,偏序,Rust当中的定义如下:对于集合X当中的元素a,b,c
\begin{itemize}
  \item 如果a<b,则!(a>b)一定成立;反之,如果a>b,则!(a<b)一定成立,即反对称性
  \item 如果a<b,b<c,则a<c,即传递性
  \item 对于X当中的所有元素,都存在a<b,或者a>b,或者a==b,三者必居其一,即完全性
\end{itemize}
如果X集合只满足前面2条,则称之为偏序;具备上述所有特征,则为全序。由于浮点数的NaN
不满足上述第3条规则,因此,Rust的浮点数属于偏序,而非全序,这回导致一个问题:浮点
数无法排序——非NaN的数值无法与NaN进行比较:
\begin{code-block}{rust}
let nan = std::f32::NAN;
let x = 0.4f32;
// 下列结果全部为false
info!("{}", nan > x);
info!("{}", nan < x);
info!("{}", nan == x);
\end{code-block}
为此,Rust设计了2个Trait表示全序与偏序:\codeinlinebg{rust}{std::cmp::Ord}(全序)以及
\codeinlinebg{rust}{std::cmd::PartialOrd}(偏序)。
PartialOrd这个Trait的partial\_cmp方法返回的是Option<Ordering>,而Ord返回的却是
Ordering。Rust的f32和f64都只实现了PartialOrd,因此,浮点类型无法进行排序,也同样无法
求取最值,如下列代码,则是无法运行的:
\begin{code-block}{rust}
let f_vec = vec![1f32, 2.0, 4.0, 0.0, -1.2];
let bigest_f = f_vec.iter().max();
\end{code-block}
对上诉代码进行编译,会直接提示如下类似的错误:
\begin{figure}[H]
  \centering
  \includegraphics[scale=0.2]{rust_float_cmp_error.png}
  \caption{浮点数的最值错误求解}
  \label{fig:rust_float_cmp_error}
\end{figure}
浮点数的排序只能通过partial\_cmp(比较相等关系)进行变换处理,如下方代码:
\label{float_sort}
\begin{code-block}{rust}
let mut f_vec = vec![1f32, 2.0, 4.0, 0.0, -1.2];
// 升序排列
f_vec.sort_by(|first, second| first.partial_cmp(second).unwrap());
// 获取排序后的最后一位
let max = f_vec.last().unwrap();
// 或者如下进行
// let max = f_vec.as_slice().last().unwrap();
// 降序排列
f_vec.sort_by(|first, second| second.partial_cmp(first).unwrap());
\end{code-block}

作为常用数据类型之一,Rust的数组也存在自己的特点,比如同类型的数组之间可以相互赋值:
\begin{code-block}{rust}
let mut array: [u32; 4] = [1, 23, 4, 5];
let array_copy: [u32; 4] = [5, 6, 7, 8];
array = array_copy;
\end{code-block}
支持数组之间的直接比较,只是数组当中的元素本身就可以进行比较才行:
\begin{code-block}{rust}
let array: [u32; 4] = [1, 23, 4, 5];
let array_copy: [u32; 4] = [5, 6, 7, 8];
info!("{:?}", array < array_copy);
\end{code-block}

Rust当中的函数也可以称之为类型的一种,并且,每个函数都有自己单独的类型,函数的类型
是fn。但是,函数的参数列表会影响fn类型的判断和表达,比如下面的例子:
\begin{code-block}{rust}
fn add_tuple(t: (u32, u32)) -> u32 {
    t.0 + t.1
}
fn add_two((x, y): (u32, u32)) -> u32 {
    x + y
}
fn add_normal(x: u32, y: u32) -> u32 {
    x + y
}
\end{code-block}
实际上,add\_tuple和add\_two这2个函数被fn类型识别成为具有相同签名的类型,因此,
在理论上,我们可以使用同一个变量,接收这2个函数的指针:
\begin{code-block}{rust}
fn main() {
    let mut func = add_tuple;
    func = add_two;
    ...
}
\end{code-block}
但是,上述代码却是错误的:虽然签名相同,但是,类型不同:
\begin{figure}[H]
  \centering
  \includegraphics[width=\linewidth]{rust_func_type.png}
  \caption{相同签名的不同函数类型}
  \label{fig:rust_func_type}
\end{figure}
解决方法,则是将其转换成通用的fn类型:
\begin{code-block}{rust}
fn main() {
    // 显示指定func的类型
    let mut func: fn((u32, u32)) -> u32 = add_tuple;
    // 使用as进行类型的转换
    // let mut func = add_tuple as fn((u32, u32)) -> u32;
    func = add_two;
    ...
}
\end{code-block}
但是,需要注意,add\_normal的功能看上去和前面两个函数的功能相同,但是,他们的
函数签名完全不同,因此,不能将其转换成func。

函数是Rust的头等公民,可以在函数/方法当中定义函数,也可以在函数/方法当中定义结构
体,甚至于定义结构体的方法和实现,以及静态变量,常量等:
\begin{code-block}{rust}
fn func_as_first(x: u32, y: u32) -> (u32, u32) {
    struct Point {
        x: u32,
        y: u32,
    };
    impl Point {
        fn area(&self) -> u32 {
            self.x * self.y
        }
        fn cycle(&self) -> u32 {
            self.x + self.y
        }
    };
    let p = Point { x: x, y: y };
    (p.area(), p.cycle())
}
\end{code-block}

常规的函数类型,都会存在返回值,这些返回值要么是特定的类型,要么就是(),即类似
C/C++的返回void。如果需要什么都不返回,则可以使用!,这种函数称之为发散函数,比如
在处理panic时,有时就需要使用发散函数:
\begin{code-block}{rust}
fn diverges() -> ! {
    panic!("This function never returns!");
}
\end{code-block}
Panic操作会直接导致软件栈展开,因此,后续的操作都不会执行,其返回的就是一个!。
发散函数的最大特点,就是可以被转换成任意一个类型,虽然执行的时候最终还是会崩溃,
如下:
\begin{code-block}{rust}
let x : i32 = diverges();
let y : String = diverges();
\end{code-block}
但是,发散函数最大的作用,在于解决编译器的类型检查:
\begin{code-block}{rust}
let p = if x {
    panic!("error");
} else {
    100
};
\end{code-block}
对于let-if而言,if-else的每个分支都必须是相同的数据类型,通过发散函数的任意类型
转换特性即!与任何类型兼容,所以上述代码才能编译通过。

所有的Rust变量,函数都是类型的一种,都可以通过一定的手段和方式,获得类型的具体信息。
常见的方式有两种,一种是使用错误信息进行推断,一种则是使用标准库函数进行获得。

通过构造一个特殊的函数,然后调用该函数,则可以获得相关的类型信息:
\begin{code-block}{rust}
// 接收一个unit参数
fn type_id(_: ()) {}

fn main() {
    let ref i = 5;
    type_id(i);
}
\end{code-block}

而另外的方式,则是使用标准库函数,不过,这个标准库函数在Rust的默认stable分支当中
是不可用的,需要在nightly分支当中进行编译使用,并且,还需要启用一些特性:
\begin{code-block}{rust}
#![feature(core_intrinsics)]
use std;
// 使用泛型参数进行不同类型的数据接收
fn print_type<T>(_arg: &T) {
    println!(
        "The type name of arg is {}",
        std::intrinsics::type_name::<T>()
    );
}
fn main() {
    let ref x = 5;
    print_type(&x);
}
\end{code-block}
编译上述代码时,则需要对编译指令进行部分的调整:
\codeinlinebg{bash}{cargo +nightly build},
然后即可实现对参数类型的打印输出。

在Rust当中,与Python不同,函数/方法并不存在默认参数,但是,结构体当中的字段,却可以
有默认值,只是,这个默认值的实现,必须和Default Trait相结合,如下:
\begin{code-block}{rust}
struct ColoredString {
    input: String,
    fg_color: String,
    bg_color: String,
}
impl Default for ColoredString {
    fn default() -> Self {
        ColoredString {
            input: String::default(),
            fg_color: String::default(),
            bg_color: String::default(),
        }
    }
}
fn main() {
    let color = ColoredString::default();
}
\end{code-block}
从上述代码当中可以看出,实际上,并不是Rust的结构体字段赋予了初始值,而是通过一个
名为default的方法,构造一个我们认为应该具有默认值的结构体。在Rust当中,常用的基本
数据类型都实现了Default Trait,可以直接使用对应的default方法。

\subsection{Trait类型与泛型再论}
关于类型,Trait也是比较重要的一个话题。在之前的示例当中,Trait全部是在具体的类型
上实现的,但是,Trait本身也可以在智能指针(Box)上实现,比如:
\begin{code-block}{rust}
trait Shape {
    fn area(self: Box<Self>) -> f64;
}
struct Circle {
    radius: f64,
}
impl Shape for Circle {
    fn area(self: Box<Self>) -> f64 {
        PI * self.radius * self.radius
    }
}
fn main() {
    let c = Box::new(Circle { radius: 4f64 });
    info!("{}", c.area());
    // 由于trait实现是在智能指针box上,因此,下面的使用是错误的
    // let c = Circle { radius: 4f64 }
    // c.area()
}
\end{code-block}
甚至在Trait上实现Trait,比如下方:
\begin{code-block}{rust}
trait Shape {
    fn area(&self) -> f64;
}
trait Round {
    fn get_radius(&self) -> f64;
}
struct Circle {
    radius: f64,
}
impl Round for Circle {
    fn get_radius(&self) -> f64 {
        self.radius
    }
}
impl Shape for dyn Round {
    fn area(&self) -> f64 {
        let radius = self.get_radius();
        PI * radius * radius
    }
}
\end{code-block}
Shape是一个Trait,Round同样也是一个Trait,Circle实现了Round,Round实现了Shape,
但是,由于Round本身是一个Trait,拥有不确定性,因此,在实现Shape的时候,需要添加
dyn关键字,提示这个Round不是普通的类型,而是一个Trait。上述代码当中,Circle间接
的实现了Shape,但是,Circle的类型无法直接使用Shape的方法,只能通过智能指针的方
式,将Circle转换成Round的类型,再进行使用,如下:
\begin{code-block}{rust}
fn main() {
    let c: Box<dyn Round> = Box::new(Circle { radius: 4f64 });
    info!("{}", c.area());
}
\end{code-block}
如果再把这个例子改得复杂一些,让Circle和Sphere同时实现Round,则我们可以使用Round
指针计算2个不同类型数据的结果:
\begin{code-block}{rust}
trait Shape {
    fn area(&self) -> f64;
}
trait Round {
    fn calc(&self) -> f64;
}
struct Circle {
    radius: f64,
}
impl Round for Circle {
    fn calc(&self) -> f64 {
        PI * self.radius * self.radius
    }
}
struct Sphere {
    radius: f64,
}
impl Round for Sphere {
    fn calc(&self) -> f64 {
        4f64 * PI * self.radius * self.radius
    }
}
impl Shape for dyn Round {
    fn area(&self) -> f64 {
        self.calc()
    }
}
fn main() {
    let circle: Box<dyn Round> = Box::new(Circle { radius: 4f64 });
    info!("The Circle area is {}", circle.area());
    let sphere: Box<dyn Round> = Box::new(Sphere { radius: 4f64 });
    info!("The Sphere area is {}", sphere.area());
}
\end{code-block}

Trait不仅仅用于实现类型,约束类型,还可以用于为其他现有的数据类型添加方法/函数,
比如:
\begin{code-block}{rust}
impl Round for i32 {
    fn calc(&self) -> f64 {
        *self as f64
    }
}
fn main() {
    let i_struct = 4i32;
    i_struct.calc();
}
\end{code-block}
这种类型的函数/方法,则称之为扩展方法/函数。从上述例子当中,我们似乎可以使用Trait
对任意类型进行函数/方法的扩展,但是,这个是存在前提的:
\begin{itemize}
  \item impl和trait的声明/定义在同一个crate当中
  \item 或者,impl的实现需要和类型的声明在同一个crate当中
\end{itemize}
如果不满足上述条件,则容易出现bug和问题,也会违反Rust的规则。

Rust的Trait支持多种特性,自然也支持继承,但是注意,Rust的结构体和enum数据类型并不
存在继承的概念。Trait的继承方式如下:
\begin{code-block}{rust}
trait Base {}
trait Derived : Base {}
\end{code-block}
当一个结构体实现了上述的Derived这个Trait,则必须同样实现Base这个Trait,否则就会
出现语法错误:
\begin{code-block}{rust}
trait Base {}
trait Derived : Base {}
struct T;
impl Derived for T {}
impl Base for T {}
\end{code-block}

Rust的Trait不仅可以包括函数的定义,同样可以直接定义函数:
\begin{code-block}{rust}
trait Page {
    fn set_page(&self) {
        info!("Page Default: 1");
    }
}
trait PerPage {
    fn set_per_page(&self) {
        info!("Per Page Default: 1");
    }
}
struct Paginate {
    page: u32,
}
impl Page for Paginate {}
impl PerPage for Paginate {}
fn main() {
    let page = Paginate { page: 8 };
    page.set_page();
    page.set_per_page();
    page.set_skip_page();
}
\end{code-block}

甚至于,Trait可以直接给结构体提供更多的组合方法:
\begin{code-block}{rust}
trait PaginateMore: Page + PerPage {
    fn set_skip_page(&self) {
        info!("Skip the page");
    }
}
fn main() {
    ...
    page.set_skip_page();
}
\end{code-block}
结构体根本不用自行实现Trait PaginateMore,就可以直接使用该Trait当中的方法。

Trait不仅仅可以用于接口实现,在Rust当中,更重要的则是类型限定,限定某些数据只能
做某些事情。比如下方的代码:
\begin{code-block}{rust}
...
fn static_dispatch<T>(t: &T) where T: Bar {
    ...
}
fn dynamic_dispatch(t : &Bar) {
    ...
}
\end{code-block}
对于实现了Trait Bar的类型来说,上述2个函数,都可以被调用,但是,从语法上,static\_dispatch
由于使用了where,表示参数必须限定在Trait Bar类型,在编译时就能够确定;而dynamic\_dispatch
则从语法上表示,输入的参数必须是Bar的对象,即Trait Object。运行时,Trait Object会根据虚表
指针从虚表当中查出正确的指针,再进行动态调用,属于在运行时确定。

但是并不是每一个Trait都可以当着Trait Object使用,这个和类型大小是否确定有关系。每一个
Trait的隐藏类型参数Self默认限定为?Sized,?Sized trait包括了所有动态大小类型以及所有
可确定大小的类型。Rust当中大部分类型都是默认可确定大小的,即<T:Sized>。当trait对象
在运行期进行动态分发时,也必须确定大小,否则无法分配内存。只有同时满足下列条件的
trait,才可以当作Trait Object使用:
\begin{itemize}
  \item Trait的Self不能被限定为Sized
  \item Trait当中的所有方法都必须是对象安全的
\end{itemize}

而所谓的对象安全,则必须满足如下的条件\colorblock{之一}:
\begin{itemize}
  \item 当Trait的Self被限定为Sized时,方法受Self:Sized约束
  \item Trait的方法签名必须\colorblock{同时满足以下3点}
  \begin{enumerate}
    \item 不包含任何泛型参数(Self)
    \item 第一个参数必须为Self类型或可解引用为Self类型
    \item Self不能出现在除第一个参数之外的其他地方
  \end{enumerate}
  \item Trait当中不能包含关联常量
\end{itemize}

比如下面的代码,就属于标准的对象安全:
\begin{code-block}{rust}
trait Bar {
    fn bax(self, x: u32);
    fn bay(&self);
    fn baz(&mut self);
}
\end{code-block}
Trait Bar不受Sized限制,Trait的方法没有额外的Self类型参数,没有泛型参数,因此是安全的。
相对应的,不安全的Trait如下:
\begin{code-block}{rust}
// 对象不安全
trait Foo {
    fn bad<T>(&self, x:T);
    fn new() -> Self;
}
// 对象安全
trait Foo {
    fn bad<T>(&self, x: T);
    fn new() -> Self
    where
        Self: Sized;
}
\end{code-block}

当然,Sized约束也可以用于Trait定义当中。比如,自行实现一个类似any的Any Trait。
\begin{code-block}{rust}
use std::ops::Fn;
trait CustomAny {
    fn custom_any<F>(&self, f: F) -> bool
    where
        Self: Sized,
        F: Fn(u32) -> bool;
}
impl CustomAny for Vec<u32> {
    fn custom_any<F>(&self, f: F) -> bool
    where
        Self: Sized,
        F: Fn(u32) -> bool,
    {
        for &x in self {
            if f(x) {
                return true;
            }
        }
        false
    }
}
fn main() {
    let v: Vec<u32> = vec![1, 2, 3];
    info!("{}", v.iter().any(|&x| x == 3));
    info!("{}", v.custom_any(|x| x == 3));
}
\end{code-block}

Trait当中不仅可以包含函数和方法,同样可以包含变量和常量,即所谓的关联变量以及关联
常量。关联常量的使用稍微有些特殊,在Trait当中可以定义关联常量,但是,使用的时候,
却是通过Trait的实现对象来使用这些关联常量的:
\begin{code-block}{rust}
trait Colorize {
    // 定义关联常量
    const FG_RED: &'static str = "31";
    const BG_YELLOW: &'static str = "43";
    fn red(self) -> ColoredString;
    fn on_yellow(self) -> ColoredString;
}
impl Colorize for ColoredString {
    fn red(self) -> ColoredString {
        ColoredString {
            // 使用关联常量,如果是Colorize::FG_RED,则会提示错误
            fg_color: String::from(ColoredString::FG_RED),
            ..self
        }
    }
    fn on_yellow(self) -> ColoredString {
        ColoredString {
            bg_color: String::from(ColoredString::BG_YELLOW),
            ..self
        }
    }
}
\end{code-block}

Trait不仅仅可以实现泛型,泛型也不仅限于Trait和<T>,对于函数/方法,也可以使用在
泛型、生命周期以及Trait当中,比如,显式的指定闭包的生命周期:
\begin{code-block}{rust}
// 将函数作为泛型参数
struct Pick<F> {
    data: (u32, u32),
    func: F,
}
impl<F> Pick<F>
where
    // for<>只能用于标记生命周期
    F: for<'f> Fn(&'f (u32, u32)) -> &'f u32,
{
    fn call(&self) -> &u32 {
        (self.func)(&self.data)
    }
}

fn max(data: &(u32, u32)) -> &u32 {
    if data.0 > data.1 {
        return &data.0;
    }
    &data.1
}
fn main() {
    let pick = Pick {
        data: (32, 34),
        func: max,
    };
    info!("{}", pick.call());
}
\end{code-block}

\subsection{常见的设计模式}
建造者模式是Rust当中最常用的设计模式之一,其主旨思想在于将可变和不可变进行分离,
一种基本的示例如下:
\begin{code-block}{rust}
use std::f64::consts;
pub struct Circle {
    radius: f64,
}
pub struct CircleBuilder {
    radius: f64,
}
impl Circle {
    pub fn new() -> CircleBuilder {
        CircleBuilder { radius: 0.0 }
    }
    pub fn area(&self) -> f64 {
        self.radius * self.radius * consts::PI
    }
}
impl CircleBuilder {
    pub fn radius(&mut self, radius: f64) -> &mut CircleBuilder {
        self.radius = radius;
        self
    }
    pub fn build(&self) -> Circle {
        Circle {
            radius: self.radius,
        }
    }
}
\end{code-block}

\subsection{排序}
Rust的整数型数组和向量(Vector)的排序是相同的,可以使用相同的方式进行,即采用
sort以及sort\_unstable进行。其中,sort是稳定排序(即不重新排序相等的元素),
sort\_unstable是不稳定排序,\colorblock{但是通常情况下速度更快},并且不会进行辅助内存的分配。
\begin{code-block}{rust}
let mut v = vec![2, 21, 12, 32, 12, 45, 90];
v.sort_unstable();
info!("The sorted vector is {:?}", v);
let mut array = [2, 23, 12, 12, 98, 100, 21];
array.sort_unstable();
info!("The sorted array is {:?}", array);
\end{code-block}
默认情况下,排序操作使用的是升序,但是可以通过定制,修改排序方式:
\begin{code-block}{rust}
let mut v = vec![2, 21, 12, 32, 12, 45, 90];
// 降序排列,可替换成v.sort_by
v.sort_unstable_by(|a, b| b.cmp(a));
info!("The sorted vector is {:?}", v);
let mut array = [2i32, -23, 12, 12, 98, -100, 21];
// 根据绝对值升序排列,可以根据其他关键字进行排序
array.sort_unstable_by_key(|k| k.abs());
info!("The sorted array is {:?}", array);
// 根据字符顺序排列,带有缓存cache,闭包函数通常只执行一次,比无缓存的快速
let mut xx = [-5i32, 4, 32, -3, 2];
xx.sort_by_cached_key(|k| k.to_string());
// 字符串排序
let mut array = ["lucifer", "titans", "asura", "garuda"];
array.sort_unstable_by_key(|item| item.to_string());
info!("The string array is {:?}", array);
let mut array = [
    "lucifer".to_string(),
    "titans".to_string(),
    "asura".to_string(),
    "garuda".to_string(),
];
// 可以转换成切片
// array[..].sort_unstable_by_key(|item| item.to_string());
// info!("The string array is {:?}", array);
array.sort_unstable_by_key(|item| item.to_string());
info!("The string array is {:?}", array);
\end{code-block}

浮点数的排序和最值操作,参见\colorunderlineref{float_sort}

除了基础数据类型可以进行排序,同样可以针对复合数据类型进行排序。在针对复合数据
类型排序时,需要实现\colorblock{Eq,PartialEq,Ord和PartialOrd}这几个trait:
\begin{code-block}{rust}
#[derive(Eq, PartialEq, Ord, PartialOrd, Debug)]
struct Student {
    name: String,
    age: u8,
}
fn main() {
    let mut stu = [
        Student {
            name: "lucifer".to_string(),
            age: 18,
        },
        Student {
            name: "garuda".to_string(),
            age: 36,
        },
    ];
    // 按照自然序列(name)
    stu.sort();
    info!("The students is {:?}", stu);
    // 根据年龄
    stu.sort_unstable_by(|first, second| first.age.cmp(&second.age));
    info!("The students is {:?}", stu);
}
\end{code-block}

\subsection{压缩与解压}
Rust可以实现文件的压缩与解压,在Linux环境下,通常使用\href{https://github.com/alexcrichton/tar-rs}{tar}(归档)
和\href{https://github.com/rust-lang/flate2-rs}{flate2}(压缩解压),比如Linux下常见的tar.gz文件的处理:
\begin{code-block}{rust}
use flate2::read::GzDecoder;
use flate2::write::GzEncoder;
use flate2::Compression;
use tar::Archive;

let path = "/root/py3.tar.gz";
let targz = match File::open(path) {
    Ok(file) => file,
    Err(error) => {
        crit!("Failed to open the file {}: {}", path, error.to_string());
    }
};
// gz文件的解码器
let tar = GzDecoder::new(targz);
// tar的管理器
let mut archive = Archive::new(tar);
// 将tar.gz解压
match archive.unpack(".") {
    Ok(_) => info!("Sucess to unpack the tar.gz file"),
    Err(error) => {
        crit!("Failed to unpark the tar.gz file: {:?}", error);
    }
}
// 创建tar.gz文件
let targz = match File::create("log.tar.gz") {
    Ok(file) => file,
    Err(error) => {
        crit!(
            "Failed to create the log.tar.gz file : {}",
            error.to_string()
        );
    }
};
// 创建gz文件的编码器,压缩算法使用默认
let encoder = GzEncoder::new(targz, Compression::default());
let mut tarfile = tar::Builder::new(encoder);
// 将文件添加到tar.gz文件当中
match tarfile.append_dir_all("log", "/var/log") {
    Ok(_) => info!("log.tar.gz created sucessful"),
    Err(error) => {
        fs::remove_file("log.tar.gz").unwrap_or_else(|why| {
            error!("Cannot remove the log.tar.gz: {:?}", why.to_string())
        });
        crit!("Failed to park the tar.gz file: {:?}", error);
    }
}
\end{code-block}

当然,归档和压缩也可以单独使用:
\begin{code-block}{rust}
use flate2::read::GzDecoder;
use flate2::write::GzEncoder;
use flate2::Compression;
use tar::Archive;
let tarf = match File::create("log.tar") {
    Ok(file) => file,
    Err(error) => {
        crit!("Failed to create the log.tar file : {}", error.to_string());
    }
};
// 注意和gz文件不一样,只是归档,则不需要创建编码器
let mut tar_file = tar::Builder::new(tarf);
match tar_file.append_dir_all("log", "/var/log") {
    Ok(_) => info!("log.tar created sucessful"),
    Err(error) => {
        fs::remove_file("log.tar")
            .unwrap_or_else(|why| error!("Cannot remove the log.tar: {:?}", why.to_string()));
        crit!("Failed to park the tar.gz file: {:?}", error);
    }
}
let path = "/root/log.tar";
let tarball = match File::open(path) {
    Ok(file) => file,
    Err(error) => {
        crit!("Failed to open the file {}: {}", path, error.to_string());
    }
};
// 同样的,解压tar文件,不需要创建解码器
let mut archive = Archive::new(tarball);
match archive.unpack(".") {
    Ok(_) => info!("Sucess to unpack the tar file"),
    Err(error) => {
        crit!("Failed to unpark the tar file: {:?}", error);
    }
}
\end{code-block}
