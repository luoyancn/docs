\section{Linux系统编程}

\subsection{IO函数}
Linux系统当中,通常需要处理IO,而IO的处理,在Linux的函数当中,主要有4个函数:
\begin{itemize}
  \item open //fcntl.h
  \item write //unistd.h
  \item read //unistd.h
  \item close //unistd.h
\end{itemize}

实现简单的touch命令的功能
\begin{code-block}{c}
#include <stdio.h>
#include <unistd.h>
#include <fcntl.h>

int main(int argc, char * argv[])
{
        // 第3个参数可以直接写为0644
        int fd = open(argv[1], O_CREAT|O_WRONLY,
                S_IRUSR|S_IWUSR|S_IRGRP|S_IROTH);
        if (0>fd)
        {
                printf("Cannot create file %s\n", argv[1]);
                return -1;
        }
        printf("Create file %s success\n", argv[1]);
        close(fd);
        return 0;
}
\end{code-block}

但是,由于Linux操作系统本身存在umask(默认为022),因此,如果上述的第3个参数写作0777,
生成的文件的权限与umask进行亦或计算之后,实际上,文件的权限还是755,并不是我们所期待的
777。如果需要保持设置的权限与生成的文件权限完全一致,需要执行如下命令:
\begin{code-block}{bash}
umask 000
# 后续再执行代码,生成文件
\end{code-block}

另外,如果只是需要打开文件,并不是创建文件,则open函数的第3个参数不需要。
除此之外,还需要注意一下,文件的打开模式
\begin{itemize}
  \item O\_TRUNC:覆盖文件
  \item O\_EXCL : 与O\_CREAT合用,如果对应文件已经存在,则提示错误
\end{itemize}

而对应的,也可以利用write函数向打开的文件句柄当中写入内容
\begin{code-block}{c}
#include <stdio.h>
#include <unistd.h>
#include <fcntl.h>

int main(int argc, char * argv[])
{
        // 第3个参数可以直接写为0644
        int fd = open(argv[1], O_CREAT|O_RDWR,
                S_IRUSR|S_IWUSR|S_IRGRP|S_IROTH);
        if (0>fd)
        {
                printf("Cannot create file %s\n", argv[1]);
                return -1;
        }
        printf("Create file %s success\n", argv[1]);

        char msg[] = "hello world";
        write(fd, msg, sizeof(msg)/sizeof(char)); //会写入一个文件结束符,特殊符号
                                                  // 如果不需要,则将长度-1即可
        close(fd);
        return 0;
}
\end{code-block}

相应的,也可以利用read函数读取打开文件的内容:
\begin{code-block}{c}
#include <stdio.h>
#include <unistd.h>
#include <fcntl.h>
#include <string.h>

int main(int argc, char * argv[])
{
        int fd = open(argv[1], O_RDONLY);
        if (0>fd)
        {
                printf("Cannot open file %s\n", argv[1]);
                return -1;
        }
        printf("Open file %s success\n", argv[1]);

        size_t read_ret = 0;
#if 0
        // 连续多次读取,并非一次性读完
        size_t total = 0;
        char readbuf[128];
        while ((read_ret=read(fd, readbuf, 127))>0) // 每次只能读取max-1,否则末尾存在特殊字符,可能出现溢出
        {
                total += read_ret;
                printf("Read %d chars \n", read_ret);
                printf("The content of file is %s \n", readbuf);
                memset(readbuf, 0, 128);
        }
        printf("The total sizeof file is %d\n", total);
#else
        // 一次性读取
        char readbuf[1024];
        read_ret=read(fd, readbuf, 1024);
        printf("Read %d chars \n", read_ret);
        printf("The content of file is %s \n", readbuf);
#endif
        close(fd);
        return 0;
}
\end{code-block}

高级一点的,我们就可以使用read和write函数来实现一个简单的文件拷贝功能。
\begin{code-block}{c}
#include <stdio.h>
#include <unistd.h>
#include <fcntl.h>
#include <string.h>

int main(int argc, char * argv[])
{
        int readrd = 0, writefd = 0;
        if (0 >= (readrd = open(argv[1], O_RDONLY)))
        {
                printf("Cannot open the source file %s\n", argv[1]);
                return -1;
        }
        if (0 >= (writefd = open(
                argv[2], O_CREAT|O_TRUNC|O_WRONLY, 0644)))
        {
                printf("Cannot create the target file %s\n", argv[2]);
                return -1;
        }

        unsigned char buffer[128];
        memset(buffer, 0, 128);

        size_t readret = 0, writeret = 0;
        while(0 < (readret = read(readrd, buffer, 127)))
        {
                if (0 > (writeret = write(writefd, buffer, readret)))
                {
                        printf("Cannot write content to write file\n");
                        return -1;
                }
                memset(buffer, 0, 128);
        }

        close(readrd);
        close(writefd);
        return 0;
}
\end{code-block}

由于读取使用的是unsigned char,因此,上述文件也可以直接拷贝二进制文件。

\subsection{标准IO函数}
Linux的IO操作包括文件IO和标准IO。所谓的文件IO,即直接调用内核提供的系统调用函数,一般需要使用头文件unistd.h当中的函数;而
标准IO,则是通过调用C的库函数,间接的调用系统调用函数,通常的,使用的头文件stdio.h当中的函数。从功能上看,标准IO与文件IO是
相同的,但是,细节上,他们存在区别。
\begin{code-block}{c}
#include <stdio.h>
#include <unistd.h>

int main(int argc, char * argv[])
{
        char  buffer[] = "hello world";
        printf("stdio %s", buffer);
        write(1, buffer, 11);
        while(1);
        return 0;
}
\end{code-block}

上述代码编译之后,运行,只有hello world能够输出,而printf的stdio hello world则无法输出。问题在于缓存。
Linux程序当中存在几种缓存:
\begin{itemize}
  \item 用户空间缓存:即想从内核读写的数据,即上述代码当中buffer
  \item 内核空间缓存:没打开一个文件,内核会在内核空间开辟一块缓存,这个称之为内核空间的缓存
  \item 库缓存:标准IO的库函数的缓存
\end{itemize}

文件IO中的写,即是将用户空间的缓存写入到内核空间缓存当中;反之,文件IO的读,则是将内核空间的缓存读写到用户空间的缓存当中。
而调用标准IO之后,数据会从用户空间写入到库缓存,当写入的数据包含\textbackslash n时,或者库缓存空间写满时,才会向内核缓存空间提交数据。
因此,如果上述代码修改为
\begin{code-block}{c}
printf("stdio %s\n", buffer); //或者直接将库缓存写满
while(1);
\end{code-block}
则会直接输出。另外,库缓存的大小默认为1024个字节。

常用fgets,gets,printf,sprintf,fprintf,fputs,puts,scanf这些函数在遇到\textbackslash n或者写满缓存时,即
调用系统调用函数,称之为行缓存函数;而fread,fwrite只有在写满缓存之后再调用系统调用函数,这些则称之为全缓存函数;
而只要调用,则会将内容和数据写入到内核当中的函数,称之为无缓存函数。fclose函数在关闭文件之前,会刷新缓存当中的数据到文件当中。

与文件IO相对应的,标准IO使用fopen函数进行文件的创建和读写。但是需要特别注意的是,实际上,fopen函数创建的函数的权限始终是
666,但是由于umask的存在,因此,fopen函数创建的文件的最终权限为644。
