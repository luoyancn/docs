\section{Linux系统编程}

\subsection{IO函数}
Linux系统当中,通常需要处理IO,而IO的处理,在Linux的函数当中,主要有4个函数:
\begin{itemize}
  \item open //fcntl.h
  \item write //unistd.h
  \item read //unistd.h
  \item close //unistd.h
\end{itemize}

实现简单的touch命令的功能
\begin{code-block}{c}
#include <stdio.h>
#include <unistd.h>
#include <fcntl.h>

int main(int argc, char * argv[])
{
        // 第3个参数可以直接写为0644
        int fd = open(argv[1], O_CREAT|O_WRONLY,
                S_IRUSR|S_IWUSR|S_IRGRP|S_IROTH);
        if (0>fd)
        {
                printf("Cannot create file %s\n", argv[1]);
                return -1;
        }
        printf("Create file %s success\n", argv[1]);
        close(fd);
        return 0;
}
\end{code-block}

但是,由于Linux操作系统本身存在umask(默认为022),因此,如果上述的第3个参数写作0777,
生成的文件的权限与umask进行亦或计算之后,实际上,文件的权限还是755,并不是我们所期待的
777。如果需要保持设置的权限与生成的文件权限完全一致,需要执行如下命令:
\begin{code-block}{bash}
umask 000
# 后续再执行代码,生成文件
\end{code-block}

Open函数只能生成普通文件,如果是管道、字符设备之类的,则无法使用open函数进行创建。
另外,如果只是需要打开文件,并不是创建文件,则open函数的第3个参数不需要。
除此之外,还需要注意一下,文件的打开模式
\begin{itemize}
  \item O\_TRUNC:覆盖文件
  \item O\_EXCL : 与O\_CREAT合用,如果对应文件已经存在,则提示错误
\end{itemize}

Open函数一旦调用,Linux内核会在内核空间打开3个文件描述符,分别是0,1,2。

而对应的,也可以利用write函数向打开的文件句柄当中写入内容
\begin{code-block}{c}
#include <stdio.h>
#include <unistd.h>
#include <fcntl.h>

int main(int argc, char * argv[])
{
        // 第3个参数可以直接写为0644
        int fd = open(argv[1], O_CREAT|O_RDWR,
                S_IRUSR|S_IWUSR|S_IRGRP|S_IROTH);
        if (0>fd)
        {
                printf("Cannot create file %s\n", argv[1]);
                return -1;
        }
        printf("Create file %s success\n", argv[1]);

        char msg[] = "hello world";
        write(fd, msg, sizeof(msg)/sizeof(char)); //会写入一个文件结束符,特殊符号
                                                  // 如果不需要,则将长度-1即可
        close(fd);
        return 0;
}
\end{code-block}

相应的,也可以利用read函数读取打开文件的内容:
\begin{code-block}{c}
#include <stdio.h>
#include <unistd.h>
#include <fcntl.h>
#include <string.h>

int main(int argc, char * argv[])
{
        int fd = open(argv[1], O_RDONLY);
        if (0>fd)
        {
                printf("Cannot open file %s\n", argv[1]);
                return -1;
        }
        printf("Open file %s success\n", argv[1]);

        size_t read_ret = 0;
#if 0
        // 连续多次读取,并非一次性读完
        size_t total = 0;
        char readbuf[128];
        while ((read_ret=read(fd, readbuf, 127))>0) // 每次只能读取max-1,否则末尾存在特殊字符,可能出现溢出
        {
                total += read_ret;
                printf("Read %d chars \n", read_ret);
                printf("The content of file is %s \n", readbuf);
                memset(readbuf, 0, 128);
        }
        printf("The total sizeof file is %d\n", total);
#else
        // 一次性读取
        char readbuf[1024];
        read_ret=read(fd, readbuf, 1024);
        printf("Read %d chars \n", read_ret);
        printf("The content of file is %s \n", readbuf);
#endif
        close(fd);
        return 0;
}
\end{code-block}

高级一点的,我们就可以使用read和write函数来实现一个简单的文件拷贝功能。
\begin{code-block}{c}
#include <stdio.h>
#include <unistd.h>
#include <fcntl.h>
#include <string.h>

int main(int argc, char * argv[])
{
        int readrd = 0, writefd = 0;
        if (0 >= (readrd = open(argv[1], O_RDONLY)))
        {
                printf("Cannot open the source file %s\n", argv[1]);
                return -1;
        }
        if (0 >= (writefd = open(
                argv[2], O_CREAT|O_TRUNC|O_WRONLY, 0644)))
        {
                printf("Cannot create the target file %s\n", argv[2]);
                return -1;
        }

        unsigned char buffer[128];
        memset(buffer, 0, 128);

        size_t readret = 0, writeret = 0;
        while(0 < (readret = read(readrd, buffer, 127)))
        {
                if (0 > (writeret = write(writefd, buffer, readret)))
                {
                        printf("Cannot write content to write file\n");
                        return -1;
                }
                memset(buffer, 0, 128);
        }

        close(readrd);
        close(writefd);
        return 0;
}
\end{code-block}

由于读取使用的是unsigned char,因此,上述文件也可以直接拷贝二进制文件。

\subsection{标准IO函数}
Linux的IO操作包括文件IO和标准IO。所谓的文件IO,即直接调用内核提供的系统调用函数,一般需要使用头文件unistd.h当中的函数;而
标准IO,则是通过调用C的库函数,间接的调用系统调用函数,通常的,使用的头文件stdio.h当中的函数。从功能上看,标准IO与文件IO是
相同的,但是,细节上,他们存在区别。
\begin{code-block}{c}
#include <stdio.h>
#include <unistd.h>

int main(int argc, char * argv[])
{
        char  buffer[] = "hello world";
        printf("stdio %s", buffer);
        write(1, buffer, 11);
        while(1);
        return 0;
}
\end{code-block}

上述代码编译之后,运行,只有hello world能够输出,而printf的stdio hello world则无法输出。问题在于缓存。
Linux程序当中存在几种缓存:
\begin{itemize}
  \item 用户空间缓存:即想从内核读写的数据,即上述代码当中buffer
  \item 内核空间缓存:没打开一个文件,内核会在内核空间开辟一块缓存,这个称之为内核空间的缓存
  \item 库缓存:标准IO的库函数的缓存
\end{itemize}

文件IO中的写,即是将用户空间的缓存写入到内核空间缓存当中;反之,文件IO的读,则是将内核空间的缓存读写到用户空间的缓存当中。
而调用标准IO之后,数据会从用户空间写入到库缓存,当写入的数据包含\textbackslash n时,或者库缓存空间写满时,才会向内核缓存空间提交数据。
因此,如果上述代码修改为
\begin{code-block}{c}
printf("stdio %s\n", buffer); //或者直接将库缓存写满
while(1);
\end{code-block}
则会直接输出。另外,库缓存的大小默认为1024个字节。

常用fgets,gets,printf,sprintf,fprintf,fputs,puts,scanf这些函数在遇到\textbackslash n或者写满缓存时,即
调用系统调用函数,称之为行缓存函数;而fread,fwrite只有在写满缓存之后再调用系统调用函数,这些则称之为全缓存函数;
而只要调用,则会将内容和数据写入到内核当中的函数,称之为无缓存函数,注意,stderr是无缓存的,而stdout则是行缓存的。
fclose函数在关闭文件之前,会刷新缓存当中的数据到文件当中。

需要注意的是fputc是缓存函数,但是,他不是行缓存函数,立即生效的话,需要使用fflush函数进行强制刷新。

除此之外,在标准IO当中,读取文件有可能会出现错误,而fgets函数读取正常时,返回读取到的内容,这个内容与fgets函数的第一个参数的结果一致,
如果读取错误,则会返回一个空指针(char)。但是无法准确判断这个错误是什么类型。判断错误的准确类型,可以使用feof和ferror函数进行判断。
前者表示读取到了文件末尾,而后一个则表示真的文件读取错误,如下代码所示:
\begin{code-block}{c}
FILE *fp = fopen("test.c")
char buffer[128];
char * read_ret = NULL;
read_ret = fgets(buffer, 128, fp);
if (NULL == read_ret)
{
        if(feof(fp))
        {
                printf("Read the end of file\n");
        }
        if(ferror(fp))
        {
                printf("Read error from the stream\n");
        }
}
\end{code-block}

与文件IO相对应的,标准IO使用fopen函数进行文件的创建和读写。但是需要特别注意的是,实际上,fopen函数创建的函数的权限始终是
666,但是由于umask的存在,因此,fopen函数创建的文件的最终权限为644。

全缓存函数fread和fwrite在使用的时候会调用syscall,写入到内核缓存当中,最后写入到硬件当中(文件)。同样的,我们也可以用fread和fwrite实现
Linux的cat命令,简单的如下:
\begin{code-block}{c}
if(NULL == (fp = fopen(argv[1], "rb")))
{
        printf("Cannot open the file %s\n", argv[1]);
        return -1;
}

unsigned char buffer[128];
memset(buffer, 0, 128);
while(0 < fread(buffer, sizeof(char), 128, fp))
{
        fwrite(buffer, sizeof(char), 128, stdout);
        memset(buffer, 0, 128);
        if(feof(fp))
        {
                printf("Read the the of file\n");
                break;
        }
}

fclose(fp); // 调用fflush,直接写入到内核缓存当中
return 0;
\end{code-block}

从执行效率上说,fgetc/fputc<fgets/fputs<fread/fwrite,主要原因在于fread基本都是在内核空间操作,效率有保证。因此,在有高效率要求的情况下,尽可能的使用fread和fwrite
作为IO的操作函数。

\subsection{目录IO}
除了文件IO和标准IO之外,Linux还提供了针对路径(目录)的IO操作函数,具体如图\nameref{fig:dirio}所示
\begin{figure}[H]
  \centering
  \includegraphics[scale=1]{dirio.png}
  \caption{Linux的目录IO函数}
  \label{fig:dirio}
\end{figure}

只是需要注意的是,mkdir函数在sys/stat.h当中,其他的函数大部分在dirent.h当中。目录的创建,可以使用如下的代码:
\begin{code-block}{c}
int ret = mkdir("zhangjl", 0777);
if(0 > ret)
{
        printf("Failed to create dir\n");
        return -1;
}
return 0;
\end{code-block}

而打开目录,则可以如下操作:
\begin{code-block}{c}
#include <dirent.h>

int main(int argc, char * argv[])
{
        DIR *dp = opendir("/root");
        if(NULL ==  dp)
        {
                printf("Failed to open dir\n");
                return -1;
        }

        closedir(dp);
        return 0;
}
\end{code-block}

读取目录内容,则可以使用readdir函数。由于readdir函数在多个头文件当中都有定义,此处应当使用dirent.h当中的函数。
具体的使用如下代码:
\begin{code-block}{c}
#include <stdio.h>
#include <dirent.h>

int main(int argc, char * argv[])
{
        DIR *dp = opendir("/root/cprograms/dirio");
        if(NULL ==  dp)
        {
                printf("Failed to open dir\n");
                return -1;
        }

        struct dirent * dir = NULL;
        while (NULL != (dir = readdir(dp)))
        {
                printf("The inode is %lu, and name is %s\n",
                        dir->d_ino, dir->d_name);
        }

        closedir(dp);
        return 0;
}
\end{code-block}
上述代码需要注意的有几点:
\begin{enumerate}
  \item readdir返回的是一个指针,而这个指针,实际上是一个链表的头指针,因此,通常情况下需要反复调用该函数,读取链表上的所有元素
  \item readdir只能返回一级文件目录当中的内容,子目录以及子目录下的子目录,则无法一次性读取
  \item rewinddir则会将readdir所得到的指针,重新定位到这个链表的头节点,也可以使用seekdir进行指定地址的跳转。
\end{enumerate}

\subsection{Linux进程通信}
首先需要明确的是,在用户空间实现进程间通信是不可能的,需要在Linux内核空间当中进行;但是线程间的通信,在用户空间就可以实现。
最明显的,线程间的通信,通过全局变量即可实现,其原因主要就是多线程之间是共享内存的,如下简单代码:
\begin{code-block}{c}
#include <stdio.h>
#include <pthread.h>
#include <unistd.h>
int main_run = 0;
void *func(void *var)
{
        int i = 0;
        //while(!main_run); //如果需要父进程执行结束之后,再执行子线程,则开启本行注释即可
        for (; i <10; i++)
        {
                usleep(100);
                printf("This is fun i=%d\n", i);
        }
}

int main(int argc, char * argv[])
{
        int i = 0;
        char buf[] = "hello world\n";
        pthread_t tid;
        int ret = 0;
        ret = pthread_create(&tid, NULL, func, (void*)buf);
        if (0 > ret)
        {
                printf("Create thread failure\n");
                return -1;
        }
        for(i = 0; i < 10; i++)
        {
                usleep(100);
                printf("this is main fun i = %d\n", i);
        }
        main_run = 1;
        while(1);
        return 0;
}
\end{code-block}

注意,多线程编译时,需要加入-pthread参数,即
\begin{code-block}{bash}
gcc -pthread -o test test.c
\end{code-block}

但是,与线程不同,进程间的每一种通信方式都是基于文件IO的思想进行设计和实现的。
\subsubsection{管道通信}
管道是一种特殊的文件,由队列来实现,遵循先进先出的顺序。与open函数类似,open函数打开的文件描述符为0,1,2,而管道函数(pipe)
打开的文件描述符则固定为3,4,分别对应fd[0]和fd[1]。
\begin{code-block}{c}
#include <stdio.h>
#include <unistd.h>

int main(int argc, char * argv[])
{
        int fd[2];
        int ret = 0;
        if (0 > (ret=pipe(fd)))
        {
                printf("Cannot create pipe \n");
                return -1;
        }
        printf("%d, %d\n", fd[0], fd[1]);
        return 0;
}
\end{code-block}

由于管道本身是特殊文件,因此,也可以对管道进行读写,但是特别需要注意的是,fd[0]只允许进行读取,而fd[1]则只允许进行写入,如下:
\begin{code-block}{c}
#include <stdio.h>
#include <unistd.h>
#include <string.h>

int main(int argc, char * argv[])
{
        int fd[2];
        int ret = 0;
        if (0 > (ret=pipe(fd)))
        {
                printf("Cannot create pipe \n");
                return -1;
        }
        char buf[] = "hello linux";
        char readbuf[128];
        memset(readbuf, 0, 128);
        size_t writed = write(fd[1], buf, sizeof(buf)/sizeof(char));
        size_t readed = read(fd[0], readbuf, writed);
        printf("Read from pipe: %s\n", readbuf);
        close(fd[0]);
        close(fd[1]);
        return 0;
}
\end{code-block}

\begin{enumerate}
  \item 管道创建在内存当中,进程结束,空间释放,管道就不存在了
  \item 管道当中的数据,一旦读取完毕,就直接从管道当中删除了
  \item 如果管道当中没有内容,则读取操作会一直阻塞;反之,如果没有读取操作,一旦缓冲写满(65536),则写入操作会阻塞
  \item 管道最大为65536字节
  \item 无名管道只能实现父子进程之间的通信
\end{enumerate}

实现父子进程的通信如下:
\begin{code-block}{c}
#include <stdio.h>
#include <unistd.h>
#include <string.h>

int main(int argc, char * argv[])
{
        int fd[2];
        int ret = pipe(fd);
        int inter = 0;
        pid_t pid;
        pid = fork();
        if (0 > ret)
        {
                printf("Cannot create pipe \n");
                return -1;
        }
        if (0 == pid)
        {
                int i = 0;
                read(fd[0], &inter, 1);
                while(!inter);
                for (;i < 5; i++)
                {
                        printf("[%d]In child\n", i);
                }
        }
        if ( 0 < pid)
        {
                int i = 0;
                for(;i < 5; i++)
                {
                        printf("[%d]In parent\n", i);
                }
                inter = 1;
                write(fd[1], &inter, 1);
        }

        close(fd[0]);
        close(fd[1]);
        return 0;
}
\end{code-block}

与无名管道相对应的,则是有名管道。所谓有名管道,其实也是一个管道,但是,他是存在于文件系统当中的,并不是仅仅只是在内存当中。
有名管道的文件,每个文件节点都含有inode编号,并且其文件为p类型(即管道类型)。管道文件只含有inode编号,不占用磁盘存储空间,与套接字,字符设备
以及块设备一样。
