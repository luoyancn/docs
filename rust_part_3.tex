\section{并行与并发}
Rust也同样支持常见的并行和并发操作,也同样分为进程,线程以及消息通信等等。

\subsection{线程}
Rust的线程操作必须使用闭包完成。在之前看到的闭包当中,通常采用的都是有参的闭包,
而在Rust的线程操作当中,则经常会遇到无参数的闭包;Rust的线程使用thread::spawn函数
进行实现:
\begin{code-block}{rust}
use std::thread;
use std::time::Duration;

fn main() {
    thread::spawn(|| {
        for i in 1..10 {
            println!("hi number {} from the spawned thread!", i);
            thread::sleep(Duration::from_millis(1));
        }
    });

    for i in 1..5 {
        println!("hi number {} from the main thread!", i);
        thread::sleep(Duration::from_millis(1));
    }
}
\end{code-block}
和其他语言的线程概念一样,当主线程结束时,所有的线程都会被终止。因此上述代码当中,
子线程(spawn)无法将所有的循环执行完成。为了达成所有进/线程执行完成之后才退出主
进/线程的目的,和其他的开发语言相同,需要在主进程当中调用join函数:
\begin{code-block}{rust}
fn main() {
    let handle = thread::spawn(|| {
        for i in 1..10 {
            println!("hi number {} from the spawned thread!", i);
            thread::sleep(Duration::from_millis(1));
        }
    });

    for i in 1..5 {
        println!("hi number {} from the main thread!", i);
        thread::sleep(Duration::from_millis(1));
    }
    handle.join().unwrap();
}
\end{code-block}
Thread::spawn的返回值是JoinHandle,是一个拥有所有权的值,当对其调用join方法时,
它会等待对应线程结束;而join的返回值是一个Result,可以按照之前介绍的方式进行处理。
同时,Join函数是一个阻塞式函数,只有当该函数运行结束之后,才会继续进行后续的操作。

多数情况下,Rust的线程不可能只会在内部运行,而和外部没有数据交互。但是,如果我们
直接使用外部数据,则会出现错误,比如下方的代码:
\begin{code-block}{rust}
fn main() {
    let v = vec![1, 2, 3];
    let handle = thread::spawn(|| {
        println!("Here's a vector: {:?}", v);
    });
    handle.join().unwrap();
}
\end{code-block}
\begin{figure}[H]
  \centering
  \includegraphics[width=\linewidth]{rust_thread_out_params.png}
  \caption{试图访问线程外部资源}
  \label{fig:rust_thread_out_params}
\end{figure}
线程使用的是闭包,从闭包的定义来说,是可以捕获并使用外部变量和数据的;但是,Rust
不知道这个线程到底会运行多长时间,因此无法知道对外部变量的引用是否一直有效,比如
下方的代码:
\begin{code-block}{rust}
fn main() {
    let v = vec![1, 2, 3];
    let handle = thread::spawn(|| {
        println!("Here's a vector: {:?}", v);
    });
    drop(v);
    handle.join().unwrap();
}
\end{code-block}
启动线程的同时,立即将v进行丢弃,线程内部无法知道v在运行阶段是否继续有效,就会
出现错误,因此,如果在线程当中使用默认的闭包模式,则无法对应的闭包是无法捕获以及
使用外部的变量和数据的。此时,则需要使用move闭包进行替换,即强制闭包获取外部变量
的所有权,而不是由Rust进行借用推断。但是需要注意,一旦使用move之后,在线程之外,
变量将无法再进行使用:
\begin{code-block}{rust}
fn main() {
    let v = vec![1, 2, 3];
    let handle = thread::spawn(move || {
        println!("Here's a vector: {:?}", v);
    });
    // 下方代码无法再进行执行
    // println!("{:?}", v);
    handle.join().unwrap();
}
\end{code-block}

\subsection{消息通信和消息传递}
每个线程做自己的事情,但是,不管什么编程语言,都需要考虑线程之间的数据交互问题。
Rust向Golang进行了学习,使用通信替换共享内存,来进行线程之间的数据传输。同样的,
Rust当中用于消息传递并发的主要工具是通道,该概念和Golang的通道概念相同。Rust的通道
分为2个角色:发送者和接收者,发送者发送消息,接收者接收消息,当发送者或者接收者任一
被丢弃时,则对应的通道被视为关闭。

Rust的通道采用mpsc::channel函数实现,mpsc表示多个生产者,单个消费者,因此,Rust
标准库实现通道的方式意味着一个通道可以有多个产生值的发送(sending)端,但只能有
一个消费这些值的接收(receiving)端。通道的实现示例如下:
\begin{code-block}{rust}
use std::sync::mpsc;
fn main() {
    let (sender, recevier) = mpsc::channel();
}
\end{code-block}
其中,函数的第一个返回值为发送者,第二个参数为接收者。使用通道发送数据通信的示例
如下:
\begin{code-block}{rust}
use std::sync::mpsc;
use std::thread;

fn main() {
    let (sender, recevier) = mpsc::channel();

    thread::spawn(move || {
        let val = "lucifer".to_string();
        match sender.send(val) {
            Ok(_) => println!("Send success"),
            Err(error) => println!("Send failed :{:?}", error),
        }
    });

    let res = match recevier.recv() {
        Ok(s) => s,
        Err(error) => {
            println!("Cannot recevie anything from sender: {:?}", error);
            "".to_string()
        }
    };
    println!("The result of channel is {}", res);
}
\end{code-block}
接收者接收消息有2种模式:默认的recv是阻塞式,返回一个Result<T, E>,当通道关闭时,
将返回Result当中的Error;而try\_recv是非阻塞式,同样是返回一个Result<T, E>,但是,
Result当中的Error表示没有接收到任何消息,可以使用for循环进行反复的尝试读取操作。
另外需要注意的是,Send函数会改变变量的所有权,当该函数执行之后,被发送的消息
(变量)将无法再使用。

但是,通道可以反复使用,而且和Golang的类似,Rust的通道也是可以进行迭代的,特别
是在接收消息时,通常采用for循环进行操作,减少了错误处理的代码,使得代码更具可读性:
\begin{code-block}{rust}
use std::sync::mpsc;
use std::thread;

fn main() {
    let (sender, recevier) = mpsc::channel();

    let handler = thread::spawn(move || {
        let vals = vec!["lucifer", "titans", "garuda"];
        for val in vals {
            match sender.send(val) {
                Ok(_) => println!("Send success"),
                Err(error) => println!("Send failed :{:?}", error),
            }
        }
    });

    for msg in recevier {
        println!("The msg is {}", msg);
    }

    match handler.join() {
        Err(error) => println!("Error{:?}", error),
        _ => (),
    }
}
\end{code-block}

同样的,由于Rust的通道默认是多生产者/单消费者,因此,可以通过多个发送端向单个接
收端发送消息。实际使用当中的多个发送端,则通常是某个发送端的克隆对象,如下:
\begin{code-block}{rust}
use std::sync::mpsc;
use std::thread;

fn main() {
    let (sender, recevier) = mpsc::channel();
    let sender_copy = sender.clone();

    let handler = thread::spawn(move || {
        let vals = vec!["lucifer", "titans", "garuda"];
        for val in vals {
            match sender.send(val) {
                Ok(_) => println!("Send success"),
                Err(error) => println!("Send failed :{:?}", error),
            }
        }
    });

    let handler_copy = thread::spawn(move || {
        let vals = vec!["zhangjl", "luoyan", "zhangzz"];
        for val in vals {
            match sender_copy.send(val) {
                Err(error) => println!("Send failed :{:?}", error),
                _ => (),
            }
        }
    });

    for msg in recevier {
        println!("The msg is {}", msg);
    }

    match handler_copy.join() {
        Err(error) => println!("Error{:?}", error),
        _ => (),
    }

    match handler.join() {
        Err(error) => println!("Error{:?}", error),
        _ => (),
    }
}
\end{code-block}

\subsection{共享状态}
在其他语言当中,有些特殊的场景,还是必须使用原有的线程并发概念——锁——来进行资源的
访问/读写控制。Rust当中同样存在锁,比较常见的就是互斥锁(互斥器,Mutex)以及原子
计数器(Arc)。在基本的操作上,互斥锁的使用和其他语言当中没有太大的区别:
\begin{code-block}{rust}
use std::sync::Mutex;
fn main() {
    let m = Mutex::new(5);
    {
        let mut num = m.lock().unwrap();
        *num = 6;
    }
    println!("m = {:?}", m);
}
\end{code-block}
注意,上述代码如果将内部大括号去除,则运行结束之后,m的状态还是锁定状态;但是,
有大括号,则表示大括号内部的段是一个有效的生命周期,当该生命周期结束之后,互斥
锁将自动释放。一旦获取了锁,就可以将返回值(在这里是num)视为一个其内部数据的
\underline{\color{red} \textbf{可变引用}}。类型系统确保了我们在使用m中的值之前
获取锁:Mutex<i32>并不是一个i32,所以必须获取锁才能使用这个i32值。

实质上,Mutex是一个智能指针,lock调用返回一个叫做MutexGuard的智能指针。这个智能
指针实现了Deref来指向其内部数据;同时也提供了一个Drop实现,使得MutexGuard离开作
用域时自动释放锁,即锁的释放是自动发生的。

但是默认情况下,Mutex是无法用于进行线程间的数据共享,如下:
\begin{code-block}{rust}
use std::rc::Rc;
use std::sync::Mutex;
use std::thread;

fn main() {
    let counter = Rc::new(Mutex::new(0));
    let mut handles = vec![];

    for _ in 0..10 {
        let counter = Rc::clone(&counter);
        let handle = thread::spawn(move || {
            let mut num = counter.lock().unwrap();

            *num += 1;
        });
        handles.push(handle);
    }

    for handle in handles {
        handle.join().unwrap();
    }

    println!("Result: {}", *counter.lock().unwrap());
}
\end{code-block}
上述代码会出现下面的类似错误:
\begin{figure}[H]
  \centering
  \includegraphics[scale=0.215]{rust_mutex_share_error.png}
  \caption{试图通过Rc共享Mutex的数据}
  \label{fig:rust_mutex_share_error}
\end{figure}
即之前提到的,Rc类型只能用于单线程/单进程环境。

而共享引用计数则需要使用Arc,它是可以安全的用于并发环境的类型,即原子引用计数,
可以在线程间进行共享所有权。Arc和Rc有相同的API,基本使用方法上类似。所有,可以直
接对上述代码进行修改:
\begin{code-block}{rust}
use std::sync::{Arc, Mutex};
use std::thread;
fn main() {
    let counter = Arc::new(Mutex::new(0));
    let mut handles = vec![];
    for _ in 0..10 {
        let counter = Arc::clone(&counter);
        let handle = thread::spawn(move || {
            let mut num = counter.lock().unwrap();
            *num += 1;
        });
        handles.push(handle);
    }
    for handle in handles {
        handle.join().unwrap();
    }
    println!("Result: {}", *counter.lock().unwrap());
}
\end{code-block}
通过这样简单的修改,成功实现了10个进程当中对同一个数值进行加法操作的功能。

\section{Match与模式匹配}
Match是Rust常用的语法糖,其用法不局限于之前所讲的范围。关于match的用法,还有很多,
并且,多数和模式匹配有关,接下来可以看一些常见的match和模式匹配的使用方式。
\begin{outline}[enumerate]
\1 多种匹配模式

在match表达式当中,可以用|匹配多个模式,表示或运算:
\begin{code-in-enumerate}{rust}
let x = 1;
match x {
    1 | 2 => println!("one or two"),
    3 => println!("three"),
    _ => println!("anything"),
}
\end{code-in-enumerate}

\1 使用..=匹配范围

..=语法允许匹配一个数值范围内的任意数据,常用于数值和字符:
\begin{code-in-enumerate}{rust}
let x = 5;
match x {
    1..=5 => println!("one through five"),
    _ => println!("something else"),
}

let y = 'c';
match y {
    'a'..='j' => println!("early ASCII letter"),
    'k'..='z' => println!("late ASCII letter"),
    _ => println!("something else"),
}
\end{code-in-enumerate}

\1 解构结构体

Let模式可以将结构体当中的字段/元素进行解构,单独或者批量赋予其他元素:
\begin{code-in-enumerate}{rust}
struct Point {
    x: i32,
    y: i32,
}
fn main() {
    let p = Point { x: 0, y: 7 };
    // 将p的x字段的值赋予a,y字段的值赋予b,a和b是整数类型,不是引用
    let Point { x: a, y: b } = p;
    // let Point {x: ref a, y: ref b} = p; 和上面类似,但是a和b是整数类型的引用
    // let Point {x: a, y: _} = p; 表示只需要将x的值赋予a,但不需要对y进行解构
    assert_eq!(0, a);
    assert_eq!(7, b);
    // let Point {x, y} = p; 将p的x字段的值赋予变量x,y字段的值赋予变量y
}
\end{code-in-enumerate}

\1 解构枚举类型

Match本身就是应枚举而生的,因此天然的可以使用它对枚举进行解构:
\begin{code-in-enumerate}{rust}
enum Message {
    Quit,
    Move { x: i32, y: i32 },
    Write(String),
    ChangeColor(i32, i32, i32),
}

fn main() {
    let msg = Message::ChangeColor(0, 160, 255);

    match msg {
        Message::Quit => {
            println!("The Quit variant has no data to destructure.")
        }
        Message::Move { x, y } => {
            println!("Move in the x direction {} and in the y direction {}", x, y);
        }
        Message::Write(text) => println!("Text message: {}", text),
        Message::ChangeColor(r, g, b) => {
            println!("Change the color to red {}, green {}, and blue {}", r, g, b)
        }
    }
}
\end{code-in-enumerate}

同样的,如果枚举当中嵌套了枚举,仍然可以使用match进行解构:
\begin{code-in-enumerate}{rust}
enum Color {
    Rgb(i32, i32, i32),
    Hsv(i32, i32, i32),
}

enum Message {
    Quit,
    Move { x: i32, y: i32 },
    Write(String),
    ChangeColor(Color),
}

fn main() {
    let msg = Message::ChangeColor(Color::Hsv(0, 160, 255));

    match msg {
        Message::ChangeColor(Color::Rgb(r, g, b)) => {
            println!("Change the color to red {}, green {}, and blue {}", r, g, b)
        }
        Message::ChangeColor(Color::Hsv(h, s, v)) => {
            println!(
                "Change the color to hue {}, saturation {}, and value {}",
                h, s, v
            )
        }
        _ => (),
    }
}
\end{code-in-enumerate}

\1 解构复合数据

用复杂的方式来混合、匹配和嵌套解构模式,解析出我们感兴趣的数据:
\begin{code-in-enumerate}{rust}
let ((feet, inches), Point {x, y}) = ((3, 10), Point { x: 3, y: -10 });
\end{code-in-enumerate}

\1 忽略不需要的元素

在Rust的当中,默认可以使用\_对不必要的变量进行忽略,通常用在match的最后分支,但是,
实际上也可以用去其他任意的模式,甚至是函数参数:
\begin{code-in-enumerate}{rust}
// 需要传入2个参数,但是忽略第一个参数
fn foo(_: i32, y: i32) {
    println!("This code only uses the y parameter: {}", y);
}

fn main() {
    foo(3, 4);
}
\end{code-in-enumerate}

除了使用\_进行忽略之外,还可以使用..语法糖进行忽略,但是针对结构体和元组存在区别:
结构体当中,忽略的是没有被列出的字段;而元组忽略的则是范围:
\begin{code-in-enumerate}{rust}
struct Point {
    x: i32,
    y: i32,
    z: i32,
}

fn main() {
    let origin = Point { x: 0, y: 0, z: 0 };
    // 将point的y进行忽略
    match origin {
        Point { x,z, .. } => println!("x is {}, z is {}", x, z),
    }

    let numbers = (2, 4, 8, 16, 32);
    match numbers {
        // 忽略元组当中除第1、2和最后一项的所有元素
        (first, second, .., last) => {
            println!("Some numbers: {}, {}, {}, ", first, second, last);
        }
    }
}
\end{code-in-enumerate}

同样的,忽略操作也可以用于闭包当中:
\begin{code-in-enumerate}{rust}
let player_scores = [("Jack", 20), ("Jane", 23), ("Jill", 18), ("John", 19)];
// 对player_scores进行迭代,忽略其中第二个元素,_可以被替换为_score
let players: Vec<_> = player_scores.iter().map(|&(player, _)| player).collect();
// 输出的结果当中将只会有字符串数据
println!("{:?}", players);
\end{code-in-enumerate}


\1 @绑定

运算符@允许我们在创建一个存放值的变量的同时测试其值是否匹配模式,比如测试字段是
否位于指定范围内,同时也希望能将其值绑定到另外的变量中以便此分支相关联的代码可以
使用它:
\begin{code-in-enumerate}{rust}
enum Message {
    Hello { id: i32 },
}

let msg = Message::Hello { id: 5 };

match msg {
    // 将变量id保存到另一个变量ip_variable当中
    Message::Hello { id: id_variable @ 3..=7 } => {
        println!("Found an id in range: {}", id_variable)
    },
    Message::Hello { id: 10..=12 } => {
        println!("Found an id in another range")
    },
    Message::Hello { id } => {
        println!("Found some other id: {}", id)
    },
}
\end{code-in-enumerate}
\end{outline}

\section{高级特征}
Rust设计不仅仅是为了开发应用程序,其设计之初,就是为了解决内存安全的问题,并且可以
广泛用于各种场景,包括C语言的专属领域:操作系统设计。在编写操作系统的过程当中,
C语言使用了很多的高级宏定义以及一些精妙的设计,而Rust同样如此。为了和硬件打交道,
Rust被设计为可以拥有直接操作硬件的能力,这些都是其高级特性的一部分。Rust的高级
特性主要包含下列内容:
\begin{enumerate}
  \item 不安全 Rust
  \item 高级 Trait
  \item 高级函数和闭包
  \item 宏
\end{enumerate}

\subsection{Unsafe}
Rust屏蔽了一系列的不安全操作来换取应用程序的稳定性和可靠性,但是,可以通过关键字
unsafe,切换到不安全的运行环境当中,并且在unsafe的代码块当中运行。常见的不安全操作
如下:
\begin{enumerate}
  \item 解引用裸指针
  \item 使用不安全的方法/函数
  \item 访问/修改可变的静态变量
  \item 实现不安全的Trait
  \item 访问union的字段
\end{enumerate}
在使用的时候,原则需要明确:保持unsafe块尽可能小,将不安全代码封装进一个安全的
抽象并提供安全API是一种常见的安全操作和手段。

所谓的裸指针,和普通的指针和智能指针相比,存在如下的区别:
\begin{enumerate}
  \item 允许忽略借用规则,可以同时拥有不可变和可变的指针,或多个指向相同位置的可变指针
  \item 不保证指向有效的内存
  \item 允许为空
  \item 不能实现任何自动清理功能
\end{enumerate}
Rust当中存在2个裸指针:分别写作*const T(不可变)和*mut T(可变),其基本的定义方式
如下:
\begin{code-block}{rust}
let mut num = 5;
let r1 = &num as *const i32; // 不可变的裸指针
let r2 = &mut num as *mut i32; // 可变的裸指针
\end{code-block}

裸指针的定义是安全的,但是,它的使用是不安全的,因此裸指针的使用必须在unsafe块
当中:
\begin{code-block}{rust}
fn main() {
    let mut num = 5;
    let r1 = &num as *const i32;
    let r2 = &mut num as *mut i32;
    unsafe {
        *r2 = 10;
        // r1,r2和num都会变更为10
        println!("{},{}", *r1, *r2);
    }
}
\end{code-block}
同样的,unsafe也可以用于定义函数/方法,不过也需要在unsafe块当中使用;但是,unsafe
的方法可以作为安全方法进行导出,在使用时,则不需要使用unsafe进行标记:
\begin{code-block}{rust}
fn main() {
    let mut num = 5;
    // 定义裸指针
    let r1 = &num as *const i32;
    let r2 = &mut num as *mut i32;
    // 使用不安全的函数/方法
    unsafe {
        unsafe_change(r1, r2);
    }
    println!("{}", num);
    safe_change(r1, r2);
    println!("{}", num);
}

// 定义不安全的函数/方法
unsafe fn unsafe_change(r1: *const i32, r2: *mut i32) {
    *r2 = 10;
    println!("{},{}", *r1, *r2);
}

// 将不安全的函数/方法封装进安全的方法当中
fn safe_change(r1: *const i32, r2: *mut i32) {
    unsafe {
        *r2 = 100;
    }
}
\end{code-block}

作为不安全的一部分,某些时候直接在Rust当中调用C语言的类库可以获得更好的性能,此时,
则同样需要在unsafe块当中使用,比如在Rust当中调用标准C的abs(绝对值)函数:
\begin{code-block}{rust}
extern "C" {
    fn abs(input: i32) -> i32;
}
fn main() {
    unsafe {
        println!("The unsafe from C: {}", abs(-200));
    }
}
\end{code-block}
上述代码出现的extern关键字,有助于创建和使用外部函数接口(Foreign Function
Interface,FFI)。外部函数接口是一个编程语言用以定义函数的方式,其允许不同(外部)
编程语言调用这些函数。Extern块中声明的函数在Rust代码中总是不安全的,

特别需要注意的是,Rust当中的可变全局变量(static)同样是不安全的,需要在unsafe
代码块当中使用;而不可变的全局常量(const和static)则不需要在unsafe块当中;另外,
全局变量同样可以是任意数据类型的:
\begin{code-block}{rust}
use std::fmt;
struct Version {
    major: u8,
    minor: u8,
}
impl fmt::Display for Version {
    fn fmt(&self, f: &mut fmt::Formatter) -> fmt::Result {
        write!(
            f,
            "The version of this bin is {}.{}",
            self.major, self.minor
        )
    }
}

// 不可变的全局常量
const __CONST_NUM__: Version = Version { major: 1, minor: 4 };
const __VERSION__: &str = "v1.4.0";
static __NAME__: &str = "lucifer";
// 可变的全局变量
static mut __COUNTER__: u8 = 1;
fn main() {
    println!("{}", __CONST_NUM__);
    println!("{}", __NAME__);
    unsafe {
        println!("{}", __COUNTER__);
    }
}
\end{code-block}

但是,并不是所有情形都适合使用unsafe,Rust本身也无法从编译器层面,保证unsafe的
代码块是完全正确的,不会出现任何错误的。比如,我们在使用裸指针*const T和*mut T
的时候,如果不够仔细,非常容易造成错误的结果:
\begin{code-block}{rust}
let mut y: u32 = 1;
let x = 1_i32;
// 将y转换成u32的裸指针,再转换成i32的裸指针,最后转换成i64的裸指针
let raw_mut = &mut y as *mut u32 as *mut i32 as *mut i64;
unsafe {
    // 对裸指针进行修改,类似于C/C++当中对指针数据的操作
    *raw_mut = -1;
}
info!("The x is {:X} and y is {:X}", x, y);
\end{code-block}
按照我们本来的设想,x会保持不变,始终为1,而y则可能变换成其他的数值,但是,实际
的结果却如下:
\begin{figure}[H]
  \centering
  \includegraphics[width=\linewidth]{rust_raw_pointer.png}
  \caption{具有潜在错误的裸指针示例}
  \label{fig:rust_raw_pointer}
\end{figure}
x变成和y一样的值的原因在于:对指向y的指针类型做了转换,让它以为自己指向的是i64
类型,恰巧x就在y旁边,y被修改的同时,就顺带把x也修改了。因此,使用unsafe必须特别
小心。

在通常的情况下,虽然可以通过引用+mut的方式,可以阻止大部分的内存不安全问题,但是
由于引用+mut的强限制性,也为带来一些比较麻烦和无奈的问题,比如下面的代码:
\begin{code-block}{rust}
#[derive(Debug)]
struct Tuple {
    first: u8,
    second: u8,
    third: u8,
}

fn main() {
    let mut t = Tuple {
        first: 0,
        second: 1,
        third: 2,
    };
    let pa = &mut t.first;
    let pb = &mut t.second;
    let pc = &mut t.third;
    *pc += 10;
    info!("{:?}", t);
}
\end{code-block}
上述代码是正确无误的,可以正常编译和运行,但是,如果我们将结构体变成数组,问题就
出现了:
\begin{code-block}{rust}
fn main() {
    let mut array_x = [1_i32, 2, 8];
    let pa = &mut array_x[0];
    let pb = &mut array_x[1];
    *pb += 10;
    info!("{:?}", t);
}
\end{code-block}
上述代码在Rust 1.50.0版本之前就会出现错误:
\begin{code-block}{bash}
error: cannot borrow `x[..]` as mutable more than once at a time
\end{code-block}
原因在于,结构体当中,pa,pb和pc指向不同的内存区域;但是在数据当中,Rust编译器会
将[\_]识别为一个整体,而\&[0], \&[1]之间都属于重叠,将pa和pb判断为存在别名关系,
即pa和pb实质上相同,违反了借用规则,因此无法通过编译。在Rust 1.50.0版本之后,编译器
针对数组进行了优化,上述代码不会再出现错误,但是,在旧版本的Rust当中,则需要采用
其他手段进行解决,即引用分割:
\begin{code-block}{bash}
let mut array_x = [1_i32, 2, 3];
// 通过split_at_mut将数组切分成2个一定不会重叠的切片
let (first, rest): (&mut [i32], &mut [i32]) = array_x.split_at_mut(1);
let (second, third): (&mut [i32], &mut [i32]) = rest.split_at_mut(1);

first[0] += 100;
second[0] += 200;
third[0] += 300;
info!("{:?}", array_x);
\end{code-block}

\subsection{高级Trait}
Trait的语法当中,使用了如下的代码形式:
\begin{code-block}{rust}
impl Iterator for Counter {
    type Item = u32;
    fn next(&mut self) -> Option<Self::Item> {
        ...
    }
}
\end{code-block}
其中,type Item表示关联数据类型,Item表示占位类型,next方法定义表明它返回
Option<Self::Item>类型的值。这个trait的实现者会指定Item的具体类型。

在Trait当中,除了默认方法,方法覆写之外,还存在着运算符重载的功能。但是,和C++
不同,Rust并不允许创建自定义的运算符,或者重载任意运算符,只有std::ops当中所列出
的运算符和相应的trait可以通过实现运算符相关的trait来实现重载,比如下面,实现Add
trait来实现对+的运算符重载:
\begin{code-block}{rust}
use std::fmt;
use std::ops::{Add, AddAssign};
struct Point {
    x: u8,
    y: u8,
}
// 实现结构体的 c = a + b
impl Add for Point {
    type Output = Point;
    fn add(self, other: Point) -> Point {
        Point {
            x: self.x + other.x,
            y: self.y + other.y,
        }
    }
}
// 实现结构体的 a = a + b;
impl AddAssign for Point {
    fn add_assign(&mut self, other: Point) {
        self.x = self.x + other.x;
        self.y = self.y + other.y;
    }
}
impl fmt::Display for Point {
    fn fmt(&self, f: &mut fmt::Formatter) -> fmt::Result {
        write!(f, "x:{}, y:{}", self.x, self.y)
    }
}
fn main() {
    let a = Point { x: 3, y: 4 };
    let b = Point { x: 5, y: 6 };
    let c = a + b;
    let mut d = Point { x: 100, y: 101 };
    d = d + c;
    println!("{}", d);
}
\end{code-block}

以Add Trait为例,其内部的实现如下:
\begin{code-block}{rust}
#[lang = "add"]
pub trait Add<Rhs = Self> {
    type Output;
    #[must_use]
    fn add(self, rhs: Rhs) -> Self::Output;
}
\end{code-block}
其中,RHS=Self这个语法叫做默认类型参数。RHS是一个泛型类型参数,它用于定义add方法
中的rhs参数。如果实现Add trait时不指定RHS的具体类型,RHS的类型将是默认的Self类型
也就是在实现Add Trait的类型。在上述例子当中,RHS就是Point这个类型。但是,也可以使用
不同的数据类型,比如下面的例子:
\begin{code-block}{rust}
struct Meters(i32);
struct Millimeters(i32);
impl Add<Meters> for Millimeters {
    type Output = Millimeters;
    fn add(self, other: Meters) -> Millimeters {
        Millimeters(self.0 + (other.0 * 1000))
    }
}
\end{code-block}
定义一个结构体米,和结构体毫米,然后定义毫米与米的加法操作,当结构体毫米与结构体
米进行相加时(注意顺序),将结果转换成毫米结果:
\begin{code-block}{rust}
let meters = Meters(1);
let mill_meters = Millimeters(10);
let mill_meters_other = mill_meters + meters;
\end{code-block}
但是,如果将上述代码的顺序更换如下:
\begin{code-block}{rust}
let mill_meters_other = meters + mill_meters;
\end{code-block}
则会出现如下的错误:
\begin{figure}[H]
  \centering
  \includegraphics[width=\linewidth]{rust_override_error.png}
  \caption{尝试进行不同类型的加法重载操作}
  \label{fig:rust_override_error}
\end{figure}
修复上述的错误也很简单,增加结构体米的加法操作重载运算符即可:
\begin{code-block}{rust}
struct Meters(i32);
struct Millimeters(i32);
impl Add<Meters> for Millimeters {
    type Output = Millimeters;
    fn add(self, other: Meters) -> Millimeters {
        Millimeters(self.0 + (other.0 * 1000))
    }
}
impl Add<Millimeters> for Meters {
    type Output = Millimeters;
    fn add(self, other: Millimeters) -> Millimeters {
        Millimeters(self.0 * 1000 + other.0)
    }
}
\end{code-block}
这样,在进行结构体米和结构体毫米之间的加法操作是,无需考虑操作数的顺序。

在之前,也提到了Deref Trait的用法,通常用于进行智能指针的解引用操作,使得智能指针
可以直接当作指定的类型使用。不过Deref Trait不仅仅可以针对智能指针,也可以对自定义
的数据类型添加其他各种操作。比较典型的例子,现在有一个Vec,其中包含的数据类型是
String,如果需要打印这个Vec<String>,则必须使用debug这个宏定义;如果不想使用这个
debug,则必须在Vec<String>上实现Display Trait,但是Display Trait是无法直接作用在
Vec<String>上的,因此我们可以采用一种方式,在一个结构体当中包含匿名的Vec<String>,
然后在这个结构体上实现Display Trait,如下:
\begin{code-block}{rust}
use std::fmt;
use std::ops::{Deref, DerefMut};
struct VecWrapper(Vec<String>);
impl Deref for VecWrapper {
    type Target = Vec<String>;
    fn deref(&self) -> &Vec<String> {
        &self.0
    }
}
impl DerefMut for VecWrapper {
    fn deref_mut(&mut self) -> &mut Vec<String> {
        &mut self.0
    }
}
impl fmt::Display for VecWrapper {
    fn fmt(&self, f: &mut fmt::Formatter) -> fmt::Result {
        write!(f, "[")?;
        for item in &self.0 {
            write!(f, "{}, ", item)?;
        }
        write!(f, "]")
    }
}
fn main() {
    let mut v = VecWrapper(vec![String::from("hello"), String::from("world")]);
    v.push("zhangjl".to_string());
    for item in &v.0 {
        println!("{}", item);
    }
    println!("{}", v);
}
\end{code-block}
在上述代码当中,使用VecWrapper将Vec<String>进行简单的封装,然后使用Deref Trait
实现对VecWrapper的解引用(包括可变和不可变),将对VecWrapper的解引用操作重定向
到直接访问Vec<String>,这样带来的好处如下:
\begin{enumerate}
  \item 无需针对VecWrapper进行额外的其他操作,即可使用所有Vec<String>的所有方法
  \item 可以如同Vec一样的进行任意的操作
  \item 可以实现VecWrapper的自定义函数/方法,但又不影响原本的Vec操作
\end{enumerate}

另外需要注意,在上述代码当中,我们再次使用了?操作符,由于write本身返回的是一个
Result类型,但是,如果write操作后面添加分号符,表示目前只考虑了正确的模式,忽略
了错误的处理,因此编译器会提示警告。为了消除这个警告,则可以使用?代替Result类型,
同时继续正确和错误分支的处理。

同样的,上述的代码VecWrapper可以做成泛型,如下:
\begin{code-block}{rust}
use std::ops::{Deref, DerefMut};
struct VecWrapper<T>(Vec<T>);
impl<T> Deref for VecWrapper<T> {
    type Target = Vec<T>;
    fn deref(&self) -> &Vec<T> {
        &self.0
    }
}
impl<T> DerefMut for VecWrapper<T> {
    fn deref_mut(&mut self) -> &mut Vec<T> {
        &mut self.0
    }
}
\end{code-block}
上述的代码只是将Vec做了一次封装,可以使用任何的Vec方法,但是,我们将无法对这个
类型实现Display Trait,因为泛型T本身是无法实现Display Trait的。

Trait的另外一个非常重要的用途就是实现继承。涉及到继承实现,不可避免的会遭遇到
函数/方法的重载/覆写,尤其是多重继承的时候。在Rust当中,同样允许不同的Trait有相
同的函数/方法定义,也同样允许一个类型实现多个Trait,比如下面的代码:
\begin{code-block}{rust}
trait Pilot {
    fn fly(&self);
    fn name();
}
trait Wizard {
    fn fly(&self);
    fn name();
}
struct Empty;
impl Pilot for Empty {
    fn fly(&self) {
        println!("This is the implement of Pilot fly method");
    }
    fn name() {
        println!("This is the name method of Pilot implement");
    }
}
impl Wizard for Empty {
    fn fly(&self) {
        println!("This is the implement of Wizard fly method");
    }
    fn name() {
        println!("This is the name method of Wizard implement");
    }
}
\end{code-block}
Trait Pilot和Wizard定义了一个同名的方法,以及一个同名的关联函数(没有self做参数),
然后结构体Empty实现了这2个Trait,但是本身没有任何的方法/函数。但是,在使用的时候,
则必须注意,一定要进行Trait的指定或者转换,否则由于存在同名函数/方法,可能导致
代码的二义性出现,从而导致错误:
\begin{code-block}{rust}
fn main() {
    let empty = Empty {};
    empty.fly();
    Empty::name();
}
\end{code-block}
\begin{figure}[H]
  \centering
  \includegraphics[width=\linewidth]{rust_same_name.png}
  \caption{实现包含同名函数/方法的多个Trait}
  \label{fig:rust_same_name}
\end{figure}
解决这种问题的方法主要有2种思路:1是增加Empty结构体自身的同名函数/方法的实现,
但这种思路相当于完全没有利用Trait的任何功能;2是对Empty进行Trait的指定,如下:
\begin{code-block}{rust}
fn main() {
    let empty = Empty {};
    Wizard::fly(&empty);
    Pilot::fly(&empty);
    <Empty as Wizard>::name();
    <Empty as Pilot>::name();
}
\end{code-block}
同样的,如果不同的Trait包含了同名的方法/函数,但是参数和返回值定义不同,在使用的
时候,也需要进行明确的指定:
\begin{code-block}{rust}
trait Pilot {
    fn fly(&self);
    fn name();
}
trait Wizard {
    fn fly(&self, name: &str);
    fn name(age: u8) -> u8;
}
...
fn main() {
    let empty = Empty {};
    Wizard::fly(&empty, "lucifer");
    <Empty as Wizard>::name(64);
}
\end{code-block}

\subsection{高级函数和闭包}
Rust的函数和闭包都有很多类似的地方,和C/C++的函数也类似,确切的说,是非常类似于
C/C++当中的函数指针,因此,Rust的函数和闭包,也可以作为函数的参数以及返回值。但是,
函数作为参数和返回值与闭包有些区别,先看使用函数作为参数与返回值,如下:
\begin{code-block}{rust}
fn newmethod() -> fn(u32) -> u32 {
    calc
}

fn fn_as_params(age: u32, f: fn(u32) -> u32) {
    println!("In the fn_as_params: {}", f(age));
}

fn calc(age: u32) -> u32 {
    age * 100
}

fn main() {
    let b = newmethod();
    println!("{}", b(32));

    fn_as_params(32, calc);
}
\end{code-block}
可以看到,函数作为参数和返回值,基本用法和C/C++当中的方式是一致的。但是,使用闭包
的情形有些区别:闭包缺少具体的大小(size)描述,如果直接传递闭包,则会因为编译器
无法知道当前参数的大小而报错,因此,当使用闭包作为参数时,需要如下进行处理:
\begin{code-block}{rust}
// 直接使用闭包
fn hello(age: u32, func: &dyn Fn(u32) -> u32) {
    println!("{}", func(age));
}

// 使用Box智能指针
fn recv(age: u32, func: Box<dyn Fn(u32) -> u32>) {
    println!("{}", func(age));
}

// 使用Fn Trait
fn asparams(age: u32, func: impl Fn(u32) -> u32) -> u32{
    func(age)
}

fn add(start: u32) -> u32 {
    start + 100
}

fn main() {
    let c = |x| x * 10;
    recv(10, Box::new(c));
    hello(8, &c);
    // 2种方式都可以
    asparams(8, &c);
    asparams(8, c);
    // Fn Trait针对函数
    asparams(8, add);
    asparams(8, &add);
}
\end{code-block}
特别提倡使用Fn Trait的方式,因为这种方式,可以处理闭包和闭包的引用,同时,同样可以
处理函数以及函数的引用,相当于是一种比较万能的方式。

同样的,使用闭包作为函数的返回值,也是需要进行额外的特殊处理:
\begin{code-block}{rust}
// 使用Box智能指针
fn newclosur() -> Box<dyn Fn(u32) -> u32> {
    Box::new(|x| x * 100)
}

// 使用Fn Trait
fn return_closur() -> impl Fn(u32) -> u32 {
    |x| x * 120
}

fn main() {
    let a = newclosur();
    println!("{}", a(12));

    let a = return_closur();
    println!("{}", a(18));
}
\end{code-block}
和上述的使用闭包的情形类似,使用Fn Trait的方式,不仅可以使用闭包作为返回值,同样
可以使用函数作为返回值。

\subsection{高级类型}
默认情况下,我们使用的Rust变量都是有类型的,而这些类型,默认情况下都需要书写完整。
如果名称太长,则会导致阅读比较麻烦,因此,Rust提供了别名。注意,Rust的别名与Golang
的别名不一样:Rust的类型别名拥有和原始类型相同的函数和方法,可以进行原始名称和
别名系统的自动转换和其他操作,但是Golang的类型别名和原始类型无法直接进行自动转换
和计算。Rust的别名如下:
\begin{code-block}{rust}
enum VeryVerboseEnumOfThingsToDoWithNumbers {
    Add,
    Subtract,
}
// 定义别名
type Operations = VeryVerboseEnumOfThingsToDoWithNumbers;
fn main() {
    // 直接使用别名
    let x = Operations::Add;
}
\end{code-block}
同样的,别名还可以用于其它场合,比如在impl的代码当中使用Self别名(注意大写):
\begin{code-block}{rust}
enum VeryVerboseEnumOfThingsToDoWithNumbers {
    Add,
    Subtract,
}
impl VeryVerboseEnumOfThingsToDoWithNumbers {
    fn run(&self, x: i32, y: i32) -> i32 {
        match self {
            // 替换原本的 VeryVerboseEnumOfThingsToDoWithNumbers::Add方式
            Self::Add => x + y,
            Self::Subtract => x - y,
        }
    }
}
\end{code-block}

Enum这种类型在之前的讲解当中,没有看到使用其当作各种类似于C/C++相同的值应用。但是,
实际上,Enum是可以直接使用变量值的方式的,同样的,Enum是可以转换成整数数值的:
\begin{code-block}{rust}
enum Number {
    Zero,
    One,
    Two,
}
enum Color {
    Red = 0xff0000,
    Green = 0x00ff00,
    Blue = 0x0000ff,
}

fn main() {
    println!("zero is {}", Number::Zero as i32);
    println!("one is {}", Number::One as i32);
    println!("roses are #{:06x}", Color::Red as i32);
    println!("violets are #{:06x}", Color::Blue as i32);
}
\end{code-block}

Rust的类型之间是可以相互进行转换的,但是Rust的类型转换必须使用显式的方式,不支持
隐式数据类型转换,数据类型转换必须使用as关键字,并且,转换也会存在数据溢出的情况:
\begin{code-block}{rust}
let deciamal = 65.81_f32;
let interer: u8 = deciamal as u8;
// 无法从浮点数直接转换成char类型
let charter: char = interer as char;
\end{code-block}
当从任何数转换成无符号数据(比如u64,u32,u8),则会在原始数据上进行加/减当前数据
类型的最大值+1,直到最后的数据处于新的无符号数据的有效数据范围内:
\begin{code-block}{rust}
let start = 1000;
let start_nev = -1000;
// 1000 -(255+1) - (255+1) - (255+1)
let res1 = start as u8;
// -1000 + (255+1) + (255+1) + (255+1) + (255 + 1)
let res2 = start_nev as u8
\end{code-block}

但是,当转换的类型不是数值类型,而是复合数据类型,则需要使用From和Into这2个Trait。
一般情况下,只要实现了From Trait,就默认的实现了Into Trait,比如下方代码:
\begin{code-block}{rust}
use std::convert::From;
struct Point {
    x: i32,
    y: i32,
}

// 实现From trait,只能从(i32,i32)到结构体,不能反向
// 同样的,Into也是只能从(i32,i32)到结构体,不能反向
impl From<(i32, i32)> for Point {
    fn from(item: (i32, i32)) -> Self {
        Point {
            x: item.0,
            y: item.1,
        }
    }
}

// 如果想实现从结构体到(i32, i32)的自动转换,则需要实现另外的From
impl From<Point> for (i32, i32) {
    fn from(item: Point) -> Self {
        (item.x, item.y)
    }
}

fn main() {
    // 显式的调用from
    let p1 = Point::from((81, 99));
    println!("{}, {}", p1.x, p1.y);

    // 显式的调用into(隐式实现的Into Trait)
    let p1: Point = (100, 101).into();
    println!("{}, {}", p1.x, p1.y);

    // 从结构体到(i32, i32)的自动转换
    let (x, y): (i32, i32) = p1.into();
    // 但是通常不这么做,而是使用模式匹配进行解决,而且模式匹配更为灵活,如下
    let Point{x:a, y: b} = p1;
}
\end{code-block}

如果复合数据类型进行转换,不能保证一定成功,则需要使用TryFrom以及TryInfo,如下:
\begin{code-block}{rust}
use std::convert::TryFrom;

struct Point {
    x: u8,
    y: u8,
}

impl TryFrom<(u8, u8)> for Point {
    type Error = ();
    fn try_from(value: (u8, u8)) -> Result<Self, Self::Error> {
        if value.0 % 2 == 0 && value.1 % 2 == 0 {
            Ok(Point {
                x: value.0,
                y: value.1,
            })
        } else {
            Err(())
        }
    }
}

fn main() {
    let tup = (32, 63);
    let p = Point::try_from(tup).expect("Invalid Params");
    println!("{}, {}", p.x, p.y);
}
\end{code-block}

作为通用的需求,将复合数据类型进行打印输出,可以采用debug宏,可以实现Display Trait,
还可以实现to\_string方法(ToString Trait):
\begin{code-block}{rust}
use std::string::ToString;

struct Point {
    x: u8,
    y: u8,
}

impl ToString for Point {
    fn to_string(&self) -> String {
        format!("Point x:{}, y:{}", self.x, self.y)
    }
}

fn main() {
    let p = Point { x: 12, y: 127 };
    println!("{}", p.to_string());
}
\end{code-block}
通常情况下,Display和ToString两者实现其一即可。除了将结构体转换成字符串,也可以将
字符串转换成结构体,而这时就需要使用FromStr这个Trait,只是普通场景下,使用不多:
\begin{code-block}{rust}
use std::num::ParseIntError;
use std::str::FromStr;
use std::string::ToString;

struct Point {
    x: u8,
    y: u8,
}

impl ToString for Point {
    fn to_string(&self) -> String {
        format!("Point x:{}, y:{}", self.x, self.y)
    }
}

impl FromStr for Point {
    type Err = ParseIntError;

    fn from_str(s: &str) -> Result<Self, Self::Err> {
        let coords: Vec<&str> = s
            .trim_matches(|p| p == '(' || p == ')')
            .split(',')
            .collect();

        let x_fromstr = coords[0].replace(" ", "").parse::<u8>()?;
        let y_fromstr = coords[1].replace(" ", "").parse::<u8>()?;

        Ok(Point {
            x: x_fromstr,
            y: y_fromstr,
        })
    }
}

fn main() {
    let p = Point::from_str("(1, 203   )").unwrap();
    println!("{}", p.to_string());
}
\end{code-block}

Rust当中没有goto语句的存在,对于某些特殊的场景,比如多层循环的内部跳出就显得不是
非常友好。因此,Rust也提供了label的机制,允许直接break对应的label,从而快速跳出
循环结构:
\begin{code-block}{rust}
fn main() {
    // 定义循环的标签
    'outer: loop {
        println!("Entered the outer loop");
        'inner: loop {
            println!("Entered the inner loop");

            // 这只是中断内部的循环
            //break;

            // 这会中断外层循环
            break 'outer;
        }
        println!("This point will never be reached");
    }
    println!("Exited the outer loop");
}
\end{code-block}

除了上述这些类型之外,在之前的Trait当中,提到了关联数据类型。关联数据类型通常用在
Trait当中,比如下方的Trait定义:
\begin{code-block}{rust}
trait Container {
    type A;
    type B;
    fn contains(&self, item_a: Self::A, item_b: Self::B) -> bool;
    fn first(&self) -> Self::A;
    fn last(&self) -> Self::B;
    fn val(&self) -> u8;
}

// 如果没有type A和type B的定义,则下面的方法必须写成如下的形式:
// fn difference<A, B, C>(container: &C) -> u8 where
//    C: Container<A, B> { ... }

fn difference<C: Container>(c1: &C, c2: &C) -> u8 {
    c1.val() - c2.val()
}

struct People<'a> {
    name: &'a str,
    age: u8,
}

impl<'a> Container for People<'a> {
    type A = &'a str;
    type B = u8;

    fn contains(&self, item_a: &str, item_b: u8) -> bool {
        self.age == item_b && self.name == item_a
    }
    fn first(&self) -> &'a str {
        self.name
    }
    fn last(&self) -> u8 {
        self.age
    }
    fn val(&self) -> u8 {
        self.age
    }
}

fn main() {
    let p = People {
        name: "zhangjl",
        age: 32,
    };
    println!("{}", p.last());
    println!("{}", p.first());
    println!("{}", p.contains("zhangjl", 32));
}
\end{code-block}

Rust当中,还存在一种非常特殊的类型:虚类型(phantom type)。这种类型通常用于函数
参数,表示在运行时不出现,但是在进行编译时进行静态检查。这种类型的参数不占据任何
存储空间,也不存在运行时行为,比如,现在设定一个英寸的结构体和一个米的结构体,
要求这2个结构体不能相加,则可以使用phantom虚类型进行解决:
\begin{code-block}{rust}
// 虚类型
use std::marker::PhantomData;
use std::ops::Add;

extern crate slog_scope;
extern crate slog_stdlog;
#[macro_use]
extern crate log;
extern crate logger;

enum Inch {}
enum Mm {}

// 使用虚类型进行定义
struct Length<Unit>(f64, PhantomData<Unit>);

impl<Unit> Add for Length<Unit> {
    type Output = Length<Unit>;

    fn add(self, other: Self) -> Self {
        Length(self.0 + other.0, PhantomData)
    }
}

fn main() {
    let logger = logger::initlogger(false, "", 0);
    let _guard = slog_scope::set_global_logger(logger);
    slog_stdlog::init().unwrap();

    let one_foot: Length<Inch> = Length(12.0, PhantomData);
    let another_foot: Length<Inch> = Length(12.0, PhantomData);
    let one_meter: Length<Mm> = Length(1000.0, PhantomData);
    let another_meter: Length<Mm> = Length(1000.0, PhantomData);

    let two_feet = one_foot + another_foot;
    let two_meters = one_meter + another_meter;

    info!("The feet is {}", two_feet.0);
    info!("The meters is {}", two_meters.0);

    // 由于2者的虚类型不同,编译期间直接提示出错
    let r = one_foot + one_meter;
}
\end{code-block}

迭代器也是常用的数据类型,在之前的介绍当中已经有基本的介绍,在这里,将使用斐波那契
数列来讲述可以无限计算下去的迭代器的实现方式,不管是有限或无限,迭代器都必须实现
Iterator这个Trait:
\begin{code-block}{rust}

// 实现结构体的clone
#[derive(Clone)]
struct Fibonacci {
    current: u64,
    nxt: u64,
}

impl Iterator for Fibonacci {
    type Item = u64;

    fn next(&mut self) -> Option<Self::Item> {
        let tmp = self.current;
        self.current = self.nxt;
        self.nxt = tmp + self.nxt;
        Some(self.nxt)
    }
}

impl Fibonacci {
    fn new() -> Fibonacci {
        Fibonacci { current: 1, nxt: 1 }
    }
}

fn main() {
    let logger = logger::initlogger(false, "", 0);
    let _guard = slog_scope::set_global_logger(logger);
    slog_stdlog::init().unwrap();

    let fib = Fibonacci::new();
    let fib_clone = fib.clone();

    // 忽略数列前10项,然后再取后续的8项
    for item in fib.skip(10).take(8) {
        info!("{}", item);
    }

    for item in fib_clone.skip(20).take(8) {
        info!("{}", item);
    }

}
\end{code-block}


\subsection{条件编译}
C/C++经常使用宏定义实现条件编译,而Rust同样可以。Rust当中,主要有2种方式:通过属性
和通过宏定义:
\begin{code-block}{rust}
// 通过cfg的属性,当目标为linux时,进行编译
#[cfg(target_os = "linux")]
fn are_you_on_linux() {
    println!("You are running linux!")
}

// 通过cfg的属性,当目标不为linux时,进行编译
#[cfg(not(target_os = "linux"))]
fn are_you_on_linux() {
    println!("You are *not* running linux!")
}

fn main() {
    are_you_on_linux();

    println!("Are you sure?");
    // 通过cfg!宏进行判断
    if cfg!(target_os = "linux") {
        println!("Yes. It's definitely linux!");
    } else {
        println!("Yes. It's definitely *not* linux!");
    }
}
\end{code-block}
Target\_os宏为Rust内置,如果想自定义编译条件或者属性,则需要如下进行操作:
\begin{outline}[enumerate]
\1 修改代码
\begin{code-in-enumerate}{rust}
#[cfg(feature = "debugs")]
fn debugs_print() {
    println!("This is the debugs_print");
}

fn main() {
    if cfg!(feature = "debugs") {
        debugs_print();
    }
    println!("Are you sure?");
}
\end{code-in-enumerate}

\1 修改cargo工程设置
\begin{code-in-enumerate}{bash}
echo >> Cargo.toml <<EOF
[features]
debugs = []
EOF
\end{code-in-enumerate}

\1 使用自定义的debugs宏进行编译
\begin{code-in-enumerate}{bash}
cargo build --features debugs

# 如果是多项目的管理方式,则需要变更为如下的方式:
# cargo build --bin adclosures --manifest-path adclosures/Cargo.toml --features debugs
\end{code-in-enumerate}

\end{outline}

\subsection{宏}
Rust也包含了宏,并且,和C/C++相比,Rust的宏会展开成为抽象语法树(AST,abstract syntax tree),
而不是普通的转换成字符串,因此,不会产生无法预料的优先权错误。最普通的宏如下:
\begin{code-block}{rust}
extern crate slog_scope;
extern crate slog_stdlog;
#[macro_use]
extern crate log;
extern crate logger;

// macro_rules! 表示后续的内容是一个宏
// greeting表示宏的名称
macro_rules! greeting {
    // () 表示该宏不接收任何参数
    () => {
        // 宏定义展开的具体内容
        info!("hello macro");
    };
}

fn main() {
    let logger = logger::initlogger(false, "", 0);
    let _guard = slog_scope::set_global_logger(logger);
    slog_stdlog::init().unwrap();

    greeting!();
}
\end{code-block}
但是,宏不可能一直是无参数的,它还包含了多种使用方式。宏的参数使用\$符号表示,并
使用指示符来注明类型,如下:
\begin{code-block}{rust}
macro_rules! create_function {
    // 宏接收一个ident指示符表示的参数,并创建一个func_name的函数
    // ident指示符表示变量名(函数名)
    ($func_name: ident) => {
        fn $func_name() {
            // stringify宏负责将ident指示符表示的参数转换成字符串
            info!("You called the {}()", stringify!($func_name));
        }
    };
}

// 使用宏创建函数,函数名为func
create_function!(func);

macro_rules! formatres {
    // 宏接收一个expr指示符表示的表达式(可以是代码块,函数/方法,其他宏)
    // expr指示符表示表达式
    // $expression表示表达式最后的执行结果
    ($expression: expr) => {
        info!("{} = {}", stringify!($expression), $expression)
    };
}

fn main() {
    // 调用func函数
    func();
    formatres!(1 + 32);
    formatres!("lucifer");
    formatres!(format!("{}, age is {}", "zhangjl", 32));
}
\end{code-block}
宏的指示符有很多,各自用于不同的场景,所有的宏指示符如下:
\begin{itemize}
  \item block
  \item expr:表达式
  \item ident:变量名/函数名
  \item item
  \item pat:模式
  \item path
  \item stmt:语句
  \item tt:标记树
  \item ty:类型
\end{itemize}

类似于方法,Rust的宏也可以进行重载,只不过,这个重载的实现比较类似于match的分支
处理流程,分割宏的分支即进行重载,则需要使用符号“:”进行:
\begin{code-block}{rust}
macro_rules! assert_bool {
    // 括号中的分号;表示调用该宏时,需要传递2条语句或者表达式
    ($left: expr; and $right: expr) => {
        info!(
            "{} and {} is {}",
            stringify!($left),
            stringify!($right),
            $left && $right
        )
    };
    // 分支之间需要使用分号;进行分割与结束
    ($left: expr; or $right: expr) => {
        info!(
            "{} or {} is {}",
            stringify!($left),
            stringify!($right),
            $left || $right
        )
    };
}

fn main() {
    assert_bool!(1 + 1 == 2; and 2 * 2 == 4 );
    assert_bool!(1 + 1 == 3; or 2 * 2 == 6 );
}
\end{code-block}

宏定义的另外一个好处就是可以处理不定参数,在处理不定参数时,需要使用+操作符以及*
操作符,+表示参数可能出现一次或多次,*则表示参数可能出现0次或多次:
\begin{code-block}{rust}
use std::cmp;
macro_rules! find_min {
    // 如果传入的只有一个参数,直接返回当前参数值
    ($x: expr) => {
        $x
    };
    // 传入多个参数,表示后续更多的参数,即x后至少还有一个参数
    ($x: expr, $($y: expr), +) => {
        // 递归调用宏本身
        cmp::min($x, find_min!($($y), +))
    };
}

fn main() {
    info!("{}", find_min!(12));
    info!("{}", find_min!(12, 65, 40 - 32));

    let a = 1;
    let b = 2;
    let c = 3;
    info!("{}", find_min!(a, a - b, c));
}
\end{code-block}
上述的宏是使用表达式模式进行的,如果采用变量模式,即使用ident模式,则上述代码需要
变更为如下:
\begin{code-block}{rust}
use std::cmp;

macro_rules! find_min {
    ($x: ident) => {
        $x
    };
    ($x: ident, $($y: ident), +) => {
        cmp::min($x, find_min!($($y), +))
    };
}

fn main() {
    // 错误的使用方式,12是一个表达式,而并非变量名
    // info!("{}", find_min!(12));
    let a = 1;
    let b = 2;
    let c = 3;
    info!("{}", find_min!(a, b, c));
}
\end{code-block}
通过对比,可以发现,在某些场景下,表达式方式比ident方式更加通用,也更加合理一些。

比较奇特的是,在Rust的宏当中,可以使用自定义的关键字,实现特殊功能,比如自定义
关键字evaluation,表示将表达式进行计算:
\begin{code-block}{rust}
macro_rules! calc {
    // 自定义关键字evalution,使用该宏时,前面必须加上evalution前缀关键词
    (evalution $e: expr) => {
        // 强制将表达式e变成数值i32类型,即将表达式e进行计算
        let val: i32 = $e;
        info!("{} = {}", stringify!($e), val);
    };
    // 当传入参数不定时
    (evalution $e: expr, $(evalution $es: expr),+) => {
        calc!(evalution $e);
        calc!($(evalution $es),+)
    }
}

fn main() {
    calc!(evalution 1 + 100);
    calc!(evalution 1+2, evalution 3 + 4, evalution 5 +6 );
    calc!(evalution 1-2, evalution 3 * 4, evalution (5 +6) * (5 - 9) );
}
\end{code-block}
由于宏的高度可定制性,因此,上述的宏代码可以变更为如下的模式,但是2者的功能完全
一样:
\begin{code-block}{rust}
macro_rules! calc {
    (evalution $e:expr) => {{
        let val: i32 = $e;
        info!("{} = {}", stringify! {$e}, val);
    }};
    (evalution $e:expr, $(evalution $es:expr),+) => {{
        calc! { evalution $e }
        calc! { $(evalution $es),+ }
    }};
}

fn main() {
    // 下面两种方式都正确
    calc!{evalution 1 + 100};
    calc!(evalution 1+2, evalution 3 + 4, evalution 5 +6 );
}
\end{code-block}

实际上,*和+不仅可以用于参数处理,也可以用于语法扩展的部分,比如,我们想实现类似
如下的一个宏:
\begin{code-block}{rust}
let empty = hashmap![];
let counts = hashmap!['A' => 0, 'C' => 0, 'G' => 0, 'T' => 0];
\end{code-block}
则宏定义大致可能如下:
\begin{code-block}{rust}
macro_rules! hashmap {
    ($key: expr => $val: expr) => {{
        let mut map = ::std::collections::HashMap::new();
        map.insert($key, $val);
        map
    }};
}
\end{code-block}
但是,到目前为止,上述的宏只能实现对一对数据的操作,无法实现任意对数据的插入操作,
因此,我们需要使用+或者*符号进行扩展,由于我们需要支持初始化一个空的hashmap,因此
选择使用*进行扩展:
\begin{code-block}{rust}
macro_rules! hashmap {
    ($key: expr => $val: expr) => {{
        let mut map = ::std::collections::HashMap::new();
        map.insert($key, $val);
        map
    }};
}
\end{code-block}
虽然参数支持了任意个数,但是,在宏体当中,map的插入操作只执行了一次,我们可以继续
使用*和+对语法部分进行扩展,扩展之后,完整的宏定义如下:
\begin{code-block}{rust}
macro_rules! hashmap {
    ($($key: expr => $val: expr), *) => {{
        let mut map = HashMap::new();
        $(map.insert($key, $val); )*
        map
    }};
}
\end{code-block}
使用时,则按照上述的使用方式即可:
\begin{code-block}{rust}
let map = hashmap!["lucifer" => 12, "titans" => 18];
let mut empty: HashMap<String, u8> = hashmap![];
\end{code-block}
默认情况下,cargo并没有提供将宏定义进行展开显示的功能,但是,我们可以通过rustc
将代码展开,确认宏定义确实是按照我们的想法进行工作的。只是需要注意,将宏定义进行
展开显示,需要使用nightly分支,因此,我们的操作基本如下:
\begin{code-block}{bash}
# 切换到nightly分支
rustup default nightly
# 对代码进行展开
rustc -Z unstable-options --pretty=expanded src/main.rs
# 如果代码需要依赖其他的非std的crate的,则应当如下执行
# rustc -Z unstable-options --pretty=expanded -L ../target/debug/deps src/main.rs
\end{code-block}
如果一切正常,则我们调用宏的代码就会被展开成如下的形式:
\begin{figure}[H]
  \centering
  \includegraphics[width=\linewidth]{rust_macro_expand.png}
  \caption{宏展开}
  \label{fig:rust_macro_expand}
\end{figure}

Rust宏的灵活性非常大,可以像C/C++一样,在宏当中嵌套/调用宏:
\begin{code-block}{rust}
macro_rules! serial_cmd {
    ($expression: expr, $port: expr, $item: expr, $timeout: expr) => {{
        let mut cmd = HEADER.to_vec();
        cmd.push($item);
        cmd.push($expression);
        match ($port).write(&cmd) {
            Ok(_) => info!(
                "Sucess {}(0x{:X>02}) the {} board, and command is {:?}",
                stringify!($expression),
                $expression,
                $item,
                &cmd
            ),
            Err(e) => error!(
                "Failed to {}(0x{:X>02}) the {} board: {}",
                stringify!($expression),
                $expression,
                $item,
                e
            ),
        }
        if 0 < $timeout {
            thread::sleep(Duration::from_secs($timeout));
        }
    }};
}

macro_rules! serial_for_all_cmd {
    ($expression: expr, $port: expr, $timeout: expr) => {{
        for item in &CODE {
            serial_cmd!($expression, $port, *item, $timeout);
        }
    }};
}
\end{code-block}
也可以直接在宏当中,插入语句块,作为宏执行的一部分:
\begin{code-block}{rust}
macro_rules! serial_for_only_one {
    ($port: expr, $location: expr, $($command: stmt),*) => {{
        serial_cmd!(POWEROFF, $port, $location, 0);
        trace!("Remove the block file to avoid the unexcepted error ...");
        let _ = fs::remove_file("/dev/sdb");

        serial_cmd!(SWITCH, $port, $location, 5);
        // 执行外部代码块
        $($command )*

        serial_cmd!(POWEROFF, $port, $location, 0);
        let _ = fs::remove_file("/dev/sdb");
    }};
}

fn main() {
    // 调用宏
    serial_for_only_one!(port, location_u8, {println!("hello")});
}
\end{code-block}

\subsection{高级函数式编程}
之前的函数式编程当中,提到了map函数,用于对数据进行处理,比如下面这种:
\begin{code-block}{rust}
let sum: u32 = c1
    .zip(c2.skip(10))
    .map(|(a, b)| a * b)
    .filter(|x| x % 3 == 0)
    .sum();
\end{code-block}

但是,实际使用当中,map还有更加广泛的用途,比如,在特定的情况下,替换match操作,
使得代码更加简单和精炼。比如,在使用match处理Option这种数据类型时,由于Option的
取值范围为Some和None,而map函数对于Option类型的处理,也恰好就是返回Some和None,
因此,可以直接使用map函数对这种Some对Some,None对None的简单映射关系进行处理,
多个不同的map进行组合,形成链式调用,相比而言,比match操作会更加简练:
\begin{code-block}{rust}
#[derive(Debug)]
enum Food {
    Apple,
    Potato,
}

#[derive(Debug)]
struct Peeled(Food);
#[derive(Debug)]
struct Chopped(Food);
#[derive(Debug)]
struct Cooked(Food);

// 常见的处理方法,使用match进行处理,并且返回一个Option
fn peel(food: Option<Food>) -> Option<Peeled> {
    match food {
        Some(food) => Some(Peeled(food)),
        None => None,
    }
}

// 使用map函数进行Option的简单映射
fn process(food: Option<Food>) -> Option<Cooked> {
    food.map(|f| Peeled(f))
        .map(|Peeled(f)| Chopped(f))
        .map(|Chopped(f)| Cooked(f))
}
\end{code-block}

然而,如果返回类型Option需要作为map函数的参数,输入到另外一个闭包或者函数当中,
则有可能出现Option<Option<T>>的结果出现,并不利于结果的解析,此时,则需要采用
and\_then进行处理,比如下方的代码:
\begin{code-block}{rust}
enum Food {
    CordonBleu,
    Steak,
    Sushi,
}

fn have_ingredients(food: Food) -> Option<Food> {
    match food {
        Food1::Sushi => None,
        _ => Some(food),
    }
}

fn have_recipe(food: Food) -> Option<Food> {
    match food {
        Food1::CordonBleu => None,
        _ => Some(food),
    }
}

// 通过map函数将上述2个函数进行连接起来,have_recipe当作一个闭包使用
// 但是,结果将变更为Option<Option<T>>
fn cookable_v1(food: Food) -> Option<Option<Food>> {
    have_ingredients(food).map(|res| have_recipe(res))
}

// 通过and_then将2个函数连接起来,形成链式调用
// have_ingredients返回的是一个Option,and_then会将其进行拆包
// 如果Option是None,则直接返回None;但是,如果是Some<T>,and_then则会将其
// 进行拆包,返回为T,而不是Some<T>
fn cookable_v2(food: Food) -> Option<Food> {
    have_ingredients(food).and_then(have_recipe)
}
\end{code-block}

Result和Option类似,但实质上,Option是Result的一个特化版本,可以将其简单的看作:
\begin{code-block}{rust}
type Option<T> = Result<T, ()>
\end{code-block}

因此,Option的map,and\_then等函数(算子)同样可以作用于Result上,比如下面的例子:
\begin{code-block}{rust}
use std::num::ParseIntError;

// 使用普通的match模式
fn multiply_v1(first_number_str: &str, second_number_str: &str) -> Result<i32, ParseIntError> {
    match first_number_str.parse::<i32>() {
        Ok(first_number)  => {
            match second_number_str.parse::<i32>() {
                Ok(second_number)  => {
                    Ok(first_number * second_number)
                },
                Err(e) => Err(e),
            }
        },
        Err(e) => Err(e),
    }
}

// 使用map与and_then模式
fn multiply_v2(first_number_str: &str, second_number_str: &str) -> Result<i32, ParseIntError> {
    // and_then将Result<T, E>拆分,如果是Err,直接返回,如果是T,即Ok(T)
    // 则进行解析为T
    first_number_str.parse::<i32>().and_then(|first_number| {
        second_number_str.parse::<i32>().map(|second_number| first_number * second_number)
    })
}
\end{code-block}

同样的,Result也可以使用别名系统,比如常见的io::Result,实际上就是Result的一个
别名特化版本:
\begin{code-block}{rust}
type Result<T> = Result<T, Error>;
\end{code-block}
因此,同样可以在代码当中使用Result的别名,对代码进行简化:
\begin{code-block}{rust}
use std::num::ParseIntError;

type AliasedResult<T> = Result<T, ParseIntError>;

fn multiply(first_number_str: &str, second_number_str: &str) -> AliasedResult<i32> {
    first_number_str.parse::<i32>().and_then(|first_number| {
        second_number_str
            .parse::<i32>()
            .map(|second_number| first_number * second_number)
    })
}

fn print(result: AliasedResult<i32>) {
    match result {
        Ok(n) => println!("n is {}", n),
        Err(e) => println!("Error: {}", e),
    }
}

fn main() {
    print(multiply("10", "2"));
    print(multiply("t", "2"));
}
\end{code-block}

由于Option和Result的特殊性,在一些特定的场合,尤其是处理错误的时候,常见的做法就是
混合Option和Result,进行混合类型的错误处理:
\begin{code-block}{rust}
use std::num::ParseIntError;

fn double_first(vec: Vec<&str>) -> Option<Result<i32, ParseIntError>> {
    // map返回Option,使用map包裹parse函数可能带来的错误信息(Result)
    vec.first().map(|first| first.parse::<i32>().map(|n| 2 * n))
}

fn double_first_v2(vec: Vec<&str>) -> Result<Option<i32>, ParseIntError> {
    let opt = vec.first().map(|first| first.parse::<i32>().map(|n| 2 * n));

    // map_or返回Result,其中,Ok子句处理opt为None的情况
    // r则处理opt为Some和Err的情况
    opt.map_or(Ok(None), |r| {
        println!("The r is error {:?}", r);
        r.map(Some)
    })
}

fn main() {
    let empty2 = vec![];

    match double_first_v2(empty2) {
        Ok(Some(x)) => println!("The result is {}", x),
        Err(e) => println!("Error is {:?}", e),
        Ok(None) => println!("None is in result"),
    }
}
\end{code-block}

\subsection{自定义错误}
Rust的错误是可以进行自行定义的,只需要实现一个Error Trait即可。Error Trait的定义
如下:
\begin{code-block}{rust}
pub trait Error: Debug + Display {
    fn source(&self) -> Option<&(dyn Error + 'static)> { ... }
    fn backtrace(&self) -> Option<&Backtrace> { ... }
    fn description(&self) -> &str { ... }
    fn cause(&self) -> Option<&dyn Error> { ... }
}
\end{code-block}
其中:
\begin{itemize}
  \item source是必须实现的函数,并且对应的错误必须实现Debug和Display Trait
  \item backtrace是只能在nightly分支当中实现的函数
  \item description被废弃,使用Display Trait或者to\_string(ToString Trait)替代
  \item cause同样被废弃,被source所取代
\end{itemize}

一个简单的例子如下:
\begin{code-block}{rust}
use std::error::Error;
use std::fmt;

// 定义自定义错误结构体
// 实现Debug Trait
#[derive(Debug)]
struct SuperError {
    msg: String,
}

// 实现Display Trait
impl fmt::Display for SuperError {
    fn fmt(&self, f: &mut fmt::Formatter) -> fmt::Result {
        write!(f, "Super Error: {}", self.msg)
    }
}

// 实现Error Trait
impl Error for SuperError {
    fn source(&self) -> Option<&(dyn Error + 'static)> {
        Some(self)
    }
}

impl SuperError {
    fn new(err: &str) -> SuperError {
        SuperError {
            msg: err.to_string(),
        }
    }
}

fn err_test() -> Result<(), SuperError> {
    Err(SuperError::new("first error"))
}

fn main() {
    match err_test() {
        // Err(SuperError{msg: e}) => println!("{}", e),
        Err(e) => println!("{}", e),
        _ => println!("no error"),
    }
}
\end{code-block}

错误和自定义错误解决的是对于错误的定义,以及对应错误的处理方式,但是,在实际的生产
使用当中,错误可能是普遍存在的,而我们需要的数据可能并不包含错误信息,而是需要
将错误从正确的结果当中剔除,比如:
\begin{code-block}{rust}
fn main() {
    let strings = vec!["tofu", "93", "18"];
    let possible_numbers: Vec<_> = strings.into_iter().map(|s| s.parse::<i32>()).collect();
    println!("Results: {:?}", possible_numbers);
}
\end{code-block}
我们的本意是将Vec当中的字符串全部格式化为数值,但是,实际的结果当中,却把包含的
错误也一同包含进来了,需要想办法将错误信息过滤掉:
\begin{code-block}{rust}
fn main() {
    let strings = vec!["tofu", "93", "18"];
    let numbers: Vec<_> = strings
        .into_iter()
        .map(|s| s.parse::<i32>())
        // filter_map进行过滤,只保留结果为ok的数据
        .filter_map(Result::ok)
        .collect();
    println!("Results: {:?}", numbers);
}
\end{code-block}

Result实现了FromIter,因此结果的向量(Vec<Result<T, E>>)可以被转换成结果包裹着
向量(Result<Vec<T>, E>)。一旦找到一个Result::Err,遍历就被终止,即满足另外一种
需求:只要任何一个错误发生,就中断当前的操作:
\begin{code-block}{rust}
fn main() {
    let strings = vec!["tofu", "93", "18"];
    // 注意numbers不再是Vec<_>,而是通过FromIter转换成了Result
    // 转换过程一旦失败,就会出现错误,中断当前的执行流程
    let numbers: Result<Vec<_>, _> = strings.into_iter().map(|s| s.parse::<i32>()).collect();
    println!("Results: {:?}", numbers);
}
\end{code-block}

但是,有的时候,我们也存在另外一种需求:将执行的正确和错误结果分类存放,以待后续
操作,此时则需要使用partition函数,对结果进行区分:
\begin{code-block}{rust}
fn main() {
    let strings = vec!["tofu", "93", "18"];
    let (numbers, errors): (Vec<_>, Vec<_>) = strings
        .into_iter()
        .map(|s| s.parse::<i32>())
        // 使用partition函数进行区分
        .partition(Result::is_ok);
    println!("Numbers: {:?}", numbers);
    println!("Errors: {:?}", errors);

    // 对后续的结果进行解构
    let numbers: Vec<_> = numbers.into_iter().map(Result::unwrap).collect();
    let errors: Vec<_> = errors.into_iter().map(Result::unwrap_err).collect();
    println!("Numbers: {:?}", numbers);
    println!("Errors: {:?}", errors);
}
\end{code-block}

\section{元编程}
Rust也包含了宏,并且,和C/C++相比,Rust的宏会展开成为抽象语法树(AST,abstract syntax tree),
而不是普通的转换成字符串,因此,不会产生无法预料的优先权错误。Rust的宏包括声明宏以及过程宏。
\subsection{声明宏}
常见的Rust宏大部分都是声明宏,最普通的宏如下:
\begin{code-block}{rust}
extern crate slog_scope;
extern crate slog_stdlog;
#[macro_use]
extern crate log;
extern crate logger;
// macro_rules! 表示后续的内容是一个宏
// greeting表示宏的名称
macro_rules! greeting {
    // () 表示该宏不接收任何参数
    () => {
        // 宏定义展开的具体内容
        info!("hello macro");
    };
}
fn main() {
    let logger = logger::initlogger(false, "", 0);
    let _guard = slog_scope::set_global_logger(logger);
    slog_stdlog::init().unwrap();
    greeting!();
}
\end{code-block}
但是,宏不可能一直是无参数的,它还包含了多种使用方式。宏的参数使用\$符号表示,并
使用指示符来注明类型,如下:
\begin{code-block}{rust}
macro_rules! create_function {
    // 宏接收一个ident指示符表示的参数,并创建一个func_name的函数
    // ident指示符表示变量名(函数名)
    ($func_name: ident) => {
        fn $func_name() {
            // stringify宏负责将ident指示符表示的参数转换成字符串
            info!("You called the {}()", stringify!($func_name));
        }
    };
}
// 使用宏创建函数,函数名为func
create_function!(func);
macro_rules! formatres {
    // 宏接收一个expr指示符表示的表达式(可以是代码块,函数/方法,其他宏)
    // expr指示符表示表达式
    // $expression表示表达式最后的执行结果
    ($expression: expr) => {
        info!("{} = {}", stringify!($expression), $expression)
    };
}
fn main() {
    // 调用func函数
    func();
    formatres!(1 + 32);
    formatres!("lucifer");
    formatres!(format!("{}, age is {}", "zhangjl", 32));
}
\end{code-block}
宏的指示符有很多,各自用于不同的场景,所有的宏指示符如下:
\begin{itemize}
  \item block:代码块,由{}限定的代码
  \item expr:表达式,会生成具体的值
  \item ident:变量名/函数名,标识符
  \item item:语言项,即组成一个Rust包的基本单位,如模块,声明,函数/类型/结构体/impl定义
  \item pat:模式
  \item path:路径,类似std::iter等
  \item stmt:语句,一般以;结尾的代码
  \item tt:标记树
  \item ty:类型
  \item meta:元数据信息,即包含在\#[...]以及\#![...]当中的信息
  \item vis:可见性,如pub
  \item lifetime:指代生命周期参数
\end{itemize}

类似于方法,Rust的宏也可以进行重载,只不过,这个重载的实现比较类似于match的分支
处理流程,分割宏的分支即进行重载,则需要使用符号“:”进行:
\begin{code-block}{rust}
macro_rules! assert_bool {
    // 括号中的分号;表示调用该宏时,需要传递2条语句或者表达式
    ($left: expr; and $right: expr) => {
        info!(
            "{} and {} is {}",
            stringify!($left),
            stringify!($right),
            $left && $right
        )
    };
    // 分支之间需要使用分号;进行分割与结束
    ($left: expr; or $right: expr) => {
        info!(
            "{} or {} is {}",
            stringify!($left),
            stringify!($right),
            $left || $right
        )
    };
}
fn main() {
    assert_bool!(1 + 1 == 2; and 2 * 2 == 4 );
    assert_bool!(1 + 1 == 3; or 2 * 2 == 6 );
}
\end{code-block}

宏定义的另外一个好处就是可以处理不定参数,在处理不定参数时,需要使用+操作符以及*
操作符,+表示参数可能出现一次或多次,*则表示参数可能出现0次或多次:
\begin{code-block}{rust}
use std::cmp;
macro_rules! find_min {
    // 如果传入的只有一个参数,直接返回当前参数值
    ($x: expr) => {
        $x
    };
    // 传入多个参数,表示后续更多的参数,即x后至少还有一个参数
    ($x: expr, $($y: expr), +) => {
        // 递归调用宏本身
        cmp::min($x, find_min!($($y), +))
    };
}
fn main() {
    info!("{}", find_min!(12));
    info!("{}", find_min!(12, 65, 40 - 32));
    let a = 1;
    let b = 2;
    let c = 3;
    info!("{}", find_min!(a, a - b, c));
}
\end{code-block}
上述的宏是使用表达式模式进行的,如果采用变量模式,即使用ident模式,则上述代码需要
变更为如下:
\begin{code-block}{rust}
use std::cmp;
macro_rules! find_min {
    ($x: ident) => {
        $x
    };
    ($x: ident, $($y: ident), +) => {
        cmp::min($x, find_min!($($y), +))
    };
}
fn main() {
    // 错误的使用方式,12是一个表达式,而并非变量名
    // info!("{}", find_min!(12));
    let a = 1;
    let b = 2;
    let c = 3;
    info!("{}", find_min!(a, b, c));
}
\end{code-block}
通过对比,可以发现,在某些场景下,表达式方式比ident方式更加通用,也更加合理一些。

比较奇特的是,在Rust的宏当中,可以使用自定义的关键字,实现特殊功能,比如自定义
关键字evaluation,表示将表达式进行计算:
\begin{code-block}{rust}
macro_rules! calc {
    // 自定义关键字evalution,使用该宏时,前面必须加上evalution前缀关键词
    (evalution $e: expr) => {
        // 强制将表达式e变成数值i32类型,即将表达式e进行计算
        let val: i32 = $e;
        info!("{} = {}", stringify!($e), val);
    };
    // 当传入参数不定时
    (evalution $e: expr, $(evalution $es: expr),+) => {
        calc!(evalution $e);
        calc!($(evalution $es),+)
    }
}
fn main() {
    calc!(evalution 1 + 100);
    calc!(evalution 1+2, evalution 3 + 4, evalution 5 +6 );
    calc!(evalution 1-2, evalution 3 * 4, evalution (5 +6) * (5 - 9) );
}
\end{code-block}
由于宏的高度可定制性,因此,上述的宏代码可以变更为如下的模式,但是2者的功能完全
一样:
\begin{code-block}{rust}
macro_rules! calc {
    (evalution $e:expr) => {{
        let val: i32 = $e;
        info!("{} = {}", stringify! {$e}, val);
    }};
    (evalution $e:expr, $(evalution $es:expr),+) => {{
        calc! { evalution $e }
        calc! { $(evalution $es),+ }
    }};
}
fn main() {
    // 下面两种方式都正确
    calc!{evalution 1 + 100};
    calc!(evalution 1+2, evalution 3 + 4, evalution 5 +6 );
}
\end{code-block}

实际上,*和+不仅可以用于参数处理,也可以用于语法扩展的部分,比如,我们想实现类似
如下的一个宏:
\begin{code-block}{rust}
let empty = hashmap![];
let counts = hashmap!['A' => 0, 'C' => 0, 'G' => 0, 'T' => 0];
\end{code-block}
则宏定义大致可能如下:
\begin{code-block}{rust}
macro_rules! hashmap {
    ($key: expr => $val: expr) => {{
        let mut map = ::std::collections::HashMap::new();
        map.insert($key, $val);
        map
    }};
}
\end{code-block}
但是,到目前为止,上述的宏只能实现对一对数据的操作,无法实现任意对数据的插入操作,
因此,我们需要使用+或者*符号进行扩展,由于我们需要支持初始化一个空的hashmap,因此
选择使用*进行扩展:
\begin{code-block}{rust}
macro_rules! hashmap {
    ($key: expr => $val: expr) => {{
        let mut map = ::std::collections::HashMap::new();
        map.insert($key, $val);
        map
    }};
}
\end{code-block}
虽然参数支持了任意个数,但是,在宏体当中,map的插入操作只执行了一次,我们可以继续
使用*和+对语法部分进行扩展,扩展之后,完整的宏定义如下:
\begin{code-block}{rust}
macro_rules! hashmap {
    ($($key: expr => $val: expr), *) => {{
        let mut map = HashMap::new();
        $(map.insert($key, $val); )*
        map
    }};
}
\end{code-block}
使用时,则按照上述的使用方式即可:
\begin{code-block}{rust}
let map = hashmap!["lucifer" => 12, "titans" => 18];
let mut empty: HashMap<String, u8> = hashmap![];
\end{code-block}
默认情况下,cargo并没有提供将宏定义进行展开显示的功能,但是,我们可以通过rustc
将代码展开,确认宏定义确实是按照我们的想法进行工作的。只是需要注意,将宏定义进行
展开显示,需要使用nightly分支,因此,我们的操作基本如下:
\begin{code-block}{bash}
# 切换到nightly分支
rustup default nightly
# 对代码进行展开
rustc -Z unstable-options --pretty=expanded src/main.rs
cargo rustc -- -Z unstable-options --pretty=expanded
# 如果代码需要依赖其他的非std的crate的,则应当如下执行
# rustc -Z unstable-options --pretty=expanded -L ../target/debug/deps src/main.rs
\end{code-block}
如果一切正常,则我们调用宏的代码就会被展开成如下的形式:
\begin{figure}[H]
  \centering
  \includegraphics[width=\linewidth]{rust_macro_expand.png}
  \caption{宏展开}
  \label{fig:rust_macro_expand}
\end{figure}

Rust宏的灵活性非常大,可以像C/C++一样,在宏当中嵌套/调用宏:
\begin{code-block}{rust}
macro_rules! serial_cmd {
    ($expression: expr, $port: expr, $item: expr, $timeout: expr) => {{
        let mut cmd = HEADER.to_vec();
        cmd.push($item);
        cmd.push($expression);
        match ($port).write(&cmd) {
            Ok(_) => info!(
                "Sucess {}(0x{:X>02}) the {} board, and command is {:?}",
                stringify!($expression),
                $expression,
                $item,
                &cmd
            ),
            Err(e) => error!(
                "Failed to {}(0x{:X>02}) the {} board: {}",
                stringify!($expression),
                $expression,
                $item,
                e
            ),
        }
        if 0 < $timeout {
            thread::sleep(Duration::from_secs($timeout));
        }
    }};
}
macro_rules! serial_for_all_cmd {
    ($expression: expr, $port: expr, $timeout: expr) => {{
        for item in &CODE {
            serial_cmd!($expression, $port, *item, $timeout);
        }
    }};
}
\end{code-block}
也可以直接在宏当中,插入语句块,作为宏执行的一部分:
\begin{code-block}{rust}
macro_rules! serial_for_only_one {
    ($port: expr, $location: expr, $($command: stmt),*) => {{
        serial_cmd!(POWEROFF, $port, $location, 0);
        trace!("Remove the block file to avoid the unexcepted error ...");
        let _ = fs::remove_file("/dev/sdb");
        serial_cmd!(SWITCH, $port, $location, 5);
        // 执行外部代码块
        $($command )*
        serial_cmd!(POWEROFF, $port, $location, 0);
        let _ = fs::remove_file("/dev/sdb");
    }};
}
fn main() {
    // 调用宏
    serial_for_only_one!(port, location_u8, {println!("hello")});
}
\end{code-block}

\subsection{宏导出}
除了在当前的crate当中使用宏之外,
宏还可以导出,宏之间也可以相互调用。宏的导出通常使用macro\_export关键字,比如:
\begin{code-block}{rust}
#[macro_export]
macro_rules! inc {
    ($x: expr) => {
       println!("{}", $x);
    };
}
\end{code-block}
然后,在其他地方,就可以直接使用这个宏。不过,有的时候,宏的实现可能需要当前包的
一些函数或者方法进行配合,则需要做如下的更改:
\begin{code-block}{rust}
// 必须将方法设置为pub,否则后续在宏定义当中,无法使用
pub fn incr(x: u32) -> u32 {
    x + 1
}
#[macro_export]
macro_rules! inc {
    ($x: expr) => {
        // $crate关键字表示当前的包
        // 当宏被导出时,自动根据上下文选择函数调用路径当中的包名
        $crate::incr($x)
    };
}
\end{code-block}
上述的导出方式,要求宏所依赖的函数,也都必须导出,否则,在外部使用宏时,无法
正常工作。

除了使用普通的函数作为宏的依赖项之外,也可以使用宏作为宏的依赖项。和普通函数一样,
如果一个宏的定义当中,依赖了另外一个宏,则必须同样当对应的依赖项导出为pub类型。
但是,如果可以使用一种额外的方式,将依赖的宏,转变为宏的内部规则进行导出:
\begin{code-block}{rust}
#[macro_export]
macro_rules! hashmap {
    /* hashmap宏的内部规则,相当于如下的一个外部宏,不管接收多少参数,一律返回
       一个空元组()
       macro_rules! unit {
       ($($input:tt),*) => {
                ()
           };
       }
       使用方式
       let res = unit!(), unit!("lucifer"), unit!("garuda", "titans")
    */
    (@unit $($x:tt)*) => (());
    /* hashmap宏的内部规则, 等价于如下的一个宏,作用是返回接收到的元素的个数
       macro_rules! count {
           // <[()]>::len()可以用于求取数组/切片的长度,使用方式如下:
           // let lenth = <[&str]>::len(&["string", "string"])
           // let lenth = <[String]>::len(&["string".to_string(), "string".to_string()])
           // let lenth = <[()]>::len(&[(), ()]) // 性能更好,因为()不占据任何内存空间
           ($($key:expr),*) => (<[()]>::len(&[$(unit! ($key)),*]));
       }
       使用方式
       let res = count!(), count!("lucifer"), count!("lucifer", "titans")
       @符号表示一个宏定义当中的内部规则,如果需要在宏当中使用宏的内部规则,
       则使用方式是 宏名!(@内部规则名 其他变量),对应到这个hashmap宏,则使用方式
       如下: hashmap!(@unit $key), hashmap!(@count $($rest),*)
    */
    (@count $($rest:expr), *) => (<[()]>::len(&[$(hashmap!(@unit $rest)),*]));
    /* $($key:expr => $value:expr),* 表达式本身可以匹配hashmap!(),hashmap!("1"=>2)
     但是,无法匹配类似hashmap!["2"=>3,]这种末尾包含,符号的模式
     $(,)* 则是用于匹配后续结尾是否带有,符号
     即hashmap!["2"=>3,]和hashmap!["2"=>3]都可以支持
    */
    ($($key:expr => $value:expr),* $(,)*) => {{
        let _cap = hashmap!(@count $($key),*);
        let mut _map = ::std::collections::HashMap::with_capacity(_cap);
        $( _map.insert($key, $value); )*
        _map
    }}
}
\end{code-block}

\subsection{过程宏}
以上提到的宏,都是声明宏,可以直接当作函数/方法使用的类型,但是,如果想实现类似于
\#[derive(Debug)]这种类型的宏,声明宏是做不到的。相对应的,这种类型的宏则被称之
为过程宏。过程宏主要用于下面3种用途:
\begin{itemize}
  \item 自定义派生属性:即类似于\#[derive(Debug)]这样的derive属性
  \item 自定义属性:即类似于实现\#[cfg()]这样的属性
  \item Bang宏:与声明宏类似,但是,是以!结尾的宏,可以当作函数/方法使用
\end{itemize}

过程宏要求必须放到proc\_macro类型的lib包当中,因此,过程宏的创建过程会稍微有一些
区别:
\begin{code-block}{bash}
cargo new --lib procmacro
echo -e "[lib]\nproc_macro=true" >> procmacro/Cargo.toml
\end{code-block}

另外,和其他的mod不太一样的是,过程宏的测试用例,不能放到相同的crate当中,必须以
外部的方式存在,因此,过程宏的文件结构大致如下:
\begin{code-block}{bash}
├── Cargo.toml
├── src
│   └── lib.rs
└── tests
    └── test.rs
\end{code-block}

实现derive方式的过程宏,其示例如下:
\begin{code-block}{rust}
// 必须如此进行使用
extern crate proc_macro;
use self::proc_macro::TokenStream;
#[proc_macro_derive(A)]
pub fn derive(input: TokenStream) -> TokenStream {
    let input = input.to_string();
    assert!(input.contains("struct A"));
    r#"
        impl A {
            pub fn a(&self) -> String {
                format!("Hello from impl A")
            }
        }
    "#
    .parse()
    .unwrap()
}
\end{code-block}
上述过程宏表示,使用\#[derive(A)]为结构体A实现一个a方法,方法直接输出一句话。相对应的,
测试用例当中的使用则应当修改如下:
\begin{code-block}{rust}
#[macro_use]
extern crate procmacro;
#[derive(A)]
struct A;
#[test]
fn test_derive_a() {
    assert_eq!("Hello from impl A", A.a());
}
\end{code-block}

而实现自定义属性宏稍微有些区别,就是必须在nightly的rust下编译,目前还没有进入到
stable分支,一个简单的示例如下:
\begin{code-block}{rust}
#![feature(register_attr)]
extern crate proc_macro;
use self::proc_macro::TokenStream;
#[proc_macro_attribute]
pub fn attr_with_args(args: TokenStream, _: TokenStream) -> TokenStream {
    let args = args.to_string();
    //let input = input.to_string();
    format!("fn foo() -> &'static str {{{}}}", args)
        .parse()
        .unwrap()
}
\end{code-block}
同样的,其测试用例如下:
\begin{code-block}{rust}
#![feature(register_attr)]
#[macro_use]
extern crate procmacro;
use procmacro::attr_with_args;
#[attr_with_args("Hello Rust")]
fn foo() {}
#[test]
fn test_foo() {
    assert_eq!("Hello Rust", foo());
}
\end{code-block}
原本的foo方法,不接收参数,同样没有返回值,但是,在attr\_with\_args这个过程宏
当中,将其强行修改为了一个返回为字符串切片的函数。

实现Bang宏的方式则如下:
\begin{code-block}{rust}
#![feature(proc_macro_hygiene)]
extern crate proc_macro;
use self::proc_macro::TokenStream;
#[proc_macro]
pub fn treemap(input: TokenStream) -> TokenStream {
    let input = input.to_string();
    let input = input.trim_end_matches(',');
    let input_v: Vec<String> = input
        .split(",")
        .map(|n| {
            let mut data = if n.contains(":") {
                n.split(":")
            } else {
                n.split("=>")
            };
            let (key, value) = (data.next().unwrap(), data.next().unwrap());
            format!("hm.insert({}, {})", key, value)
        })
        .collect();
    let count: usize = input.len();
    let token = format!(
        "{{
        let mut hm = ::std::collections::HashMap::with_capacity({});
        {}
        hm
    }}",
        count,
        input_v
            .iter()
            .map(|n| format!("{};", n))
            .collect::<String>()
    );
    token.parse().unwrap()
}
\end{code-block}

Bang宏可以如同声明宏一样的进行使用,其使用方式如下:
\begin{code-block}{rust}
#[macro_use]
extern crate procmacro;
#[test]
fn test_treemap() {
    let hm = treemap! {"a":1, "b": 2};
    assert_eq!(hm["a"], 1);
    let hm = treemap! {"a" => 1, "b" => 4};
    assert_eq!(hm["b"], 4);
}
\end{code-block}

过程宏的本质是在函数/方法当中,使用TokenStream重构,本质还是一个特殊的函数/方法。
因此,过程宏不需要像声明宏一样的进行export,但是,必须将过程宏的函数声明为pub,
生成的过程宏才可以被外部使用。

\subsection{语法树}
编写真正可用的过程宏实际上比上面的例子要复杂很多,但不管如何变化,Rust的宏都是依赖于
语法树结构的,而过程宏的实现方式/过程,就是对解析的语法树进行处理的过程。关于语法树
的解析和读取,通常采用的是第三方的Rust Crate进行操作,目前比较常用的是\href{https://github.com/dtolnay/quote}{Quote},
\href{https://github.com/dtolnay/syn}{Syn}以及\href{https://github.com/alexcrichton/proc-macro2}{Proc-macro2}。
在编写真正的过程宏时,通常都需要上述3个crate的协助,需要在Cargo.toml当中添加如下的内容:
\begin{code-block}{toml}
[dependencies]
quote = "1.0.9"
syn = {version = "1.0.72", features = ["full", "extra-traits", "visit"]}
proc-macro2 = "1.0.26"
[lib]
proc-macro = true
\end{code-block}
另外,当lib当中设置\codeinlinebg{toml}{proc-macro=true}之后,
则对应的crate只能导出过程宏,不能导出其他的类型数据。

所有的编程语言都离不开词法分析,Rust同样如此。在Rust当中,通常使用TokenStream进行词法分析,
解析代码内容,编写过程宏离不开对TokenStream的解析。在调试过程宏的时候,由于标准输出
不可用,因此通常只能通过标准错误输出进行信息的打印。通常情况下,都是使用\codeinlinebg{rust}{eprint!}
或者\codeinlinebg{rust}{eprintln!}进行过程宏的调试输出。
一个简单的过程宏示例如下,当然,由于我们进行了输出,也可以看到Rust的语法树的大致结构:
\begin{code-block}{rust}
use proc_macro::TokenStream;
#[proc_macro_attribute]
pub fn test_proc_macro(attr: TokenStream, item: TokenStream) -> TokenStream {
    eprintln!("{:#?}", attr);
    eprintln!("{:#?}", item);
    item
}
\end{code-block}
调用的时候,需要在其他的crate当中引入这个crate:
\begin{code-block}{rust}
use procmacros::test_proc_macro;
fn main() {
    ...
}
#[test_proc_macro("lucifer")]
fn hello() {
    info!("hello");
}
\end{code-block}

默认情况下,代码需要经过编译,才能判断是否存在问题,不过,rust提供了1种简便的思路
来检测代码是否存在问题:\codeinlinebg{bash}{cargo check}。
该指令不会对代码进行实质的编译动作,但是会对过程宏进行预处理(即将其转变成正常的Rust代码),
因此,会得到类似如下的一些输出:
\begin{figure}[H]
  \centering
  \includegraphics[width=\linewidth]{rust_cargo_check.png}
  \caption{代码检测与预处理}
  \label{fig:rust_cargo_check}
\end{figure}

除了使用上述指令之外,也可以采用宏展开的方式,但是默认的宏展开方式需要使用nightly
分支,第三方工具cargo-expand则可以支持在stable分支直接展开。
但是,需要注意expand指令只能在源码(即rs文件)所在路径执行:\codeinlinebg{bash}{cargo expand}。
然后会得到类似如下的输出:
\begin{figure}[H]
  \centering
  \includegraphics[width=\linewidth]{rust_cargo_expand.png}
  \caption{宏代码展开}
  \label{fig:rust_cargo_expand}
\end{figure}
通过expand指令,可以将代码当中的宏代码全部转换为正常的Rust代码,从而方便进行阅读
识别和调试修改。

在编译混合有过程宏的Rust代码时,其基本流程是先展开过程宏,将其处理成普通的Rust代码,
然后再合并这些代码,最后再进行编译。从\colorunderlineref{fig:rust_cargo_check}所示当中,
可以看到有很多的特殊的标记,这些标记共同组成了Rust的抽象语法树结构:
\begin{itemize}
  \item Ident:标识符
  \item span:表示对应的元素在代码当中出现的位置(字节顺序)
  \item Group:组,表示语法树的组别
  \item delimiter:分隔符
  \item stream:表示每一组的内容(TokenStream)
  \item punct:标点符号
  \item literal:字符字面量
\end{itemize}

TokenStream只是一系列符号的组合,与语义无关,因此,如果将之前代码的过程宏调用填充
入无意义的数据,过程宏的处理同样不会有什么问题:
\begin{code-block}{rust}
#[test_proc_macro(!&)@)(*&$9)]
fn hello() {
    info!("hello");
}
\end{code-block}
上述代码在编译阶段的结果输出大致如下:
\begin{figure}[H]
  \centering
  \includegraphics[width=\linewidth]{rust_token_stream.png}
  \caption{TokenStream}
  \label{fig:rust_token_stream}
\end{figure}
从上述结果可以看到,TokenStream还是一种比较低级的处理形式,如果手工写一个TokenStream,
极易出现错误,因此需要使用上文提到的syn和quote,将TokenStream转换成具有语义信息
抽象度更高的数据结构:抽象语法树。将上述的过程宏代码改写如下:
\begin{code-block}{rust}
use proc_macro::TokenStream;
use proc_macro2;
use quote::quote;
use syn::{self, parse_macro_input, spanned::Spanned, AttributeArgs, Item};
// 标记该过程宏为属性模式
#[proc_macro_attribute]
pub fn test_proc_macro_ast(attr: TokenStream, item: TokenStream) -> TokenStream {
    // 将属性转换成语法数进行输出,即#[test_proc_macro(!&)@)(*&$9)]这部分代码
    eprintln!("{:#?}", parse_macro_input!(attr as AttributeArgs));
    // 将真正的代码转换成语法树输出,即被#[...]所修饰的代码
    let body_ast = parse_macro_input!(item as Item);
    eprintln!("{:#?}", body_ast);
    // 将语法树转换成TokenStream返回给编译器
    // quote返回的并不是TokenStream,而是proc_macro2::TokenStream类型,必须转换
    // #body_ast并不是Rust的合法语法,而是quote的自定义语法格式
    quote!(#body_ast).into()
}
\end{code-block}
如果对上述代码进行check,则会发现其输出结果大致如下:
\begin{figure}[H]
  \centering
  \includegraphics[width=\linewidth]{rust_ast.png}
  \caption{抽象语法树}
  \label{fig:rust_ast}
\end{figure}
注意,此时的输出就不再是纯粹的TokenStream(无语义)了,而是带有语义分析的抽象语法树
结构。抽象语法树可以完整的检查代码当中的语法逻辑问题,因此,如果像之前的代码,
使用不符合语法的方式调用这个过程宏,则代码的预编译阶段就无法通过,编译器会直接提示错误。
而这个错误,则是由\codeinlinebg{rust}{parse_macro_input!}这个
宏提示出来的。

\subsection{过程宏案例-派生过程宏Builder}
\label{builder}
假设当前有一个结构体如下:
\begin{code-block}{rust}
#[derive(Builder)]
pub struct Command {
    executable: String,
    #[builder(each = "arg")]
    args: Vec<String>,
    current_dir: Option<String>,
}
\end{code-block}
在该结构体上应用一个名为Builder的过程宏,使得在编译阶段最终生成的代码如下:
\begin{code-block}{rust}
pub struct Command {
    executable: String,
    #[builder(each = "arg")]
    args: Vec<String>,
    current_dir: Option<String>,
}
pub struct CommandBuilder {
    executable: Option<String>,
    args: Option<Vec<String>>,
    env: Option<Vec<String>>,
    current_dir: Option<String>,
}
impl Command {
    pub fn builder() -> CommandBuilder {
        CommandBuilder {
            executable: None,
            args: None,
            env: None,
            current_dir: None,
        }
    }
}
\end{code-block}
即利用\codeinlinebg{rust}{#[derive(Builder)]}宏
对任意结构体实现工厂模式代码的自动生成。

为实现这个Builder宏,首先实现其基本的结构:
\begin{code-block}{rust}
use proc_macro::TokenStream;
use proc_macro2;
use quote::quote;
use syn::{self, parse_macro_input, spanned::Spanned, AttributeArgs, Item};
// 派生宏,不再使用proc_macro_attribute(属性宏)
#[proc_macro_derive(Builder)]
pub fn derive(input: TokenStream) -> TokenStream {
    // 读取输出,转换成语法树
    let st = parse_macro_input!(input as syn::DeriveInput);
    TokenStream::new()
}
\end{code-block}

然后实现一个真正的语法树展开函数:
\begin{code-block}{rust}
fn do_expand(st: &syn::DeriveInput) -> syn::Result<proc_macro2::TokenStream> {
    // 获取语法树ident信息,即结构体名称(字面量)
    let struct_name_literal = st.ident.to_string();
    // 构造新的结构体名称
    let builder_name_literal = format!("{}Builder", struct_name_literal);
    // 构造新的结构体的标识符(不是string)
    // 第一个为标识符的字面量,第二个为位置信息
    let builder_name_ident = syn::Ident::new(&builder_name_literal, st.span());
    let struct_name_ident = &st.ident;
    let ret = quote!(
        // #builder_name_ident 表示使用之前的 builder_name_ident替换当前位置的内容
        // 是quote宏的自定义语法格式
        pub struct #builder_name_ident {
        }
        impl #struct_name_ident {
            pub fn builder() -> #builder_name_ident {
                #builder_name_ident {
                }
            }
        }
    );
    Ok(ret)
}
\end{code-block}
注意,上述代码当中的st(语法树)结构是一个\codeinlinebg{rust}{syn::DeriveInput}结构体\footnote{参考:\url{https://docs.rs/syn/1.0.72/syn/struct.DeriveInput.html}},其内在结构如下:
\begin{code-block}{rust}
pub struct DeriveInput {
    pub attrs: Vec<Attribute>, // 结构体/函数的属性,即#[]部分
    pub vis: Visibility,       // 可见性,pub,private
    pub ident: Ident,          // 标识符
    pub generics: Generics,    // 泛型
    pub data: Data,            // 字段,其类型可以是struct和enum以及联合体Union等
}
\end{code-block}
可以针对该结构体进行解析,从而得到语法树的各个元素。

有了do\_expand函数进行语法树的展开和修改之后,再回过头来修改框架的实现:
\begin{code-block}{rust}
#[proc_macro_derive(Builder)]
pub fn derive(input: TokenStream) -> TokenStream {
    let st = parse_macro_input!(input as syn::DeriveInput);
    match do_expand(&st) {
        // 转换成TokenStream
        Ok(token_stream) => token_stream.into(),
        // 将错误转换成编译器能够识别的TokenStream
        Err(error) => error.to_compile_error().into(),
    }
}
\end{code-block}

到此时,一个Builder属性宏的基本框架已经有了,通过\codeinlinebg{bash}{cargo expand}将
代码展开,最终得到的结果大致如下,当然,目前还缺乏结构体的字段内容:
\begin{figure}[H]
  \centering
  \includegraphics[width=\linewidth]{rust_expand.png}
  \caption{Builder宏的展开结果}
  \label{fig:rust_expand}
\end{figure}

需要注意的是,本例使用的是派生式的过程宏,而不是属性式的过程宏。属性式的过程宏
可以对其装饰的代码进行直接的修改,而派生式的过程宏则无法对代码进行直接的修改,
而是转为将代码追加在原始代码后面。

接下来开始对结构体字段进行填充。在填充之前,首先对语法树DeriveInput进行简单的一些
介绍,该结构体当中存在一个Data字段\footnote{参考:\url{https://docs.rs/syn/1.0.72/syn/enum.Data.html}},
而这个字段的详细定义,则如下:
\begin{code-block}{rust}
pub enum Data {
    Struct(DataStruct), // 针对结构体
    Enum(DataEnum),     // 针对Enum
    Union(DataUnion),   // 针对union
}
\end{code-block}
由于本例当中,关注的是结构体,因此,需要重点关注\codeinlinebg{rust}{DataStruct}
这个字段\footnote{参考:\url{https://docs.rs/syn/1.0.72/syn/struct.DataStruct.html}},该字段的具体定义如下:
\begin{code-block}{rust}
pub struct DataStruct {
    pub struct_token: Struct,       // 对应struct字面量
    pub fields: Fields,             // 对应struct的字段,filed
    pub semi_token: Option<Semi>,   // 分号,可选
}
\end{code-block}
结构体的字段变量,放在了\codeinlinebg{rust}{Fields}当中,
这个结构体的定义则大致如下\footnote{参考:\url{https://docs.rs/syn/1.0.72/syn/enum.Fields.html}}
\begin{code-block}{rust}
pub enum Fields {
    Named(FieldsNamed),         // 有名字段
    Unnamed(FieldsUnnamed),     // 无名字段
    Unit,                       // 元组,类似()
}
\end{code-block}
将有名字段的结构体继续进行展开,最终,我们会得到一个名为\codeinlinebg{rust}{Field}
的结构体\footnote{Field定义:\url{https://docs.rs/syn/1.0.72/syn/struct.Field.html}},
这个结构体的定义大致如下:
\begin{code-block}{rust}
pub struct Field {
    pub attrs: Vec<Attribute>,          // 字段属性
    pub vis: Visibility,                // 字段可见性
    pub ident: Option<Ident>,           // 字段名称
    pub colon_token: Option<Colon>,     // 冒号
    pub ty: Type,                       // 字段类型
}
\end{code-block}
而对结构体字段的填充过程,实际上就是用代码实现,最终找到上述结构体,并对其继续
构造的一个过程。由于整个语法树结构比较复杂,单独使用一个函数来实现对结构体的解析
和构造,获取结构体的字段定义\footnote{关于使用Token!替换Comma的说明:\url{https://docs.rs/syn/1.0.72/syn/token/struct.Comma.html}}:
\begin{code-block}{rust}
// get_filed_from_derive_input方法当中的result,实际上是一个
// Punctuated<Field, Comma>对象,其中Comma表示逗号。
// 不过,syn的文档描述当中,说明了最好不要直接使用Comma这种变量,
// 而是使用Token![,]这种宏表示
type StructFields = syn::punctuated::Punctuated<syn::Field, syn::Token![,]>;

fn get_filed_from_derive_input(st: &syn::DeriveInput) -> syn::Result<&StructFields> {
    // 根据上面结构体的定义,对输入的语法树节点进行解析
    // 但是,只匹配有名字段
    if let syn::Data::Struct(syn::DataStruct {
        fields: syn::Fields::Named(syn::FieldsNamed { ref named, .. }),
        ..
    }) = st.data
    {
        return Ok(named);
    }

    // 如果没有,则返回一个编译器可以使用的错误信息
    // 返回错误在源代码当中的位置信息
    Err(syn::Error::new_spanned(
        st,
        "Must define on Struct, Not on Enum",
    ))
}
\end{code-block}

接着构造一个函数来产生结构体的字段定义:
\begin{code-block}{rust}
fn generate_builder_struct_fields_def(
    st: &syn::DeriveInput,
) -> syn::Result<proc_macro2::TokenStream> {
    // 获取语法树处理之后得到的所有字段
    let fields = get_filed_from_derive_input(st)?;

    // 获得字段(语法树的标识符)
    let idents: Vec<_> = fields.iter().map(|f| &f.ident).collect();

    // 获得标识符的类型信息
    let types: Vec<_> = fields.iter().map(|f| &f.ty).collect();

    // 使用quote宏进行构造
    let ret = quote! {
        // #(),* 表示重复操作,操作的就是#()当中的内容
        // #indets和#types表示使用对应的变量进行替换
        // 这里使用的是Option的绝对路径,目的是防止和用户自定义的其他类型发生冲突
        #(#idents: std::option::Option<#types>), *
    };
    Ok(ret)
}
\end{code-block}
通过上述的函数,我们就可以定义结构体的字段了。接下来是对结构体字段的初始化,
这个操作也采用一个函数进行:
\begin{code-block}{rust}
fn generate_builder_struct_fields_init(
    st: &syn::DeriveInput,
) -> syn::Result<Vec<proc_macro2::TokenStream>> {
    // 获取语法树当中的所有字段
    let fields = get_filed_from_derive_input(st)?;

    let init_data: Vec<_> = fields
        .iter()
        .map(|f| {
            let ident = &f.ident;
            quote! {
                // 对ident全部设置为None
                #ident: std::option::Option::None
            }
        })
        .collect();
    Ok(init_data)
}
\end{code-block}

最后,将上述的2个函数结合到\codeinlinebg{rust}{do_expand}函数当中,改造之后的函数则如下:
\begin{code-block}{rust}
fn do_expand(st: &syn::DeriveInput) -> syn::Result<proc_macro2::TokenStream> {
    ...
    let builder_struct_fields_def = generate_builder_struct_fields_def(st)?;

    let builder_struct_fields_init = generate_builder_struct_fields_init(st)?;

    let ret = quote!(
        pub struct #builder_name_ident {
            #builder_struct_fields_def
        }

        impl #struct_name_ident {
            pub fn builder() -> #builder_name_ident {
                #builder_name_ident {
                    // quote的宏语法
                    // #(),* 表示重复操作,操作的就是#()当中的内容
                    #(#builder_struct_fields_init),*
                }
            }
        }

    );
    ...
    Ok(ret)
}
\end{code-block}

在面向对象的设计思路当中,通常还对结构体/类添加相关的getter/setter等方法,同样的,
我们也可以利用过程宏实现这类操作,比如,使用一个函数实现Builder的setter方法:
\begin{code-block}{rust}
fn generate_setter(st: &syn::DeriveInput) -> syn::Result<proc_macro2::TokenStream> {
    let fields = get_filed_from_derive_input(st)?;

    let idents: Vec<_> = fields.iter().map(|f| &f.ident).collect();
    let types: Vec<_> = fields.iter().map(|f| &f.ty).collect();

    // 构造一个可变的tokenstream
    let mut final_token_stream = proc_macro2::TokenStream::new();

    // 并行迭代idents和types
    for (ident, type_) in idents.iter().zip(types.iter()) {
        // 生成setter的tokenstream
        let token_stream_slice = quote! {

            fn #ident(&mut self, #ident: #type_) -> Self {
                self.#ident = std::option::Option::Some(#ident);
                self
            }

        };

        // 追加到最终的tokenstream当中
        final_token_stream.extend(token_stream_slice);
    }
    Ok(final_token_stream)
}
\end{code-block}

继续改造之后比较完整的\codeinlinebg{rust}{do_expand}函数如下:
\begin{code-block}{rust}
fn do_expand(st: &syn::DeriveInput) -> syn::Result<proc_macro2::TokenStream> {
    ...
    let setter_functions = generate_setter(st)?;
    let ret = quote!(
        ...
        impl #builder_name_ident {
            #setter_functions
        }
    );
    Ok(ret)
}
\end{code-block}
上述代码通过\codeinlinebg{bash}{cargo expand}之后,
其结果大致如下:
\begin{figure}[H]
  \centering
  \includegraphics[width=\linewidth]{rust_expand_2.png}
  \caption{Builder宏的展开结果2}
  \label{fig:rust_expand_2}
\end{figure}

作为Builder模式,Builder结构体必然还需要一个类似于build的函数或者方法,来生成真正的
结构体。在类似于build这样的函数/方法当中,通常需要完成2件事情:1是检查builder结构
的所有字段是否可以填充原始结构体,即字段是否都存在;2是生成一个原始结构体。同样的,
可以利用函数/方法来实现对字段的检查,以及字段的初始化:
\begin{code-block}{rust}
fn check_fileds(st: &syn::DeriveInput) -> syn::Result<proc_macro2::TokenStream> {
    let fields = get_filed_from_derive_input(st)?;
    let idents: Vec<_> = fields.iter().map(|f| &f.ident).collect();
    let mut final_check_stream = proc_macro2::TokenStream::new();
    for ident in idents.iter() {
        let check_stream_slice = quote! {
            // 检测字段是否为空
            if self.#ident.is_none() {
                let err_msg = format!("{} field is missing", stringify!(#ident));
                // 返回错误
                return std::result::Result::Err(err_msg.into());
            }
        };
        final_check_stream.extend(check_stream_slice);
    }
    Ok(final_check_stream)
}
fn build_target_fields(st: &syn::DeriveInput) -> syn::Result<proc_macro2::TokenStream> {
    let fields = get_filed_from_derive_input(st)?;
    let idents: Vec<_> = fields.iter().map(|f| &f.ident).collect();
    let mut final_init_stream = proc_macro2::TokenStream::new();
    for ident in idents.iter() {
        let init_stream_slice = quote! {
            #ident: self.#ident.clone().unwrap(),
        };
        final_init_stream.extend(init_stream_slice);
    }
    Ok(final_init_stream)
}
\end{code-block}

到此为止,基本上,利用过程宏,实现Builder模式就大致完成了。将所有的代码整合在一起,
整个过程宏的完整代码如下:
\begin{code-block}{rust}
use proc_macro::TokenStream;
use proc_macro2;
use quote::quote;
use syn::{self, parse_macro_input, spanned::Spanned};

#[proc_macro_derive(Builder)]
pub fn derive(input: TokenStream) -> TokenStream {
    let st = parse_macro_input!(input as syn::DeriveInput);
    match do_expand(&st) {
        Ok(token_stream) => token_stream.into(),
        Err(error) => error.to_compile_error().into(),
    }
}

fn do_expand(st: &syn::DeriveInput) -> syn::Result<proc_macro2::TokenStream> {
    let struct_name_literal = st.ident.to_string();
    let builder_name_literal = format!("{}Builder", struct_name_literal);
    let builder_name_ident = syn::Ident::new(&builder_name_literal, st.span());
    let struct_name_ident = &st.ident;

    let builder_struct_fields_def = generate_builder_struct_fields_def(st)?;
    let builder_struct_fields_init = generate_builder_struct_fields_init(st)?;

    let setter_functions = generate_setter(st)?;
    let checked_res = check_fileds(st)?;
    let build_res = build_target_fields(st)?;

    let ret = quote!(
        pub struct #builder_name_ident {
            #builder_struct_fields_def
        }

        impl #struct_name_ident {
            pub fn builder() -> #builder_name_ident {
                #builder_name_ident {
                    #(#builder_struct_fields_init),*
                }
            }
        }

        impl #builder_name_ident {
            #setter_functions
            pub fn build(&mut self) -> std::result::Result<#struct_name_ident, std::boxed::Box<dyn std::error::Error>>{
                #checked_res
                Ok(#struct_name_ident {
                    #build_res
                })
            }
        }

    );
    Ok(ret)
}

fn check_fileds(st: &syn::DeriveInput) -> syn::Result<proc_macro2::TokenStream> {
    let fields = get_filed_from_derive_input(st)?;
    let idents: Vec<_> = fields.iter().map(|f| &f.ident).collect();
    let mut final_check_stream = proc_macro2::TokenStream::new();

    for ident in idents.iter() {
        let check_stream_slice = quote! {
            if self.#ident.is_none() {
                let err_msg = format!("{} field is missing", stringify!(#ident));
                return std::result::Result::Err(err_msg.into());
            }
        };
        final_check_stream.extend(check_stream_slice);
    }
    Ok(final_check_stream)
}

fn build_target_fields(st: &syn::DeriveInput) -> syn::Result<proc_macro2::TokenStream> {
    let fields = get_filed_from_derive_input(st)?;
    let idents: Vec<_> = fields.iter().map(|f| &f.ident).collect();
    let mut final_init_stream = proc_macro2::TokenStream::new();
    for ident in idents.iter() {
        let init_stream_slice = quote! {
            #ident: self.#ident.clone().unwrap(),
        };
        final_init_stream.extend(init_stream_slice);
    }
    Ok(final_init_stream)
}

fn generate_setter(st: &syn::DeriveInput) -> syn::Result<proc_macro2::TokenStream> {
    let fields = get_filed_from_derive_input(st)?;
    let idents: Vec<_> = fields.iter().map(|f| &f.ident).collect();
    let types: Vec<_> = fields.iter().map(|f| &f.ty).collect();

    let mut final_token_stream = proc_macro2::TokenStream::new();
    for (ident, type_) in idents.iter().zip(types.iter()) {
        let token_stream_slice = quote! {
            pub fn #ident(&mut self, #ident: #type_) -> & mut Self {
                self.#ident = std::option::Option::Some(#ident);
                self
            }
        };
        final_token_stream.extend(token_stream_slice);
    }
    Ok(final_token_stream)
}

type StructFields = syn::punctuated::Punctuated<syn::Field, syn::Token![,]>;

fn get_filed_from_derive_input(st: &syn::DeriveInput) -> syn::Result<&StructFields> {
    if let syn::Data::Struct(syn::DataStruct {
        fields: syn::Fields::Named(syn::FieldsNamed { ref named, .. }),
        ..
    }) = st.data
    {
        return Ok(named);
    }
    Err(syn::Error::new_spanned(
        st,
        "Must define on Struct, Not on Enum",
    ))
}

fn generate_builder_struct_fields_def(
    st: &syn::DeriveInput,
) -> syn::Result<proc_macro2::TokenStream> {
    let fields = get_filed_from_derive_input(st)?;
    let idents: Vec<_> = fields.iter().map(|f| &f.ident).collect();
    let types: Vec<_> = fields.iter().map(|f| &f.ty).collect();

    let ret = quote! {
        #(#idents: std::option::Option<#types>), *
    };

    Ok(ret)
}

fn generate_builder_struct_fields_init(
    st: &syn::DeriveInput,
) -> syn::Result<Vec<proc_macro2::TokenStream>> {
    let fields = get_filed_from_derive_input(st)?;
    let init_data: Vec<_> = fields
        .iter()
        .map(|f| {
            let ident = &f.ident;
            quote! {
                #ident: std::option::Option::None
            }
        })
        .collect();
    Ok(init_data)
}
\end{code-block}

使用该宏的代码则如下:
\begin{code-block}{rust}
#[derive(Debug, Builder)]
pub struct Command {
    executable: String,
    args: Vec<String>,
    env: Vec<String>,
    current_dir: String,
}
fn main() {
    let builder = Command::builder()
        .executable("lucifer".to_owned())
        .args(vec![])
        .env(vec![])
        .current_dir("target".to_owned())
        .build()
        .unwrap();
    info!("{:#?}", builder);
}
\end{code-block}

\subsection{过程宏案例-派生过程宏Builder深入}
通常情况下,Rust的结构体并不是要求所有的字段都必须有值,或者必须初始化,存在可选的
字段,因此,可以继续对Builder过程宏进行改造。比如将上例当中的
\codeinlinebg{rust}{Command}结构体
定义修改为如下:
\begin{code-block}{rust}
pub struct Command {
    executable: String,
    args: Vec<String>,
    env: Vec<String>,
    current_dir: Option<String>,
}
\end{code-block}
即\codeinlinebg{rust}{current_dir}
为可选字段,而Builder生成的结构体\codeinlinebg{rust}{CommandBuilder}
则如下:
\begin{code-block}{rust}
pub struct CommandBuilder {
    executable: Option<String>,
    args: Option<Vec<String>>,
    env: Option<Vec<String>>,
    current_dir: Option<String>,
}
\end{code-block}

而这样的结构体,如果将其展开为语法树,原始结构体的\codeinlinebg{rust}{current_dir}
字段形式大致如下:
\begin{code-block}{json}
Path(
    TypePath {
        qself: None,
        path: Path {
            leading_colon: None,
            segments: [
                PathSegment {
                    ident: Ident {
                        ident: "Option",
                        span: #0 bytes(1337..1343),
                    },
                    arguments: AngleBracketed(
                        AngleBracketedGenericArguments {
                            colon2_token: None,
                            lt_token: Lt,
                            args: [
                                Type(
                                    Path(
                                        TypePath {
                                            qself: None,
                                            path: Path {
                                                leading_colon: None,
                                                segments: [
                                                    PathSegment {
                                                        ident: Ident {
                                                            ident: "String",
                                                            span: #0 bytes(1344..1350),
                                                        },
                                                        arguments: None,
                                                    },
                                                ],
                                            },
                                        },
                                    ),
                                ),
                            ],
                            gt_token: Gt,
                        },
                    ),
                },
            ],
        },
    },
)
\end{code-block}
相比之下,普通的字段类型,其语法树结构可能如下:
\begin{code-block}{rust}
Path(
    TypePath {
        qself: None,
        path: Path {
            leading_colon: None,
            segments: [
                PathSegment {
                    ident: Ident {
                        ident: "String",
                        span: #0 bytes(1267..1273),
                    },
                    arguments: None,
                },
            ],
        },
    },
)
\end{code-block}
可以看到,2者的区别比较大。如果还是沿用上一个示例的方式进行Builder的构建,会出现
这样一个问题:Builder宏所生成的结构体会被处理成如下的形式:
\begin{code-block}{rust}
pub struct CommandBuilder {
    executable: std::option::Option<String>,
    args: std::option::Option<Vec<String>>,
    env: std::option::Option<Vec<String>>,
    current_dir: std::option::Option<Option<String>>,   // 被多层Option封装
}
\end{code-block}
上面的形式显然无法满足我们的需求。因此,为达到上述目的,需要对字段的类型进行进
一步的处理。对于过程宏而言(syn包),字段的类型是一个结构体,其定义大致如下\footnote{类型的定义:\url{https://docs.rs/syn/1.0.72/syn/enum.Type.html}}:
\begin{code-block}{rust}
pub enum Type {
    Array(TypeArray),
    BareFn(TypeBareFn),
    Group(TypeGroup),
    ImplTrait(TypeImplTrait),
    Infer(TypeInfer),
    Macro(TypeMacro),
    Never(TypeNever),
    Paren(TypeParen),
    Path(TypePath),
    Ptr(TypePtr),
    Reference(TypeReference),
    Slice(TypeSlice),
    TraitObject(TypeTraitObject),
    Tuple(TypeTuple),
    Verbatim(TokenStream),
    // some variants omitted
}
\end{code-block}

需要做的,就是对字段类型的语法树进行解构,一步一步的按照需求,准确的定位Option所标识的
字段,并将其的内层类型取出来。因此,需要对上一章节\colorunderlineref{builder}的代码进行重构。

首先需要实现的,就是对可选字段(Option)的语法树定位和处理,并且返回对应的TokenStream:
\begin{code-block}{rust}
fn get_option_fields(st: &syn::Type) -> Option<&syn::Type> {
    // 对type语法树进行解构
    if let syn::Type::Path(syn::TypePath {
        path: syn::Path { segments, .. },
        ..
    }) = st
    {
        if let Some(segment) = segments.last() {
            // 找到Option字段
            if segment.ident.to_string() == "Option" {
                {
                    if let syn::PathArguments::AngleBracketed(
                        syn::AngleBracketedGenericArguments { ref args, .. },
                    ) = segment.arguments
                    {
                        // 获得真实的数据类型
                        if let Some(syn::GenericArgument::Type(inner_type)) = args.first() {
                            return Some(inner_type);
                        }
                    }
                }
            }
        }
    }
    None
}
\end{code-block}
然后,根据需求,将原有的代码进行改写,比如,Builder的结构体字段定义函数,需要修改
\begin{code-block}{rust}
fn generate_builder_struct_fields_def(
    st: &syn::DeriveInput,
) -> syn::Result<proc_macro2::TokenStream> {
    ...
    let types: Vec<_> = fields
        .iter()
        .map(|f| {
            if let Some(ty) = get_option_fields(&f.ty) {
                ty
            } else {
                &f.ty
            }
        })
        .collect();
    ...
}
\end{code-block}
Builder的setter方法需要修改:
\begin{code-block}{rust}
fn generate_setter(st: &syn::DeriveInput) -> syn::Result<proc_macro2::TokenStream> {
    ...
    for (ident, type_) in idents.iter().zip(types.iter()) {
        let token_stream_slice = if let Some(inner_type) = get_option_fields(type_) {
            quote! {
                pub fn #ident(&mut self, #ident: #inner_type) -> & mut Self {
                    self.#ident = std::option::Option::Some(#ident);
                    self
                }
            }
        } else {
            quote! {
                pub fn #ident(&mut self, #ident: #type_) -> & mut Self {
                    self.#ident = std::option::Option::Some(#ident);
                    self
                }
            }
        };
        final_token_stream.extend(token_stream_slice);
    }
    ...
}
\end{code-block}
同样的,Builder的字段检查和初始化方法也需要修改:
\begin{code-block}{rust}
fn check_fileds(st: &syn::DeriveInput) -> syn::Result<proc_macro2::TokenStream> {
    let fields = get_filed_from_derive_input(st)?;
    let idents: Vec<_> = fields.iter().map(|f| &f.ident).collect();
    let types: Vec<_> = fields.iter().map(|f| &f.ty).collect();
    let mut final_check_stream = proc_macro2::TokenStream::new();
    for (ident, type_) in idents.iter().zip(types.iter()) {
        if get_option_fields(type_).is_some() {
            continue;
        }
        let check_stream_slice = quote! {
            if self.#ident.is_none() {
                let err_msg = format!("{} field is missing", stringify!(#ident));
                return std::result::Result::Err(err_msg.into());
            }
        };
        final_check_stream.extend(check_stream_slice);
    }
    Ok(final_check_stream)
}

fn build_target_fields(st: &syn::DeriveInput) -> syn::Result<proc_macro2::TokenStream> {
    let fields = get_filed_from_derive_input(st)?;
    let idents: Vec<_> = fields.iter().map(|f| &f.ident).collect();
    let types: Vec<_> = fields.iter().map(|f| &f.ty).collect();
    let mut final_init_stream = proc_macro2::TokenStream::new();
    for (ident, types_) in idents.iter().zip(types.iter()) {
        let init_stream_slice = if get_option_fields(types_).is_none() {
            quote! {
                #ident: self.#ident.clone().unwrap(),
            }
        } else {
            quote! {
                #ident: self.#ident.clone(),
            }
        };
        final_init_stream.extend(init_stream_slice);
    }
    Ok(final_init_stream)
}
\end{code-block}

改造之后的完整代码如下:
\begin{code-block}{rust}
use proc_macro::TokenStream;
use proc_macro2;
use quote::quote;
use syn::{self, parse_macro_input, spanned::Spanned};
#[proc_macro_derive(Builder)]
pub fn derive(input: TokenStream) -> TokenStream {
    let st = parse_macro_input!(input as syn::DeriveInput);
    match do_expand(&st) {
        Ok(token_stream) => token_stream.into(),
        Err(error) => error.to_compile_error().into(),
    }
}
fn do_expand(st: &syn::DeriveInput) -> syn::Result<proc_macro2::TokenStream> {
    let struct_name_literal = st.ident.to_string();
    let builder_name_literal = format!("{}Builder", struct_name_literal);
    let builder_name_ident = syn::Ident::new(&builder_name_literal, st.span());
    let struct_name_ident = &st.ident;
    let builder_struct_fields_def = generate_builder_struct_fields_def(st)?;
    let builder_struct_fields_init = generate_builder_struct_fields_init(st)?;
    let setter_functions = generate_setter(st)?;
    let checked_res = check_fileds(st)?;
    let build_res = build_target_fields(st)?;
    let ret = quote!(
        pub struct #builder_name_ident {
            #builder_struct_fields_def
        }
        impl #struct_name_ident {
            pub fn builder() -> #builder_name_ident {
                #builder_name_ident {
                    #(#builder_struct_fields_init),*
                }
            }
        }
        impl #builder_name_ident {
            #setter_functions
            pub fn build(&mut self) -> std::result::Result<#struct_name_ident, std::boxed::Box<dyn std::error::Error>>{
                #checked_res
                Ok(#struct_name_ident {
                    #build_res
                })
            }
        }
    );
    Ok(ret)
}
fn check_fileds(st: &syn::DeriveInput) -> syn::Result<proc_macro2::TokenStream> {
    let fields = get_filed_from_derive_input(st)?;
    let idents: Vec<_> = fields.iter().map(|f| &f.ident).collect();
    let types: Vec<_> = fields.iter().map(|f| &f.ty).collect();
    let mut final_check_stream = proc_macro2::TokenStream::new();
    for (ident, type_) in idents.iter().zip(types.iter()) {
        if get_option_fields(type_).is_some() {
            continue;
        }
        let check_stream_slice = quote! {
            if self.#ident.is_none() {
                let err_msg = format!("{} field is missing", stringify!(#ident));
                return std::result::Result::Err(err_msg.into());
            }
        };
        final_check_stream.extend(check_stream_slice);
    }
    Ok(final_check_stream)
}
fn build_target_fields(st: &syn::DeriveInput) -> syn::Result<proc_macro2::TokenStream> {
    let fields = get_filed_from_derive_input(st)?;
    let idents: Vec<_> = fields.iter().map(|f| &f.ident).collect();
    let types: Vec<_> = fields.iter().map(|f| &f.ty).collect();
    let mut final_init_stream = proc_macro2::TokenStream::new();
    for (ident, types_) in idents.iter().zip(types.iter()) {
        let init_stream_slice = if get_option_fields(types_).is_none() {
            quote! {
                #ident: self.#ident.clone().unwrap(),
            }
        } else {
            quote! {
                #ident: self.#ident.clone(),
            }
        };
        final_init_stream.extend(init_stream_slice);
    }
    Ok(final_init_stream)
}
fn generate_setter(st: &syn::DeriveInput) -> syn::Result<proc_macro2::TokenStream> {
    let fields = get_filed_from_derive_input(st)?;
    let idents: Vec<_> = fields.iter().map(|f| &f.ident).collect();
    let types: Vec<_> = fields.iter().map(|f| &f.ty).collect();
    let mut final_token_stream = proc_macro2::TokenStream::new();
    for (ident, type_) in idents.iter().zip(types.iter()) {
        let token_stream_slice = if let Some(inner_type) = get_option_fields(type_) {
            quote! {
                pub fn #ident(&mut self, #ident: #inner_type) -> & mut Self {
                    self.#ident = std::option::Option::Some(#ident);
                    self
                }
            }
        } else {
            quote! {
                pub fn #ident(&mut self, #ident: #type_) -> & mut Self {
                    self.#ident = std::option::Option::Some(#ident);
                    self
                }
            }
        };
        final_token_stream.extend(token_stream_slice);
    }
    Ok(final_token_stream)
}
type StructFields = syn::punctuated::Punctuated<syn::Field, syn::Token![,]>;
fn get_filed_from_derive_input(st: &syn::DeriveInput) -> syn::Result<&StructFields> {
    if let syn::Data::Struct(syn::DataStruct {
        fields: syn::Fields::Named(syn::FieldsNamed { ref named, .. }),
        ..
    }) = st.data
    {
        return Ok(named);
    }
    Err(syn::Error::new_spanned(
        st,
        "Must define on Struct, Not on Enum",
    ))
}
fn generate_builder_struct_fields_def(
    st: &syn::DeriveInput,
) -> syn::Result<proc_macro2::TokenStream> {
    let fields = get_filed_from_derive_input(st)?;
    let idents: Vec<_> = fields.iter().map(|f| &f.ident).collect();
    let types: Vec<_> = fields
        .iter()
        .map(|f| {
            if let Some(ty) = get_option_fields(&f.ty) {
                ty
            } else {
                &f.ty
            }
        })
        .collect();
    let ret = quote! {
        #(#idents: std::option::Option<#types>), *
    };
    Ok(ret)
}
fn get_option_fields(st: &syn::Type) -> Option<&syn::Type> {
    if let syn::Type::Path(syn::TypePath {
        path: syn::Path { segments, .. },
        ..
    }) = st
    {
        if let Some(segment) = segments.last() {
            if segment.ident.to_string() == "Option" {
                {
                    if let syn::PathArguments::AngleBracketed(
                        syn::AngleBracketedGenericArguments { ref args, .. },
                    ) = segment.arguments
                    {
                        if let Some(syn::GenericArgument::Type(inner_type)) = args.first() {
                            return Some(inner_type);
                        }
                    }
                }
            }
        }
    }
    None
}
fn generate_builder_struct_fields_init(
    st: &syn::DeriveInput,
) -> syn::Result<Vec<proc_macro2::TokenStream>> {
    let fields = get_filed_from_derive_input(st)?;
    let init_data: Vec<_> = fields
        .iter()
        .map(|f| {
            let ident = &f.ident;
            quote! {
                #ident: std::option::Option::None
            }
        })
        .collect();
    Ok(init_data)
}
\end{code-block}

完成上述操作之后,在主函数当中,就可以以多种方式进行builder的初始化:
\begin{code-block}{rust}
fn main(){
    let builder = Command::builder()
        .executable("lucifer".to_owned())
        .args(vec![])
        .env(vec![])
        .current_dir("target".to_owned())
        .build()
        .unwrap();
    info!("{:#?}", builder);
    let builder = Command::builder()
        .executable("lucifer".to_owned())
        .args(vec![])
        .env(vec!["titans".to_owned(), "garuda".to_owned()])
        .build()
        .unwrap();
    info!("{:#?}", builder);
}
#[derive(Debug, Builder)]
pub struct Command {
    executable: String,
    args: Vec<String>,
    env: Vec<String>,
    current_dir: Option<String>,
}
\end{code-block}

\subsection{过程宏案例-派生过程宏Builder的派生属性}
\label{builder_attribute}
如果继续进行深入,考虑下面的一种形式:
\begin{code-block}{rust}
#[derive(Builder)]
pub struct Command {
    executable: String,
    #[builder(each = "arg")]
    args: Vec<String>,
    #[builder(each = "env")]
    env: Vec<String>,
    current_dir: Option<String>,
}
fn main() {
    ...
    let builder = Command::builder()
        .executable("lucifer".to_owned())
        .arg("first")
        .arg("second")
        .env(vec!["titans".to_owned(), "garuda".to_owned()])
        .build()
        .unwrap();
}
\end{code-block}
即,使用\codeinlinebg{rust}{#[builder(each = "arg")]}
这样的代码对结构体的字段进行标记,标记之后,原本对结构体进行初始化,使用的是一个vec结构(args),
后续可以使用单个元素(arg)进行追加的方式进行初始化。通常来说,实现这种宏,都应该要求builder当中的标签
和真实的属性名之间最好有区别。如果要求标签和属性名相同,则需要进行额外的其他特殊处理,即需要避免生成一个一次性通过列表进行赋值的方法。
以上述代码为例,如果使用builder宏装饰env字段,并且标签也是env,那么,针对env字段,最好的方式就是不要实现一个
使用列表直接复制的函数,即\codeinlinebg{rust}{builder.env("something")}与
\codeinlinebg{rust}{builder.env(vec!["nothing", "ok"])}这2种形式
无法并存,只能存在第一个。另外的重点是,这个builder宏,只能在被使用了
\codeinlinebg{rust}{#[derive(Builder)]}
修饰之后的结构体当中使用,而不能单独使用,因此,类似builder的宏被称之为惰性属性
宏,其完整的定义形式,一般是\codeinline{rust}{#[proc_macro_derive(Builder, attributes(builder))]}
这种。

由于上述需求的存在,在处理arg和env参数时,就无法再套用之前例子当中的Option类型来
包裹这两个参数了,换言之,对于增加了标签\codeinline{rust}{#[builder(each="arg")]}的
字段,就不能将其在Builder当中处理成Option类型,而是必须保留成Vec类型,即该结构所对应
的Builder结构体,其展开的内容应当如下所示:
\begin{code-block}{rust}
pub struct CommandEachBuilder {
    executable: std::option::Option<String>,
    args: Vec<String>,
    env: Vec<String>,
    current_dir: std::option::Option<String>,
}
\end{code-block}
这点在后续的处理当中非常重要。

惰性属性宏与常见的属性过程宏基本上是一样的,因此,在处理惰性属性宏之前,先看看
常见的属性是什么样式的。简单的查看结构体的属性语法树结构,可以如下进行操作:
\begin{code-block}{rust}
#[proc_macro_derive(Test)]
pub fn do_test(input: TokenStream) -> TokenStream {
    let st = parse_macro_input!(input as syn::DeriveInput);
    let attr = st.attrs.first().unwrap();
    eprintln!("{:#?}", attr);
    proc_macro2::TokenStream::new().into()
}
...
#[derive(Test)]
#[blog::com(Bar)]
pub struct CommandEach {
    executable: String,
    args: Vec<String>,
    ...
}
fn main() {
    ...
}
\end{code-block}
通过eprint函数,可以得到关于属性Attribute的详细信息:
\begin{code-block}{json}
Attribute {
    pound_token: Pound,
    style: Outer,
    bracket_token: Bracket,
    path: Path {
        leading_colon: None,
        segments: [
            PathSegment {
                ident: Ident {
                    ident: "blog",
                    span: #0 bytes(1402..1406),
                },
                arguments: None,
            },
            Colon2,
            PathSegment {
                ident: Ident {
                    ident: "com",
                    span: #0 bytes(1408..1411),
                },
                arguments: None,
            },
        ],
    },
    tokens: TokenStream [
        Group {
            delimiter: Parenthesis,
            stream: TokenStream [
                Ident {
                    ident: "Bar",
                    span: #0 bytes(1412..1415),
                },
            ],
            span: #0 bytes(1411..1416),
        },
    ],
}
\end{code-block}
在上述的输出当中,需要关注的有几个点:
\begin{enumerate}
  \item 属性的style值为Outer,如果对结构体的装饰变成了\codeinlinebg{rust}{#![derive]}这种,则style会变成Inner
  \item Rust会把属性拆分成2部分,path和tokens,其中path表示路径,tokens则是一个原始的TokenStream,并没有变成语法树节点,仅仅只是做了分词处理,没有任何语义,需要自行解析,因此,可以在其中加入自己规定的语法结构
  \item path指导编译器决定如何处理后面的部分
  \item tokens如果本身是一个符合Rust语法规范的结构,可以采用\codeinlinebg{rust}{parse_meta}方法将path和tokens解析成为\codeinlinebg{rust}{syn::Meta}数据类型,使之成为独立的语法树节点
\end{enumerate}

关于\codeinlinebg{rust}{syn::Meta}\footnote{类型定义:\url{https://docs.rs/syn/1.0.73/syn/enum.Meta.html}}
数据类型,其中包含3种:
\begin{enumerate}
  \item \codeinlinebg{rust}{syn::Meta::Path}表示一个路径,\codeinlinebg{rust}{#[A]}只有一个小节,但是A也是一个Path;\codeinlinebg{rust}{#[A::B::C]}也是一个Path,但是会被拆分成多个PathSegment。
  \item \codeinlinebg{rust}{syn::Meta::List}表示一个列表,这个列表必须由一个前置路径和一个括号标记组成,括号内部的内容通过逗号分割为多个条目(组成列表),每个条目又是一个\codeinlinebg{rust}{syn::Meta}
数据类型。比如\codeinlinebg{rust}{#[Foo(AA,BB,CC)]}当中,Foo就是前置路径,而AA等则是列表项,同时,也是\codeinlinebg{rust}{syn::Meta}数据类型;
而\codeinlinebg{rust}{#[Foo(AAA,BBB(CCC,DDD))]}表示列表可以嵌套,Foo是全局的前置路径,而BBB是内层\codeinlinebg{rust}{syn::Meta::List}的前置路径。
  \item \codeinlinebg{rust}{syn::Meta::NameValue}则是常见的键值对,key部分是一个\codeinlinebg{rust}{syn::Path}类型,而value则通常是字符串。比如
\codeinlinebg{rust}{#[xxx="yyy"]}。默认情况下,kv对只能有一对,如果需要有多个kv对,则必须使用list类型,即\codeinline{rust}{#[Foo(x="y",w="z")]}这种形式。
\end{enumerate}
只要是attr符合上述标准,则可以使用\codeinlinebg{rust}{parse_meta}函数对其进行解析,比如下面的例子:
\begin{code-block}{rust}
#[proc_macro_derive(Test)]
pub fn do_test(input: TokenStream) -> TokenStream {
    let st = parse_macro_input!(input as syn::DeriveInput);
    let attr = st.attrs.first().unwrap();
    // 直接解析attr,将attr解析成Rust的语法树节点
    let meta = attr.parse_meta();
    eprintln!("{:#?}", meta);
    proc_macro2::TokenStream::new().into()
}
...
#[derive(Test)]
#[Foo(x = "lucifer")]
pub struct CommandEach {
    executable: String,
    args: Vec<String>,
...
}
\end{code-block}
上述代码通过\codeinlinebg{bash}{cargo check}之后,可以得到类似如下的结果:
\begin{code-block}{json}
Ok(
    List(
        MetaList {
            path: Path {
                leading_colon: None,
                segments: [
                    PathSegment {
                        ident: Ident {
                            ident: "Foo",
                            span: #0 bytes(1414..1417),
                        },
                        arguments: None,
                    },
                ],
            },
            paren_token: Paren,
            nested: [
                Meta(
                    NameValue(
                        MetaNameValue {
                            path: Path {
                                leading_colon: None,
                                segments: [
                                    PathSegment {
                                        ident: Ident {
                                            ident: "x",
                                            span: #0 bytes(1418..1419),
                                        },
                                        arguments: None,
                                    },
                                ],
                            },
                            eq_token: Eq,
                            lit: Str(
                                LitStr {
                                    token: "lucifer",
                                },
                            ),
                        },
                    ),
                ),
            ],
        },
    ),
)
\end{code-block}

在Rust的语法树当中,如果对一个结构体的字段进行展开,其基本
的语法树结构可能如下:
\begin{code-block}{json}
Field {
    attrs: [],
    vis: Inherited,
    ident: Some(
        Ident {
            ident: "executable",
            span: #0 bytes(1429..1439),
        },
    ),
    colon_token: Some(
        Colon,
    ),
    ty: Path(
        TypePath {
            qself: None,
            path: Path {
                leading_colon: None,
                segments: [
                    PathSegment {
                        ident: Ident {
                            ident: "String",
                            span: #0 bytes(1441..1447),
                        },
                        arguments: None,
                    },
                ],
            },
        },
    ),
}
\end{code-block}
即,每个结构体字段都带有attrs(属性)标记。上述语法树为不带属性标记的字段,而
如果是类似本节开头的部分,在env和args上设置了属性标记,该字段的语法树会变为如下的
模式:
\begin{code-block}{json}
Field {
    attrs: [
        Attribute {
            pound_token: Pound,
            style: Outer,
            bracket_token: Bracket,
            path: Path {
                leading_colon: None,
                segments: [
                    PathSegment {
                        ident: Ident {
                            ident: "builder",
                            span: #0 bytes(1455..1462),
                        },
                        arguments: None,
                    },
                ],
            },
            tokens: TokenStream [
                Group {
                    delimiter: Parenthesis,
                    stream: TokenStream [
                        Ident {
                            ident: "each",
                            span: #0 bytes(1463..1467),
                        },
                        Punct {
                            ch: '=',
                            spacing: Alone,
                            span: #0 bytes(1468..1469),
                        },
                        Literal {
                            kind: Str,
                            symbol: "arg",
                            suffix: None,
                            span: #0 bytes(1470..1475),
                        },
                    ],
                    span: #0 bytes(1462..1476),
                },
            ],
        },
    ],
    ...
},
\end{code-block}
需要注意的是,上述语法树内容,全部表示的是结构体字段的属性标记,并非结构体字段
本身的语法树结构,可以看到,其结构和原始的Attribute是相同的。属性标签是Rust当中
常用的标签,也非常灵活。

回到本案例的需求,针对包含有属性标签字段的结构体,首先需要解决的,就是找到这些
带有标签的结构体字段:
\begin{code-block}{rust}
fn get_attr_field_ident(field: &syn::Field) -> Option<syn::Ident> {
    for attr in &field.attrs {
        if let Ok(syn::Meta::List(syn::MetaList {
            ref path,
            ref nested,
            ..
        })) = attr.parse_meta()
        {
            if let Some(__path__) = path.segments.first() {
                if __path__.ident == "builder" {
                    if let Some(syn::NestedMeta::Meta(syn::Meta::NameValue(__dict__))) =
                        nested.first()
                    {
                        if __dict__.path.is_ident("each") {
                            if let syn::Lit::Str(ref arg_token) = __dict__.lit {
                                return Some(syn::Ident::new(
                                    arg_token.value().as_str(),
                                    attr.span(),
                                ));
                            }
                        }
                    }
                }
            }
        }
    }
    None
}
\end{code-block}
上述代码的作用,就是遍历字段的属性(Attr),对其进行解构,如果存在\codeinline{rust}{builder="each"}这种模式的标签,则返回
找到的标签字段(即本例当中的each字段标签),解构的内容参照之前的属性被解析成Meta的相关内容。

获得字段的标签属性之后,接下来就是对结构体的字段进行处理。由于对字段添加了标签,
原有的处理方式已经不太适用了,新的处理方式需要考虑至少3种不同的情况:
\begin{enumerate}
  \item 必需的原始类型被Option包裹
  \item 保留原始的Vec类型
  \item 使用Option包裹可选的原始类型
\end{enumerate}
在开始处理这些结构体字段之前,需要首先了解每一种结构体字段的语法树结构大致是什么
样子,才可以有的放矢,比如,不带标签的标量数据类型(即常见的String,int等没有使用
Vec或者Option包裹的数据类型)的语法树大致如下:
\begin{code-block}{json}
Field {
    attrs: [],
    vis: Inherited,
    ident: Some(
        Ident {
            ident: "executable",
            span: #0 bytes(1686..1696),
        },
    ),
    colon_token: Some(
        Colon,
    ),
    ty: Path(
        TypePath {
            qself: None,
            path: Path {
                leading_colon: None,
                segments: [
                    PathSegment {
                        ident: Ident {
                            ident: "String",
                            span: #0 bytes(1698..1704),
                        },
                        arguments: None,
                    },
                ],
            },
        },
    ),
}
\end{code-block}
同样的,不带标签的矢量数据类型(Vec或者Option)的语法树结构大致如下:
\begin{code-block}{json}
Field {
    attrs: [],
    vis: Inherited,
    ident: Some(
        Ident {
            ident: "others",
            span: #0 bytes(1813..1819),
        },
    ),
    colon_token: Some(
        Colon,
    ),
    ty: Path(
        TypePath {
            qself: None,
            path: Path {
                leading_colon: None,
                segments: [
                    PathSegment {
                        ident: Ident {
                            ident: "Vec",
                            span: #0 bytes(1821..1824),
                        },
                        arguments: AngleBracketed(
                            AngleBracketedGenericArguments {
                                colon2_token: None,
                                lt_token: Lt,
                                args: [
                                    Type(
                                        Path(
                                            TypePath {
                                                qself: None,
                                                path: Path {
                                                    leading_colon: None,
                                                    segments: [
                                                        PathSegment {
                                                            ident: Ident {
                                                                ident: "String",
                                                                span: #0 bytes(1825..1831),
                                                            },
                                                            arguments: None,
                                                        },
                                                    ],
                                                },
                                            },
                                        ),
                                    ),
                                ],
                                gt_token: Gt,
                            },
                        ),
                    },
                ],
            },
        },
    ),
}
\end{code-block}
带有标签的标量数据类型,其语法树结构则可能如下:
\begin{code-block}{json}
Field {
    attrs: [
        Attribute {
            pound_token: Pound,
            style: Outer,
            bracket_token: Bracket,
            path: Path {
                leading_colon: None,
                segments: [
                    PathSegment {
                        ident: Ident {
                            ident: "builder",
                            span: #0 bytes(1688..1695),
                        },
                        arguments: None,
                    },
                ],
            },
            tokens: TokenStream [
                Group {
                    delimiter: Parenthesis,
                    stream: TokenStream [
                        Ident {
                            ident: "each",
                            span: #0 bytes(1696..1700),
                        },
                        Punct {
                            ch: '=',
                            spacing: Alone,
                            span: #0 bytes(1701..1702),
                        },
                        Literal {
                            kind: Str,
                            symbol: "arg",
                            suffix: None,
                            span: #0 bytes(1703..1708),
                        },
                    ],
                    span: #0 bytes(1695..1709),
                },
            ],
        },
    ],
    vis: Inherited,
    ident: Some(
        Ident {
            ident: "executable",
            span: #0 bytes(1715..1725),
        },
    ),
    colon_token: Some(
        Colon,
    ),
    ty: Path(
        TypePath {
            qself: None,
            path: Path {
                leading_colon: None,
                segments: [
                    PathSegment {
                        ident: Ident {
                            ident: "String",
                            span: #0 bytes(1727..1733),
                        },
                        arguments: None,
                    },
                ],
            },
        },
    ),
}
\end{code-block}
带有标签的矢量数据类型的语法树结构则是如下类似:
\begin{code-block}{json}
Field {
    attrs: [
        Attribute {
            pound_token: Pound,
            style: Outer,
            bracket_token: Bracket,
            path: Path {
                leading_colon: None,
                segments: [
                    PathSegment {
                        ident: Ident {
                            ident: "builder",
                            span: #0 bytes(1741..1748),
                        },
                        arguments: None,
                    },
                ],
            },
            tokens: TokenStream [
                Group {
                    delimiter: Parenthesis,
                    stream: TokenStream [
                        Ident {
                            ident: "each",
                            span: #0 bytes(1749..1753),
                        },
                        Punct {
                            ch: '=',
                            spacing: Alone,
                            span: #0 bytes(1754..1755),
                        },
                        Literal {
                            kind: Str,
                            symbol: "arg",
                            suffix: None,
                            span: #0 bytes(1756..1761),
                        },
                    ],
                    span: #0 bytes(1748..1762),
                },
            ],
        },
    ],
    vis: Inherited,
    ident: Some(
        Ident {
            ident: "args",
            span: #0 bytes(1768..1772),
        },
    ),
    colon_token: Some(
        Colon,
    ),
    ty: Path(
        TypePath {
            qself: None,
            path: Path {
                leading_colon: None,
                segments: [
                    PathSegment {
                        ident: Ident {
                            ident: "Vec",
                            span: #0 bytes(1774..1777),
                        },
                        arguments: AngleBracketed(
                            AngleBracketedGenericArguments {
                                colon2_token: None,
                                lt_token: Lt,
                                args: [
                                    Type(
                                        Path(
                                            TypePath {
                                                qself: None,
                                                path: Path {
                                                    leading_colon: None,
                                                    segments: [
                                                        PathSegment {
                                                            ident: Ident {
                                                                ident: "String",
                                                                span: #0 bytes(1778..1784),
                                                            },
                                                            arguments: None,
                                                        },
                                                    ],
                                                },
                                            },
                                        ),
                                    ),
                                ],
                                gt_token: Gt,
                            },
                        ),
                    },
                ],
            },
        },
    ),
}
\end{code-block}

有了以上对于结构体字段语法树结构的了解,处理结构体字段的代码可以大致如下:
\begin{code-block}{rust}
fn get_generic_fields_type_each<'a>(
    st: &'a syn::Type,
    outer_ident_name: &str,
) -> Option<&'a syn::Type> {
    if let syn::Type::Path(syn::TypePath {
        path: syn::Path { segments, .. },
        ..
    }) = st
    {
        if let Some(segment) = segments.last() {
            // 解析原始结构体的字段类型(type),针对使用Option和Vec描述的类型
            // 返回其内部的真实数据类型
            if segment.ident.to_string() == outer_ident_name {
                if let syn::PathArguments::AngleBracketed(syn::AngleBracketedGenericArguments {
                    args,
                    ..
                }) = &segment.arguments
                {
                    if let Some(syn::GenericArgument::Type(inner_type)) = args.first() {
                        return Some(inner_type);
                    }
                }
            }
        }
    }
    None
}
fn generate_builder_struct_fields_def_each(
    fields: &StructFields,
) -> syn::Result<proc_macro2::TokenStream> {
    let idents: Vec<_> = fields.iter().map(|f| &f.ident).collect();
    let types: Vec<_> = fields
        .iter()
        .map(|f| {
            // 如果原始结构体字段的数据类型本身是使用Option封装的,则提取其内部数据类型
            if let Some(inner_type) = get_generic_fields_type_each(&f.ty, "Option") {
                quote!(std::option::Option<#inner_type>)
            } else if get_attr_field_ident(f).is_some() {
                // 如果原始字段上存在标签,在本例当中的做法,是将其默认识别成vec类型
                // 因此,不再使用Option进行包裹
                let origin_type = &f.ty;
                quote!(#origin_type)
            } else {
                // 其他的继续使用Option进行包裹,即使原始字段是Vec类型
                let origin_type = &f.ty;
                quote!(std::option::Option<#origin_type>)
            }
        })
        .collect();
    Ok(quote!( #(#idents: #types), *))
}
\end{code-block}

获取到结构体的字段之后,接下来就是对这些字段进行初始化操作:
\begin{code-block}{rust}
fn generate_builder_struct_fields_init_each(
    fields: &StructFields,
) -> syn::Result<Vec<proc_macro2::TokenStream>> {
    let init_data: Vec<_> = fields
        .iter()
        .map(|f| {
            let ident = &f.ident;
            // 如果原始结构体的字段上存在标签,该字段将被识别为Vec类型
            // 直接使用Vec进行初始化
            if get_attr_field_ident(f).is_some() {
                quote!(#ident: std::vec::Vec::new())
            } else {
                // 否则,使用Option进行填充
                quote!(#ident: std::option::Option::None)
            }
        })
        .collect();
    Ok(init_data)
}
\end{code-block}

接下来,就是生成Builder结构体的setter函数:
\begin{code-block}{rust}
fn generate_setter_each(fields: &StructFields) -> syn::Result<proc_macro2::TokenStream> {
    let idents: Vec<_> = fields.iter().map(|f| &f.ident).collect();
    let types: Vec<_> = fields.iter().map(|f| &f.ty).collect();
    let mut final_token_stream = proc_macro2::TokenStream::new();
    for (idx, (ident, type_)) in idents.iter().zip(types.iter()).enumerate() {
        let mut tokenstream_piece;
        // 如果原始字段是Option类型
        if let Some(inner_type) = get_generic_fields_type_each(type_, "Option") {
            tokenstream_piece = quote! {
                pub fn #ident(&mut self, #ident: #inner_type) -> & mut Self {
                    // 使用Option的内部数据类型对字段进行初始化
                    self.#ident = std::option::Option::Some(#ident);
                    self
                }
            };
        // 如果原始字段上存在标签
        } else if let Some(ref builder_for_each) = get_attr_field_ident(&fields[idx]) {
            // 检测当前字段是否是Vec
            let inner_type = get_generic_fields_type_each(type_, "Vec").ok_or(syn::Error::new(
                fields[idx].span(),
                "each field must be specified with Vec field",
            ))?;
            tokenstream_piece = quote! {
                pub fn #builder_for_each(&mut self, #builder_for_each: #inner_type) -> & mut Self {
                    self.#ident.push(#builder_for_each);
                    self
                }
            };
            // 如果标签名称和字段名称不同,还需要生成一个字段本身的setter方法
            if builder_for_each != ident.as_ref().unwrap() {
                tokenstream_piece.extend(quote! {
                    pub fn #ident(&mut self, #ident: #type_) -> & mut Self {
                        self.#ident = #ident.clone();
                        self
                    }
                });
            }
        } else {
            tokenstream_piece = quote! {
                pub fn #ident(&mut self, #ident: #type_) -> & mut Self {
                    self.#ident = std::option::Option::Some(#ident);
                    self
                }
            };
        };
        final_token_stream.extend(tokenstream_piece);
    }
    Ok(final_token_stream)
}
\end{code-block}

最后的重点,则是生成本需求的build方法,实现Rust的建造者模式:
\begin{code-block}{rust}
fn generate_builder_function(
    fields: &StructFields,
    origin_struct_ident: &syn::Ident,
) -> syn::Result<proc_macro2::TokenStream> {
    let idents: Vec<_> = fields.iter().map(|f| &f.ident).collect();
    let types: Vec<_> = fields.iter().map(|f| &f.ty).collect();
    let mut check_pieces = Vec::new();
    for (idx, (__ident__, __type__)) in idents.iter().zip(types.iter()).enumerate() {
        // 如果字段不是Option或者不存在标签,表示该字段为必需字段,必须进行初始化
        if get_generic_fields_type_each(__type__, "Option").is_none()
            && get_attr_field_ident(&fields[idx]).is_none()
        {
            check_pieces.push(quote! {
                if self.#__ident__.is_none() {
                    let err = format!("{} field missing", stringify!(#__ident__));
                    return std::result::Result::Err(err.into())
                }
            });
        }
    }
    let mut fill_result = Vec::new();
    for (idx, (__ident__, __type__)) in idents.iter().zip(types.iter()).enumerate() {
        // 如果字段存在标签,直接将字段进行拷贝
        if get_attr_field_ident(&fields[idx]).is_some() {
            fill_result.push(quote!(#__ident__: self.#__ident__.clone()));
        // 如果原始字段是Option类型,则将builder的类型进行解包,返回真实的数据类型
        } else if get_generic_fields_type_each(__type__, "Option").is_none() {
            fill_result.push(quote!(#__ident__: self.#__ident__.clone().unwrap()));
        } else {
            fill_result.push(quote!(#__ident__: self.#__ident__.clone()));
        }
    }
    let final_token = quote! {
        pub fn build(&mut self) -> std::result::Result<#origin_struct_ident, std::boxed::Box<dyn std::error::Error>>{
            #(#check_pieces)*
            Ok(#origin_struct_ident {
                #(#fill_result),*
            })
        }
    };
    Ok(final_token)
}
\end{code-block}

有了以上的基础,我们将其有机的结合起来:
\begin{code-block}{rust}
use proc_macro::TokenStream;
use proc_macro2;
use quote::quote;
use syn::{self, parse_macro_input, spanned::Spanned};
type StructFields = syn::punctuated::Punctuated<syn::Field, syn::Token![,]>;
#[proc_macro_derive(BuilderEach, attributes(builder))]
pub fn deriveach(input: TokenStream) -> TokenStream {
    let st = parse_macro_input!(input as syn::DeriveInput);
    match do_expand_each(&st) {
        Ok(token_stream) => token_stream.into(),
        Err(error) => error.to_compile_error().into(),
    }
}
fn do_expand_each(st: &syn::DeriveInput) -> syn::Result<proc_macro2::TokenStream> {
    let struct_name_literal = st.ident.to_string();
    let builder_name_literal = format!("{}Builder", struct_name_literal);
    let builder_name_ident = syn::Ident::new(&builder_name_literal, st.span());
    let struct_name_ident = &st.ident;
    let fields = get_filed_from_derive_input_each(st)?;
    let builder_struct_fields_def = generate_builder_struct_fields_def_each(fields)?;
    let builder_struct_fields_init = generate_builder_struct_fields_init_each(fields)?;
    let setter_functions = generate_setter_each(fields)?;
    let build_function = generate_builder_function(fields, &struct_name_ident)?;
    let ret = quote!(
        pub struct #builder_name_ident {
            #builder_struct_fields_def
        }
        impl #struct_name_ident {
            pub fn builder() -> #builder_name_ident {
                #builder_name_ident {
                    #(#builder_struct_fields_init),*
                }
            }
        }
        impl #builder_name_ident {
            #setter_functions
            #build_function
        }
    );
    Ok(ret)
}
fn get_filed_from_derive_input_each(st: &syn::DeriveInput) -> syn::Result<&StructFields> {
    if let syn::Data::Struct(syn::DataStruct {
        fields: syn::Fields::Named(syn::FieldsNamed { ref named, .. }),
        ..
    }) = st.data
    {
        return Ok(named);
    }
    Err(syn::Error::new_spanned(
        st,
        "Must define on Struct, Not on Enum",
    ))
}
fn get_attr_field_ident(field: &syn::Field) -> Option<syn::Ident> {...}
fn get_generic_fields_type_each<'a>(st: &'a syn::Type, outer_ident_name: &str,
) -> Option<&'a syn::Type> {...}
fn generate_builder_struct_fields_def_each(fields: &StructFields,
) -> syn::Result<proc_macro2::TokenStream> {...}
fn generate_builder_struct_fields_init_each(fields: &StructFields,
) -> syn::Result<Vec<proc_macro2::TokenStream>> {...}
fn generate_setter_each(fields: &StructFields) -> syn::Result<proc_macro2::TokenStream> {...}
fn generate_builder_function(fields: &StructFields, origin_struct_ident: &syn::Ident,
) -> syn::Result<proc_macro2::TokenStream> {...}
\end{code-block}

最终,我们看看其使用方式以及结果:
\begin{code-block}{rust}
#[derive(Debug, BuilderEach)]
pub struct CommandEach {
    executable: String,
    #[builder(each = "arg")]
    args: Vec<String>,
    #[builder(each = "env")]
    env: Vec<String>,
    others: Vec<String>,
    current_dir: Option<String>,
}
fn main() {
    let builder = CommandEach::builder()
        .executable("lucifer".to_owned())
        .arg("lucifer".to_owned())
        .arg("titans".to_owned())
        .env("zhangjl".to_owned())
        .env("luoyan".to_owned())
        .others(vec![])
        .build()
        .unwrap();
    info!("{:#?}", builder);
    let builder = CommandEach::builder()
        .executable("lucifer".to_owned())
        .others(vec![])
        .build()
        .unwrap();
    info!("{:#?}", builder);
}
\end{code-block}

如果通过\codeinlinebg{bash}{cargo expand}将代码进行展开,得到的结果类似如下:
\begin{figure}[H]
  \centering
  \includegraphics[width=\linewidth]{rust_label_expand.png}
  \caption{带标签的结构体展开}
  \label{fig:rust_label_expand}
\end{figure}
而上述代码的执行结果,则大致如下:
\begin{figure}[H]
  \centering
  \includegraphics[width=\linewidth]{rust_label_result.png}
  \caption{带标签的结构体的执行结果}
  \label{fig:rust_label_result}
\end{figure}
可以看到,CommandEach和Command的结果是截然不同的。

现在结构体支持标签了,但是,新的问题又出现了:上述的过程宏要求的标签是\codeinline{rust}{#[builder(each = "...")]}
这种模式,如果标签写错了,即each写成了其他的字符,从程序安全的角度,应当给予足够
的错误信息提示,因此,之前的获取标签的函数需要进行改写:
\begin{code-block}{rust}
fn get_attr_field_ident(field: &syn::Field) -> syn::Result<Option<syn::Ident>> {
    for attr in &field.attrs {
        // 获得标签的语法树节点
        if let Ok(syn::Meta::List(ref list)) = attr.parse_meta() {
            // 二次解构
            let syn::MetaList {
                ref path,
                ref nested,
                ..
            } = list;
            if let Some(__path__) = path.segments.first() {
                if __path__.ident == "builder" {
                    if let Some(syn::NestedMeta::Meta(syn::Meta::NameValue(__dict__))) =
                        nested.first()
                    {
                        if __dict__.path.is_ident("each") {
                            if let syn::Lit::Str(ref arg_token) = __dict__.lit {
                                return Ok(Some(syn::Ident::new(
                                    arg_token.value().as_str(),
                                    attr.span(),
                                )));
                            }
                        } else {
                            // 如果builder当中的标签不是each,则返回一个错误
                            // 错误的范围则限定在标签的语法树节点
                            return Err(syn::Error::new_spanned(
                                list,
                                r#"expected `builder(each = "...")`"#,
                            ));
                        }
                    }
                }
            }
        }
    }
    Ok(None)
}
\end{code-block}

由于需要返回错误信息,因此,也需要对其他使用到该函数的地方进行修改。对于普通的
使用方式,直接在该方法的后面添加\codeinlinebg{rust}{?}即可,而如果是闭包,则需要
稍微注意一下。调用了\codeinlinebg{rust}{get_attr_field_ident}函数的代码修改如下:
\begin{code-block}{rust}
fn generate_builder_function(
    fields: &StructFields,
    origin_struct_ident: &syn::Ident,
) -> syn::Result<proc_macro2::TokenStream> {
    ...
    for (idx, (__ident__, __type__)) in idents.iter().zip(types.iter()).enumerate() {
        if get_generic_fields_type_each(__type__, "Option").is_none()
            && get_attr_field_ident(&fields[idx])?.is_none()
        ...
    }
    let mut fill_result = Vec::new();
    for (idx, (__ident__, __type__)) in idents.iter().zip(types.iter()).enumerate() {
        if get_attr_field_ident(&fields[idx])?.is_some() {
            fill_result.push(quote!(#__ident__: self.#__ident__.clone()));
        }
        ...
    }
    ...
    Ok(final_token)
}
fn generate_setter_each(fields: &StructFields) -> syn::Result<proc_macro2::TokenStream> {
    ...
    for (idx, (ident, type_)) in idents.iter().zip(types.iter()).enumerate() {
        let mut tokenstream_piece;
        if let Some(inner_type) = get_generic_fields_type_each(type_, "Option") {
            ...
        } else if let Some(ref builder_for_each) = get_attr_field_ident(&fields[idx])? {
            ...
        }
    }
}
fn generate_builder_struct_fields_init_each(
    fields: &StructFields,
) -> syn::Result<Vec<proc_macro2::TokenStream>> {
    let init_data: syn::Result<Vec<proc_macro2::TokenStream>> = fields
        .iter()
        .map(|f| {
            let ident = &f.ident;
            if get_attr_field_ident(f)?.is_some() {
                Ok(quote!(#ident: std::vec::Vec::new()))
            } else {
                Ok(quote!(#ident: std::option::Option::None))
            }
        })
        .collect();
    Ok(init_data?)
}
fn generate_builder_struct_fields_def_each(
    fields: &StructFields,
) -> syn::Result<proc_macro2::TokenStream> {
    let idents: Vec<_> = fields.iter().map(|f| &f.ident).collect();
    let types: syn::Result<Vec<_>> = fields
        .iter()
        .map(|f| {
            if let Some(inner_type) = get_generic_fields_type_each(&f.ty, "Option") {
                Ok(quote!(std::option::Option<#inner_type>))
            } else if get_attr_field_ident(f)?.is_some() {
                let origin_type = &f.ty;
                Ok(quote!(#origin_type))
            } else {
                let origin_type = &f.ty;
                Ok(quote!(std::option::Option<#origin_type>))
            }
        })
        .collect();
    let __types__ = types?;
    Ok(quote!( #(#idents: #__types__), *))
}
\end{code-block}
其余代码保持不变,这样,如果在使用该过程宏时写错了标签,即类似如下:
\begin{code-block}{rust}
#[derive(BuilderEach)]
pub struct CommandEach {
    executable: String,
    #[builder(each = "arg")]
    args: Vec<String>,
    #[builder(eac = "env")]
    env: Vec<String>,
    ...
}
\end{code-block}
则代码在编译的时候,就会出现如下比较明确的错误信息,当然了,正确的标签则不会提示错误:
\begin{figure}[H]
  \centering
  \includegraphics[width=\linewidth]{rust_label_error.png}
  \caption{处理有错误的标签}
  \label{fig:rust_label_error}
\end{figure}
\subsection{过程宏案例-自定义Debug}

\input{rust_part_7}



