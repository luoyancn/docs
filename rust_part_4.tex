\subsection{高级函数式编程}
之前的函数式编程当中,提到了map函数,用于对数据进行处理,比如下面这种:
\begin{code-block}{rust}
let sum: u32 = c1
    .zip(c2.skip(10))
    .map(|(a, b)| a * b)
    .filter(|x| x % 3 == 0)
    .sum();
\end{code-block}

但是,实际使用当中,map还有更加广泛的用途,比如,在特定的情况下,替换match操作,
使得代码更加简单和精炼。比如,在使用match处理Option这种数据类型时,由于Option的
取值范围为Some和None,而map函数对于Option类型的处理,也恰好就是返回Some和None,
因此,可以直接使用map函数对这种Some对Some,None对None的简单映射关系进行处理,
多个不同的map进行组合,形成链式调用,相比而言,比match操作会更加简练:
\begin{code-block}{rust}
#[derive(Debug)]
enum Food {
    Apple,
    Potato,
}

#[derive(Debug)]
struct Peeled(Food);
#[derive(Debug)]
struct Chopped(Food);
#[derive(Debug)]
struct Cooked(Food);

// 常见的处理方法,使用match进行处理,并且返回一个Option
fn peel(food: Option<Food>) -> Option<Peeled> {
    match food {
        Some(food) => Some(Peeled(food)),
        None => None,
    }
}

// 使用map函数进行Option的简单映射
fn process(food: Option<Food>) -> Option<Cooked> {
    food.map(|f| Peeled(f))
        .map(|Peeled(f)| Chopped(f))
        .map(|Chopped(f)| Cooked(f))
}
\end{code-block}

然而,如果返回类型Option需要作为map函数的参数,输入到另外一个闭包或者函数当中,
则有可能出现Option<Option<T>>的结果出现,并不利于结果的解析,此时,则需要采用
and\_then进行处理,比如下方的代码:
\begin{code-block}{rust}
enum Food {
    CordonBleu,
    Steak,
    Sushi,
}

fn have_ingredients(food: Food) -> Option<Food> {
    match food {
        Food1::Sushi => None,
        _ => Some(food),
    }
}

fn have_recipe(food: Food) -> Option<Food> {
    match food {
        Food1::CordonBleu => None,
        _ => Some(food),
    }
}

// 通过map函数将上述2个函数进行连接起来,have_recipe当作一个闭包使用
// 但是,结果将变更为Option<Option<T>>
fn cookable_v1(food: Food) -> Option<Option<Food>> {
    have_ingredients(food).map(|res| have_recipe(res))
}

// 通过and_then将2个函数连接起来,形成链式调用
// have_ingredients返回的是一个Option,and_then会将其进行拆包
// 如果Option是None,则直接返回None;但是,如果是Some<T>,and_then则会将其
// 进行拆包,返回为T,而不是Some<T>
fn cookable_v2(food: Food) -> Option<Food> {
    have_ingredients(food).and_then(have_recipe)
}
\end{code-block}

Result和Option类似,但实质上,Option是Result的一个特化版本,可以将其简单的看作:
\begin{code-block}{rust}
type Option<T> = Result<T, ()>
\end{code-block}

因此,Option的map,and\_then等函数(算子)同样可以作用于Result上,比如下面的例子:
\begin{code-block}{rust}
use std::num::ParseIntError;

// 使用普通的match模式
fn multiply_v1(first_number_str: &str, second_number_str: &str) -> Result<i32, ParseIntError> {
    match first_number_str.parse::<i32>() {
        Ok(first_number)  => {
            match second_number_str.parse::<i32>() {
                Ok(second_number)  => {
                    Ok(first_number * second_number)
                },
                Err(e) => Err(e),
            }
        },
        Err(e) => Err(e),
    }
}

// 使用map与and_then模式
fn multiply_v2(first_number_str: &str, second_number_str: &str) -> Result<i32, ParseIntError> {
    // and_then将Result<T, E>拆分,如果是Err,直接返回,如果是T,即Ok(T)
    // 则进行解析为T
    first_number_str.parse::<i32>().and_then(|first_number| {
        second_number_str.parse::<i32>().map(|second_number| first_number * second_number)
    })
}
\end{code-block}

同样的,Result也可以使用别名系统,比如常见的io::Result,实际上就是Result的一个
别名特化版本:
\begin{code-block}{rust}
type Result<T> = Result<T, Error>;
\end{code-block}
因此,同样可以在代码当中使用Result的别名,对代码进行简化:
\begin{code-block}{rust}
use std::num::ParseIntError;

type AliasedResult<T> = Result<T, ParseIntError>;

fn multiply(first_number_str: &str, second_number_str: &str) -> AliasedResult<i32> {
    first_number_str.parse::<i32>().and_then(|first_number| {
        second_number_str
            .parse::<i32>()
            .map(|second_number| first_number * second_number)
    })
}

fn print(result: AliasedResult<i32>) {
    match result {
        Ok(n) => println!("n is {}", n),
        Err(e) => println!("Error: {}", e),
    }
}

fn main() {
    print(multiply("10", "2"));
    print(multiply("t", "2"));
}
\end{code-block}

由于Option和Result的特殊性,在一些特定的场合,尤其是处理错误的时候,常见的做法就是
混合Option和Result,进行混合类型的错误处理:
\begin{code-block}{rust}
use std::num::ParseIntError;

fn double_first(vec: Vec<&str>) -> Option<Result<i32, ParseIntError>> {
    // map返回Option,使用map包裹parse函数可能带来的错误信息(Result)
    vec.first().map(|first| first.parse::<i32>().map(|n| 2 * n))
}

fn double_first_v2(vec: Vec<&str>) -> Result<Option<i32>, ParseIntError> {
    let opt = vec.first().map(|first| first.parse::<i32>().map(|n| 2 * n));

    // map_or返回Result,其中,Ok子句处理opt为None的情况
    // r则处理opt为Some和Err的情况
    opt.map_or(Ok(None), |r| {
        println!("The r is error {:?}", r);
        r.map(Some)
    })
}

fn main() {
    let empty2 = vec![];

    match double_first_v2(empty2) {
        Ok(Some(x)) => println!("The result is {}", x),
        Err(e) => println!("Error is {:?}", e),
        Ok(None) => println!("None is in result"),
    }
}
\end{code-block}

\subsection{自定义错误}
Rust的错误是可以进行自行定义的,只需要实现一个Error Trait即可。Error Trait的定义
如下:
\begin{code-block}{rust}
pub trait Error: Debug + Display {
    fn source(&self) -> Option<&(dyn Error + 'static)> { ... }
    fn backtrace(&self) -> Option<&Backtrace> { ... }
    fn description(&self) -> &str { ... }
    fn cause(&self) -> Option<&dyn Error> { ... }
}
\end{code-block}
其中:
\begin{itemize}
  \item source是必须实现的函数,并且对应的错误必须实现Debug和Display Trait
  \item backtrace是只能在nightly分支当中实现的函数
  \item description被废弃,使用Display Trait或者to\_string(ToString Trait)替代
  \item cause同样被废弃,被source所取代
\end{itemize}

一个简单的例子如下:
\begin{code-block}{rust}
use std::error::Error;
use std::fmt;

// 定义自定义错误结构体
// 实现Debug Trait
#[derive(Debug)]
struct SuperError {
    msg: String,
}

// 实现Display Trait
impl fmt::Display for SuperError {
    fn fmt(&self, f: &mut fmt::Formatter) -> fmt::Result {
        write!(f, "Super Error: {}", self.msg)
    }
}

// 实现Error Trait
impl Error for SuperError {
    fn source(&self) -> Option<&(dyn Error + 'static)> {
        Some(self)
    }
}

impl SuperError {
    fn new(err: &str) -> SuperError {
        SuperError {
            msg: err.to_string(),
        }
    }
}

fn err_test() -> Result<(), SuperError> {
    Err(SuperError::new("first error"))
}

fn main() {
    match err_test() {
        // Err(SuperError{msg: e}) => println!("{}", e),
        Err(e) => println!("{}", e),
        _ => println!("no error"),
    }
}
\end{code-block}

错误和自定义错误解决的是对于错误的定义,以及对应错误的处理方式,但是,在实际的生产
使用当中,错误可能是普遍存在的,而我们需要的数据可能并不包含错误信息,而是需要
将错误从正确的结果当中剔除,比如:
\begin{code-block}{rust}
fn main() {
    let strings = vec!["tofu", "93", "18"];
    let possible_numbers: Vec<_> = strings.into_iter().map(|s| s.parse::<i32>()).collect();
    println!("Results: {:?}", possible_numbers);
}
\end{code-block}
我们的本意是将Vec当中的字符串全部格式化为数值,但是,实际的结果当中,却把包含的
错误也一同包含进来了,需要想办法将错误信息过滤掉:
\begin{code-block}{rust}
fn main() {
    let strings = vec!["tofu", "93", "18"];
    let numbers: Vec<_> = strings
        .into_iter()
        .map(|s| s.parse::<i32>())
        // filter_map进行过滤,只保留结果为ok的数据
        .filter_map(Result::ok)
        .collect();
    println!("Results: {:?}", numbers);
}
\end{code-block}

Result实现了FromIter,因此结果的向量(Vec<Result<T, E>>)可以被转换成结果包裹着
向量(Result<Vec<T>, E>)。一旦找到一个Result::Err,遍历就被终止,即满足另外一种
需求:只要任何一个错误发生,就中断当前的操作:
\begin{code-block}{rust}
fn main() {
    let strings = vec!["tofu", "93", "18"];
    // 注意numbers不再是Vec<_>,而是通过FromIter转换成了Result
    // 转换过程一旦失败,就会出现错误,中断当前的执行流程
    let numbers: Result<Vec<_>, _> = strings.into_iter().map(|s| s.parse::<i32>()).collect();
    println!("Results: {:?}", numbers);
}
\end{code-block}

但是,有的时候,我们也存在另外一种需求:将执行的正确和错误结果分类存放,以待后续
操作,此时则需要使用partition函数,对结果进行区分:
\begin{code-block}{rust}
fn main() {
    let strings = vec!["tofu", "93", "18"];
    let (numbers, errors): (Vec<_>, Vec<_>) = strings
        .into_iter()
        .map(|s| s.parse::<i32>())
        // 使用partition函数进行区分
        .partition(Result::is_ok);
    println!("Numbers: {:?}", numbers);
    println!("Errors: {:?}", errors);

    // 对后续的结果进行解构
    let numbers: Vec<_> = numbers.into_iter().map(Result::unwrap).collect();
    let errors: Vec<_> = errors.into_iter().map(Result::unwrap_err).collect();
    println!("Numbers: {:?}", numbers);
    println!("Errors: {:?}", errors);
}
\end{code-block}

\section{死灵书与实践}

\subsection{随机数实践}
Rust的随机数模块并不包含在标准库当中,需要使用rand这个crate,其基本的使用如下:
\begin{code-block}{rust}
use rand::distributions::{Distribution, Uniform};
use rand::seq::IteratorRandom;
use rand::Rng;

fn main() {
    let mut rng = rand::thread_rng();
    // 生成随机数
    info!("The float64 rand number is {}", rng.gen::<f64>());
    info!("The u32 rand number is {}", rng.gen::<u32>());
    info!("The i32 rand number is {}", rng.gen::<i32>());
    info!("The u8 rand number is {}", rng.gen::<u8>());
    // 从指定区间生成随机数
    info!("The range rand number is {}", rng.gen_range(0..100));
    info!(
        "The range rand float number is {}",
        rng.gen_range(10.0..50.0)
    );
    // 从[0, 5]生成随机数
    info!("The range rand number is {}", rng.gen_range(0..=5));

    // 定义[1, 7)的均匀分布
    let die = Uniform::from(1..7);
    // 从该分布当中生成采样
    let throw = die.sample(&mut rng);
    info!("The sample of uniform is {:>width$}", throw, width = 5);

    // 生成多个随机数
    let tuple: (u8, u8, u8) = rng.gen();
    info!("The tuple of random is {:?}", tuple);
    // 生成随机数组
    let array: [u8; 6] = rng.gen();
    info!("The array of random is {:?}", array);
    let mut exsit_array: [u8; 5] = [1, 2, 34, 5, 6];
    // 使用随机数填充已存在的数组
    rng.fill(&mut exsit_array);
    info!("The array of random is {:?}", exsit_array);

    // 从均匀分布当中随机采样3个数据
    // 得到的结果可能出现重复的情况
    let samples: Vec<u8> = (&mut rng).sample_iter(die).take(3).collect();
    info!("The samples of sample range 1..7 is {:?}", samples);

    let v = vec![1, 2, 3, 4, 5];
    // 从vec当中采样4个数据,得到的结果不会重复
    let sample = v.iter().choose_multiple(&mut rng, 4);
    info!("The samples of sample range 1..5 is {:?}", sample);
    let sample: Vec<u8> = (1..=10).choose_multiple(&mut rng, 4);
    info!("The samples of sample range 1..10 is {:?}", sample);
}
\end{code-block}

Rust的rand crate不仅可以生成随机数,也可以生成自定义的随机数据,比如:
\begin{code-block}{rust}
use rand::distributions::{Distribution, Standard, Uniform};
use rand::seq::IteratorRandom;
use rand::Rng;

struct Point {
    x: u8,
    y: u8,
}

impl fmt::Display for Point {
    fn fmt(&self, f: &mut fmt::Formatter) -> fmt::Result {
        write!(f, "x: {}, y: {}", self.x, self.y)
    }
}

// 在 Point 类型之上,对Standard实现Distribution trait,使得Point可以被gen函数随机生成
impl Distribution<Point> for Standard {
    // 默认的实现方法
    fn sample<R: Rng + ?Sized>(&self, rng: &mut R) -> Point {
        let (rand_x, rand_y) = rng.gen();
        Point {
            x: rand_x,
            y: rand_y,
        }
    }
}

fn main() {
    let mut rng = rand::thread_rng();
    let rand_point = rng.gen::<Point>();
    info!("The rand_point is {}", rand_point);
}
\end{code-block}

同样的,可以生成随机的字符串:
\begin{code-block}{rust}
use rand::distributions::{Alphanumeric, Distribution, Standard, Uniform};
use rand::seq::IteratorRandom;
use rand::Rng;

fn main() {
    let mut rng = rand::thread_rng();
    let rand_string: String = (&mut rng)
        // 从a-z,A-Z以及0-9当中进行选择
        .sample_iter(&Alphanumeric)
        // 获取其中的10个元素
        .take(10)
        // 默认的结果是char类型,需要继续转换成String
        .map(char::from)
        .collect();
    info!("The rand_string is {}", rand_string);
}
\end{code-block}

如果默认的字符集不满足要求,还可以自定义字符集,比如下面的示例:
\begin{code-block}{rust}
use rand::distributions::{Alphanumeric, Distribution, Standard, Uniform};
use rand::seq::IteratorRandom;
use rand::Rng;
const CHARSET: &[u8] = b"ABCDEFGHIJKLMNOPQRSTUVWXYZ\
    abcdefghijklmnopqrstuvwxyz\
    0123456789)(*&^%$#@!~";
const PASSWORD_LEN: usize = 10;

fn main() {
    let mut rng = rand::thread_rng();
    let password: String = (0..PASSWORD_LEN)
        .map(|_| {
            let idx = rng.gen_range(0..CHARSET.len());
            CHARSET[idx] as char
        })
        .collect();
    info!("The password is {}", password);

    // 也可以更换成之前的采样函数,看起来更为精炼
    let passwd: String = CHARSET
        .choose_multiple(&mut rng, 10)
        .map(|r| *r as char)
        .collect();
    info!("The password is {}", passwd);
}
\end{code-block}

同样的,针对自定义的数据类型,同样可以采用采样方法,进行随机数据的提取:
\begin{code-block}{rust}
use rand::distributions::{Alphanumeric, Distribution, Standard, Uniform};
use rand::seq::IteratorRandom;
use rand::Rng;

#[derive(Debug)]
struct Person {
    name: String,
    age: u8,
}

fn main() {
    let mut rng = rand::thread_rng();
    let persons = vec![
        Person {
            name: "lucifer".to_string(),
            age: 18,
        },
        Person {
            name: "titans".to_string(),
            age: 19,
        },
        Person {
            name: "garuda".to_string(),
            age: 36,
        },
    ];

    // 从person的vec当中,随机抽取2个元素
    let rand_person: Vec<_> = persons.choose_multiple(&mut rng, 2).collect();
    info!("The rand person is {:?}", rand_person);
}
\end{code-block}

\subsection{类型再论}
Rust的类型比较多,char,字符串,整数,浮点数等等。这些基础类型和其他语言比较类似,
但是也包含了自己的特点:比如,char类型占据4个字节,可以存放任何一个unicode字符;
对于ASCII字符,只需要一个字节即可,而一个字节的数据,则可以放在u8类型的数据当中,
因此,对于ASCII类型的字符串/字符数组,可以使用u8类型(即单字节)的数组进行存放,
这样,占用的资源空间会比char的数组小:
\begin{code-block}{rust}
fn main() {
    // 字符串前面的b,表示将对应的字面量存放在u8类型当中
    let s: &[u8] = b"hello";
    info!("{:?}", s);
}
\end{code-block}
同时,Rust支持的整数类型比较广泛,包括8bit,16bit,32bit,64bit,最大可以支持到
128bit;而特殊的isize和usize,则是和平台相关。如果平台是32位的,则isize和usize为
32位,如果是64位,则其数据宽度为64位。

整个Rust的类型当中,只有空类型占据的空间是最小的,都是0。Rust的空类型包括单元类型
(unit,即空元组)以及空结构体:
\begin{code-block}{rust}
// empty是空元组类型
let empty : () = ();

// 空结构体
struct Empty();
\end{code-block}
为了查看类型所占用的空间,可以使用size\_of函数进行查看:
\begin{code-block}{rust}
use std::mem;

struct Empty();
fn main() {
    info!("The Empty struct size is {}", mem::size_of::<Empty>());
    // 查看空元组所占据的内存大小
    info!("The none tuple size is {}", mem::size_of::<()>());
}
\end{code-block}

在Rust当中,浮点类型是非常特殊的数据类型。浮点类型当中,存在一个特殊的值:NaN,
即非法的浮点数值,因为该数据的存在,浮点数不具备全序关系(total order)。所谓的
全序,偏序,Rust当中的定义如下:对于集合X当中的元素a,b,c
\begin{itemize}
  \item 如果a<b,则!(a>b)一定成立;反之,如果a>b,则!(a<b)一定成立,即反对称性
  \item 如果a<b,b<c,则a<c,即传递性
  \item 对于X当中的所有元素,都存在a<b,或者a>b,或者a==b,三者必居其一,即完全性
\end{itemize}
如果X集合只满足前面2条,则称之为偏序;具备上述所有特征,则为全序。由于浮点数的NaN
不满足上述第3条规则,因此,Rust的浮点数属于偏序,而非全序,这回导致一个问题:浮点
数无法排序——非NaN的数值无法与NaN进行比较:
\begin{code-block}{rust}
let nan = std::f32::NAN;
let x = 0.4f32;
// 下列结果全部为false
info!("{}", nan > x);
info!("{}", nan < x);
info!("{}", nan == x);
\end{code-block}
为此,Rust设计了2个Trait表示全序与偏序:\mintinline[breaklines=true,breakanywhere,breaksymbolleft=,breakanywheresymbolpre=,]{rust}{std::cmp::Ord}(全序)以及
\mintinline[breaklines=true,breakanywhere,breaksymbolleft=,breakanywheresymbolpre=,]{rust}{std::cmd::PartialOrd}(偏序)。
PartialOrd这个Trait的partial\_cmp方法返回的是Option<Ordering>,而Ord返回的却是
Ordering。Rust的f32和f64都只实现了PartialOrd,因此,浮点类型无法进行排序,也同样无法
求取最值,如下列代码,则是无法运行的:
\begin{code-block}{rust}
let f_vec = vec![1f32, 2.0, 4.0, 0.0, -1.2];
let bigest_f = f_vec.iter().max();
\end{code-block}
对上诉代码进行编译,会直接提示如下类似的错误:
\begin{figure}[H]
  \centering
  \includegraphics[scale=0.2]{rust_float_cmp_error.png}
  \caption{浮点数的最值错误求解}
  \label{fig:rust_float_cmp_error}
\end{figure}
浮点数的排序只能通过partial\_cmp(比较相等关系)进行变换处理,如下方代码:
\begin{code-block}{rust}
let mut f_vec = vec![1f32, 2.0, 4.0, 0.0, -1.2];
// 升序排列
f_vec.sort_by(|first, second| first.partial_cmp(second).unwrap());
// 获取排序后的最后一位
let max = f_vec.last().unwrap();
// 或者如下进行
// let max = f_vec.as_slice().last().unwrap();
// 降序排列
f_vec.sort_by(|first, second| second.partial_cmp(first).unwrap());
\end{code-block}

作为常用数据类型之一,Rust的数组也存在自己的特点,比如同类型的数组之间可以相互赋值:
\begin{code-block}{rust}
let mut array: [u32; 4] = [1, 23, 4, 5];
let array_copy: [u32; 4] = [5, 6, 7, 8];
array = array_copy;
\end{code-block}
支持数组之间的直接比较,只是数组当中的元素本身就可以进行比较才行:
\begin{code-block}{rust}
let array: [u32; 4] = [1, 23, 4, 5];
let array_copy: [u32; 4] = [5, 6, 7, 8];
info!("{:?}", array < array_copy);
\end{code-block}

Rust当中的函数也可以称之为类型的一种,并且,每个函数都有自己单独的类型,函数的类型
是fn。但是,函数的参数列表会影响fn类型的判断和表达,比如下面的例子:
\begin{code-block}{rust}
fn add_tuple(t: (u32, u32)) -> u32 {
    t.0 + t.1
}

fn add_two((x, y): (u32, u32)) -> u32 {
    x + y
}

fn add_normal(x: u32, y: u32) -> u32 {
    x + y
}

\end{code-block}
实际上,add\_tuple和add\_two这2个函数被fn类型识别成为具有相同签名的类型,因此,
在理论上,我们可以使用同一个变量,接收这2个函数的指针:
\begin{code-block}{rust}
fn main() {
    let mut func = add_tuple;
    func = add_two;
    ...
}
\end{code-block}
但是,上述代码却是错误的:虽然签名相同,但是,类型不同:
\begin{figure}[H]
  \centering
  \includegraphics[width=\linewidth]{rust_func_type.png}
  \caption{相同签名的不同函数类型}
  \label{fig:rust_func_type}
\end{figure}
解决方法,则是将其转换成通用的fn类型:
\begin{code-block}{rust}
fn main() {
    // 显示指定func的类型
    let mut func: fn((u32, u32)) -> u32 = add_tuple;
    // 使用as进行类型的转换
    // let mut func = add_tuple as fn((u32, u32)) -> u32;
    func = add_two;
    ...
}
\end{code-block}
但是,需要注意,add\_normal的功能看上去和前面两个函数的功能相同,但是,他们的
函数签名完全不同,因此,不能将其转换成func。

函数是Rust的头等公民,可以在函数/方法当中定义函数,也可以在函数/方法当中定义结构
体,甚至于定义结构体的方法和实现,以及静态变量,常量等:
\begin{code-block}{rust}
fn func_as_first(x: u32, y: u32) -> (u32, u32) {
    struct Point {
        x: u32,
        y: u32,
    };

    impl Point {
        fn area(&self) -> u32 {
            self.x * self.y
        }
        fn cycle(&self) -> u32 {
            self.x + self.y
        }
    };

    let p = Point { x: x, y: y };
    (p.area(), p.cycle())
}
\end{code-block}

常规的函数类型,都会存在返回值,这些返回值要么是特定的类型,要么就是(),即类似
C/C++的返回void。如果需要什么都不返回,则可以使用!,这种函数称之为发散函数,比如
在处理panic时,有时就需要使用发散函数:
\begin{code-block}{rust}
fn diverges() -> ! {
    panic!("This function never returns!");
}
\end{code-block}
Panic操作会直接导致软件栈展开,因此,后续的操作都不会执行,其返回的就是一个!。
发散函数的最大特点,就是可以被转换成任意一个类型,虽然执行的时候最终还是会崩溃,
如下:
\begin{code-block}{rust}
let x : i32 = diverges();
let y : String = diverges();
\end{code-block}
但是,发散函数最大的作用,在于解决编译器的类型检查:
\begin{code-block}{rust}
let p = if x {
    panic!("error");
} else {
    100
};
\end{code-block}
对于let-if而言,if-else的每个分支都必须是相同的数据类型,通过发散函数的任意类型
转换特性即!与任何类型兼容,所以上述代码才能编译通过。

所有的Rust变量,函数都是类型的一种,都可以通过一定的手段和方式,获得类型的具体信息。
常见的方式有两种,一种是使用错误信息进行推断,一种则是使用标准库函数进行获得。

通过构造一个特殊的函数,然后调用该函数,则可以获得相关的类型信息:
\begin{code-block}{rust}
// 接收一个unit参数
fn type_id(_: ()) {}

fn main() {
    let ref i = 5;
    type_id(i);
}
\end{code-block}

而另外的方式,则是使用标准库函数,不过,这个标准库函数在Rust的默认stable分支当中
是不可用的,需要在nightly分支当中进行编译使用,并且,还需要启用一些特性:
\begin{code-block}{rust}
#![feature(core_intrinsics)]
use std;

// 使用泛型参数进行不同类型的数据接收
fn print_type<T>(_arg: &T) {
    println!(
        "The type name of arg is {}",
        std::intrinsics::type_name::<T>()
    );
}

fn main() {
    let ref x = 5;
    print_type(&x);
}
\end{code-block}
编译上述代码时,则需要对编译指令进行部分的调整:
\mintinline[breaklines=true,breakanywhere,breaksymbolleft=,breakanywheresymbolpre=,]{bash}{cargo +nightly build},
然后即可实现对参数类型的打印输出。

在Rust当中,与Python不同,函数/方法并不存在默认参数,但是,结构体当中的字段,却可以
有默认值,只是,这个默认值的实现,必须和Default Trait相结合,如下:
\begin{code-block}{rust}
struct ColoredString {
    input: String,
    fg_color: String,
    bg_color: String,
}

impl Default for ColoredString {
    fn default() -> Self {
        ColoredString {
            input: String::default(),
            fg_color: String::default(),
            bg_color: String::default(),
        }
    }
}

fn main() {
    let color = ColoredString::default();
}
\end{code-block}
从上述代码当中可以看出,实际上,并不是Rust的结构体字段赋予了初始值,而是通过一个
名为default的方法,构造一个我们认为应该具有默认值的结构体。在Rust当中,常用的基本
数据类型都实现了Default Trait,可以直接使用对应的default方法。

\subsection{Trait类型与泛型再论}
关于类型,Trait也是比较重要的一个话题。在之前的示例当中,Trait全部是在具体的类型
上实现的,但是,Trait本身也可以在智能指针(Box)上实现,比如:
\begin{code-block}{rust}
trait Shape {
    fn area(self: Box<Self>) -> f64;
}
struct Circle {
    radius: f64,
}
impl Shape for Circle {
    fn area(self: Box<Self>) -> f64 {
        PI * self.radius * self.radius
    }
}
fn main() {
    let c = Box::new(Circle { radius: 4f64 });
    info!("{}", c.area());
    // 由于trait实现是在智能指针box上,因此,下面的使用是错误的
    // let c = Circle { radius: 4f64 }
    // c.area()
}
\end{code-block}
甚至在Trait上实现Trait,比如下方:
\begin{code-block}{rust}
trait Shape {
    fn area(&self) -> f64;
}
trait Round {
    fn get_radius(&self) -> f64;
}
struct Circle {
    radius: f64,
}
impl Round for Circle {
    fn get_radius(&self) -> f64 {
        self.radius
    }
}
impl Shape for dyn Round {
    fn area(&self) -> f64 {
        let radius = self.get_radius();
        PI * radius * radius
    }
}
\end{code-block}
Shape是一个Trait,Round同样也是一个Trait,Circle实现了Round,Round实现了Shape,
但是,由于Round本身是一个Trait,拥有不确定性,因此,在实现Shape的时候,需要添加
dyn关键字,提示这个Round不是普通的类型,而是一个Trait。上述代码当中,Circle间接
的实现了Shape,但是,Circle的类型无法直接使用Shape的方法,只能通过智能指针的方
式,将Circle转换成Round的类型,再进行使用,如下:
\begin{code-block}{rust}
fn main() {
    let c: Box<dyn Round> = Box::new(Circle { radius: 4f64 });
    info!("{}", c.area());
}
\end{code-block}
如果再把这个例子改得复杂一些,让Circle和Sphere同时实现Round,则我们可以使用Round
指针计算2个不同类型数据的结果:
\begin{code-block}{rust}
trait Shape {
    fn area(&self) -> f64;
}
trait Round {
    fn calc(&self) -> f64;
}
struct Circle {
    radius: f64,
}
impl Round for Circle {
    fn calc(&self) -> f64 {
        PI * self.radius * self.radius
    }
}
struct Sphere {
    radius: f64,
}
impl Round for Sphere {
    fn calc(&self) -> f64 {
        4f64 * PI * self.radius * self.radius
    }
}
impl Shape for dyn Round {
    fn area(&self) -> f64 {
        self.calc()
    }
}
fn main() {
    let circle: Box<dyn Round> = Box::new(Circle { radius: 4f64 });
    info!("The Circle area is {}", circle.area());
    let sphere: Box<dyn Round> = Box::new(Sphere { radius: 4f64 });
    info!("The Sphere area is {}", sphere.area());
}
\end{code-block}

Trait不仅仅用于实现类型,约束类型,还可以用于为其他现有的数据类型添加方法/函数,
比如:
\begin{code-block}{rust}
impl Round for i32 {
    fn calc(&self) -> f64 {
        *self as f64
    }
}
fn main() {
    let i_struct = 4i32;
    i_struct.calc();
}
\end{code-block}
这种类型的函数/方法,则称之为扩展方法/函数。从上述例子当中,我们似乎可以使用Trait
对任意类型进行函数/方法的扩展,但是,这个是存在前提的:
\begin{itemize}
  \item impl和trait的声明/定义在同一个crate当中
  \item 或者,impl的实现需要和类型的声明在同一个crate当中
\end{itemize}
如果不满足上述条件,则容易出现bug和问题,也会违反Rust的规则。

Rust的Trait支持多种特性,自然也支持继承,但是注意,Rust的结构体和enum数据类型并不
存在继承的概念。Trait的继承方式如下:
\begin{code-block}{rust}
trait Base {}
trait Derived : Base {}
\end{code-block}
当一个结构体实现了上述的Derived这个Trait,则必须同样实现Base这个Trait,否则就会
出现语法错误:
\begin{code-block}{rust}
trait Base {}
trait Derived : Base {}
struct T;
impl Derived for T {}
impl Base for T {}
\end{code-block}

Rust的Trait不仅可以包括函数的定义,同样可以直接定义函数:
\begin{code-block}{rust}
trait Page {
    fn set_page(&self) {
        info!("Page Default: 1");
    }
}
trait PerPage {
    fn set_per_page(&self) {
        info!("Per Page Default: 1");
    }
}
struct Paginate {
    page: u32,
}
impl Page for Paginate {}
impl PerPage for Paginate {}
fn main() {
    let page = Paginate { page: 8 };
    page.set_page();
    page.set_per_page();
    page.set_skip_page();
}
\end{code-block}

甚至于,Trait可以直接给结构体提供更多的组合方法:
\begin{code-block}{rust}
trait PaginateMore: Page + PerPage {
    fn set_skip_page(&self) {
        info!("Skip the page");
    }
}
fn main() {
    ...
    page.set_skip_page();
}
\end{code-block}
结构体根本不用自行实现Trait PaginateMore,就可以直接使用该Trait当中的方法。

Trait不仅仅可以用于接口实现,在Rust当中,更重要的则是类型限定,限定某些数据只能
做某些事情。比如下方的代码:
\begin{code-block}{rust}
...
fn static_dispatch<T>(t: &T) where T: Bar {
    ...
}
fn dynamic_dispatch(t : &Bar) {
    ...
}
\end{code-block}
对于实现了Trait Bar的类型来说,上述2个函数,都可以被调用,但是,从语法上,static\_dispatch
由于使用了where,表示参数必须限定在Trait Bar类型,在编译时就能够确定;而dynamic\_dispatch
则从语法上表示,输入的参数必须是Bar的对象,即Trait Object。运行时,Trait Object会根据虚表
指针从虚表当中查出正确的指针,再进行动态调用,属于在运行时确定。

但是并不是每一个Trait都可以当着Trait Object使用,这个和类型大小是否确定有关系。每一个
Trait的隐藏类型参数Self默认限定为?Sized,?Sized trait包括了所有动态大小类型以及所有
可确定大小的类型。Rust当中大部分类型都是默认可确定大小的,即<T:Sized>。当trait对象
在运行期进行动态分发时,也必须确定大小,否则无法分配内存。只有同时满足下列条件的
trait,才可以当作Trait Object使用:
\begin{itemize}
  \item Trait的Self不能被限定为Sized
  \item Trait当中的所有方法都必须是对象安全的
\end{itemize}

而所谓的对象安全,则必须满足如下的条件\underline{\color{red} \textbf{之一}}:
\begin{itemize}
  \item 当Trait的Self被限定为Sized时,方法受Self:Sized约束
  \item Trait的方法签名必须\underline{\color{red} \textbf{同时满足以下3点}}
  \begin{enumerate}
    \item 不包含任何泛型参数(Self)
    \item 第一个参数必须为Self类型或可解引用为Self类型
    \item Self不能出现在除第一个参数之外的其他地方
  \end{enumerate}
  \item Trait当中不能包含关联常量
\end{itemize}

比如下面的代码,就属于标准的对象安全:
\begin{code-block}{rust}
trait Bar {
    fn bax(self, x: u32);
    fn bay(&self);
    fn baz(&mut self);
}
\end{code-block}
Trait Bar不受Sized限制,Trait的方法没有额外的Self类型参数,没有泛型参数,因此是安全的。
相对应的,不安全的Trait如下:
\begin{code-block}{rust}
// 对象不安全
trait Foo {
    fn bad<T>(&self, x:T);
    fn new() -> Self;
}

// 对象安全
trait Foo {
    fn bad<T>(&self, x: T);
    fn new() -> Self
    where
        Self: Sized;
}
\end{code-block}

当然,Sized约束也可以用于Trait定义当中。比如,自行实现一个类似any的Any Trait。
\begin{code-block}{rust}
use std::ops::Fn;
trait CustomAny {
    fn custom_any<F>(&self, f: F) -> bool
    where
        Self: Sized,
        F: Fn(u32) -> bool;
}

impl CustomAny for Vec<u32> {
    fn custom_any<F>(&self, f: F) -> bool
    where
        Self: Sized,
        F: Fn(u32) -> bool,
    {
        for &x in self {
            if f(x) {
                return true;
            }
        }
        false
    }
}

fn main() {
    let v: Vec<u32> = vec![1, 2, 3];
    info!("{}", v.iter().any(|&x| x == 3));
    info!("{}", v.custom_any(|x| x == 3));
}
\end{code-block}

Trait当中不仅可以包含函数和方法,同样可以包含变量和常量,即所谓的关联变量以及关联
常量。关联常量的使用稍微有些特殊,在Trait当中可以定义关联常量,但是,使用的时候,
却是通过Trait的实现对象来使用这些关联常量的:
\begin{code-block}{rust}
trait Colorize {
    // 定义关联常量
    const FG_RED: &'static str = "31";
    const BG_YELLOW: &'static str = "43";
    fn red(self) -> ColoredString;
    fn on_yellow(self) -> ColoredString;
}

impl Colorize for ColoredString {
    fn red(self) -> ColoredString {
        ColoredString {
            // 使用关联常量,如果是Colorize::FG_RED,则会提示错误
            fg_color: String::from(ColoredString::FG_RED),
            ..self
        }
    }
    fn on_yellow(self) -> ColoredString {
        ColoredString {
            bg_color: String::from(ColoredString::BG_YELLOW),
            ..self
        }
    }
}
\end{code-block}

Trait不仅仅可以实现泛型,泛型也不仅限于Trait和<T>,对于函数/方法,也可以使用在
泛型、生命周期以及Trait当中,比如,显式的指定闭包的生命周期:
\begin{code-block}{rust}
// 将函数作为泛型参数
struct Pick<F> {
    data: (u32, u32),
    func: F,
}

impl<F> Pick<F>
where
    // for<>只能用于标记生命周期
    F: for<'f> Fn(&'f (u32, u32)) -> &'f u32,
{
    fn call(&self) -> &u32 {
        (self.func)(&self.data)
    }
}

fn max(data: &(u32, u32)) -> &u32 {
    if data.0 > data.1 {
        return &data.0;
    }
    &data.1
}

fn main() {
    let pick = Pick {
        data: (32, 34),
        func: max,
    };
    info!("{}", pick.call());
}
\end{code-block}

\subsection{常见的设计模式}
建造者模式是Rust当中最常用的设计模式之一,其主旨思想在于将可变和不可变进行分离,
一种基本的示例如下:
\begin{code-block}{rust}
use std::f64::consts;

pub struct Circle {
    radius: f64,
}

pub struct CircleBuilder {
    radius: f64,
}

impl Circle {
    pub fn new() -> CircleBuilder {
        CircleBuilder { radius: 0.0 }
    }

    pub fn area(&self) -> f64 {
        self.radius * self.radius * consts::PI
    }
}

impl CircleBuilder {
    pub fn radius(&mut self, radius: f64) -> &mut CircleBuilder {
        self.radius = radius;
        self
    }
    pub fn build(&self) -> Circle {
        Circle {
            radius: self.radius,
        }
    }
}
\end{code-block}

\subsection{并行与并发再论}
虽然Rust本身的线程/进程管理非常完善,但是,thread::spawn产生的线程没有名称,并且
其栈空间大小默认为2M,如果需要需要针对线程/进程进行粒度更细的操作,比如自定义
线程名称,自定义线程的资源等等,此时,就需要使用thread::Builder进行修改,具体示例
如下:
\begin{code-block}{rust}
let mut v_thread = vec![];
for id in 1..5 {
    let thread_name = format!("child-{}", id);
    let size: usize = 1024;
    // 定义线程的名称,设置线程占用的栈大小为1M(1024)
    let builder = Builder::new().name(thread_name).stack_size(size);
    // builder.spawn返回的是Result<JoinHander, std::io::Error>
    // 需要进行处理,取出真正的线程句柄
    match builder.spawn(move || {
        info!(
            "In the child: {}, and the child name is {}",
            id,
            current().name().unwrap()
        );
    }) {
        Ok(child) => v_thread.push(child),
        Err(error) => error!("Cannot create the thread {} because: {:?}", id, error),
    };
}

// 其他的同普通的线程,
for child in v_thread {
    child.join().unwrap();
}
\end{code-block}

由于线程包含自己的资源空间,因此,存在一个特殊的存储空间——线程本地存储(Thread Local Storage,TLS),
存放在该区域的资源,其他线程无法访问,而是每个线程独占的数据:
\begin{code-block}{rust}
use std::cell::RefCell;
use std::thread;

fn main() {
    // 在线程本地存储定义一个FOO变量,最终的类型是thread::LocalKey
    thread_local!(static FOO: RefCell<u32> = RefCell::new(1));
    // 提供了一个with方法,可以通过给该方法传入闭包
    // 来操作线程本地存储中包含的变量
    FOO.with(|f| {
        info!("The f borrow is {}", *f.borrow());
        *f.borrow_mut() = 2;
    });

    let handler = thread::spawn(move || {
        // 子线程也有一个线程本地存储实例FOO,为主线程的副本
        // 也可以使用thread_local!宏在该子线程中重新创建一个LocalKey实例
        FOO.with(|f| {
            info!("In the handler thread The f borrow is {}", *f.borrow());
            *f.borrow_mut() = 3;
        });
    });

    // 主线程当中FOO实例并没有被子线程修改为3
    // thread local!宏定义单个线程内的一些独享数据
    FOO.with(|f| {
        info!("The f borrow is {}", *f.borrow());
    });

    handler.join().unwrap();
}
\end{code-block}

在同步原语支持方面,Rust也有自己的实现方式,通过使用std::thread当中的park函数提供
阻塞线程的能力,但并不能永久的阻塞线程,存在时间限制;而std::thread::part\_timeout
则可以显式的指定阻塞的超时时间;std::thread::Thread::unpark则可以将阻塞的线程重启;
如果需要让出当前线程的时间片,则需要使用std::thread::yeild\_now,让其他线程进行执行。
简单的阻塞例子如下:
\begin{code-block}{rust}
use std::thread::{self, Builder};
use std::time::Duration;

fn main() {
    let parked_thread = Builder::new()
        .spawn(|| {
            info!("Parking the thread ...");
            // 阻塞当前线程
            thread::park();
            info!("Thread parked");
        })
        .unwrap();
    thread::sleep(Duration::from_secs(5));
    info!("Unparking the thread");
    // 从JoinHandle中得到具体的线程
    parked_thread.thread().unpark();
    // 将该线程重新启动,该线程会继续沿着之前暂停的上下文执行
    parked_thread.join().unwrap();
}
\end{code-block}

除了常见的互斥锁(Mutex)之外,Rust也支持读写锁(RwLock)。读写锁的基本示例如下:
\begin{code-block}{rust}
use std::sync::RwLock;

fn main() {
    let rw_lock = RwLock::new(5);
    // 读写锁的使用必须使用{}进行区分,即便是单独使用读或者写也是一样
    // 通过代码块{},让读写锁自动释放,否则会出现死锁
    {
        let read_1 = rw_lock.read().unwrap();
        let read_2 = rw_lock.read().unwrap();
        info!("The read_1 is {}, and read_2 is {}", read_1, read_2);
    }
    {
        let mut write = rw_lock.write().unwrap();
        *write = 100;
    }
    info!("The data is {:?}", rw_lock);
}
\end{code-block}

而针对于同步的需求,Rust提供了屏障(Barrier)和条件变量(Condition Variable)原语。
屏障,是要求所有的条件全部满足之后,再进行后续操作,即在满足某个条件前,阻塞全部的
线程,通常用于线程同步,如下:
\begin{code-block}{rust}
use std::sync::{Arc, Barrier};
use std::thread;

fn main() {
    let mut vec = vec![];
    let barrier = Arc::new(Barrier::new(5));
    for id in 0..5 {
        let barrier_copy = barrier.clone();
        vec.push(thread::spawn(move || {
            info!("Thread {} Waiting the other threads...", id);
            // wait阻塞了所有的线程,当所有线程的wait之前部分全部执行完成之后
            // wait操作才算执行完成,才会执行每个线程后续的操作
            barrier_copy.wait();
            info!("{} After wait...", id);
        }));
    }

    for handler in vec {
        handler.join().unwrap();
    }
}
\end{code-block}

而条件变量与屏障稍微的区别在于,它不是阻塞所有的线程,而是在满足特定条件前,阻塞
一个得到了互斥锁的线程,如下:
\begin{code-block}{rust}
use std::sync::{Arc, Condvar, Mutex};
use std::thread;
use std::time::Duration;

fn main() {
    // 生成包含互斥锁的条件变量condvar
    let pair = Arc::new(((Mutex::new(false)), Condvar::new()));
    let pair_clone = pair.clone();

    let handler = thread::spawn(move || {
        let &(ref lock, ref cvar) = &*pair_clone;
        // 获得互斥锁
        let mut started = lock.lock().unwrap();
        info!("In the child thread");
        thread::sleep(Duration::from_secs(5));
        *started = true;
        // 通知主线程
        cvar.notify_one();
    });

    let &(ref lock, ref cvar) = &*pair;
    let mut started = lock.lock().unwrap();
    while !*started {
        info!("Waiting for the started singal {} ...", started);
        // 使用条件变量的wait阻塞当前线程,一直到cvar退出
        started = cvar.wait(started).unwrap();
        info!("Started singal finished {} ...", started);
    }
    handler.join().unwrap();
}
\end{code-block}
相比于单纯的互斥锁必须多次出入临界区才能获取到某个状态的信息,条件变量减少了系统
资源的浪费,但是需要注意,每个条件变量每次只能和一个互斥锁(体)一起使用。

除了使用锁、屏障以及条件变量,关于同步的问题,还可以使用原子操作。Rust目前只提供了
4个原子操作类型:AtomicBool、Atomiclsize、AtomicPtr和AtomicUsize。需要注意,虽然原子
操作类型本身可以保证操作的原子性,但是其本身并没有提供跨线程的共享方法,如果需要
使得原子数据类型也可以在线程间共享,则应当使用Arc进行封装,比如下面,使用原子类型
实现一个自旋锁:
\begin{code-block}{rust}
use std::sync::atomic::{AtomicUsize, Ordering};
use std::sync::Arc;

fn main() {
    let spinlock = Arc::new(AtomicUsize::new(1));
    let spinlock_clone = spinlock.clone();
    let handler = thread::spawn(move || {
        // 将原子类型的数据设置为0
        spinlock_clone.store(0, Ordering::SeqCst);
    });

    // 使用spinlock的load方法读取其内部原子类型的值,如果不为0,
    // 则不停地循环测试锁的状态,直到其状态被置为0为止
    // 所谓“自旋”就是指在语义上表示这种不断循环获取锁状态的行为
    while spinlock.load(Ordering::SeqCst) != 0 {}
    handler.join().unwrap();
}
\end{code-block}
代码当中的Ordering表示内存参数顺序,可以通过该参数来控制底层线程执行顺序。默认的,
Rust支持5种内存顺序,归为3大类:
\begin{itemize}
  \item 排序一致性顺序——SeqCst:最简单直观,要求必须先存储,后读取,在多线程环境下,所有的原子写操作都必须在读操作之前完成,强行指定了线程的执行顺序,保证了多线程中所有操作的全局一致性,但是存在性能损耗,其实质类似于餐厅点餐,相当于强制要求所有需要结账的客人,必须等所有点单的客户完成之后才可以结账
  \item 自由顺序——Relaxed:和SeqCst相反,完全不会对线程的顺序进行干涉,线程只进行原子操作,但是,线程之间会存在竞态条件,使用这种内存顺序会比较危险,只有在明确了解当前使用场景且必须使用它的情况下(比如只有读操作),才可使用自由顺序
  \item 获取-释放顺序——Release,Acquire和AcqRel: 是除排序一致性顺序之外的优先选择,默认情况下,不会对全部线程进行统一强制性的执行顺序要求,store表示释放(release),而load表示获取(acquire),通过这2种操作的协作实现线程同步。Release表示使用该顺序的store操作,之前所有的操作对于使用Acquire顺序的load操作都可见;反之,使用使用Aquire顺序的load操作,对于使用Release的store操作都是可见的;AcqRel表示读时使用Acquire顺序的load操作,写时使用Release顺序的store操作。获取释放顺序虽然不像排序一致性顺序那样对全局线程统一排序,但是它让每个线程都能接固定的顺序执行。
\end{itemize}

\subsection{常见错误处理方法}
由于很多代码都是第三方的,而Rust本身也在不断的发展,有可能出现版本不兼容或者特性
不兼容的情况,此时,则需要进行相关的修改。比如下面的一种错误:
\begin{figure}[H]
  \centering
  \includegraphics[width=\linewidth]{rust_feature_error.png}
  \caption{缺少特性支持编译失败}
  \label{fig:rust_feature_error}
\end{figure}
遇到这种错误,则需要直接修改对应的类库的源代码。以上述错误为例,编译的help表示
\mintinline[breaklines=true,breakanywhere,breaksymbolleft=,breakanywheresymbolpre=,]{bash}{add `#![feature(array_value_iter_slice)]` to the crate attributes to enable},
则我们应当在对应的crate的lib.rs的头部当中,添加内容如下:
\begin{figure}[H]
  \centering
  \includegraphics[width=\linewidth]{rust_feature_add.png}
  \caption{增加特性支持}
  \label{fig:rust_feature_add}
\end{figure}
