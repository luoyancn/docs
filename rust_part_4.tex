\subsection{高级函数式编程}
之前的函数式编程当中,提到了map函数,用于对数据进行处理,比如下面这种:
\begin{code-block}{rust}
let sum: u32 = c1
    .zip(c2.skip(10))
    .map(|(a, b)| a * b)
    .filter(|x| x % 3 == 0)
    .sum();
\end{code-block}

但是,实际使用当中,map还有更加广泛的用途,比如,在特定的情况下,替换match操作,
使得代码更加简单和精炼。比如,在使用match处理Option这种数据类型时,由于Option的
取值范围为Some和None,而map函数对于Option类型的处理,也恰好就是返回Some和None,
因此,可以直接使用map函数对这种Some对Some,None对None的简单映射关系进行处理,
多个不同的map进行组合,形成链式调用,相比而言,比match操作会更加简练:
\begin{code-block}{rust}
#[derive(Debug)]
enum Food {
    Apple,
    Potato,
}

#[derive(Debug)]
struct Peeled(Food);
#[derive(Debug)]
struct Chopped(Food);
#[derive(Debug)]
struct Cooked(Food);

// 常见的处理方法,使用match进行处理,并且返回一个Option
fn peel(food: Option<Food>) -> Option<Peeled> {
    match food {
        Some(food) => Some(Peeled(food)),
        None => None,
    }
}

// 使用map函数进行Option的简单映射
fn process(food: Option<Food>) -> Option<Cooked> {
    food.map(|f| Peeled(f))
        .map(|Peeled(f)| Chopped(f))
        .map(|Chopped(f)| Cooked(f))
}
\end{code-block}

然而,如果返回类型Option需要作为map函数的参数,输入到另外一个闭包或者函数当中,
则有可能出现Option<Option<T>>的结果出现,并不利于结果的解析,此时,则需要采用
and\_then进行处理,比如下方的代码:
\begin{code-block}{rust}
enum Food {
    CordonBleu,
    Steak,
    Sushi,
}

fn have_ingredients(food: Food) -> Option<Food> {
    match food {
        Food1::Sushi => None,
        _ => Some(food),
    }
}

fn have_recipe(food: Food) -> Option<Food> {
    match food {
        Food1::CordonBleu => None,
        _ => Some(food),
    }
}

// 通过map函数将上述2个函数进行连接起来,have_recipe当作一个闭包使用
// 但是,结果将变更为Option<Option<T>>
fn cookable_v1(food: Food) -> Option<Option<Food>> {
    have_ingredients(food).map(|res| have_recipe(res))
}

// 通过and_then将2个函数连接起来,形成链式调用
// have_ingredients返回的是一个Option,and_then会将其进行拆包
// 如果Option是None,则直接返回None;但是,如果是Some<T>,and_then则会将其
// 进行拆包,返回为T,而不是Some<T>
fn cookable_v2(food: Food) -> Option<Food> {
    have_ingredients(food).and_then(have_recipe)
}
\end{code-block}

Result和Option类似,但实质上,Option是Result的一个特化版本,可以将其简单的看作:
\begin{code-block}{rust}
type Option<T> = Result<T, ()>
\end{code-block}

因此,Option的map,and\_then等函数(算子)同样可以作用于Result上,比如下面的例子:
\begin{code-block}{rust}
use std::num::ParseIntError;

// 使用普通的match模式
fn multiply_v1(first_number_str: &str, second_number_str: &str) -> Result<i32, ParseIntError> {
    match first_number_str.parse::<i32>() {
        Ok(first_number)  => {
            match second_number_str.parse::<i32>() {
                Ok(second_number)  => {
                    Ok(first_number * second_number)
                },
                Err(e) => Err(e),
            }
        },
        Err(e) => Err(e),
    }
}

// 使用map与and_then模式
fn multiply_v2(first_number_str: &str, second_number_str: &str) -> Result<i32, ParseIntError> {
    // and_then将Result<T, E>拆分,如果是Err,直接返回,如果是T,即Ok(T)
    // 则进行解析为T
    first_number_str.parse::<i32>().and_then(|first_number| {
        second_number_str.parse::<i32>().map(|second_number| first_number * second_number)
    })
}
\end{code-block}

同样的,Result也可以使用别名系统,比如常见的io::Result,实际上就是Result的一个
别名特化版本:
\begin{code-block}{rust}
type Result<T> = Result<T, Error>;
\end{code-block}
因此,同样可以在代码当中使用Result的别名,对代码进行简化:
\begin{code-block}{rust}
use std::num::ParseIntError;

type AliasedResult<T> = Result<T, ParseIntError>;

fn multiply(first_number_str: &str, second_number_str: &str) -> AliasedResult<i32> {
    first_number_str.parse::<i32>().and_then(|first_number| {
        second_number_str
            .parse::<i32>()
            .map(|second_number| first_number * second_number)
    })
}

fn print(result: AliasedResult<i32>) {
    match result {
        Ok(n) => println!("n is {}", n),
        Err(e) => println!("Error: {}", e),
    }
}

fn main() {
    print(multiply("10", "2"));
    print(multiply("t", "2"));
}
\end{code-block}

由于Option和Result的特殊性,在一些特定的场合,尤其是处理错误的时候,常见的做法就是
混合Option和Result,进行混合类型的错误处理:
\begin{code-block}{rust}
use std::num::ParseIntError;

fn double_first(vec: Vec<&str>) -> Option<Result<i32, ParseIntError>> {
    // map返回Option,使用map包裹parse函数可能带来的错误信息(Result)
    vec.first().map(|first| first.parse::<i32>().map(|n| 2 * n))
}

fn double_first_v2(vec: Vec<&str>) -> Result<Option<i32>, ParseIntError> {
    let opt = vec.first().map(|first| first.parse::<i32>().map(|n| 2 * n));

    // map_or返回Result,其中,Ok子句处理opt为None的情况
    // r则处理opt为Some和Err的情况
    opt.map_or(Ok(None), |r| {
        println!("The r is error {:?}", r);
        r.map(Some)
    })
}

fn main() {
    let empty2 = vec![];

    match double_first_v2(empty2) {
        Ok(Some(x)) => println!("The result is {}", x),
        Err(e) => println!("Error is {:?}", e),
        Ok(None) => println!("None is in result"),
    }
}
\end{code-block}

\subsection{自定义错误}
Rust的错误是可以进行自行定义的,只需要实现一个Error Trait即可。Error Trait的定义
如下:
\begin{code-block}{rust}
pub trait Error: Debug + Display {
    fn source(&self) -> Option<&(dyn Error + 'static)> { ... }
    fn backtrace(&self) -> Option<&Backtrace> { ... }
    fn description(&self) -> &str { ... }
    fn cause(&self) -> Option<&dyn Error> { ... }
}
\end{code-block}
其中:
\begin{itemize}
  \item source是必须实现的函数,并且对应的错误必须实现Debug和Display Trait
  \item backtrace是只能在nightly分支当中实现的函数
  \item description被废弃,使用Display Trait或者to\_string(ToString Trait)替代
  \item cause同样被废弃,被source所取代
\end{itemize}

一个简单的例子如下:
\begin{code-block}{rust}
use std::error::Error;
use std::fmt;

// 定义自定义错误结构体
// 实现Debug Trait
#[derive(Debug)]
struct SuperError {
    msg: String,
}

// 实现Display Trait
impl fmt::Display for SuperError {
    fn fmt(&self, f: &mut fmt::Formatter) -> fmt::Result {
        write!(f, "Super Error: {}", self.msg)
    }
}

// 实现Error Trait
impl Error for SuperError {
    fn source(&self) -> Option<&(dyn Error + 'static)> {
        Some(self)
    }
}

impl SuperError {
    fn new(err: &str) -> SuperError {
        SuperError {
            msg: err.to_string(),
        }
    }
}

fn err_test() -> Result<(), SuperError> {
    Err(SuperError::new("first error"))
}

fn main() {
    match err_test() {
        // Err(SuperError{msg: e}) => println!("{}", e),
        Err(e) => println!("{}", e),
        _ => println!("no error"),
    }
}
\end{code-block}

错误和自定义错误解决的是对于错误的定义,以及对应错误的处理方式,但是,在实际的生产
使用当中,错误可能是普遍存在的,而我们需要的数据可能并不包含错误信息,而是需要
将错误从正确的结果当中剔除,比如:
\begin{code-block}{rust}
fn main() {
    let strings = vec!["tofu", "93", "18"];
    let possible_numbers: Vec<_> = strings.into_iter().map(|s| s.parse::<i32>()).collect();
    println!("Results: {:?}", possible_numbers);
}
\end{code-block}
我们的本意是将Vec当中的字符串全部格式化为数值,但是,实际的结果当中,却把包含的
错误也一同包含进来了,需要想办法将错误信息过滤掉:
\begin{code-block}{rust}
fn main() {
    let strings = vec!["tofu", "93", "18"];
    let numbers: Vec<_> = strings
        .into_iter()
        .map(|s| s.parse::<i32>())
        // filter_map进行过滤,只保留结果为ok的数据
        .filter_map(Result::ok)
        .collect();
    println!("Results: {:?}", numbers);
}
\end{code-block}

Result实现了FromIter,因此结果的向量(Vec<Result<T, E>>)可以被转换成结果包裹着
向量(Result<Vec<T>, E>)。一旦找到一个Result::Err,遍历就被终止,即满足另外一种
需求:只要任何一个错误发生,就中断当前的操作:
\begin{code-block}{rust}
fn main() {
    let strings = vec!["tofu", "93", "18"];
    // 注意numbers不再是Vec<_>,而是通过FromIter转换成了Result
    // 转换过程一旦失败,就会出现错误,中断当前的执行流程
    let numbers: Result<Vec<_>, _> = strings.into_iter().map(|s| s.parse::<i32>()).collect();
    println!("Results: {:?}", numbers);
}
\end{code-block}

但是,有的时候,我们也存在另外一种需求:将执行的正确和错误结果分类存放,以待后续
操作,此时则需要使用partition函数,对结果进行区分:
\begin{code-block}{rust}
fn main() {
    let strings = vec!["tofu", "93", "18"];
    let (numbers, errors): (Vec<_>, Vec<_>) = strings
        .into_iter()
        .map(|s| s.parse::<i32>())
        // 使用partition函数进行区分
        .partition(Result::is_ok);
    println!("Numbers: {:?}", numbers);
    println!("Errors: {:?}", errors);

    // 对后续的结果进行解构
    let numbers: Vec<_> = numbers.into_iter().map(Result::unwrap).collect();
    let errors: Vec<_> = errors.into_iter().map(Result::unwrap_err).collect();
    println!("Numbers: {:?}", numbers);
    println!("Errors: {:?}", errors);
}
\end{code-block}
