\subsection{高级函数式编程}
之前的函数式编程当中,提到了map函数,用于对数据进行处理,比如下面这种:
\begin{code-block}{rust}
let sum: u32 = c1
    .zip(c2.skip(10))
    .map(|(a, b)| a * b)
    .filter(|x| x % 3 == 0)
    .sum();
\end{code-block}

但是,实际使用当中,map还有更加广泛的用途,比如,在特定的情况下,替换match操作,
使得代码更加简单和精炼。比如,在使用match处理Option这种数据类型时,由于Option的
取值范围为Some和None,而map函数对于Option类型的处理,也恰好就是返回Some和None,
因此,可以直接使用map函数对这种Some对Some,None对None的简单映射关系进行处理,
多个不同的map进行组合,形成链式调用,相比而言,比match操作会更加简练:
\begin{code-block}{rust}
#[derive(Debug)]
enum Food {
    Apple,
    Potato,
}

#[derive(Debug)]
struct Peeled(Food);
#[derive(Debug)]
struct Chopped(Food);
#[derive(Debug)]
struct Cooked(Food);

// 常见的处理方法,使用match进行处理,并且返回一个Option
fn peel(food: Option<Food>) -> Option<Peeled> {
    match food {
        Some(food) => Some(Peeled(food)),
        None => None,
    }
}

// 使用map函数进行Option的简单映射
fn process(food: Option<Food>) -> Option<Cooked> {
    food.map(|f| Peeled(f))
        .map(|Peeled(f)| Chopped(f))
        .map(|Chopped(f)| Cooked(f))
}
\end{code-block}

然而,如果返回类型Option需要作为map函数的参数,输入到另外一个闭包或者函数当中,
则有可能出现Option<Option<T>>的结果出现,并不利于结果的解析,此时,则需要采用
and\_then进行处理,比如下方的代码:
\begin{code-block}{rust}
enum Food {
    CordonBleu,
    Steak,
    Sushi,
}

fn have_ingredients(food: Food) -> Option<Food> {
    match food {
        Food1::Sushi => None,
        _ => Some(food),
    }
}

fn have_recipe(food: Food) -> Option<Food> {
    match food {
        Food1::CordonBleu => None,
        _ => Some(food),
    }
}

// 通过map函数将上述2个函数进行连接起来,have_recipe当作一个闭包使用
// 但是,结果将变更为Option<Option<T>>
fn cookable_v1(food: Food) -> Option<Option<Food>> {
    have_ingredients(food).map(|res| have_recipe(res))
}

// 通过and_then将2个函数连接起来,形成链式调用
// have_ingredients返回的是一个Option,and_then会将其进行拆包
// 如果Option是None,则直接返回None;但是,如果是Some<T>,and_then则会将其
// 进行拆包,返回为T,而不是Some<T>
fn cookable_v2(food: Food) -> Option<Food> {
    have_ingredients(food).and_then(have_recipe)
}
\end{code-block}

Result和Option类似,但实质上,Option是Result的一个特化版本,可以将其简单的看作:
\begin{code-block}{rust}
type Option<T> = Result<T, ()>
\end{code-block}

因此,Option的map,and\_then等函数(算子)同样可以作用于Result上,比如下面的例子:
\begin{code-block}{rust}
use std::num::ParseIntError;

// 使用普通的match模式
fn multiply_v1(first_number_str: &str, second_number_str: &str) -> Result<i32, ParseIntError> {
    match first_number_str.parse::<i32>() {
        Ok(first_number)  => {
            match second_number_str.parse::<i32>() {
                Ok(second_number)  => {
                    Ok(first_number * second_number)
                },
                Err(e) => Err(e),
            }
        },
        Err(e) => Err(e),
    }
}

// 使用map与and_then模式
fn multiply_v2(first_number_str: &str, second_number_str: &str) -> Result<i32, ParseIntError> {
    // and_then将Result<T, E>拆分,如果是Err,直接返回,如果是T,即Ok(T)
    // 则进行解析为T
    first_number_str.parse::<i32>().and_then(|first_number| {
        second_number_str.parse::<i32>().map(|second_number| first_number * second_number)
    })
}
\end{code-block}

同样的,Result也可以使用别名系统,比如常见的io::Result,实际上就是Result的一个
别名特化版本:
\begin{code-block}{rust}
type Result<T> = Result<T, Error>;
\end{code-block}
因此,同样可以在代码当中使用Result的别名,对代码进行简化:
\begin{code-block}{rust}
use std::num::ParseIntError;

type AliasedResult<T> = Result<T, ParseIntError>;

fn multiply(first_number_str: &str, second_number_str: &str) -> AliasedResult<i32> {
    first_number_str.parse::<i32>().and_then(|first_number| {
        second_number_str
            .parse::<i32>()
            .map(|second_number| first_number * second_number)
    })
}

fn print(result: AliasedResult<i32>) {
    match result {
        Ok(n) => println!("n is {}", n),
        Err(e) => println!("Error: {}", e),
    }
}

fn main() {
    print(multiply("10", "2"));
    print(multiply("t", "2"));
}
\end{code-block}

由于Option和Result的特殊性,在一些特定的场合,尤其是处理错误的时候,常见的做法就是
混合Option和Result,进行混合类型的错误处理:
\begin{code-block}{rust}
use std::num::ParseIntError;

fn double_first(vec: Vec<&str>) -> Option<Result<i32, ParseIntError>> {
    // map返回Option,使用map包裹parse函数可能带来的错误信息(Result)
    vec.first().map(|first| first.parse::<i32>().map(|n| 2 * n))
}

fn double_first_v2(vec: Vec<&str>) -> Result<Option<i32>, ParseIntError> {
    let opt = vec.first().map(|first| first.parse::<i32>().map(|n| 2 * n));

    // map_or返回Result,其中,Ok子句处理opt为None的情况
    // r则处理opt为Some和Err的情况
    opt.map_or(Ok(None), |r| {
        println!("The r is error {:?}", r);
        r.map(Some)
    })
}

fn main() {
    let empty2 = vec![];

    match double_first_v2(empty2) {
        Ok(Some(x)) => println!("The result is {}", x),
        Err(e) => println!("Error is {:?}", e),
        Ok(None) => println!("None is in result"),
    }
}
\end{code-block}

\subsection{自定义错误}
Rust的错误是可以进行自行定义的,只需要实现一个Error Trait即可。Error Trait的定义
如下:
\begin{code-block}{rust}
pub trait Error: Debug + Display {
    fn source(&self) -> Option<&(dyn Error + 'static)> { ... }
    fn backtrace(&self) -> Option<&Backtrace> { ... }
    fn description(&self) -> &str { ... }
    fn cause(&self) -> Option<&dyn Error> { ... }
}
\end{code-block}
其中:
\begin{itemize}
  \item source是必须实现的函数,并且对应的错误必须实现Debug和Display Trait
  \item backtrace是只能在nightly分支当中实现的函数
  \item description被废弃,使用Display Trait或者to\_string(ToString Trait)替代
  \item cause同样被废弃,被source所取代
\end{itemize}

一个简单的例子如下:
\begin{code-block}{rust}
use std::error::Error;
use std::fmt;

// 定义自定义错误结构体
// 实现Debug Trait
#[derive(Debug)]
struct SuperError {
    msg: String,
}

// 实现Display Trait
impl fmt::Display for SuperError {
    fn fmt(&self, f: &mut fmt::Formatter) -> fmt::Result {
        write!(f, "Super Error: {}", self.msg)
    }
}

// 实现Error Trait
impl Error for SuperError {
    fn source(&self) -> Option<&(dyn Error + 'static)> {
        Some(self)
    }
}

impl SuperError {
    fn new(err: &str) -> SuperError {
        SuperError {
            msg: err.to_string(),
        }
    }
}

fn err_test() -> Result<(), SuperError> {
    Err(SuperError::new("first error"))
}

fn main() {
    match err_test() {
        // Err(SuperError{msg: e}) => println!("{}", e),
        Err(e) => println!("{}", e),
        _ => println!("no error"),
    }
}
\end{code-block}

错误和自定义错误解决的是对于错误的定义,以及对应错误的处理方式,但是,在实际的生产
使用当中,错误可能是普遍存在的,而我们需要的数据可能并不包含错误信息,而是需要
将错误从正确的结果当中剔除,比如:
\begin{code-block}{rust}
fn main() {
    let strings = vec!["tofu", "93", "18"];
    let possible_numbers: Vec<_> = strings.into_iter().map(|s| s.parse::<i32>()).collect();
    println!("Results: {:?}", possible_numbers);
}
\end{code-block}
我们的本意是将Vec当中的字符串全部格式化为数值,但是,实际的结果当中,却把包含的
错误也一同包含进来了,需要想办法将错误信息过滤掉:
\begin{code-block}{rust}
fn main() {
    let strings = vec!["tofu", "93", "18"];
    let numbers: Vec<_> = strings
        .into_iter()
        .map(|s| s.parse::<i32>())
        // filter_map进行过滤,只保留结果为ok的数据
        .filter_map(Result::ok)
        .collect();
    println!("Results: {:?}", numbers);
}
\end{code-block}

Result实现了FromIter,因此结果的向量(Vec<Result<T, E>>)可以被转换成结果包裹着
向量(Result<Vec<T>, E>)。一旦找到一个Result::Err,遍历就被终止,即满足另外一种
需求:只要任何一个错误发生,就中断当前的操作:
\begin{code-block}{rust}
fn main() {
    let strings = vec!["tofu", "93", "18"];
    // 注意numbers不再是Vec<_>,而是通过FromIter转换成了Result
    // 转换过程一旦失败,就会出现错误,中断当前的执行流程
    let numbers: Result<Vec<_>, _> = strings.into_iter().map(|s| s.parse::<i32>()).collect();
    println!("Results: {:?}", numbers);
}
\end{code-block}

但是,有的时候,我们也存在另外一种需求:将执行的正确和错误结果分类存放,以待后续
操作,此时则需要使用partition函数,对结果进行区分:
\begin{code-block}{rust}
fn main() {
    let strings = vec!["tofu", "93", "18"];
    let (numbers, errors): (Vec<_>, Vec<_>) = strings
        .into_iter()
        .map(|s| s.parse::<i32>())
        // 使用partition函数进行区分
        .partition(Result::is_ok);
    println!("Numbers: {:?}", numbers);
    println!("Errors: {:?}", errors);

    // 对后续的结果进行解构
    let numbers: Vec<_> = numbers.into_iter().map(Result::unwrap).collect();
    let errors: Vec<_> = errors.into_iter().map(Result::unwrap_err).collect();
    println!("Numbers: {:?}", numbers);
    println!("Errors: {:?}", errors);
}
\end{code-block}

\section{元编程}
Rust也包含了宏,并且,和C/C++相比,Rust的宏会展开成为抽象语法树(AST,abstract syntax tree),
而不是普通的转换成字符串,因此,不会产生无法预料的优先权错误。Rust的宏包括声明宏以及过程宏。
\subsection{声明宏}
常见的Rust宏大部分都是声明宏,最普通的宏如下:
\begin{code-block}{rust}
extern crate slog_scope;
extern crate slog_stdlog;
#[macro_use]
extern crate log;
extern crate logger;
// macro_rules! 表示后续的内容是一个宏
// greeting表示宏的名称
macro_rules! greeting {
    // () 表示该宏不接收任何参数
    () => {
        // 宏定义展开的具体内容
        info!("hello macro");
    };
}
fn main() {
    let logger = logger::initlogger(false, "", 0);
    let _guard = slog_scope::set_global_logger(logger);
    slog_stdlog::init().unwrap();
    greeting!();
}
\end{code-block}
但是,宏不可能一直是无参数的,它还包含了多种使用方式。宏的参数使用\$符号表示,并
使用指示符来注明类型,如下:
\begin{code-block}{rust}
macro_rules! create_function {
    // 宏接收一个ident指示符表示的参数,并创建一个func_name的函数
    // ident指示符表示变量名(函数名)
    ($func_name: ident) => {
        fn $func_name() {
            // stringify宏负责将ident指示符表示的参数转换成字符串
            info!("You called the {}()", stringify!($func_name));
        }
    };
}
// 使用宏创建函数,函数名为func
create_function!(func);
macro_rules! formatres {
    // 宏接收一个expr指示符表示的表达式(可以是代码块,函数/方法,其他宏)
    // expr指示符表示表达式
    // $expression表示表达式最后的执行结果
    ($expression: expr) => {
        info!("{} = {}", stringify!($expression), $expression)
    };
}
fn main() {
    // 调用func函数
    func();
    formatres!(1 + 32);
    formatres!("lucifer");
    formatres!(format!("{}, age is {}", "zhangjl", 32));
}
\end{code-block}
宏的指示符有很多,各自用于不同的场景,所有的宏指示符如下:
\begin{itemize}
  \item block:代码块,由{}限定的代码
  \item expr:表达式,会生成具体的值
  \item ident:变量名/函数名,标识符
  \item item:语言项,即组成一个Rust包的基本单位,如模块,声明,函数/类型/结构体/impl定义
  \item pat:模式
  \item path:路径,类似std::iter等
  \item stmt:语句,一般以;结尾的代码
  \item tt:标记树
  \item ty:类型
  \item meta:元数据信息,即包含在\#[...]以及\#![...]当中的信息
  \item vis:可见性,如pub
  \item lifetime:指代生命周期参数
\end{itemize}

类似于方法,Rust的宏也可以进行重载,只不过,这个重载的实现比较类似于match的分支
处理流程,分割宏的分支即进行重载,则需要使用符号“:”进行:
\begin{code-block}{rust}
macro_rules! assert_bool {
    // 括号中的分号;表示调用该宏时,需要传递2条语句或者表达式
    ($left: expr; and $right: expr) => {
        info!(
            "{} and {} is {}",
            stringify!($left),
            stringify!($right),
            $left && $right
        )
    };
    // 分支之间需要使用分号;进行分割与结束
    ($left: expr; or $right: expr) => {
        info!(
            "{} or {} is {}",
            stringify!($left),
            stringify!($right),
            $left || $right
        )
    };
}
fn main() {
    assert_bool!(1 + 1 == 2; and 2 * 2 == 4 );
    assert_bool!(1 + 1 == 3; or 2 * 2 == 6 );
}
\end{code-block}

宏定义的另外一个好处就是可以处理不定参数,在处理不定参数时,需要使用+操作符以及*
操作符,+表示参数可能出现一次或多次,*则表示参数可能出现0次或多次:
\begin{code-block}{rust}
use std::cmp;
macro_rules! find_min {
    // 如果传入的只有一个参数,直接返回当前参数值
    ($x: expr) => {
        $x
    };
    // 传入多个参数,表示后续更多的参数,即x后至少还有一个参数
    ($x: expr, $($y: expr), +) => {
        // 递归调用宏本身
        cmp::min($x, find_min!($($y), +))
    };
}
fn main() {
    info!("{}", find_min!(12));
    info!("{}", find_min!(12, 65, 40 - 32));
    let a = 1;
    let b = 2;
    let c = 3;
    info!("{}", find_min!(a, a - b, c));
}
\end{code-block}
上述的宏是使用表达式模式进行的,如果采用变量模式,即使用ident模式,则上述代码需要
变更为如下:
\begin{code-block}{rust}
use std::cmp;
macro_rules! find_min {
    ($x: ident) => {
        $x
    };
    ($x: ident, $($y: ident), +) => {
        cmp::min($x, find_min!($($y), +))
    };
}
fn main() {
    // 错误的使用方式,12是一个表达式,而并非变量名
    // info!("{}", find_min!(12));
    let a = 1;
    let b = 2;
    let c = 3;
    info!("{}", find_min!(a, b, c));
}
\end{code-block}
通过对比,可以发现,在某些场景下,表达式方式比ident方式更加通用,也更加合理一些。

比较奇特的是,在Rust的宏当中,可以使用自定义的关键字,实现特殊功能,比如自定义
关键字evaluation,表示将表达式进行计算:
\begin{code-block}{rust}
macro_rules! calc {
    // 自定义关键字evalution,使用该宏时,前面必须加上evalution前缀关键词
    (evalution $e: expr) => {
        // 强制将表达式e变成数值i32类型,即将表达式e进行计算
        let val: i32 = $e;
        info!("{} = {}", stringify!($e), val);
    };
    // 当传入参数不定时
    (evalution $e: expr, $(evalution $es: expr),+) => {
        calc!(evalution $e);
        calc!($(evalution $es),+)
    }
}
fn main() {
    calc!(evalution 1 + 100);
    calc!(evalution 1+2, evalution 3 + 4, evalution 5 +6 );
    calc!(evalution 1-2, evalution 3 * 4, evalution (5 +6) * (5 - 9) );
}
\end{code-block}
由于宏的高度可定制性,因此,上述的宏代码可以变更为如下的模式,但是2者的功能完全
一样:
\begin{code-block}{rust}
macro_rules! calc {
    (evalution $e:expr) => {{
        let val: i32 = $e;
        info!("{} = {}", stringify! {$e}, val);
    }};
    (evalution $e:expr, $(evalution $es:expr),+) => {{
        calc! { evalution $e }
        calc! { $(evalution $es),+ }
    }};
}
fn main() {
    // 下面两种方式都正确
    calc!{evalution 1 + 100};
    calc!(evalution 1+2, evalution 3 + 4, evalution 5 +6 );
}
\end{code-block}

实际上,*和+不仅可以用于参数处理,也可以用于语法扩展的部分,比如,我们想实现类似
如下的一个宏:
\begin{code-block}{rust}
let empty = hashmap![];
let counts = hashmap!['A' => 0, 'C' => 0, 'G' => 0, 'T' => 0];
\end{code-block}
则宏定义大致可能如下:
\begin{code-block}{rust}
macro_rules! hashmap {
    ($key: expr => $val: expr) => {{
        let mut map = ::std::collections::HashMap::new();
        map.insert($key, $val);
        map
    }};
}
\end{code-block}
但是,到目前为止,上述的宏只能实现对一对数据的操作,无法实现任意对数据的插入操作,
因此,我们需要使用+或者*符号进行扩展,由于我们需要支持初始化一个空的hashmap,因此
选择使用*进行扩展:
\begin{code-block}{rust}
macro_rules! hashmap {
    ($key: expr => $val: expr) => {{
        let mut map = ::std::collections::HashMap::new();
        map.insert($key, $val);
        map
    }};
}
\end{code-block}
虽然参数支持了任意个数,但是,在宏体当中,map的插入操作只执行了一次,我们可以继续
使用*和+对语法部分进行扩展,扩展之后,完整的宏定义如下:
\begin{code-block}{rust}
macro_rules! hashmap {
    ($($key: expr => $val: expr), *) => {{
        let mut map = HashMap::new();
        $(map.insert($key, $val); )*
        map
    }};
}
\end{code-block}
使用时,则按照上述的使用方式即可:
\begin{code-block}{rust}
let map = hashmap!["lucifer" => 12, "titans" => 18];
let mut empty: HashMap<String, u8> = hashmap![];
\end{code-block}
默认情况下,cargo并没有提供将宏定义进行展开显示的功能,但是,我们可以通过rustc
将代码展开,确认宏定义确实是按照我们的想法进行工作的。只是需要注意,将宏定义进行
展开显示,需要使用nightly分支,因此,我们的操作基本如下:
\begin{code-block}{bash}
# 切换到nightly分支
rustup default nightly
# 对代码进行展开
rustc -Z unstable-options --pretty=expanded src/main.rs
cargo rustc -- -Z unstable-options --pretty=expanded
# 如果代码需要依赖其他的非std的crate的,则应当如下执行
# rustc -Z unstable-options --pretty=expanded -L ../target/debug/deps src/main.rs
\end{code-block}
如果一切正常,则我们调用宏的代码就会被展开成如下的形式:
\begin{figure}[H]
  \centering
  \includegraphics[width=\linewidth]{rust_macro_expand.png}
  \caption{宏展开}
  \label{fig:rust_macro_expand}
\end{figure}

Rust宏的灵活性非常大,可以像C/C++一样,在宏当中嵌套/调用宏:
\begin{code-block}{rust}
macro_rules! serial_cmd {
    ($expression: expr, $port: expr, $item: expr, $timeout: expr) => {{
        let mut cmd = HEADER.to_vec();
        cmd.push($item);
        cmd.push($expression);
        match ($port).write(&cmd) {
            Ok(_) => info!(
                "Sucess {}(0x{:X>02}) the {} board, and command is {:?}",
                stringify!($expression),
                $expression,
                $item,
                &cmd
            ),
            Err(e) => error!(
                "Failed to {}(0x{:X>02}) the {} board: {}",
                stringify!($expression),
                $expression,
                $item,
                e
            ),
        }
        if 0 < $timeout {
            thread::sleep(Duration::from_secs($timeout));
        }
    }};
}
macro_rules! serial_for_all_cmd {
    ($expression: expr, $port: expr, $timeout: expr) => {{
        for item in &CODE {
            serial_cmd!($expression, $port, *item, $timeout);
        }
    }};
}
\end{code-block}
也可以直接在宏当中,插入语句块,作为宏执行的一部分:
\begin{code-block}{rust}
macro_rules! serial_for_only_one {
    ($port: expr, $location: expr, $($command: stmt),*) => {{
        serial_cmd!(POWEROFF, $port, $location, 0);
        trace!("Remove the block file to avoid the unexcepted error ...");
        let _ = fs::remove_file("/dev/sdb");
        serial_cmd!(SWITCH, $port, $location, 5);
        // 执行外部代码块
        $($command )*
        serial_cmd!(POWEROFF, $port, $location, 0);
        let _ = fs::remove_file("/dev/sdb");
    }};
}
fn main() {
    // 调用宏
    serial_for_only_one!(port, location_u8, {println!("hello")});
}
\end{code-block}

\subsection{宏导出}
除了在当前的crate当中使用宏之外,
宏还可以导出,宏之间也可以相互调用。宏的导出通常使用macro\_export关键字,比如:
\begin{code-block}{rust}
#[macro_export]
macro_rules! inc {
    ($x: expr) => {
       println!("{}", $x);
    };
}
\end{code-block}
然后,在其他地方,就可以直接使用这个宏。不过,有的时候,宏的实现可能需要当前包的
一些函数或者方法进行配合,则需要做如下的更改:
\begin{code-block}{rust}
// 必须将方法设置为pub,否则后续在宏定义当中,无法使用
pub fn incr(x: u32) -> u32 {
    x + 1
}
#[macro_export]
macro_rules! inc {
    ($x: expr) => {
        // $crate关键字表示当前的包
        // 当宏被导出时,自动根据上下文选择函数调用路径当中的包名
        $crate::incr($x)
    };
}
\end{code-block}
上述的导出方式,要求宏所依赖的函数,也都必须导出,否则,在外部使用宏时,无法
正常工作。

除了使用普通的函数作为宏的依赖项之外,也可以使用宏作为宏的依赖项。和普通函数一样,
如果一个宏的定义当中,依赖了另外一个宏,则必须同样当对应的依赖项导出为pub类型。
但是,如果可以使用一种额外的方式,将依赖的宏,转变为宏的内部规则进行导出:
\begin{code-block}{rust}
#[macro_export]
macro_rules! hashmap {
    /* hashmap宏的内部规则,相当于如下的一个外部宏,不管接收多少参数,一律返回
       一个空元组()
       macro_rules! unit {
       ($($input:tt),*) => {
                ()
           };
       }
       使用方式
       let res = unit!(), unit!("lucifer"), unit!("garuda", "titans")
    */
    (@unit $($x:tt)*) => (());
    /* hashmap宏的内部规则, 等价于如下的一个宏,作用是返回接收到的元素的个数
       macro_rules! count {
           // <[()]>::len()可以用于求取数组/切片的长度,使用方式如下:
           // let lenth = <[&str]>::len(&["string", "string"])
           // let lenth = <[String]>::len(&["string".to_string(), "string".to_string()])
           // let lenth = <[()]>::len(&[(), ()]) // 性能更好,因为()不占据任何内存空间
           ($($key:expr),*) => (<[()]>::len(&[$(unit! ($key)),*]));
       }
       使用方式
       let res = count!(), count!("lucifer"), count!("lucifer", "titans")
       @符号表示一个宏定义当中的内部规则,如果需要在宏当中使用宏的内部规则,
       则使用方式是 宏名!(@内部规则名 其他变量),对应到这个hashmap宏,则使用方式
       如下: hashmap!(@unit $key), hashmap!(@count $($rest),*)
    */
    (@count $($rest:expr), *) => (<[()]>::len(&[$(hashmap!(@unit $rest)),*]));
    /* $($key:expr => $value:expr),* 表达式本身可以匹配hashmap!(),hashmap!("1"=>2)
     但是,无法匹配类似hashmap!["2"=>3,]这种末尾包含,符号的模式
     $(,)* 则是用于匹配后续结尾是否带有,符号
     即hashmap!["2"=>3,]和hashmap!["2"=>3]都可以支持
    */
    ($($key:expr => $value:expr),* $(,)*) => {{
        let _cap = hashmap!(@count $($key),*);
        let mut _map = ::std::collections::HashMap::with_capacity(_cap);
        $( _map.insert($key, $value); )*
        _map
    }}
}
\end{code-block}

\subsection{过程宏}
以上提到的宏,都是声明宏,可以直接当作函数/方法使用的类型,但是,如果想实现类似于
\#[derive(Debug)]这种类型的宏,声明宏是做不到的。相对应的,这种类型的宏则被称之
为过程宏。过程宏主要用于下面3种用途:
\begin{itemize}
  \item 自定义派生属性:即类似于\#[derive(Debug)]这样的derive属性
  \item 自定义属性:即类似于实现\#[cfg()]这样的属性
  \item Bang宏:与声明宏类似,但是,是以!结尾的宏,可以当作函数/方法使用
\end{itemize}

过程宏要求必须放到proc\_macro类型的lib包当中,因此,过程宏的创建过程会稍微有一些
区别:
\begin{code-block}{bash}
cargo new --lib procmacro
echo -e "[lib]\nproc_macro=true" >> procmacro/Cargo.toml
\end{code-block}

另外,和其他的mod不太一样的是,过程宏的测试用例,不能放到相同的crate当中,必须以
外部的方式存在,因此,过程宏的文件结构大致如下:
\begin{code-block}{bash}
├── Cargo.toml
├── src
│   └── lib.rs
└── tests
    └── test.rs
\end{code-block}

实现derive方式的过程宏,其示例如下:
\begin{code-block}{rust}
// 必须如此进行使用
extern crate proc_macro;
use self::proc_macro::TokenStream;
#[proc_macro_derive(A)]
pub fn derive(input: TokenStream) -> TokenStream {
    let input = input.to_string();
    assert!(input.contains("struct A"));
    r#"
        impl A {
            pub fn a(&self) -> String {
                format!("Hello from impl A")
            }
        }
    "#
    .parse()
    .unwrap()
}
\end{code-block}
上述过程宏表示,使用\#[derive(A)]为结构体A实现一个a方法,方法直接输出一句话。相对应的,
测试用例当中的使用则应当修改如下:
\begin{code-block}{rust}
#[macro_use]
extern crate procmacro;
#[derive(A)]
struct A;
#[test]
fn test_derive_a() {
    assert_eq!("Hello from impl A", A.a());
}
\end{code-block}

而实现自定义属性宏稍微有些区别,就是必须在nightly的rust下编译,目前还没有进入到
stable分支,一个简单的示例如下:
\begin{code-block}{rust}
#![feature(register_attr)]
extern crate proc_macro;
use self::proc_macro::TokenStream;
#[proc_macro_attribute]
pub fn attr_with_args(args: TokenStream, _: TokenStream) -> TokenStream {
    let args = args.to_string();
    //let input = input.to_string();
    format!("fn foo() -> &'static str {{{}}}", args)
        .parse()
        .unwrap()
}
\end{code-block}
同样的,其测试用例如下:
\begin{code-block}{rust}
#![feature(register_attr)]
#[macro_use]
extern crate procmacro;
use procmacro::attr_with_args;
#[attr_with_args("Hello Rust")]
fn foo() {}
#[test]
fn test_foo() {
    assert_eq!("Hello Rust", foo());
}
\end{code-block}
原本的foo方法,不接收参数,同样没有返回值,但是,在attr\_with\_args这个过程宏
当中,将其强行修改为了一个返回为字符串切片的函数。

实现Bang宏的方式则如下:
\begin{code-block}{rust}
#![feature(proc_macro_hygiene)]
extern crate proc_macro;
use self::proc_macro::TokenStream;
#[proc_macro]
pub fn treemap(input: TokenStream) -> TokenStream {
    let input = input.to_string();
    let input = input.trim_end_matches(',');
    let input_v: Vec<String> = input
        .split(",")
        .map(|n| {
            let mut data = if n.contains(":") {
                n.split(":")
            } else {
                n.split("=>")
            };
            let (key, value) = (data.next().unwrap(), data.next().unwrap());
            format!("hm.insert({}, {})", key, value)
        })
        .collect();
    let count: usize = input.len();
    let token = format!(
        "{{
        let mut hm = ::std::collections::HashMap::with_capacity({});
        {}
        hm
    }}",
        count,
        input_v
            .iter()
            .map(|n| format!("{};", n))
            .collect::<String>()
    );
    token.parse().unwrap()
}
\end{code-block}

Bang宏可以如同声明宏一样的进行使用,其使用方式如下:
\begin{code-block}{rust}
#[macro_use]
extern crate procmacro;
#[test]
fn test_treemap() {
    let hm = treemap! {"a":1, "b": 2};
    assert_eq!(hm["a"], 1);
    let hm = treemap! {"a" => 1, "b" => 4};
    assert_eq!(hm["b"], 4);
}
\end{code-block}

过程宏的本质是在函数/方法当中,使用TokenStream重构,本质还是一个特殊的函数/方法。
因此,过程宏不需要像声明宏一样的进行export,但是,必须将过程宏的函数声明为pub,
生成的过程宏才可以被外部使用。

\subsection{语法树}
编写真正可用的过程宏实际上比上面的例子要复杂很多,但不管如何变化,Rust的宏都是依赖于
语法树结构的,而过程宏的实现方式/过程,就是对解析的语法树进行处理的过程。关于语法树
的解析和读取,通常采用的是第三方的Rust Crate进行操作,目前比较常用的是\href{https://github.com/dtolnay/quote}{Quote},
\href{https://github.com/dtolnay/syn}{Syn}以及\href{https://github.com/alexcrichton/proc-macro2}{Proc-macro2}。
在编写真正的过程宏时,通常都需要上述3个crate的协助,需要在Cargo.toml当中添加如下的内容:
\begin{code-block}{toml}
[dependencies]
quote = "1.0.9"
syn = {version = "1.0.72", features = ["full", "extra-traits", "visit"]}
proc-macro2 = "1.0.26"
[lib]
proc-macro = true
\end{code-block}
另外,当lib当中设置\codeinlinebg{toml}{proc-macro=true}之后,
则对应的crate只能导出过程宏,不能导出其他的类型数据。

所有的编程语言都离不开词法分析,Rust同样如此。在Rust当中,通常使用TokenStream进行词法分析,
解析代码内容,编写过程宏离不开对TokenStream的解析。在调试过程宏的时候,由于标准输出
不可用,因此通常只能通过标准错误输出进行信息的打印。通常情况下,都是使用\codeinlinebg{rust}{eprint!}
或者\codeinlinebg{rust}{eprintln!}进行过程宏的调试输出。
一个简单的过程宏示例如下,当然,由于我们进行了输出,也可以看到Rust的语法树的大致结构:
\begin{code-block}{rust}
use proc_macro::TokenStream;
#[proc_macro_attribute]
pub fn test_proc_macro(attr: TokenStream, item: TokenStream) -> TokenStream {
    eprintln!("{:#?}", attr);
    eprintln!("{:#?}", item);
    item
}
\end{code-block}
调用的时候,需要在其他的crate当中引入这个crate:
\begin{code-block}{rust}
use procmacros::test_proc_macro;
fn main() {
    ...
}
#[test_proc_macro("lucifer")]
fn hello() {
    info!("hello");
}
\end{code-block}

默认情况下,代码需要经过编译,才能判断是否存在问题,不过,rust提供了1种简便的思路
来检测代码是否存在问题:\codeinlinebg{bash}{cargo check}。
该指令不会对代码进行实质的编译动作,但是会对过程宏进行预处理(即将其转变成正常的Rust代码),
因此,会得到类似如下的一些输出:
\begin{figure}[H]
  \centering
  \includegraphics[width=\linewidth]{rust_cargo_check.png}
  \caption{代码检测与预处理}
  \label{fig:rust_cargo_check}
\end{figure}

除了使用上述指令之外,也可以采用宏展开的方式,但是默认的宏展开方式需要使用nightly
分支,第三方工具cargo-expand则可以支持在stable分支直接展开。
但是,需要注意expand指令只能在源码(即rs文件)所在路径执行:\codeinlinebg{bash}{cargo expand}。
然后会得到类似如下的输出:
\begin{figure}[H]
  \centering
  \includegraphics[width=\linewidth]{rust_cargo_expand.png}
  \caption{宏代码展开}
  \label{fig:rust_cargo_expand}
\end{figure}
通过expand指令,可以将代码当中的宏代码全部转换为正常的Rust代码,从而方便进行阅读
识别和调试修改。

在编译混合有过程宏的Rust代码时,其基本流程是先展开过程宏,将其处理成普通的Rust代码,
然后再合并这些代码,最后再进行编译。从\colorunderlineref{fig:rust_cargo_check}所示当中,
可以看到有很多的特殊的标记,这些标记共同组成了Rust的抽象语法树结构:
\begin{itemize}
  \item Ident:标识符
  \item span:表示对应的元素在代码当中出现的位置(字节顺序)
  \item Group:组,表示语法树的组别
  \item delimiter:分隔符
  \item stream:表示每一组的内容(TokenStream)
  \item punct:标点符号
  \item literal:字符字面量
\end{itemize}

TokenStream只是一系列符号的组合,与语义无关,因此,如果将之前代码的过程宏调用填充
入无意义的数据,过程宏的处理同样不会有什么问题:
\begin{code-block}{rust}
#[test_proc_macro(!&)@)(*&$9)]
fn hello() {
    info!("hello");
}
\end{code-block}
上述代码在编译阶段的结果输出大致如下:
\begin{figure}[H]
  \centering
  \includegraphics[width=\linewidth]{rust_token_stream.png}
  \caption{TokenStream}
  \label{fig:rust_token_stream}
\end{figure}
从上述结果可以看到,TokenStream还是一种比较低级的处理形式,如果手工写一个TokenStream,
极易出现错误,因此需要使用上文提到的syn和quote,将TokenStream转换成具有语义信息
抽象度更高的数据结构:抽象语法树。将上述的过程宏代码改写如下:
\begin{code-block}{rust}
use proc_macro::TokenStream;
use proc_macro2;
use quote::quote;
use syn::{self, parse_macro_input, spanned::Spanned, AttributeArgs, Item};
// 标记该过程宏为属性模式
#[proc_macro_attribute]
pub fn test_proc_macro_ast(attr: TokenStream, item: TokenStream) -> TokenStream {
    // 将属性转换成语法数进行输出,即#[test_proc_macro(!&)@)(*&$9)]这部分代码
    eprintln!("{:#?}", parse_macro_input!(attr as AttributeArgs));
    // 将真正的代码转换成语法树输出,即被#[...]所修饰的代码
    let body_ast = parse_macro_input!(item as Item);
    eprintln!("{:#?}", body_ast);
    // 将语法树转换成TokenStream返回给编译器
    // quote返回的并不是TokenStream,而是proc_macro2::TokenStream类型,必须转换
    // #body_ast并不是Rust的合法语法,而是quote的自定义语法格式
    quote!(#body_ast).into()
}
\end{code-block}
如果对上述代码进行check,则会发现其输出结果大致如下:
\begin{figure}[H]
  \centering
  \includegraphics[width=\linewidth]{rust_ast.png}
  \caption{抽象语法树}
  \label{fig:rust_ast}
\end{figure}
注意,此时的输出就不再是纯粹的TokenStream(无语义)了,而是带有语义分析的抽象语法树
结构。抽象语法树可以完整的检查代码当中的语法逻辑问题,因此,如果像之前的代码,
使用不符合语法的方式调用这个过程宏,则代码的预编译阶段就无法通过,编译器会直接提示错误。
而这个错误,则是由\codeinlinebg{rust}{parse_macro_input!}这个
宏提示出来的。

\subsection{过程宏案例-派生过程宏Builder}
\label{builder}
假设当前有一个结构体如下:
\begin{code-block}{rust}
#[derive(Builder)]
pub struct Command {
    executable: String,
    #[builder(each = "arg")]
    args: Vec<String>,
    current_dir: Option<String>,
}
\end{code-block}
在该结构体上应用一个名为Builder的过程宏,使得在编译阶段最终生成的代码如下:
\begin{code-block}{rust}
pub struct Command {
    executable: String,
    #[builder(each = "arg")]
    args: Vec<String>,
    current_dir: Option<String>,
}
pub struct CommandBuilder {
    executable: Option<String>,
    args: Option<Vec<String>>,
    env: Option<Vec<String>>,
    current_dir: Option<String>,
}
impl Command {
    pub fn builder() -> CommandBuilder {
        CommandBuilder {
            executable: None,
            args: None,
            env: None,
            current_dir: None,
        }
    }
}
\end{code-block}
即利用\codeinlinebg{rust}{#[derive(Builder)]}宏
对任意结构体实现工厂模式代码的自动生成。

为实现这个Builder宏,首先实现其基本的结构:
\begin{code-block}{rust}
use proc_macro::TokenStream;
use proc_macro2;
use quote::quote;
use syn::{self, parse_macro_input, spanned::Spanned, AttributeArgs, Item};
// 派生宏,不再使用proc_macro_attribute(属性宏)
#[proc_macro_derive(Builder)]
pub fn derive(input: TokenStream) -> TokenStream {
    // 读取输出,转换成语法树
    let st = parse_macro_input!(input as syn::DeriveInput);
    TokenStream::new()
}
\end{code-block}

然后实现一个真正的语法树展开函数:
\begin{code-block}{rust}
fn do_expand(st: &syn::DeriveInput) -> syn::Result<proc_macro2::TokenStream> {
    // 获取语法树ident信息,即结构体名称(字面量)
    let struct_name_literal = st.ident.to_string();
    // 构造新的结构体名称
    let builder_name_literal = format!("{}Builder", struct_name_literal);
    // 构造新的结构体的标识符(不是string)
    // 第一个为标识符的字面量,第二个为位置信息
    let builder_name_ident = syn::Ident::new(&builder_name_literal, st.span());
    let struct_name_ident = &st.ident;
    let ret = quote!(
        // #builder_name_ident 表示使用之前的 builder_name_ident替换当前位置的内容
        // 是quote宏的自定义语法格式
        pub struct #builder_name_ident {
        }
        impl #struct_name_ident {
            pub fn builder() -> #builder_name_ident {
                #builder_name_ident {
                }
            }
        }
    );
    Ok(ret)
}
\end{code-block}
注意,上述代码当中的st(语法树)结构是一个\codeinlinebg{rust}{syn::DeriveInput}结构体\footnote{参考:\url{https://docs.rs/syn/1.0.72/syn/struct.DeriveInput.html}},其内在结构如下:
\begin{code-block}{rust}
pub struct DeriveInput {
    pub attrs: Vec<Attribute>, // 结构体/函数的属性,即#[]部分
    pub vis: Visibility,       // 可见性,pub,private
    pub ident: Ident,          // 标识符
    pub generics: Generics,    // 泛型
    pub data: Data,            // 字段,其类型可以是struct和enum以及联合体Union等
}
\end{code-block}
可以针对该结构体进行解析,从而得到语法树的各个元素。

有了do\_expand函数进行语法树的展开和修改之后,再回过头来修改框架的实现:
\begin{code-block}{rust}
#[proc_macro_derive(Builder)]
pub fn derive(input: TokenStream) -> TokenStream {
    let st = parse_macro_input!(input as syn::DeriveInput);
    match do_expand(&st) {
        // 转换成TokenStream
        Ok(token_stream) => token_stream.into(),
        // 将错误转换成编译器能够识别的TokenStream
        Err(error) => error.to_compile_error().into(),
    }
}
\end{code-block}

到此时,一个Builder属性宏的基本框架已经有了,通过\codeinlinebg{bash}{cargo expand}将
代码展开,最终得到的结果大致如下,当然,目前还缺乏结构体的字段内容:
\begin{figure}[H]
  \centering
  \includegraphics[width=\linewidth]{rust_expand.png}
  \caption{Builder宏的展开结果}
  \label{fig:rust_expand}
\end{figure}

需要注意的是,本例使用的是派生式的过程宏,而不是属性式的过程宏。属性式的过程宏
可以对其装饰的代码进行直接的修改,而派生式的过程宏则无法对代码进行直接的修改,
而是转为将代码追加在原始代码后面。

接下来开始对结构体字段进行填充。在填充之前,首先对语法树DeriveInput进行简单的一些
介绍,该结构体当中存在一个Data字段\footnote{参考:\url{https://docs.rs/syn/1.0.72/syn/enum.Data.html}},
而这个字段的详细定义,则如下:
\begin{code-block}{rust}
pub enum Data {
    Struct(DataStruct), // 针对结构体
    Enum(DataEnum),     // 针对Enum
    Union(DataUnion),   // 针对union
}
\end{code-block}
由于本例当中,关注的是结构体,因此,需要重点关注\codeinlinebg{rust}{DataStruct}
这个字段\footnote{参考:\url{https://docs.rs/syn/1.0.72/syn/struct.DataStruct.html}},该字段的具体定义如下:
\begin{code-block}{rust}
pub struct DataStruct {
    pub struct_token: Struct,       // 对应struct字面量
    pub fields: Fields,             // 对应struct的字段,filed
    pub semi_token: Option<Semi>,   // 分号,可选
}
\end{code-block}
结构体的字段变量,放在了\codeinlinebg{rust}{Fields}当中,
这个结构体的定义则大致如下\footnote{参考:\url{https://docs.rs/syn/1.0.72/syn/enum.Fields.html}}
\begin{code-block}{rust}
pub enum Fields {
    Named(FieldsNamed),         // 有名字段
    Unnamed(FieldsUnnamed),     // 无名字段
    Unit,                       // 元组,类似()
}
\end{code-block}
将有名字段的结构体继续进行展开,最终,我们会得到一个名为\codeinlinebg{rust}{Field}
的结构体\footnote{Field定义:\url{https://docs.rs/syn/1.0.72/syn/struct.Field.html}},
这个结构体的定义大致如下:
\begin{code-block}{rust}
pub struct Field {
    pub attrs: Vec<Attribute>,          // 字段属性
    pub vis: Visibility,                // 字段可见性
    pub ident: Option<Ident>,           // 字段名称
    pub colon_token: Option<Colon>,     // 冒号
    pub ty: Type,                       // 字段类型
}
\end{code-block}
而对结构体字段的填充过程,实际上就是用代码实现,最终找到上述结构体,并对其继续
构造的一个过程。由于整个语法树结构比较复杂,单独使用一个函数来实现对结构体的解析
和构造,获取结构体的字段定义\footnote{关于使用Token!替换Comma的说明:\url{https://docs.rs/syn/1.0.72/syn/token/struct.Comma.html}}:
\begin{code-block}{rust}
// get_filed_from_derive_input方法当中的result,实际上是一个
// Punctuated<Field, Comma>对象,其中Comma表示逗号。
// 不过,syn的文档描述当中,说明了最好不要直接使用Comma这种变量,
// 而是使用Token![,]这种宏表示
type StructFields = syn::punctuated::Punctuated<syn::Field, syn::Token![,]>;

fn get_filed_from_derive_input(st: &syn::DeriveInput) -> syn::Result<&StructFields> {
    // 根据上面结构体的定义,对输入的语法树节点进行解析
    // 但是,只匹配有名字段
    if let syn::Data::Struct(syn::DataStruct {
        fields: syn::Fields::Named(syn::FieldsNamed { ref named, .. }),
        ..
    }) = st.data
    {
        return Ok(named);
    }

    // 如果没有,则返回一个编译器可以使用的错误信息
    // 返回错误在源代码当中的位置信息
    Err(syn::Error::new_spanned(
        st,
        "Must define on Struct, Not on Enum",
    ))
}
\end{code-block}

接着构造一个函数来产生结构体的字段定义:
\begin{code-block}{rust}
fn generate_builder_struct_fields_def(
    st: &syn::DeriveInput,
) -> syn::Result<proc_macro2::TokenStream> {
    // 获取语法树处理之后得到的所有字段
    let fields = get_filed_from_derive_input(st)?;

    // 获得字段(语法树的标识符)
    let idents: Vec<_> = fields.iter().map(|f| &f.ident).collect();

    // 获得标识符的类型信息
    let types: Vec<_> = fields.iter().map(|f| &f.ty).collect();

    // 使用quote宏进行构造
    let ret = quote! {
        // #(),* 表示重复操作,操作的就是#()当中的内容
        // #indets和#types表示使用对应的变量进行替换
        // 这里使用的是Option的绝对路径,目的是防止和用户自定义的其他类型发生冲突
        #(#idents: std::option::Option<#types>), *
    };
    Ok(ret)
}
\end{code-block}
通过上述的函数,我们就可以定义结构体的字段了。接下来是对结构体字段的初始化,
这个操作也采用一个函数进行:
\begin{code-block}{rust}
fn generate_builder_struct_fields_init(
    st: &syn::DeriveInput,
) -> syn::Result<Vec<proc_macro2::TokenStream>> {
    // 获取语法树当中的所有字段
    let fields = get_filed_from_derive_input(st)?;

    let init_data: Vec<_> = fields
        .iter()
        .map(|f| {
            let ident = &f.ident;
            quote! {
                // 对ident全部设置为None
                #ident: std::option::Option::None
            }
        })
        .collect();
    Ok(init_data)
}
\end{code-block}

最后,将上述的2个函数结合到\codeinlinebg{rust}{do_expand}函数当中,改造之后的函数则如下:
\begin{code-block}{rust}
fn do_expand(st: &syn::DeriveInput) -> syn::Result<proc_macro2::TokenStream> {
    ...
    let builder_struct_fields_def = generate_builder_struct_fields_def(st)?;

    let builder_struct_fields_init = generate_builder_struct_fields_init(st)?;

    let ret = quote!(
        pub struct #builder_name_ident {
            #builder_struct_fields_def
        }

        impl #struct_name_ident {
            pub fn builder() -> #builder_name_ident {
                #builder_name_ident {
                    // quote的宏语法
                    // #(),* 表示重复操作,操作的就是#()当中的内容
                    #(#builder_struct_fields_init),*
                }
            }
        }

    );
    ...
    Ok(ret)
}
\end{code-block}

在面向对象的设计思路当中,通常还对结构体/类添加相关的getter/setter等方法,同样的,
我们也可以利用过程宏实现这类操作,比如,使用一个函数实现Builder的setter方法:
\begin{code-block}{rust}
fn generate_setter(st: &syn::DeriveInput) -> syn::Result<proc_macro2::TokenStream> {
    let fields = get_filed_from_derive_input(st)?;

    let idents: Vec<_> = fields.iter().map(|f| &f.ident).collect();
    let types: Vec<_> = fields.iter().map(|f| &f.ty).collect();

    // 构造一个可变的tokenstream
    let mut final_token_stream = proc_macro2::TokenStream::new();

    // 并行迭代idents和types
    for (ident, type_) in idents.iter().zip(types.iter()) {
        // 生成setter的tokenstream
        let token_stream_slice = quote! {

            fn #ident(&mut self, #ident: #type_) -> Self {
                self.#ident = std::option::Option::Some(#ident);
                self
            }

        };

        // 追加到最终的tokenstream当中
        final_token_stream.extend(token_stream_slice);
    }
    Ok(final_token_stream)
}
\end{code-block}

继续改造之后比较完整的\codeinlinebg{rust}{do_expand}函数如下:
\begin{code-block}{rust}
fn do_expand(st: &syn::DeriveInput) -> syn::Result<proc_macro2::TokenStream> {
    ...
    let setter_functions = generate_setter(st)?;
    let ret = quote!(
        ...
        impl #builder_name_ident {
            #setter_functions
        }
    );
    Ok(ret)
}
\end{code-block}
上述代码通过\codeinlinebg{bash}{cargo expand}之后,
其结果大致如下:
\begin{figure}[H]
  \centering
  \includegraphics[width=\linewidth]{rust_expand_2.png}
  \caption{Builder宏的展开结果2}
  \label{fig:rust_expand_2}
\end{figure}

作为Builder模式,Builder结构体必然还需要一个类似于build的函数或者方法,来生成真正的
结构体。在类似于build这样的函数/方法当中,通常需要完成2件事情:1是检查builder结构
的所有字段是否可以填充原始结构体,即字段是否都存在;2是生成一个原始结构体。同样的,
可以利用函数/方法来实现对字段的检查,以及字段的初始化:
\begin{code-block}{rust}
fn check_fileds(st: &syn::DeriveInput) -> syn::Result<proc_macro2::TokenStream> {
    let fields = get_filed_from_derive_input(st)?;
    let idents: Vec<_> = fields.iter().map(|f| &f.ident).collect();
    let mut final_check_stream = proc_macro2::TokenStream::new();
    for ident in idents.iter() {
        let check_stream_slice = quote! {
            // 检测字段是否为空
            if self.#ident.is_none() {
                let err_msg = format!("{} field is missing", stringify!(#ident));
                // 返回错误
                return std::result::Result::Err(err_msg.into());
            }
        };
        final_check_stream.extend(check_stream_slice);
    }
    Ok(final_check_stream)
}
fn build_target_fields(st: &syn::DeriveInput) -> syn::Result<proc_macro2::TokenStream> {
    let fields = get_filed_from_derive_input(st)?;
    let idents: Vec<_> = fields.iter().map(|f| &f.ident).collect();
    let mut final_init_stream = proc_macro2::TokenStream::new();
    for ident in idents.iter() {
        let init_stream_slice = quote! {
            #ident: self.#ident.clone().unwrap(),
        };
        final_init_stream.extend(init_stream_slice);
    }
    Ok(final_init_stream)
}
\end{code-block}

到此为止,基本上,利用过程宏,实现Builder模式就大致完成了。将所有的代码整合在一起,
整个过程宏的完整代码如下:
\begin{code-block}{rust}
use proc_macro::TokenStream;
use proc_macro2;
use quote::quote;
use syn::{self, parse_macro_input, spanned::Spanned};

#[proc_macro_derive(Builder)]
pub fn derive(input: TokenStream) -> TokenStream {
    let st = parse_macro_input!(input as syn::DeriveInput);
    match do_expand(&st) {
        Ok(token_stream) => token_stream.into(),
        Err(error) => error.to_compile_error().into(),
    }
}

fn do_expand(st: &syn::DeriveInput) -> syn::Result<proc_macro2::TokenStream> {
    let struct_name_literal = st.ident.to_string();
    let builder_name_literal = format!("{}Builder", struct_name_literal);
    let builder_name_ident = syn::Ident::new(&builder_name_literal, st.span());
    let struct_name_ident = &st.ident;

    let builder_struct_fields_def = generate_builder_struct_fields_def(st)?;
    let builder_struct_fields_init = generate_builder_struct_fields_init(st)?;

    let setter_functions = generate_setter(st)?;
    let checked_res = check_fileds(st)?;
    let build_res = build_target_fields(st)?;

    let ret = quote!(
        pub struct #builder_name_ident {
            #builder_struct_fields_def
        }

        impl #struct_name_ident {
            pub fn builder() -> #builder_name_ident {
                #builder_name_ident {
                    #(#builder_struct_fields_init),*
                }
            }
        }

        impl #builder_name_ident {
            #setter_functions
            pub fn build(&mut self) -> std::result::Result<#struct_name_ident, std::boxed::Box<dyn std::error::Error>>{
                #checked_res
                Ok(#struct_name_ident {
                    #build_res
                })
            }
        }

    );
    Ok(ret)
}

fn check_fileds(st: &syn::DeriveInput) -> syn::Result<proc_macro2::TokenStream> {
    let fields = get_filed_from_derive_input(st)?;
    let idents: Vec<_> = fields.iter().map(|f| &f.ident).collect();
    let mut final_check_stream = proc_macro2::TokenStream::new();

    for ident in idents.iter() {
        let check_stream_slice = quote! {
            if self.#ident.is_none() {
                let err_msg = format!("{} field is missing", stringify!(#ident));
                return std::result::Result::Err(err_msg.into());
            }
        };
        final_check_stream.extend(check_stream_slice);
    }
    Ok(final_check_stream)
}

fn build_target_fields(st: &syn::DeriveInput) -> syn::Result<proc_macro2::TokenStream> {
    let fields = get_filed_from_derive_input(st)?;
    let idents: Vec<_> = fields.iter().map(|f| &f.ident).collect();
    let mut final_init_stream = proc_macro2::TokenStream::new();
    for ident in idents.iter() {
        let init_stream_slice = quote! {
            #ident: self.#ident.clone().unwrap(),
        };
        final_init_stream.extend(init_stream_slice);
    }
    Ok(final_init_stream)
}

fn generate_setter(st: &syn::DeriveInput) -> syn::Result<proc_macro2::TokenStream> {
    let fields = get_filed_from_derive_input(st)?;
    let idents: Vec<_> = fields.iter().map(|f| &f.ident).collect();
    let types: Vec<_> = fields.iter().map(|f| &f.ty).collect();

    let mut final_token_stream = proc_macro2::TokenStream::new();
    for (ident, type_) in idents.iter().zip(types.iter()) {
        let token_stream_slice = quote! {
            pub fn #ident(&mut self, #ident: #type_) -> & mut Self {
                self.#ident = std::option::Option::Some(#ident);
                self
            }
        };
        final_token_stream.extend(token_stream_slice);
    }
    Ok(final_token_stream)
}

type StructFields = syn::punctuated::Punctuated<syn::Field, syn::Token![,]>;

fn get_filed_from_derive_input(st: &syn::DeriveInput) -> syn::Result<&StructFields> {
    if let syn::Data::Struct(syn::DataStruct {
        fields: syn::Fields::Named(syn::FieldsNamed { ref named, .. }),
        ..
    }) = st.data
    {
        return Ok(named);
    }
    Err(syn::Error::new_spanned(
        st,
        "Must define on Struct, Not on Enum",
    ))
}

fn generate_builder_struct_fields_def(
    st: &syn::DeriveInput,
) -> syn::Result<proc_macro2::TokenStream> {
    let fields = get_filed_from_derive_input(st)?;
    let idents: Vec<_> = fields.iter().map(|f| &f.ident).collect();
    let types: Vec<_> = fields.iter().map(|f| &f.ty).collect();

    let ret = quote! {
        #(#idents: std::option::Option<#types>), *
    };

    Ok(ret)
}

fn generate_builder_struct_fields_init(
    st: &syn::DeriveInput,
) -> syn::Result<Vec<proc_macro2::TokenStream>> {
    let fields = get_filed_from_derive_input(st)?;
    let init_data: Vec<_> = fields
        .iter()
        .map(|f| {
            let ident = &f.ident;
            quote! {
                #ident: std::option::Option::None
            }
        })
        .collect();
    Ok(init_data)
}
\end{code-block}

使用该宏的代码则如下:
\begin{code-block}{rust}
#[derive(Debug, Builder)]
pub struct Command {
    executable: String,
    args: Vec<String>,
    env: Vec<String>,
    current_dir: String,
}
fn main() {
    let builder = Command::builder()
        .executable("lucifer".to_owned())
        .args(vec![])
        .env(vec![])
        .current_dir("target".to_owned())
        .build()
        .unwrap();
    info!("{:#?}", builder);
}
\end{code-block}

\subsection{过程宏案例-派生过程宏Builder深入}
通常情况下,Rust的结构体并不是要求所有的字段都必须有值,或者必须初始化,存在可选的
字段,因此,可以继续对Builder过程宏进行改造。比如将上例当中的
\codeinlinebg{rust}{Command}结构体
定义修改为如下:
\begin{code-block}{rust}
pub struct Command {
    executable: String,
    args: Vec<String>,
    env: Vec<String>,
    current_dir: Option<String>,
}
\end{code-block}
即\codeinlinebg{rust}{current_dir}
为可选字段,而Builder生成的结构体\codeinlinebg{rust}{CommandBuilder}
则如下:
\begin{code-block}{rust}
pub struct CommandBuilder {
    executable: Option<String>,
    args: Option<Vec<String>>,
    env: Option<Vec<String>>,
    current_dir: Option<String>,
}
\end{code-block}

而这样的结构体,如果将其展开为语法树,原始结构体的\codeinlinebg{rust}{current_dir}
字段形式大致如下:
\begin{code-block}{json}
Path(
    TypePath {
        qself: None,
        path: Path {
            leading_colon: None,
            segments: [
                PathSegment {
                    ident: Ident {
                        ident: "Option",
                        span: #0 bytes(1337..1343),
                    },
                    arguments: AngleBracketed(
                        AngleBracketedGenericArguments {
                            colon2_token: None,
                            lt_token: Lt,
                            args: [
                                Type(
                                    Path(
                                        TypePath {
                                            qself: None,
                                            path: Path {
                                                leading_colon: None,
                                                segments: [
                                                    PathSegment {
                                                        ident: Ident {
                                                            ident: "String",
                                                            span: #0 bytes(1344..1350),
                                                        },
                                                        arguments: None,
                                                    },
                                                ],
                                            },
                                        },
                                    ),
                                ),
                            ],
                            gt_token: Gt,
                        },
                    ),
                },
            ],
        },
    },
)
\end{code-block}
相比之下,普通的字段类型,其语法树结构可能如下:
\begin{code-block}{rust}
Path(
    TypePath {
        qself: None,
        path: Path {
            leading_colon: None,
            segments: [
                PathSegment {
                    ident: Ident {
                        ident: "String",
                        span: #0 bytes(1267..1273),
                    },
                    arguments: None,
                },
            ],
        },
    },
)
\end{code-block}
可以看到,2者的区别比较大。如果还是沿用上一个示例的方式进行Builder的构建,会出现
这样一个问题:Builder宏所生成的结构体会被处理成如下的形式:
\begin{code-block}{rust}
pub struct CommandBuilder {
    executable: std::option::Option<String>,
    args: std::option::Option<Vec<String>>,
    env: std::option::Option<Vec<String>>,
    current_dir: std::option::Option<Option<String>>,   // 被多层Option封装
}
\end{code-block}
上面的形式显然无法满足我们的需求。因此,为达到上述目的,需要对字段的类型进行进
一步的处理。对于过程宏而言(syn包),字段的类型是一个结构体,其定义大致如下\footnote{类型的定义:\url{https://docs.rs/syn/1.0.72/syn/enum.Type.html}}:
\begin{code-block}{rust}
pub enum Type {
    Array(TypeArray),
    BareFn(TypeBareFn),
    Group(TypeGroup),
    ImplTrait(TypeImplTrait),
    Infer(TypeInfer),
    Macro(TypeMacro),
    Never(TypeNever),
    Paren(TypeParen),
    Path(TypePath),
    Ptr(TypePtr),
    Reference(TypeReference),
    Slice(TypeSlice),
    TraitObject(TypeTraitObject),
    Tuple(TypeTuple),
    Verbatim(TokenStream),
    // some variants omitted
}
\end{code-block}

需要做的,就是对字段类型的语法树进行解构,一步一步的按照需求,准确的定位Option所标识的
字段,并将其的内层类型取出来。因此,需要对上一章节\colorunderlineref{builder}的代码进行重构。

首先需要实现的,就是对可选字段(Option)的语法树定位和处理,并且返回对应的TokenStream:
\begin{code-block}{rust}
fn get_option_fields(st: &syn::Type) -> Option<&syn::Type> {
    // 对type语法树进行解构
    if let syn::Type::Path(syn::TypePath {
        path: syn::Path { segments, .. },
        ..
    }) = st
    {
        if let Some(segment) = segments.last() {
            // 找到Option字段
            if segment.ident.to_string() == "Option" {
                {
                    if let syn::PathArguments::AngleBracketed(
                        syn::AngleBracketedGenericArguments { ref args, .. },
                    ) = segment.arguments
                    {
                        // 获得真实的数据类型
                        if let Some(syn::GenericArgument::Type(inner_type)) = args.first() {
                            return Some(inner_type);
                        }
                    }
                }
            }
        }
    }
    None
}
\end{code-block}
然后,根据需求,将原有的代码进行改写,比如,Builder的结构体字段定义函数,需要修改
\begin{code-block}{rust}
fn generate_builder_struct_fields_def(
    st: &syn::DeriveInput,
) -> syn::Result<proc_macro2::TokenStream> {
    ...
    let types: Vec<_> = fields
        .iter()
        .map(|f| {
            if let Some(ty) = get_option_fields(&f.ty) {
                ty
            } else {
                &f.ty
            }
        })
        .collect();
    ...
}
\end{code-block}
Builder的setter方法需要修改:
\begin{code-block}{rust}
fn generate_setter(st: &syn::DeriveInput) -> syn::Result<proc_macro2::TokenStream> {
    ...
    for (ident, type_) in idents.iter().zip(types.iter()) {
        let token_stream_slice = if let Some(inner_type) = get_option_fields(type_) {
            quote! {
                pub fn #ident(&mut self, #ident: #inner_type) -> & mut Self {
                    self.#ident = std::option::Option::Some(#ident);
                    self
                }
            }
        } else {
            quote! {
                pub fn #ident(&mut self, #ident: #type_) -> & mut Self {
                    self.#ident = std::option::Option::Some(#ident);
                    self
                }
            }
        };
        final_token_stream.extend(token_stream_slice);
    }
    ...
}
\end{code-block}
同样的,Builder的字段检查和初始化方法也需要修改:
\begin{code-block}{rust}
fn check_fileds(st: &syn::DeriveInput) -> syn::Result<proc_macro2::TokenStream> {
    let fields = get_filed_from_derive_input(st)?;
    let idents: Vec<_> = fields.iter().map(|f| &f.ident).collect();
    let types: Vec<_> = fields.iter().map(|f| &f.ty).collect();
    let mut final_check_stream = proc_macro2::TokenStream::new();
    for (ident, type_) in idents.iter().zip(types.iter()) {
        if get_option_fields(type_).is_some() {
            continue;
        }
        let check_stream_slice = quote! {
            if self.#ident.is_none() {
                let err_msg = format!("{} field is missing", stringify!(#ident));
                return std::result::Result::Err(err_msg.into());
            }
        };
        final_check_stream.extend(check_stream_slice);
    }
    Ok(final_check_stream)
}

fn build_target_fields(st: &syn::DeriveInput) -> syn::Result<proc_macro2::TokenStream> {
    let fields = get_filed_from_derive_input(st)?;
    let idents: Vec<_> = fields.iter().map(|f| &f.ident).collect();
    let types: Vec<_> = fields.iter().map(|f| &f.ty).collect();
    let mut final_init_stream = proc_macro2::TokenStream::new();
    for (ident, types_) in idents.iter().zip(types.iter()) {
        let init_stream_slice = if get_option_fields(types_).is_none() {
            quote! {
                #ident: self.#ident.clone().unwrap(),
            }
        } else {
            quote! {
                #ident: self.#ident.clone(),
            }
        };
        final_init_stream.extend(init_stream_slice);
    }
    Ok(final_init_stream)
}
\end{code-block}

改造之后的完整代码如下:
\begin{code-block}{rust}
use proc_macro::TokenStream;
use proc_macro2;
use quote::quote;
use syn::{self, parse_macro_input, spanned::Spanned};
#[proc_macro_derive(Builder)]
pub fn derive(input: TokenStream) -> TokenStream {
    let st = parse_macro_input!(input as syn::DeriveInput);
    match do_expand(&st) {
        Ok(token_stream) => token_stream.into(),
        Err(error) => error.to_compile_error().into(),
    }
}
fn do_expand(st: &syn::DeriveInput) -> syn::Result<proc_macro2::TokenStream> {
    let struct_name_literal = st.ident.to_string();
    let builder_name_literal = format!("{}Builder", struct_name_literal);
    let builder_name_ident = syn::Ident::new(&builder_name_literal, st.span());
    let struct_name_ident = &st.ident;
    let builder_struct_fields_def = generate_builder_struct_fields_def(st)?;
    let builder_struct_fields_init = generate_builder_struct_fields_init(st)?;
    let setter_functions = generate_setter(st)?;
    let checked_res = check_fileds(st)?;
    let build_res = build_target_fields(st)?;
    let ret = quote!(
        pub struct #builder_name_ident {
            #builder_struct_fields_def
        }
        impl #struct_name_ident {
            pub fn builder() -> #builder_name_ident {
                #builder_name_ident {
                    #(#builder_struct_fields_init),*
                }
            }
        }
        impl #builder_name_ident {
            #setter_functions
            pub fn build(&mut self) -> std::result::Result<#struct_name_ident, std::boxed::Box<dyn std::error::Error>>{
                #checked_res
                Ok(#struct_name_ident {
                    #build_res
                })
            }
        }
    );
    Ok(ret)
}
fn check_fileds(st: &syn::DeriveInput) -> syn::Result<proc_macro2::TokenStream> {
    let fields = get_filed_from_derive_input(st)?;
    let idents: Vec<_> = fields.iter().map(|f| &f.ident).collect();
    let types: Vec<_> = fields.iter().map(|f| &f.ty).collect();
    let mut final_check_stream = proc_macro2::TokenStream::new();
    for (ident, type_) in idents.iter().zip(types.iter()) {
        if get_option_fields(type_).is_some() {
            continue;
        }
        let check_stream_slice = quote! {
            if self.#ident.is_none() {
                let err_msg = format!("{} field is missing", stringify!(#ident));
                return std::result::Result::Err(err_msg.into());
            }
        };
        final_check_stream.extend(check_stream_slice);
    }
    Ok(final_check_stream)
}
fn build_target_fields(st: &syn::DeriveInput) -> syn::Result<proc_macro2::TokenStream> {
    let fields = get_filed_from_derive_input(st)?;
    let idents: Vec<_> = fields.iter().map(|f| &f.ident).collect();
    let types: Vec<_> = fields.iter().map(|f| &f.ty).collect();
    let mut final_init_stream = proc_macro2::TokenStream::new();
    for (ident, types_) in idents.iter().zip(types.iter()) {
        let init_stream_slice = if get_option_fields(types_).is_none() {
            quote! {
                #ident: self.#ident.clone().unwrap(),
            }
        } else {
            quote! {
                #ident: self.#ident.clone(),
            }
        };
        final_init_stream.extend(init_stream_slice);
    }
    Ok(final_init_stream)
}
fn generate_setter(st: &syn::DeriveInput) -> syn::Result<proc_macro2::TokenStream> {
    let fields = get_filed_from_derive_input(st)?;
    let idents: Vec<_> = fields.iter().map(|f| &f.ident).collect();
    let types: Vec<_> = fields.iter().map(|f| &f.ty).collect();
    let mut final_token_stream = proc_macro2::TokenStream::new();
    for (ident, type_) in idents.iter().zip(types.iter()) {
        let token_stream_slice = if let Some(inner_type) = get_option_fields(type_) {
            quote! {
                pub fn #ident(&mut self, #ident: #inner_type) -> & mut Self {
                    self.#ident = std::option::Option::Some(#ident);
                    self
                }
            }
        } else {
            quote! {
                pub fn #ident(&mut self, #ident: #type_) -> & mut Self {
                    self.#ident = std::option::Option::Some(#ident);
                    self
                }
            }
        };
        final_token_stream.extend(token_stream_slice);
    }
    Ok(final_token_stream)
}
type StructFields = syn::punctuated::Punctuated<syn::Field, syn::Token![,]>;
fn get_filed_from_derive_input(st: &syn::DeriveInput) -> syn::Result<&StructFields> {
    if let syn::Data::Struct(syn::DataStruct {
        fields: syn::Fields::Named(syn::FieldsNamed { ref named, .. }),
        ..
    }) = st.data
    {
        return Ok(named);
    }
    Err(syn::Error::new_spanned(
        st,
        "Must define on Struct, Not on Enum",
    ))
}
fn generate_builder_struct_fields_def(
    st: &syn::DeriveInput,
) -> syn::Result<proc_macro2::TokenStream> {
    let fields = get_filed_from_derive_input(st)?;
    let idents: Vec<_> = fields.iter().map(|f| &f.ident).collect();
    let types: Vec<_> = fields
        .iter()
        .map(|f| {
            if let Some(ty) = get_option_fields(&f.ty) {
                ty
            } else {
                &f.ty
            }
        })
        .collect();
    let ret = quote! {
        #(#idents: std::option::Option<#types>), *
    };
    Ok(ret)
}
fn get_option_fields(st: &syn::Type) -> Option<&syn::Type> {
    if let syn::Type::Path(syn::TypePath {
        path: syn::Path { segments, .. },
        ..
    }) = st
    {
        if let Some(segment) = segments.last() {
            if segment.ident.to_string() == "Option" {
                {
                    if let syn::PathArguments::AngleBracketed(
                        syn::AngleBracketedGenericArguments { ref args, .. },
                    ) = segment.arguments
                    {
                        if let Some(syn::GenericArgument::Type(inner_type)) = args.first() {
                            return Some(inner_type);
                        }
                    }
                }
            }
        }
    }
    None
}
fn generate_builder_struct_fields_init(
    st: &syn::DeriveInput,
) -> syn::Result<Vec<proc_macro2::TokenStream>> {
    let fields = get_filed_from_derive_input(st)?;
    let init_data: Vec<_> = fields
        .iter()
        .map(|f| {
            let ident = &f.ident;
            quote! {
                #ident: std::option::Option::None
            }
        })
        .collect();
    Ok(init_data)
}
\end{code-block}

完成上述操作之后,在主函数当中,就可以以多种方式进行builder的初始化:
\begin{code-block}{rust}
fn main(){
    let builder = Command::builder()
        .executable("lucifer".to_owned())
        .args(vec![])
        .env(vec![])
        .current_dir("target".to_owned())
        .build()
        .unwrap();
    info!("{:#?}", builder);
    let builder = Command::builder()
        .executable("lucifer".to_owned())
        .args(vec![])
        .env(vec!["titans".to_owned(), "garuda".to_owned()])
        .build()
        .unwrap();
    info!("{:#?}", builder);
}
#[derive(Debug, Builder)]
pub struct Command {
    executable: String,
    args: Vec<String>,
    env: Vec<String>,
    current_dir: Option<String>,
}
\end{code-block}

\subsection{过程宏案例-派生过程宏Builder的派生属性}
\label{builder_attribute}
如果继续进行深入,考虑下面的一种形式:
\begin{code-block}{rust}
#[derive(Builder)]
pub struct Command {
    executable: String,
    #[builder(each = "arg")]
    args: Vec<String>,
    #[builder(each = "env")]
    env: Vec<String>,
    current_dir: Option<String>,
}
fn main() {
    ...
    let builder = Command::builder()
        .executable("lucifer".to_owned())
        .arg("first")
        .arg("second")
        .env(vec!["titans".to_owned(), "garuda".to_owned()])
        .build()
        .unwrap();
}
\end{code-block}
即,使用\codeinlinebg{rust}{#[builder(each = "arg")]}
这样的代码对结构体的字段进行标记,标记之后,原本对结构体进行初始化,使用的是一个vec结构(args),
后续可以使用单个元素(arg)进行追加的方式进行初始化。通常来说,实现这种宏,都应该要求builder当中的标签
和真实的属性名之间最好有区别。如果要求标签和属性名相同,则需要进行额外的其他特殊处理,即需要避免生成一个一次性通过列表进行赋值的方法。
以上述代码为例,如果使用builder宏装饰env字段,并且标签也是env,那么,针对env字段,最好的方式就是不要实现一个
使用列表直接复制的函数,即\codeinlinebg{rust}{builder.env("something")}与
\codeinlinebg{rust}{builder.env(vec!["nothing", "ok"])}这2种形式
无法并存,只能存在第一个。另外的重点是,这个builder宏,只能在被使用了
\codeinlinebg{rust}{#[derive(Builder)]}
修饰之后的结构体当中使用,而不能单独使用,因此,类似builder的宏被称之为惰性属性
宏,其完整的定义形式,一般是\codeinline{rust}{#[proc_macro_derive(Builder, attributes(builder))]}
这种。

由于上述需求的存在,在处理arg和env参数时,就无法再套用之前例子当中的Option类型来
包裹这两个参数了,换言之,对于增加了标签\codeinline{rust}{#[builder(each="arg")]}的
字段,就不能将其在Builder当中处理成Option类型,而是必须保留成Vec类型,即该结构所对应
的Builder结构体,其展开的内容应当如下所示:
\begin{code-block}{rust}
pub struct CommandEachBuilder {
    executable: std::option::Option<String>,
    args: Vec<String>,
    env: Vec<String>,
    current_dir: std::option::Option<String>,
}
\end{code-block}
这点在后续的处理当中非常重要。

惰性属性宏与常见的属性过程宏基本上是一样的,因此,在处理惰性属性宏之前,先看看
常见的属性是什么样式的。简单的查看结构体的属性语法树结构,可以如下进行操作:
\begin{code-block}{rust}
#[proc_macro_derive(Test)]
pub fn do_test(input: TokenStream) -> TokenStream {
    let st = parse_macro_input!(input as syn::DeriveInput);
    let attr = st.attrs.first().unwrap();
    eprintln!("{:#?}", attr);
    proc_macro2::TokenStream::new().into()
}
...
#[derive(Test)]
#[blog::com(Bar)]
pub struct CommandEach {
    executable: String,
    args: Vec<String>,
    ...
}
fn main() {
    ...
}
\end{code-block}
通过eprint函数,可以得到关于属性Attribute的详细信息:
\begin{code-block}{json}
Attribute {
    pound_token: Pound,
    style: Outer,
    bracket_token: Bracket,
    path: Path {
        leading_colon: None,
        segments: [
            PathSegment {
                ident: Ident {
                    ident: "blog",
                    span: #0 bytes(1402..1406),
                },
                arguments: None,
            },
            Colon2,
            PathSegment {
                ident: Ident {
                    ident: "com",
                    span: #0 bytes(1408..1411),
                },
                arguments: None,
            },
        ],
    },
    tokens: TokenStream [
        Group {
            delimiter: Parenthesis,
            stream: TokenStream [
                Ident {
                    ident: "Bar",
                    span: #0 bytes(1412..1415),
                },
            ],
            span: #0 bytes(1411..1416),
        },
    ],
}
\end{code-block}
在上述的输出当中,需要关注的有几个点:
\begin{enumerate}
  \item 属性的style值为Outer,如果对结构体的装饰变成了\codeinlinebg{rust}{#![derive]}这种,则style会变成Inner
  \item Rust会把属性拆分成2部分,path和tokens,其中path表示路径,tokens则是一个原始的TokenStream,并没有变成语法树节点,仅仅只是做了分词处理,没有任何语义,需要自行解析,因此,可以在其中加入自己规定的语法结构
  \item path指导编译器决定如何处理后面的部分
  \item tokens如果本身是一个符合Rust语法规范的结构,可以采用\codeinlinebg{rust}{parse_meta}方法将path和tokens解析成为\codeinlinebg{rust}{syn::Meta}数据类型,使之成为独立的语法树节点
\end{enumerate}

关于\codeinlinebg{rust}{syn::Meta}\footnote{类型定义:\url{https://docs.rs/syn/1.0.73/syn/enum.Meta.html}}
数据类型,其中包含3种:
\begin{enumerate}
  \item \codeinlinebg{rust}{syn::Meta::Path}表示一个路径,\codeinlinebg{rust}{#[A]}只有一个小节,但是A也是一个Path;\codeinlinebg{rust}{#[A::B::C]}也是一个Path,但是会被拆分成多个PathSegment。
  \item \codeinlinebg{rust}{syn::Meta::List}表示一个列表,这个列表必须由一个前置路径和一个括号标记组成,括号内部的内容通过逗号分割为多个条目(组成列表),每个条目又是一个\codeinlinebg{rust}{syn::Meta}
数据类型。比如\codeinlinebg{rust}{#[Foo(AA,BB,CC)]}当中,Foo就是前置路径,而AA等则是列表项,同时,也是\codeinlinebg{rust}{syn::Meta}数据类型;
而\codeinlinebg{rust}{#[Foo(AAA,BBB(CCC,DDD))]}表示列表可以嵌套,Foo是全局的前置路径,而BBB是内层\codeinlinebg{rust}{syn::Meta::List}的前置路径。
  \item \codeinlinebg{rust}{syn::Meta::NameValue}则是常见的键值对,key部分是一个\codeinlinebg{rust}{syn::Path}类型,而value则通常是字符串。比如
\codeinlinebg{rust}{#[xxx="yyy"]}。默认情况下,kv对只能有一对,如果需要有多个kv对,则必须使用list类型,即\codeinline{rust}{#[Foo(x="y",w="z")]}这种形式。
\end{enumerate}
只要是attr符合上述标准,则可以使用\codeinlinebg{rust}{parse_meta}函数对其进行解析,比如下面的例子:
\begin{code-block}{rust}
#[proc_macro_derive(Test)]
pub fn do_test(input: TokenStream) -> TokenStream {
    let st = parse_macro_input!(input as syn::DeriveInput);
    let attr = st.attrs.first().unwrap();
    // 直接解析attr,将attr解析成Rust的语法树节点
    let meta = attr.parse_meta();
    eprintln!("{:#?}", meta);
    proc_macro2::TokenStream::new().into()
}
...
#[derive(Test)]
#[Foo(x = "lucifer")]
pub struct CommandEach {
    executable: String,
    args: Vec<String>,
...
}
\end{code-block}
上述代码通过\codeinlinebg{bash}{cargo check}之后,可以得到类似如下的结果:
\begin{code-block}{json}
Ok(
    List(
        MetaList {
            path: Path {
                leading_colon: None,
                segments: [
                    PathSegment {
                        ident: Ident {
                            ident: "Foo",
                            span: #0 bytes(1414..1417),
                        },
                        arguments: None,
                    },
                ],
            },
            paren_token: Paren,
            nested: [
                Meta(
                    NameValue(
                        MetaNameValue {
                            path: Path {
                                leading_colon: None,
                                segments: [
                                    PathSegment {
                                        ident: Ident {
                                            ident: "x",
                                            span: #0 bytes(1418..1419),
                                        },
                                        arguments: None,
                                    },
                                ],
                            },
                            eq_token: Eq,
                            lit: Str(
                                LitStr {
                                    token: "lucifer",
                                },
                            ),
                        },
                    ),
                ),
            ],
        },
    ),
)
\end{code-block}

在Rust的语法树当中,如果对一个结构体的字段进行展开,其基本
的语法树结构可能如下:
\begin{code-block}{json}
Field {
    attrs: [],
    vis: Inherited,
    ident: Some(
        Ident {
            ident: "executable",
            span: #0 bytes(1429..1439),
        },
    ),
    colon_token: Some(
        Colon,
    ),
    ty: Path(
        TypePath {
            qself: None,
            path: Path {
                leading_colon: None,
                segments: [
                    PathSegment {
                        ident: Ident {
                            ident: "String",
                            span: #0 bytes(1441..1447),
                        },
                        arguments: None,
                    },
                ],
            },
        },
    ),
}
\end{code-block}
即,每个结构体字段都带有attrs(属性)标记。上述语法树为不带属性标记的字段,而
如果是类似本节开头的部分,在env和args上设置了属性标记,该字段的语法树会变为如下的
模式:
\begin{code-block}{json}
Field {
    attrs: [
        Attribute {
            pound_token: Pound,
            style: Outer,
            bracket_token: Bracket,
            path: Path {
                leading_colon: None,
                segments: [
                    PathSegment {
                        ident: Ident {
                            ident: "builder",
                            span: #0 bytes(1455..1462),
                        },
                        arguments: None,
                    },
                ],
            },
            tokens: TokenStream [
                Group {
                    delimiter: Parenthesis,
                    stream: TokenStream [
                        Ident {
                            ident: "each",
                            span: #0 bytes(1463..1467),
                        },
                        Punct {
                            ch: '=',
                            spacing: Alone,
                            span: #0 bytes(1468..1469),
                        },
                        Literal {
                            kind: Str,
                            symbol: "arg",
                            suffix: None,
                            span: #0 bytes(1470..1475),
                        },
                    ],
                    span: #0 bytes(1462..1476),
                },
            ],
        },
    ],
    ...
},
\end{code-block}
需要注意的是,上述语法树内容,全部表示的是结构体字段的属性标记,并非结构体字段
本身的语法树结构,可以看到,其结构和原始的Attribute是相同的。属性标签是Rust当中
常用的标签,也非常灵活。

回到本案例的需求,针对包含有属性标签字段的结构体,首先需要解决的,就是找到这些
带有标签的结构体字段:
\begin{code-block}{rust}
fn get_attr_field_ident(field: &syn::Field) -> Option<syn::Ident> {
    for attr in &field.attrs {
        if let Ok(syn::Meta::List(syn::MetaList {
            ref path,
            ref nested,
            ..
        })) = attr.parse_meta()
        {
            if let Some(__path__) = path.segments.first() {
                if __path__.ident == "builder" {
                    if let Some(syn::NestedMeta::Meta(syn::Meta::NameValue(__dict__))) =
                        nested.first()
                    {
                        if __dict__.path.is_ident("each") {
                            if let syn::Lit::Str(ref arg_token) = __dict__.lit {
                                return Some(syn::Ident::new(
                                    arg_token.value().as_str(),
                                    attr.span(),
                                ));
                            }
                        }
                    }
                }
            }
        }
    }
    None
}
\end{code-block}
上述代码的作用,就是遍历字段的属性(Attr),对其进行解构,如果存在\codeinline{rust}{builder="each"}这种模式的标签,则返回
找到的标签字段(即本例当中的each字段标签),解构的内容参照之前的属性被解析成Meta的相关内容。

获得字段的标签属性之后,接下来就是对结构体的字段进行处理。由于对字段添加了标签,
原有的处理方式已经不太适用了,新的处理方式需要考虑至少3种不同的情况:
\begin{enumerate}
  \item 必需的原始类型被Option包裹
  \item 保留原始的Vec类型
  \item 使用Option包裹可选的原始类型
\end{enumerate}
在开始处理这些结构体字段之前,需要首先了解每一种结构体字段的语法树结构大致是什么
样子,才可以有的放矢,比如,不带标签的标量数据类型(即常见的String,int等没有使用
Vec或者Option包裹的数据类型)的语法树大致如下:
\begin{code-block}{json}
Field {
    attrs: [],
    vis: Inherited,
    ident: Some(
        Ident {
            ident: "executable",
            span: #0 bytes(1686..1696),
        },
    ),
    colon_token: Some(
        Colon,
    ),
    ty: Path(
        TypePath {
            qself: None,
            path: Path {
                leading_colon: None,
                segments: [
                    PathSegment {
                        ident: Ident {
                            ident: "String",
                            span: #0 bytes(1698..1704),
                        },
                        arguments: None,
                    },
                ],
            },
        },
    ),
}
\end{code-block}
同样的,不带标签的矢量数据类型(Vec或者Option)的语法树结构大致如下:
\begin{code-block}{json}
Field {
    attrs: [],
    vis: Inherited,
    ident: Some(
        Ident {
            ident: "others",
            span: #0 bytes(1813..1819),
        },
    ),
    colon_token: Some(
        Colon,
    ),
    ty: Path(
        TypePath {
            qself: None,
            path: Path {
                leading_colon: None,
                segments: [
                    PathSegment {
                        ident: Ident {
                            ident: "Vec",
                            span: #0 bytes(1821..1824),
                        },
                        arguments: AngleBracketed(
                            AngleBracketedGenericArguments {
                                colon2_token: None,
                                lt_token: Lt,
                                args: [
                                    Type(
                                        Path(
                                            TypePath {
                                                qself: None,
                                                path: Path {
                                                    leading_colon: None,
                                                    segments: [
                                                        PathSegment {
                                                            ident: Ident {
                                                                ident: "String",
                                                                span: #0 bytes(1825..1831),
                                                            },
                                                            arguments: None,
                                                        },
                                                    ],
                                                },
                                            },
                                        ),
                                    ),
                                ],
                                gt_token: Gt,
                            },
                        ),
                    },
                ],
            },
        },
    ),
}
\end{code-block}
带有标签的标量数据类型,其语法树结构则可能如下:
\begin{code-block}{json}
Field {
    attrs: [
        Attribute {
            pound_token: Pound,
            style: Outer,
            bracket_token: Bracket,
            path: Path {
                leading_colon: None,
                segments: [
                    PathSegment {
                        ident: Ident {
                            ident: "builder",
                            span: #0 bytes(1688..1695),
                        },
                        arguments: None,
                    },
                ],
            },
            tokens: TokenStream [
                Group {
                    delimiter: Parenthesis,
                    stream: TokenStream [
                        Ident {
                            ident: "each",
                            span: #0 bytes(1696..1700),
                        },
                        Punct {
                            ch: '=',
                            spacing: Alone,
                            span: #0 bytes(1701..1702),
                        },
                        Literal {
                            kind: Str,
                            symbol: "arg",
                            suffix: None,
                            span: #0 bytes(1703..1708),
                        },
                    ],
                    span: #0 bytes(1695..1709),
                },
            ],
        },
    ],
    vis: Inherited,
    ident: Some(
        Ident {
            ident: "executable",
            span: #0 bytes(1715..1725),
        },
    ),
    colon_token: Some(
        Colon,
    ),
    ty: Path(
        TypePath {
            qself: None,
            path: Path {
                leading_colon: None,
                segments: [
                    PathSegment {
                        ident: Ident {
                            ident: "String",
                            span: #0 bytes(1727..1733),
                        },
                        arguments: None,
                    },
                ],
            },
        },
    ),
}
\end{code-block}
带有标签的矢量数据类型的语法树结构则是如下类似:
\begin{code-block}{json}
Field {
    attrs: [
        Attribute {
            pound_token: Pound,
            style: Outer,
            bracket_token: Bracket,
            path: Path {
                leading_colon: None,
                segments: [
                    PathSegment {
                        ident: Ident {
                            ident: "builder",
                            span: #0 bytes(1741..1748),
                        },
                        arguments: None,
                    },
                ],
            },
            tokens: TokenStream [
                Group {
                    delimiter: Parenthesis,
                    stream: TokenStream [
                        Ident {
                            ident: "each",
                            span: #0 bytes(1749..1753),
                        },
                        Punct {
                            ch: '=',
                            spacing: Alone,
                            span: #0 bytes(1754..1755),
                        },
                        Literal {
                            kind: Str,
                            symbol: "arg",
                            suffix: None,
                            span: #0 bytes(1756..1761),
                        },
                    ],
                    span: #0 bytes(1748..1762),
                },
            ],
        },
    ],
    vis: Inherited,
    ident: Some(
        Ident {
            ident: "args",
            span: #0 bytes(1768..1772),
        },
    ),
    colon_token: Some(
        Colon,
    ),
    ty: Path(
        TypePath {
            qself: None,
            path: Path {
                leading_colon: None,
                segments: [
                    PathSegment {
                        ident: Ident {
                            ident: "Vec",
                            span: #0 bytes(1774..1777),
                        },
                        arguments: AngleBracketed(
                            AngleBracketedGenericArguments {
                                colon2_token: None,
                                lt_token: Lt,
                                args: [
                                    Type(
                                        Path(
                                            TypePath {
                                                qself: None,
                                                path: Path {
                                                    leading_colon: None,
                                                    segments: [
                                                        PathSegment {
                                                            ident: Ident {
                                                                ident: "String",
                                                                span: #0 bytes(1778..1784),
                                                            },
                                                            arguments: None,
                                                        },
                                                    ],
                                                },
                                            },
                                        ),
                                    ),
                                ],
                                gt_token: Gt,
                            },
                        ),
                    },
                ],
            },
        },
    ),
}
\end{code-block}

有了以上对于结构体字段语法树结构的了解,处理结构体字段的代码可以大致如下:
\begin{code-block}{rust}
fn get_generic_fields_type_each<'a>(
    st: &'a syn::Type,
    outer_ident_name: &str,
) -> Option<&'a syn::Type> {
    if let syn::Type::Path(syn::TypePath {
        path: syn::Path { segments, .. },
        ..
    }) = st
    {
        if let Some(segment) = segments.last() {
            // 解析原始结构体的字段类型(type),针对使用Option和Vec描述的类型
            // 返回其内部的真实数据类型
            if segment.ident.to_string() == outer_ident_name {
                if let syn::PathArguments::AngleBracketed(syn::AngleBracketedGenericArguments {
                    args,
                    ..
                }) = &segment.arguments
                {
                    if let Some(syn::GenericArgument::Type(inner_type)) = args.first() {
                        return Some(inner_type);
                    }
                }
            }
        }
    }
    None
}
fn generate_builder_struct_fields_def_each(
    fields: &StructFields,
) -> syn::Result<proc_macro2::TokenStream> {
    let idents: Vec<_> = fields.iter().map(|f| &f.ident).collect();
    let types: Vec<_> = fields
        .iter()
        .map(|f| {
            // 如果原始结构体字段的数据类型本身是使用Option封装的,则提取其内部数据类型
            if let Some(inner_type) = get_generic_fields_type_each(&f.ty, "Option") {
                quote!(std::option::Option<#inner_type>)
            } else if get_attr_field_ident(f).is_some() {
                // 如果原始字段上存在标签,在本例当中的做法,是将其默认识别成vec类型
                // 因此,不再使用Option进行包裹
                let origin_type = &f.ty;
                quote!(#origin_type)
            } else {
                // 其他的继续使用Option进行包裹,即使原始字段是Vec类型
                let origin_type = &f.ty;
                quote!(std::option::Option<#origin_type>)
            }
        })
        .collect();
    Ok(quote!( #(#idents: #types), *))
}
\end{code-block}

获取到结构体的字段之后,接下来就是对这些字段进行初始化操作:
\begin{code-block}{rust}
fn generate_builder_struct_fields_init_each(
    fields: &StructFields,
) -> syn::Result<Vec<proc_macro2::TokenStream>> {
    let init_data: Vec<_> = fields
        .iter()
        .map(|f| {
            let ident = &f.ident;
            // 如果原始结构体的字段上存在标签,该字段将被识别为Vec类型
            // 直接使用Vec进行初始化
            if get_attr_field_ident(f).is_some() {
                quote!(#ident: std::vec::Vec::new())
            } else {
                // 否则,使用Option进行填充
                quote!(#ident: std::option::Option::None)
            }
        })
        .collect();
    Ok(init_data)
}
\end{code-block}

接下来,就是生成Builder结构体的setter函数:
\begin{code-block}{rust}
fn generate_setter_each(fields: &StructFields) -> syn::Result<proc_macro2::TokenStream> {
    let idents: Vec<_> = fields.iter().map(|f| &f.ident).collect();
    let types: Vec<_> = fields.iter().map(|f| &f.ty).collect();
    let mut final_token_stream = proc_macro2::TokenStream::new();
    for (idx, (ident, type_)) in idents.iter().zip(types.iter()).enumerate() {
        let mut tokenstream_piece;
        // 如果原始字段是Option类型
        if let Some(inner_type) = get_generic_fields_type_each(type_, "Option") {
            tokenstream_piece = quote! {
                pub fn #ident(&mut self, #ident: #inner_type) -> & mut Self {
                    // 使用Option的内部数据类型对字段进行初始化
                    self.#ident = std::option::Option::Some(#ident);
                    self
                }
            };
        // 如果原始字段上存在标签
        } else if let Some(ref builder_for_each) = get_attr_field_ident(&fields[idx]) {
            // 检测当前字段是否是Vec
            let inner_type = get_generic_fields_type_each(type_, "Vec").ok_or(syn::Error::new(
                fields[idx].span(),
                "each field must be specified with Vec field",
            ))?;
            tokenstream_piece = quote! {
                pub fn #builder_for_each(&mut self, #builder_for_each: #inner_type) -> & mut Self {
                    self.#ident.push(#builder_for_each);
                    self
                }
            };
            // 如果标签名称和字段名称不同,还需要生成一个字段本身的setter方法
            if builder_for_each != ident.as_ref().unwrap() {
                tokenstream_piece.extend(quote! {
                    pub fn #ident(&mut self, #ident: #type_) -> & mut Self {
                        self.#ident = #ident.clone();
                        self
                    }
                });
            }
        } else {
            tokenstream_piece = quote! {
                pub fn #ident(&mut self, #ident: #type_) -> & mut Self {
                    self.#ident = std::option::Option::Some(#ident);
                    self
                }
            };
        };
        final_token_stream.extend(tokenstream_piece);
    }
    Ok(final_token_stream)
}
\end{code-block}

最后的重点,则是生成本需求的build方法,实现Rust的建造者模式:
\begin{code-block}{rust}
fn generate_builder_function(
    fields: &StructFields,
    origin_struct_ident: &syn::Ident,
) -> syn::Result<proc_macro2::TokenStream> {
    let idents: Vec<_> = fields.iter().map(|f| &f.ident).collect();
    let types: Vec<_> = fields.iter().map(|f| &f.ty).collect();
    let mut check_pieces = Vec::new();
    for (idx, (__ident__, __type__)) in idents.iter().zip(types.iter()).enumerate() {
        // 如果字段不是Option或者不存在标签,表示该字段为必需字段,必须进行初始化
        if get_generic_fields_type_each(__type__, "Option").is_none()
            && get_attr_field_ident(&fields[idx]).is_none()
        {
            check_pieces.push(quote! {
                if self.#__ident__.is_none() {
                    let err = format!("{} field missing", stringify!(#__ident__));
                    return std::result::Result::Err(err.into())
                }
            });
        }
    }
    let mut fill_result = Vec::new();
    for (idx, (__ident__, __type__)) in idents.iter().zip(types.iter()).enumerate() {
        // 如果字段存在标签,直接将字段进行拷贝
        if get_attr_field_ident(&fields[idx]).is_some() {
            fill_result.push(quote!(#__ident__: self.#__ident__.clone()));
        // 如果原始字段是Option类型,则将builder的类型进行解包,返回真实的数据类型
        } else if get_generic_fields_type_each(__type__, "Option").is_none() {
            fill_result.push(quote!(#__ident__: self.#__ident__.clone().unwrap()));
        } else {
            fill_result.push(quote!(#__ident__: self.#__ident__.clone()));
        }
    }
    let final_token = quote! {
        pub fn build(&mut self) -> std::result::Result<#origin_struct_ident, std::boxed::Box<dyn std::error::Error>>{
            #(#check_pieces)*
            Ok(#origin_struct_ident {
                #(#fill_result),*
            })
        }
    };
    Ok(final_token)
}
\end{code-block}

有了以上的基础,我们将其有机的结合起来:
\begin{code-block}{rust}
use proc_macro::TokenStream;
use proc_macro2;
use quote::quote;
use syn::{self, parse_macro_input, spanned::Spanned};
type StructFields = syn::punctuated::Punctuated<syn::Field, syn::Token![,]>;
#[proc_macro_derive(BuilderEach, attributes(builder))]
pub fn deriveach(input: TokenStream) -> TokenStream {
    let st = parse_macro_input!(input as syn::DeriveInput);
    match do_expand_each(&st) {
        Ok(token_stream) => token_stream.into(),
        Err(error) => error.to_compile_error().into(),
    }
}
fn do_expand_each(st: &syn::DeriveInput) -> syn::Result<proc_macro2::TokenStream> {
    let struct_name_literal = st.ident.to_string();
    let builder_name_literal = format!("{}Builder", struct_name_literal);
    let builder_name_ident = syn::Ident::new(&builder_name_literal, st.span());
    let struct_name_ident = &st.ident;
    let fields = get_filed_from_derive_input_each(st)?;
    let builder_struct_fields_def = generate_builder_struct_fields_def_each(fields)?;
    let builder_struct_fields_init = generate_builder_struct_fields_init_each(fields)?;
    let setter_functions = generate_setter_each(fields)?;
    let build_function = generate_builder_function(fields, &struct_name_ident)?;
    let ret = quote!(
        pub struct #builder_name_ident {
            #builder_struct_fields_def
        }
        impl #struct_name_ident {
            pub fn builder() -> #builder_name_ident {
                #builder_name_ident {
                    #(#builder_struct_fields_init),*
                }
            }
        }
        impl #builder_name_ident {
            #setter_functions
            #build_function
        }
    );
    Ok(ret)
}
fn get_filed_from_derive_input_each(st: &syn::DeriveInput) -> syn::Result<&StructFields> {
    if let syn::Data::Struct(syn::DataStruct {
        fields: syn::Fields::Named(syn::FieldsNamed { ref named, .. }),
        ..
    }) = st.data
    {
        return Ok(named);
    }
    Err(syn::Error::new_spanned(
        st,
        "Must define on Struct, Not on Enum",
    ))
}
fn get_attr_field_ident(field: &syn::Field) -> Option<syn::Ident> {...}
fn get_generic_fields_type_each<'a>(st: &'a syn::Type, outer_ident_name: &str,
) -> Option<&'a syn::Type> {...}
fn generate_builder_struct_fields_def_each(fields: &StructFields,
) -> syn::Result<proc_macro2::TokenStream> {...}
fn generate_builder_struct_fields_init_each(fields: &StructFields,
) -> syn::Result<Vec<proc_macro2::TokenStream>> {...}
fn generate_setter_each(fields: &StructFields) -> syn::Result<proc_macro2::TokenStream> {...}
fn generate_builder_function(fields: &StructFields, origin_struct_ident: &syn::Ident,
) -> syn::Result<proc_macro2::TokenStream> {...}
\end{code-block}

最终,我们看看其使用方式以及结果:
\begin{code-block}{rust}
#[derive(Debug, BuilderEach)]
pub struct CommandEach {
    executable: String,
    #[builder(each = "arg")]
    args: Vec<String>,
    #[builder(each = "env")]
    env: Vec<String>,
    others: Vec<String>,
    current_dir: Option<String>,
}
fn main() {
    let builder = CommandEach::builder()
        .executable("lucifer".to_owned())
        .arg("lucifer".to_owned())
        .arg("titans".to_owned())
        .env("zhangjl".to_owned())
        .env("luoyan".to_owned())
        .others(vec![])
        .build()
        .unwrap();
    info!("{:#?}", builder);
    let builder = CommandEach::builder()
        .executable("lucifer".to_owned())
        .others(vec![])
        .build()
        .unwrap();
    info!("{:#?}", builder);
}
\end{code-block}

如果通过\codeinlinebg{bash}{cargo expand}将代码进行展开,得到的结果类似如下:
\begin{figure}[H]
  \centering
  \includegraphics[width=\linewidth]{rust_label_expand.png}
  \caption{带标签的结构体展开}
  \label{fig:rust_label_expand}
\end{figure}
而上述代码的执行结果,则大致如下:
\begin{figure}[H]
  \centering
  \includegraphics[width=\linewidth]{rust_label_result.png}
  \caption{带标签的结构体的执行结果}
  \label{fig:rust_label_result}
\end{figure}
可以看到,CommandEach和Command的结果是截然不同的。

现在结构体支持标签了,但是,新的问题又出现了:上述的过程宏要求的标签是\codeinline{rust}{#[builder(each = "...")]}
这种模式,如果标签写错了,即each写成了其他的字符,从程序安全的角度,应当给予足够
的错误信息提示,因此,之前的获取标签的函数需要进行改写:
\begin{code-block}{rust}
fn get_attr_field_ident(field: &syn::Field) -> syn::Result<Option<syn::Ident>> {
    for attr in &field.attrs {
        // 获得标签的语法树节点
        if let Ok(syn::Meta::List(ref list)) = attr.parse_meta() {
            // 二次解构
            let syn::MetaList {
                ref path,
                ref nested,
                ..
            } = list;
            if let Some(__path__) = path.segments.first() {
                if __path__.ident == "builder" {
                    if let Some(syn::NestedMeta::Meta(syn::Meta::NameValue(__dict__))) =
                        nested.first()
                    {
                        if __dict__.path.is_ident("each") {
                            if let syn::Lit::Str(ref arg_token) = __dict__.lit {
                                return Ok(Some(syn::Ident::new(
                                    arg_token.value().as_str(),
                                    attr.span(),
                                )));
                            }
                        } else {
                            // 如果builder当中的标签不是each,则返回一个错误
                            // 错误的范围则限定在标签的语法树节点
                            return Err(syn::Error::new_spanned(
                                list,
                                r#"expected `builder(each = "...")`"#,
                            ));
                        }
                    }
                }
            }
        }
    }
    Ok(None)
}
\end{code-block}

由于需要返回错误信息,因此,也需要对其他使用到该函数的地方进行修改。对于普通的
使用方式,直接在该方法的后面添加\codeinlinebg{rust}{?}即可,而如果是闭包,则需要
稍微注意一下。调用了\codeinlinebg{rust}{get_attr_field_ident}函数的代码修改如下:
\begin{code-block}{rust}
fn generate_builder_function(
    fields: &StructFields,
    origin_struct_ident: &syn::Ident,
) -> syn::Result<proc_macro2::TokenStream> {
    ...
    for (idx, (__ident__, __type__)) in idents.iter().zip(types.iter()).enumerate() {
        if get_generic_fields_type_each(__type__, "Option").is_none()
            && get_attr_field_ident(&fields[idx])?.is_none()
        ...
    }
    let mut fill_result = Vec::new();
    for (idx, (__ident__, __type__)) in idents.iter().zip(types.iter()).enumerate() {
        if get_attr_field_ident(&fields[idx])?.is_some() {
            fill_result.push(quote!(#__ident__: self.#__ident__.clone()));
        }
        ...
    }
    ...
    Ok(final_token)
}
fn generate_setter_each(fields: &StructFields) -> syn::Result<proc_macro2::TokenStream> {
    ...
    for (idx, (ident, type_)) in idents.iter().zip(types.iter()).enumerate() {
        let mut tokenstream_piece;
        if let Some(inner_type) = get_generic_fields_type_each(type_, "Option") {
            ...
        } else if let Some(ref builder_for_each) = get_attr_field_ident(&fields[idx])? {
            ...
        }
    }
}
fn generate_builder_struct_fields_init_each(
    fields: &StructFields,
) -> syn::Result<Vec<proc_macro2::TokenStream>> {
    let init_data: syn::Result<Vec<proc_macro2::TokenStream>> = fields
        .iter()
        .map(|f| {
            let ident = &f.ident;
            if get_attr_field_ident(f)?.is_some() {
                Ok(quote!(#ident: std::vec::Vec::new()))
            } else {
                Ok(quote!(#ident: std::option::Option::None))
            }
        })
        .collect();
    Ok(init_data?)
}
fn generate_builder_struct_fields_def_each(
    fields: &StructFields,
) -> syn::Result<proc_macro2::TokenStream> {
    let idents: Vec<_> = fields.iter().map(|f| &f.ident).collect();
    let types: syn::Result<Vec<_>> = fields
        .iter()
        .map(|f| {
            if let Some(inner_type) = get_generic_fields_type_each(&f.ty, "Option") {
                Ok(quote!(std::option::Option<#inner_type>))
            } else if get_attr_field_ident(f)?.is_some() {
                let origin_type = &f.ty;
                Ok(quote!(#origin_type))
            } else {
                let origin_type = &f.ty;
                Ok(quote!(std::option::Option<#origin_type>))
            }
        })
        .collect();
    let __types__ = types?;
    Ok(quote!( #(#idents: #__types__), *))
}
\end{code-block}
其余代码保持不变,这样,如果在使用该过程宏时写错了标签,即类似如下:
\begin{code-block}{rust}
#[derive(BuilderEach)]
pub struct CommandEach {
    executable: String,
    #[builder(each = "arg")]
    args: Vec<String>,
    #[builder(eac = "env")]
    env: Vec<String>,
    ...
}
\end{code-block}
则代码在编译的时候,就会出现如下比较明确的错误信息,当然了,正确的标签则不会提示错误:
\begin{figure}[H]
  \centering
  \includegraphics[width=\linewidth]{rust_label_error.png}
  \caption{处理有错误的标签}
  \label{fig:rust_label_error}
\end{figure}
\subsection{过程宏案例-自定义Debug}

\section{多线程}
Rust也同样支持常见的并行和并发操作,也同样分为进程,线程以及消息通信等等。

\subsection{线程}
Rust的线程操作必须使用闭包完成。在之前看到的闭包当中,通常采用的都是有参的闭包,
而在Rust的线程操作当中,则经常会遇到无参数的闭包;Rust的线程使用thread::spawn函数
进行实现:
\begin{code-block}{rust}
use std::thread;
use std::time::Duration;
fn main() {
    thread::spawn(|| {
        for i in 1..10 {
            println!("hi number {} from the spawned thread!", i);
            thread::sleep(Duration::from_millis(1));
        }
    });
    for i in 1..5 {
        println!("hi number {} from the main thread!", i);
        thread::sleep(Duration::from_millis(1));
    }
}
\end{code-block}
和其他语言的线程概念一样,当主线程结束时,所有的线程都会被终止。因此上述代码当中,
子线程(spawn)无法将所有的循环执行完成。为了达成所有进/线程执行完成之后才退出主
进/线程的目的,和其他的开发语言相同,需要在主进程当中调用join函数:
\begin{code-block}{rust}
fn main() {
    let handle = thread::spawn(|| {
        for i in 1..10 {
            println!("hi number {} from the spawned thread!", i);
            thread::sleep(Duration::from_millis(1));
        }
    });
    for i in 1..5 {
        println!("hi number {} from the main thread!", i);
        thread::sleep(Duration::from_millis(1));
    }
    handle.join().unwrap();
}
\end{code-block}
Thread::spawn的返回值是JoinHandle,是一个拥有所有权的值,当对其调用join方法时,
它会等待对应线程结束;而join的返回值是一个Result,可以按照之前介绍的方式进行处理。
同时,Join函数是一个阻塞式函数,只有当该函数运行结束之后,才会继续进行后续的操作。

多数情况下,Rust的线程不可能只会在内部运行,而和外部没有数据交互。但是,如果我们
直接使用外部数据,则会出现错误,比如下方的代码:
\begin{code-block}{rust}
fn main() {
    let v = vec![1, 2, 3];
    let handle = thread::spawn(|| {
        println!("Here's a vector: {:?}", v);
    });
    handle.join().unwrap();
}
\end{code-block}
\begin{figure}[H]
  \centering
  \includegraphics[width=\linewidth]{rust_thread_out_params.png}
  \caption{试图访问线程外部资源}
  \label{fig:rust_thread_out_params}
\end{figure}
线程使用的是闭包,从闭包的定义来说,是可以捕获并使用外部变量和数据的;但是,Rust
不知道这个线程到底会运行多长时间,因此无法知道对外部变量的引用是否一直有效,比如
下方的代码:
\begin{code-block}{rust}
fn main() {
    let v = vec![1, 2, 3];
    let handle = thread::spawn(|| {
        println!("Here's a vector: {:?}", v);
    });
    drop(v);
    handle.join().unwrap();
}
\end{code-block}
启动线程的同时,立即将v进行丢弃,线程内部无法知道v在运行阶段是否继续有效,就会
出现错误,因此,如果在线程当中使用默认的闭包模式,则无法对应的闭包是无法捕获以及
使用外部的变量和数据的。此时,则需要使用move闭包进行替换,即强制闭包获取外部变量
的所有权,而不是由Rust进行借用推断。但是需要注意,一旦使用move之后,在线程之外,
变量将无法再进行使用:
\begin{code-block}{rust}
fn main() {
    let v = vec![1, 2, 3];
    let handle = thread::spawn(move || {
        println!("Here's a vector: {:?}", v);
    });
    // 下方代码无法再进行执行
    // println!("{:?}", v);
    handle.join().unwrap();
}
\end{code-block}

\subsection{消息通信和消息传递}
每个线程做自己的事情,但是,不管什么编程语言,都需要考虑线程之间的数据交互问题。
Rust向Golang进行了学习,使用通信替换共享内存,来进行线程之间的数据传输。同样的,
Rust当中用于消息传递并发的主要工具是通道,该概念和Golang的通道概念相同。Rust的通道
分为2个角色:发送者和接收者,发送者发送消息,接收者接收消息,当发送者或者接收者任一
被丢弃时,则对应的通道被视为关闭。

Rust的通道采用mpsc::channel函数实现,mpsc表示多个生产者,单个消费者,因此,Rust
标准库实现通道的方式意味着一个通道可以有多个产生值的发送(sending)端,但只能有
一个消费这些值的接收(receiving)端。通道的实现示例如下:
\begin{code-block}{rust}
use std::sync::mpsc;

fn main() {
    let (sender, recevier) = mpsc::channel();
}
\end{code-block}
其中,函数的第一个返回值为发送者,第二个参数为接收者。使用通道发送数据通信的示例
如下:
\begin{code-block}{rust}
use std::sync::mpsc;
use std::thread;

fn main() {
    let (sender, recevier) = mpsc::channel();

    thread::spawn(move || {
        let val = "lucifer".to_string();

        match sender.send(val) {
            Ok(_) => println!("Send success"),
            Err(error) => println!("Send failed :{:?}", error),
        }
    });

    let res = match recevier.recv() {
        Ok(s) => s,
        Err(error) => {
            println!("Cannot recevie anything from sender: {:?}", error);
            "".to_string()
        }
    };

    println!("The result of channel is {}", res);
}
\end{code-block}
接收者接收消息有2种模式:默认的recv是阻塞式,返回一个Result<T, E>,当通道关闭时,
将返回Result当中的Error;而try\_recv是非阻塞式,同样是返回一个Result<T, E>,但是,
Result当中的Error表示没有接收到任何消息,可以使用for循环进行反复的尝试读取操作。
另外需要注意的是,Send函数会改变变量的所有权,当该函数执行之后,被发送的消息
(变量)将无法再使用。

但是,通道可以反复使用,而且和Golang的类似,Rust的通道也是可以进行迭代的,特别
是在接收消息时,通常采用for循环进行操作,减少了错误处理的代码,使得代码更具可读性:
\begin{code-block}{rust}
use std::sync::mpsc;
use std::thread;

fn main() {
    let (sender, recevier) = mpsc::channel();

    let handler = thread::spawn(move || {
        let vals = vec!["lucifer", "titans", "garuda"];
        for val in vals {
            match sender.send(val) {
                Ok(_) => println!("Send success"),
                Err(error) => println!("Send failed :{:?}", error),
            }
        }
    });

    for msg in recevier {
        println!("The msg is {}", msg);
    }

    match handler.join() {
        Err(error) => println!("Error{:?}", error),
        _ => (),
    }
}
\end{code-block}

同样的,由于Rust的通道默认是多生产者/单消费者,因此,可以通过多个发送端向单个接
收端发送消息。实际使用当中的多个发送端,则通常是某个发送端的克隆对象,如下:
\begin{code-block}{rust}
use std::sync::mpsc;
use std::thread;

fn main() {
    let (sender, recevier) = mpsc::channel();
    let sender_copy = sender.clone();

    let handler = thread::spawn(move || {
        let vals = vec!["lucifer", "titans", "garuda"];

        for val in vals {
            match sender.send(val) {
                Ok(_) => println!("Send success"),
                Err(error) => println!("Send failed :{:?}", error),
            }
        }
    });
    let handler_copy = thread::spawn(move || {
        let vals = vec!["zhangjl", "luoyan", "zhangzz"];

        for val in vals {
            match sender_copy.send(val) {
                Err(error) => println!("Send failed :{:?}", error),
                _ => (),
            }
        }
    });
    for msg in recevier {
        println!("The msg is {}", msg);
    }
    match handler_copy.join() {
        Err(error) => println!("Error{:?}", error),
        _ => (),
    }
    match handler.join() {
        Err(error) => println!("Error{:?}", error),
        _ => (),
    }
}
\end{code-block}

\subsection{共享状态}
在其他语言当中,有些特殊的场景,还是必须使用原有的线程并发概念——锁——来进行资源的
访问/读写控制。Rust当中同样存在锁,比较常见的就是互斥锁(互斥器,Mutex)以及原子
计数器(Arc)。在基本的操作上,互斥锁的使用和其他语言当中没有太大的区别:
\begin{code-block}{rust}
use std::sync::Mutex;
fn main() {
    let m = Mutex::new(5);
    {
        let mut num = m.lock().unwrap();
        *num = 6;
    }
    println!("m = {:?}", m);
}
\end{code-block}
注意,上述代码如果将内部大括号去除,则运行结束之后,m的状态还是锁定状态;但是,
有大括号,则表示大括号内部的段是一个有效的生命周期,当该生命周期结束之后,互斥
锁将自动释放。一旦获取了锁,就可以将返回值(在这里是num)视为一个其内部数据的
\colorblock{可变引用}。类型系统确保了我们在使用m中的值之前
获取锁:Mutex<i32>并不是一个i32,所以必须获取锁才能使用这个i32值。

实质上,Mutex是一个智能指针,lock调用返回一个叫做MutexGuard的智能指针。这个智能
指针实现了Deref来指向其内部数据;同时也提供了一个Drop实现,使得MutexGuard离开作
用域时自动释放锁,即锁的释放是自动发生的。

但是默认情况下,Mutex是无法用于进行线程间的数据共享,如下:
\begin{code-block}{rust}
use std::rc::Rc;
use std::sync::Mutex;
use std::thread;
fn main() {
    let counter = Rc::new(Mutex::new(0));
    let mut handles = vec![];
    for _ in 0..10 {
        let counter = Rc::clone(&counter);
        let handle = thread::spawn(move || {
            let mut num = counter.lock().unwrap();
            *num += 1;
        });
        handles.push(handle);
    }
    for handle in handles {
        handle.join().unwrap();
    }
    println!("Result: {}", *counter.lock().unwrap());
}
\end{code-block}
上述代码会出现下面的类似错误:
\begin{figure}[H]
  \centering
  \includegraphics[width=\linewidth]{rust_mutex_share_error.png}
  \caption{试图通过Rc共享Mutex的数据}
  \label{fig:rust_mutex_share_error}
\end{figure}
即之前提到的,Rc类型只能用于单线程/单进程环境。

而共享引用计数则需要使用Arc,它是可以安全的用于并发环境的类型,即原子引用计数,
可以在线程间进行共享所有权。Arc和Rc有相同的API,基本使用方法上类似。所有,可以直
接对上述代码进行修改:
\begin{code-block}{rust}
use std::sync::{Arc, Mutex};
use std::thread;
fn main() {
    let counter = Arc::new(Mutex::new(0));
    let mut handles = vec![];
    for _ in 0..10 {
        let counter = Arc::clone(&counter);
        let handle = thread::spawn(move || {
            let mut num = counter.lock().unwrap();
            *num += 1;
        });
        handles.push(handle);
    }
    for handle in handles {
        handle.join().unwrap();
    }
    println!("Result: {}", *counter.lock().unwrap());
}
\end{code-block}
通过这样简单的修改,成功实现了10个进程当中对同一个数值进行加法操作的功能。

虽然Rust本身的线程/进程管理非常完善,但是,thread::spawn产生的线程没有名称,并且
其栈空间大小默认为2M,如果需要需要针对线程/进程进行粒度更细的操作,比如自定义
线程名称,自定义线程的资源等等,此时,就需要使用thread::Builder进行修改,具体示例
如下:
\begin{code-block}{rust}
let mut v_thread = vec![];
for id in 1..5 {
    let thread_name = format!("child-{}", id);
    let size: usize = 1024;
    // 定义线程的名称,设置线程占用的栈大小为1M(1024)
    let builder = Builder::new().name(thread_name).stack_size(size);
    // builder.spawn返回的是Result<JoinHander, std::io::Error>
    // 需要进行处理,取出真正的线程句柄
    match builder.spawn(move || {
        info!(
            "In the child: {}, and the child name is {}",
            id,
            current().name().unwrap()
        );
    }) {
        Ok(child) => v_thread.push(child),
        Err(error) => error!("Cannot create the thread {} because: {:?}", id, error),
    };
}
// 其他的同普通的线程,
for child in v_thread {
    child.join().unwrap();
}
\end{code-block}

由于线程包含自己的资源空间,因此,存在一个特殊的存储空间——线程本地存储(Thread Local Storage,TLS),
存放在该区域的资源,其他线程无法访问,而是每个线程独占的数据:
\begin{code-block}{rust}
use std::cell::RefCell;
use std::thread;
fn main() {
    // 在线程本地存储定义一个FOO变量,最终的类型是thread::LocalKey
    thread_local!(static FOO: RefCell<u32> = RefCell::new(1));
    // 提供了一个with方法,可以通过给该方法传入闭包
    // 来操作线程本地存储中包含的变量
    FOO.with(|f| {
        info!("The f borrow is {}", *f.borrow());
        *f.borrow_mut() = 2;
    });
    let handler = thread::spawn(move || {
        // 子线程也有一个线程本地存储实例FOO,为主线程的副本
        // 也可以使用thread_local!宏在该子线程中重新创建一个LocalKey实例
        FOO.with(|f| {
            info!("In the handler thread The f borrow is {}", *f.borrow());
            *f.borrow_mut() = 3;
        });
    });
    // 主线程当中FOO实例并没有被子线程修改为3
    // thread local!宏定义单个线程内的一些独享数据
    FOO.with(|f| {
        info!("The f borrow is {}", *f.borrow());
    });
    handler.join().unwrap();
}
\end{code-block}

在同步原语支持方面,Rust也有自己的实现方式,通过使用std::thread当中的park函数提供
阻塞线程的能力,但并不能永久的阻塞线程,存在时间限制;而std::thread::part\_timeout
则可以显式的指定阻塞的超时时间;std::thread::Thread::unpark则可以将阻塞的线程重启;
如果需要让出当前线程的时间片,则需要使用std::thread::yeild\_now,让其他线程进行执行。
简单的阻塞例子如下:
\begin{code-block}{rust}
use std::thread::{self, Builder};
use std::time::Duration;
fn main() {
    let parked_thread = Builder::new()
        .spawn(|| {
            info!("Parking the thread ...");
            // 阻塞当前线程
            thread::park();
            info!("Thread parked");
        })
        .unwrap();
    thread::sleep(Duration::from_secs(5));
    info!("Unparking the thread");
    // 从JoinHandle中得到具体的线程
    parked_thread.thread().unpark();
    // 将该线程重新启动,该线程会继续沿着之前暂停的上下文执行
    parked_thread.join().unwrap();
}
\end{code-block}

除了常见的互斥锁(Mutex)之外,Rust也支持读写锁(RwLock)。读写锁的基本示例如下:
\begin{code-block}{rust}
use std::sync::RwLock;
fn main() {
    let rw_lock = RwLock::new(5);
    // 读写锁的使用必须使用{}进行区分,即便是单独使用读或者写也是一样
    // 通过代码块{},让读写锁自动释放,否则会出现死锁
    {
        let read_1 = rw_lock.read().unwrap();
        let read_2 = rw_lock.read().unwrap();
        info!("The read_1 is {}, and read_2 is {}", read_1, read_2);
    }
    {
        let mut write = rw_lock.write().unwrap();
        *write = 100;
    }
    info!("The data is {:?}", rw_lock);
}
\end{code-block}

而针对于同步的需求,Rust提供了屏障(Barrier)和条件变量(Condition Variable)原语。
屏障,是要求所有的条件全部满足之后,再进行后续操作,即在满足某个条件前,阻塞全部的
线程,通常用于线程同步,如下:
\begin{code-block}{rust}
use std::sync::{Arc, Barrier};
use std::thread;
fn main() {
    let mut vec = vec![];
    let barrier = Arc::new(Barrier::new(5));
    for id in 0..5 {
        let barrier_copy = barrier.clone();
        vec.push(thread::spawn(move || {
            info!("Thread {} Waiting the other threads...", id);
            // wait阻塞了所有的线程,当所有线程的wait之前部分全部执行完成之后
            // wait操作才算执行完成,才会执行每个线程后续的操作
            barrier_copy.wait();
            info!("{} After wait...", id);
        }));
    }
    for handler in vec {
        handler.join().unwrap();
    }
}
\end{code-block}

而条件变量与屏障稍微的区别在于,它不是阻塞所有的线程,而是在满足特定条件前,阻塞
一个得到了互斥锁的线程,如下:
\begin{code-block}{rust}
use std::sync::{Arc, Condvar, Mutex};
use std::thread;
use std::time::Duration;
fn main() {
    // 生成包含互斥锁的条件变量condvar
    let pair = Arc::new(((Mutex::new(false)), Condvar::new()));
    let pair_clone = pair.clone();
    let handler = thread::spawn(move || {
        let &(ref lock, ref cvar) = &*pair_clone;
        // 获得互斥锁
        let mut started = lock.lock().unwrap();
        info!("In the child thread");
        thread::sleep(Duration::from_secs(5));
        *started = true;
        // 通知主线程
        cvar.notify_one();
    });
    let &(ref lock, ref cvar) = &*pair;
    let mut started = lock.lock().unwrap();
    while !*started {
        info!("Waiting for the started singal {} ...", started);
        // 使用条件变量的wait阻塞当前线程,一直到cvar退出
        started = cvar.wait(started).unwrap();
        info!("Started singal finished {} ...", started);
    }
    handler.join().unwrap();
}
\end{code-block}
相比于单纯的互斥锁必须多次出入临界区才能获取到某个状态的信息,条件变量减少了系统
资源的浪费,但是需要注意,每个条件变量每次只能和一个互斥锁(体)一起使用。

除了使用锁、屏障以及条件变量,关于同步的问题,还可以使用原子操作。Rust目前只提供了
4个原子操作类型:AtomicBool、Atomiclsize、AtomicPtr和AtomicUsize。需要注意,虽然原子
操作类型本身可以保证操作的原子性,但是其本身并没有提供跨线程的共享方法,如果需要
使得原子数据类型也可以在线程间共享,则应当使用Arc进行封装,比如下面,使用原子类型
实现一个自旋锁:
\begin{code-block}{rust}
use std::sync::atomic::{AtomicUsize, Ordering};
use std::sync::Arc;
fn main() {
    let spinlock = Arc::new(AtomicUsize::new(1));
    let spinlock_clone = spinlock.clone();
    let handler = thread::spawn(move || {
        // 将原子类型的数据设置为0
        spinlock_clone.store(0, Ordering::SeqCst);
    });
    // 使用spinlock的load方法读取其内部原子类型的值,如果不为0,
    // 则不停地循环测试锁的状态,直到其状态被置为0为止
    // 所谓“自旋”就是指在语义上表示这种不断循环获取锁状态的行为
    while spinlock.load(Ordering::SeqCst) != 0 {}
    handler.join().unwrap();
}
\end{code-block}
代码当中的Ordering表示内存参数顺序,可以通过该参数来控制底层线程执行顺序。默认的,
Rust支持5种内存顺序,归为3大类:
\begin{itemize}
  \item 排序一致性顺序——SeqCst:最简单直观,要求必须先存储,后读取,在多线程环境下,所有的原子写操作都必须在读操作之前完成,强行指定了线程的执行顺序,保证了多线程中所有操作的全局一致性,但是存在性能损耗,其实质类似于餐厅点餐,相当于强制要求所有需要结账的客人,必须等所有点单的客户完成之后才可以结账
  \item 自由顺序——Relaxed:和SeqCst相反,完全不会对线程的顺序进行干涉,线程只进行原子操作,但是,线程之间会存在竞态条件,使用这种内存顺序会比较危险,只有在明确了解当前使用场景且必须使用它的情况下(比如只有读操作),才可使用自由顺序
  \item 获取-释放顺序——Release,Acquire和AcqRel: 是除排序一致性顺序之外的优先选择,默认情况下,不会对全部线程进行统一强制性的执行顺序要求,store表示释放(release),而load表示获取(acquire),通过这2种操作的协作实现线程同步。Release表示使用该顺序的store操作,之前所有的操作对于使用Acquire顺序的load操作都可见;反之,使用使用Aquire顺序的load操作,对于使用Release的store操作都是可见的;AcqRel表示读时使用Acquire顺序的load操作,写时使用Release顺序的store操作。获取释放顺序虽然不像排序一致性顺序那样对全局线程统一排序,但是它让每个线程都能接固定的顺序执行。
\end{itemize}

在此之前,已经谈到Rust支持channel通信来解决多线程环境所遇到的问题,比如之前的小例子:
\begin{code-block}{rust}
use std::thread;
use std::sync::mpsc::channel;
fn main() {
    let (tx, rx) = channel();
    let handler = thread::spawn(move || {
        tx.send(10).unwrap();
    });
    let res = rx.recv().unwrap();
    handler.join().unwrap();
}
\end{code-block}
像这种只有2个线程间通信的channel,称之为流通道,在流通道的内部,默认使用的是单生产者
单消费者的模式来提升性能。在此之前,我们看到多个发送者(生产者)单个接收者(消费者)
模式的通道,则称之为共享通道。而由于统一使用的channel函数生成通道,这样的通道又
称之为异步通道,即所有的操作都可以异步的进行处理,不会出现线程阻塞的情况。

同步通道的例子则如下:
\begin{code-block}{rust}
use std::thread;
use std::sync::mpsc::sync_channel;
fn main() {
    // 创建缓冲区为1的同步通道
    let (tx, rx) = sync_channel(1);
    tx.send(1).unwrap();
    let handler = thread::spawn(move || {
        tx.send(2).unwrap();
    });
    let res1 = rx.recv().unwrap();
    info!("The result is {}", res1);
    let res2 = rx.recv().unwrap();
    info!("The result2 is {}", res2);
    handler.join().unwrap();
}
\end{code-block}
在上述代码当中,由于channel的缓冲区设置为1,所以,当第一条信息被消费(recv)之前,
后续的消息发送会被一直阻塞,直到缓冲区可用为止。

虽然channel解决了很多的多线程同步和共享问题,但是,channel并没有解决死锁的问题,
当设计不周到的时候,channel同样会出现死锁的问题:
\begin{code-block}{rust}
use std::thread;
use std::sync::mpsc::channel;
fn main() {
    let (tx, rx) = channel();
    let mut handlers = vec![];
    for i in 0..5 {
        let tc = tx.clone();
        let handler = thread::spawn(move || {
            tc.send(i).unwrap();
        });
        handlers.push(handler);
    }
    // 如果注释下面代码,主线程将一直不退出
    // drop(tx);
    for j in rx.iter() {
        info!("{:?}", j);
    }
    for handler in handlers {
        handler.join().unwrap();
    }
}
\end{code-block}
因为rx的iter方法会阻塞线程,只要tx还没有被析构,该迭代器就会一直等待新的消息,
只有tx被析构之后,迭代器才能返回None,从而结束退出main主线程。由于上述代码的tx
一直没有析构,所以迭代器依旧会进行等待,但是tx也没有发送信息,从而造成死锁的状态。
显式调用drop之后,死锁将不会存在。

\section{异步编程(Future)的基本原理}
Rust目前的版本当中,异步/同步的支持比较完善,通常需要使用Future进行实现,
其中futures\footnote{\url{https://github.com/rust-lang/futures-rs}}提供了一个比较基本的异步编程实现,
Tokio\footnote{\url{https://github.com/tokio-rs/tokio}}也提供了比较完整的平台支持,
Async-Std\footnote{\url{https://github.com/async-rs/async-std}}也提供了
相关的支持。多线程的劣势主要体现在操作系统调度开销,难度较大,线程切换以及
跨线程共享数据会产生很多的额外开销,这些就是异步并发(async/await)发挥作用的
重要场景。

\subsection{Future的基本原理与实现机制}
所谓的Futrue,从字面上讲,就是一些将来完成的操作,也就是一些并不是当前立即结束
的操作,以此指代Rust的异步操作。Rust的异步操作实现基于轮询机制,每个异步任务分成
了3个阶段:
\begin{enumerate}
  \item 轮询阶段(Poll):一个Future被轮询之后,会开始执行,直到被阻塞。轮询Future通常被称为执行器(executor)
  \item 等待阶段:事件源(即需要使用Future的对象,通常称为reactor)注册等待事件发生,并确保当对应的事件发生或准备好时唤醒对应的Future
  \item 唤醒阶段:事件发生,相应的Future被唤醒,执行器执行,直至完成
\end{enumerate}

例如对大文件进行操作,如果采用普通的文件打开方式,操作大文件,需要将文件内容打开
(全部在内存展开)之后,才能操作;而异步的文件打开,则相当于将文件句柄返回给调用
者,而后续的内容,在调用者需要进行使用时,再加载到内存当中。两者的具体的情况如下
所示:
\begin{figure}[H]
  \centering
  % 禁止svg当中的特殊文字转换
  \includesvg[inkscapelatex=false, width=\linewidth]{rust_async}
  \caption{同步与异步执行的区别\protect\footnotemark}
  \label{fig:rust_async}
\end{figure}
\footnotetext{同/异步:\url{https://os.phil-opp.com/async-await/async-example.svg}}
上述例子当中,轮询阶段就相当于询问文件是否被打开,而等待阶段,则相当于等待文件的
内容加载内存当中,唤醒,则是调用者访问文件内容。

Rust的异步实现依赖于Future。在标准库当中,Rust提供了Future的
Trait以及\codeinlinebg{rust}{async}和\codeinlinebg{rust}{await}2个异步并发的关键字,
但是并没有提供具体的实现(即运行时),运行时则是由其他的库提供的,常见的就包括
上面提到的Async-Std和Tokio,以及Rust官方提供的futures(不在Rust的标准库当中)。
Rust的异步运行时可以分为2部分:执行器(executor)和反应器(reactor),
这2部分则是由Waker(唤醒器)进行交互。

以futures为例,一个简单的异步应用大致如下:
\begin{code-block}{rust}
extern crate futures;
async fn foo() -> u8 { 8 }
async fn hello() {
    let res = foo().await;
    println!("The res is {}", res);
}

fn main() {
    let future = hello();
    futures::executor::block_on(future);
}
\end{code-block}
在上述的例子当中,使用关键字\codeinlinebg{rust}{async}修饰的函数都是异步函数,这些
函数都会返回一个Future Trait,如果不使用\codeinlinebg{rust}{async}关键字,实际上
也是可以实现异步函数的定义的,只不过,会相对比较麻烦,并且需要显式的使用Future,
比如下面的代码:
\begin{code-block}{rust}
use futures::future::{self, Future};
fn async_read_file(name: &str) -> impl Future<Output = String> {
    future::ready(String::from(name))
}
fn async_with_lifetime<'a>(input: &'a u8) -> impl Future<Output = u8> + 'a {
    future::ready(*input)
}
\end{code-block}

Rust的Future Trait的定义如下\footnote{定义:\url{https://doc.rust-lang.org/std/future/trait.Future.html}}:
\begin{code-block}{rust}
pub trait Future {
    type Output;
    fn poll(self: Pin<&mut Self>, cx: &mut Context) -> Poll<Self::Output>;
}
\end{code-block}
其中,Output指的是异步函数的返回数据类型,而\codeinlinebg{rust}{poll}函数则是整个
Future工作机制的核心。该函数的实质是一个有限状态机,返回的Poll实际上是状态的描述。
Poll的定义如下:
\begin{code-block}{rust}
pub enum Poll<T> {
    Ready(T),
    Pending,
}
\end{code-block}
当异步函数的执行结果完成并且可用时,\codeinlinebg{rust}{poll}函数将返回一个Ready
包裹的结果,表示完成;如果执行还没有结束,则会返回一个Pending,表示执行还将继续,
相关的数据还没有准备完成,需要继续等待或者轮询,并且从当前的context当中,克隆一个
waker,一旦future状态有新的变化,waker函数将被唤醒。注意,如果状态不是Ready,poll函数
会被再次调用(间隔一定时间),直到返回Ready之后,poll函数将不再被调用。如果用普通的
Rust代码进行表示,其内在的逻辑可以模拟如下(但是性能非常差):
\begin{code-block}{rust}
let future = async_read_file("foo.txt");
let file_content = loop {
    match future.poll(…) {
        Poll::Ready(value) => break value,
        Poll::Pending => {}, // do something
    }
}
\end{code-block}

而另一种思路,则是使用future的组合,比如下方的例子:
\begin{code-block}{rust}
use std::future::{self,Future};
use std::task::{Context, Poll};
use std::pin::Pin;
fn main() {
    let _ = file_len();
}
struct StringLen<F> {
    inner_future: F,
}
impl<F> Future for StringLen<F> where F: Future<Output = String>{
    type Output = usize;
    fn poll(mut self: Pin<&mut Self>, cx: &mut Context<'_>) -> Poll<Self::Output> {
        match self.inner_future.poll(cx) {
            Poll::Ready(s) => Poll::Ready(s.len()),
            Poll::Pending => Poll::Pending,
        }
    }
}
fn string_len(string: impl Future<Output = String>)
    -> impl Future<Output = usize>
{
    StringLen {
        inner_future: string,
    }
}
fn file_len() -> impl Future<Output = usize> {
    let file_content_future = async_read_file("foo.txt");
    string_len(file_content_future)
}
fn async_read_file(name: &str) -> impl Future<Output = String> {
    future::ready(String::from(name))
}
\end{code-block}
\begin{attention}
上述代码只是演示了一种Future可能的处理方式,但是,代码本身不可使用,原因在于在
上述代码当中还没有处理pin(固定)。
\end{attention}

通过future的组合,可以实现非常高效的代码,但是,在某些情况下,由于Rust的类型系统
以及基于闭包的接口,会使得future的组合使用起来比较难,比如下面的代码:
\begin{code-block}{rust}
use futures::future::{self, Either, Future, FutureExt};
fn main() {
    let _ = example(100);
}
fn example(min_len: usize) -> impl Future<Output = String> {
    async_read_file("foo.txt").then(move |content| {
        if content.len() < min_len {
            Either::Left(async_read_file("bar.txt").map(|s| content + &s))
        } else {
            Either::Right(future::ready(content))
        }
    })
}
fn async_read_file(name: &str) -> impl Future<Output = String> {
    future::ready(String::from(name))
}
\end{code-block}
上述代码的功能很简单,读取foo.txt文件,然后使用\codeinlinebg{rust}{then}连接第二个
future:如果foo.txt的内容长度小于给定的最小值,则读取另一个文件bar.txt的内容,并将
其长度追加到返回结果当中,否则,只返回foo.txt的内容。上述操作当中,必须使用\codeinlinebg{rust}{move}
关键字,否则会存在一个生命周期的错误;另外,if/else必须返回相同的类型,但是,
上述代码当中,if返回的是\codeinlinebg{rust}{futures::future::Map},而else返回的是\codeinlinebg{rust}{futures::future::Ready},
必须使用\codeinlinebg{rust}{Either}将结果进行封装,转换成所期望的future。

\subsection{Future的async/await模式}
从上面的各种例子都可以看到,使用普通的方式实现Rust的异步编程可能会涉及到非常复杂的
代码,于是,Rust官方使用\codeinlinebg{rust}{async/await}等关键字,来简化异步编程的实现,
屏蔽了繁琐的实现细节。使用上述两个关键字之后,Rust的编译器会在编译过程当中进行
自动转换,转换成Future的模式,比如:
\begin{code-block}{rust}
async fn foo() -> u32 {
    0
}
// 编译器内部会自动转换成这种模式
fn foo() -> impl Future<Output = u32> {
    future::ready(0)
}
\end{code-block}

但是,这个转换过程对于开发者而言是透明无感的,以上面代码为例,使用async/await
进行改写之后的结果大致如下:
\begin{code-block}{rust}
async fn example(min_len: usize) -> String {
    let content = async_read_file("foo.txt").await;
    if content.len() < min_len {
        content + &async_read_file("bar.txt").await
    } else {
        content
    }
}
\end{code-block}
可以看到,代码更加的简练,并且逻辑上也是非常清晰。

但是,不管怎么变换,Future的内在还是一个有限状态机。\codeinlinebg{rust}{async}负责
将对应的函数转换成有限状态机,而\codeinlinebg{rust}{await}调用代表了不同的状态。
在上述代码当中,编译器将创建一个具有4个状态的状态机:
\begin{figure}[H]
  \centering
  \includesvg[inkscapelatex=false, width=\linewidth]{async-state-machine-states}
  \caption{异步状态机\protect\footnotemark}
  \label{fig:async-state-machine-states}
\end{figure}
\footnotetext{状态机:\url{https://os.phil-opp.com/async-await/async-state-machine-states.svg}}
\begin{enumerate}
  \item Start/End表示函数执行的开始和结束
  \item \colorblock{Waiting on foo.txt}状态表示该函数当前正在等待第一个async\_read\_file结果,表示一个暂停点
  \item \colorblock{Waiting on bar.txt}状态表示函数等待第二个async\_read\_file结果,同样也是一个暂停点
\end{enumerate}

有限状态机通过调用\codeinlinebg{rust}{poll}函数实现状态的切换,从而实现Future Trait:
\begin{figure}[H]
  \centering
  \begin{minipage}{\textwidth}
  \includesvg[inkscapelatex=false, width=\linewidth]{async-state-machine-basic}
  \caption{状态机切换\protect\footnotemark}
  \label{fig:async-state-machine-basic}
  \end{minipage}
\end{figure}
\footnotetext{状态转换:\url{https://os.phil-opp.com/async-await/async-state-machine-basic.svg}}

为了从最后一个等待状态继续,状态机必须在自身内部保持对当前状态的跟踪。此外,还必须
保存将在下一次poll调用当中需要被使用到的全部变量。幸运的是Rust编译器知道什么时候
使用哪些变量,因此,它可以自动生成包含了所需变量的结构体,注意,是\colorblock{自动生成}。
同样以上面的代码为例,如果深入到Rust编译器内部,看到的代码可能会是下面的样子:
\begin{code-block}{rust}
async fn example(min_len: usize) -> String {
    let content = async_read_file("foo.txt").await;
    if content.len() < min_len {
        content + &async_read_file("bar.txt").await
    } else {
        content
    }
}
// 注意,下面的代码是编译器自动生成的,不是开发者手动编写的
struct StartState {
    min_len: usize,
}
struct WaitingOnFooTxtState {
    min_len: usize,
    foo_txt_future: impl Future<Output = String>,
}
struct WaitingOnBarTxtState {
    content: String,
    bar_txt_future: impl Future<Output = String>,
}
struct EndState {}
\end{code-block}

在\colorblock{start}和\colorblock{Waiting on foo.txt}状态下,需要保存min\_len参数,
是因为在后面需要和content的长度进行比较,\colorblock{Waiting on foo.txt}状态保存
了另外一个foo\_txt\_future,代表了async\_read\_file的调用所返回的future,而这个
future会被状态机继续轮询,因此需要保留;\colorblock{Waiting on bar.txt}状态包含
内容变量,因为在bar.txt准备好后,字符串连接操作需要它。它还存储一个bar\_txt\_future,
表示bar.txt的正在进行的加载。该结构不包含min\_len变量,因为在content.len()比较之
后不再需要它。 在\colorblock{stop}状态下,不存储任何变量,因为该函数已经运行完成。

同样的,编译器也会自动生成状态机相关的代码,以上面的代码为例,Rust编译器会在内部
生成类似如下的状态机代码:
\begin{code-block}{rust}
enum ExampleStateMachine {
    Start(StartState),
    WaitingOnFooTxt(WaitingOnFooTxtState),
    WaitingOnBarTxt(WaitingOnBarTxtState),
    End(EndState),
}
\end{code-block}

使用编译器已经生成的各种状态结构体,利用enum将其进行封装成顶层的状态机,在此基础上,
为了完成/实现状态之前的切换,Rust编译器会根据上面的example函数自动生成其Future Trait
的实现,注意,下列代码也是Rust自动生成的,但并不代表编译器生成的代码就一定是下面
的样子:
\begin{code-block}{rust}
impl Future for ExampleStateMachine {
    type Output = String; // return type of `example`
    fn poll(self: Pin<&mut Self>, cx: &mut Context) -> Poll<Self::Output> {
        loop {
            match self { // TODO: handle pinning
                ExampleStateMachine::Start(state) => {…}
                ExampleStateMachine::WaitingOnFooTxt(state) => {…}
                ExampleStateMachine::WaitingOnBarTxt(state) => {…}
                ExampleStateMachine::End(state) => {…}
            }
        }
    }
}
\end{code-block}

状态机的Future实现关联数据类型Output为String,原因是异步函数example的返回类型是
String;同样的,由于Future Trait的定义当中包含了poll函数,因此,编译器又不辞辛劳
的生成或者说是实现了poll函数,在该函数体当中,使用loop-match对当前的状态进行轮询
和切换。当然,为了简单起见,这些编译器生成的代码都没有包含pin(固定,后续会专门
进行讲解),所有权以及生命周期等,因此,这些代码应当被视为不可运行的伪代码,而
编译器生成的真实代码,毫无疑问是可以处理上述的所有特性和问题的。

为了更进一步的理解Future Trait的内部实现机制,接下来会针对match的不同分支进行分析:
\begin{outline}[enumerate]
\1 Start分支

Rust编译器生成的Start分支的代码可能如下:
\begin{code-block}{rust}
ExampleStateMachine::Start(state) => {
    // from body of `example`
    let foo_txt_future = async_read_file("foo.txt");
    // `.await` operation
    let state = WaitingOnFooTxtState {
        min_len: state.min_len,
        foo_txt_future,
    };
    *self = ExampleStateMachine::WaitingOnFooTxt(state);
}
\end{code-block}
在example函数开始的时候,状态机实际上处于Start状态,在这种状态下,会执行example函数
体当中的代码,直到遇到第一个await关键字,即执行\codeinlinebg{rust}{async_read_file("foo.txt")}
这句,而为了处理该函数后面的await操作,编译器需要将状态机的状态修改为WaitingOnFooTxtState。
由于loop-match操作,状态机会跳入到WaitingOnFooTxtState分支。

\1 WaitingOnFooTxtState分支

同样的,该分支的代码(编译器自动生成)的可能如下:
\begin{code-block}{rust}
ExampleStateMachine::WaitingOnFooTxt(state) => {
    match state.foo_txt_future.poll(cx) {
        Poll::Pending => return Poll::Pending,
        Poll::Ready(content) => {
            // from body of `example`
            if content.len() < state.min_len {
                let bar_txt_future = async_read_file("bar.txt");
                // `.await` operation
                let state = WaitingOnBarTxtState { content, bar_txt_future, };
                *self = ExampleStateMachine::WaitingOnBarTxt(state);
            } else {
                *self = ExampleStateMachine::End(EndState));
                return Poll::Ready(content);
            }
        }
    }
}
\end{code-block}
注意,从该分支开始,poll函数被第一次调用,这是因为我们要获得foo\_txt\_future这个
future的真实结果,如果该结果没有准备好,则退出循环,返回pending状态;由于本次
循环当中,状态机实例(self)还是处于WaitingOnFooTxt状态,因此,下一次的轮询调用
仍然会执行当前的分支,不会进行状态切换(即进入其他的分支)。当foo\_txt\_future
准备就绪之后,结果将赋给content变量,如果content的长度满足最小长度,则状态机的
状态将被切换成EndState,直接返回一个Ready,结束轮询;否则,将异步读取bar.txt文件,
并再次通过await关键字将状态进行切换成为WaitingOnBarTxtState,而后续的轮询操作将进入
WaitingOnBarTxt这个分支。该分支的执行情况与本分支的类似。

\1 End分支

End分支应该是最为简单的分支,编译器生成的代码可能如下:
\begin{code-block}{rust}
ExampleStateMachine::End(_) => {
    panic!("poll called after Poll::Ready was returned");
}
\end{code-block}

实际上,一旦状态机变成了End状态,或者返回了Ready状态,就不能也不应该再次进行轮询,
而是直接退出循环和轮询,因此,如果是在End状态再次调用了poll函数,应当直接进行panic
处理。
\end{outline}

通过上述的模拟和分析大致理解编译器生成的状态机以及Future Trait的实现机制可能是
什么样的,实际上,编译器可以使用多种不同的方式实现或者生成Futrure Trait的代码,
目前看到的例子是基于生成器实现的。

目前还缺少异步函数本身的处理,要知道,原始的异步函数定义本身是下面的样式:
\begin{code-block}{rust}
async fn example(min_len: usize) -> String
\end{code-block}
由于完整的函数体是由状态机实现的(编译器内部/编译器视角),因此,编译器目前对原始
的异步函数做的唯一事情就是初始化状态机并返回它。编译器自动生成的相关代码可能如下:
\begin{code-block}{rust}
fn example(min_len: usize) -> ExampleStateMachine {
    ExampleStateMachine::Start(StartState { min_len, })
}
\end{code-block}
这个函数不再包含async关键字,取而代之的,则是返回状态机类型(ExampleStateMachine),
毫无疑问这个类型实现了Future Trait。如同上面的示例显示的一样,状态机在Start状态
当中被构造,并且使用min\_len参数初始化相应的状态结构体,一个Future在Rust编译器当
中是如何实现以及转换的大致过程就如同上述描述的过程。

\subsection{Future的pinning(锚点/固定)}
在之前的介绍当中已经提到了pinning,那pinning到底是什么。不过,先决条件是,需要了解
自引用结构(Self-Referential Structs)。所谓自引用结构,就是一个结构体当中包含了一个
字段,该字段直接引用了当前结构体当中的其他字段,或者该字段直接指向了当前结构体当中
的某个字段。

比如上面的Rust编译器实现的状态机,当其进行状态转换需要将每个暂停点的局部变量存储
在一个结构中,就属于一种简单的自引用结构的实际应用。对于类似于example这样的小函数,
一般不会出现问题,然而,当出现变量相互引用时,情况会变得更加的复杂,比如下面的代码:
\begin{code-block}{rust}
async fn pin_example() -> i32 {
    let array = [1, 2, 3];
    let element = &array[2];
    async_write_file("foo.txt", element.to_string()).await;
    *element
}
\end{code-block}

该函数创建了一个数组,然后创建了一个对数组最后一个元素的引用(地址),将其存储
在一个变量当中,并将该变量转换成字符串异步的写入一个文件当中,最后再返回该变量
所引用的数据。由于该函数只使用了一个await操作,因此,Rust编译器生成的状态机只有
3种:start,end和“wait on write”;由于该函数无参数传入,因此,start状态的结构体
是空的,与end状态的结构体类似,但是,“wait on write”的状态结构体会存在比较大的变化:
\begin{code-block}{rust}
struct WaitingOnWriteState {
    array: [1, 2, 3],
    element: 0x1001c, // array数组最后一个元素的地址
}
\end{code-block}

因为要返回array的元素值,并且返回的元素由element所代表的地址所引用,因此编译器
需要同时存储结构体的array和element。由于element本身代表了一个地址引用,存储的是
一个指向被引用元素的指针(内存地址),在这里使用0x1001c作为示例内存地址。该地址
指向数组array的最后一个元素,因此,它的实际内容取决于结构体在内存当中的位置。像
这种具有这种指向内部的结构体被称为\colorblock{自引用结构},这种结构体可以使用自身的一个字段
来引用或者指向自己。

一般情况下,自引用结构在使用时和普通的结构并没有太大的区别,但是它在内存排列和管理
上,会带来一些问题,下图是上面的自引用结构体的内存示意图:
\begin{figure}[H]
  \centering
  \includesvg[inkscapelatex=false, width=\linewidth]{self-referential-struct}
  \caption{自引用结构的内存分布示意图\protect\footnotemark}
  \label{fig:self-referential-struct}
\end{figure}
\footnotetext{内存分布:\url{https://os.phil-opp.com/async-await/self-referential-struct.svg}}
如图,结构体当中的array的首地址是0x10014(每个元素4个长度),而element字段的地址
是0x10020,但是element存储的地址是0x1001c(即array最后一个元素的地址)。一切都表现
得很好。不过,一旦结构体在内存当中的地址发生了变化(比如将该结构体当作参数传递,
或者赋值给其他变量),结果就不一样了:
\begin{figure}[H]
  \centering
  \includesvg[inkscapelatex=false, width=\linewidth]{self-referential-struct-moved}
  \caption{移动后的内存分布示意图\protect\footnotemark}
  \label{fig:self-referential-struct-moved}
\end{figure}
\footnotetext{移动后的内存分布:\url{https://os.phil-opp.com/async-await/self-referential-struct-moved.svg}}
字段array的首地址已经变成了0x10024,但是,element字段当中存储的地址还是0x1001c,
这就导致使用element字段引用array的元素时,指向的地址是一个未经初始化的地址,即
指针悬空(相当于C/C++当中常见的野指针),这会直接导致在状态机的下一次轮询的时候
发生未定义的行为。

很显然,这种行为在Rust当中是不安全的,也是编译器严格禁止的,因此需要解决这种不
安全的行为。从现有的理论和实践当中,主要的做法有3种:
\begin{itemize}
  \item 移动时更新内部的指针

  其思想是在结构体在内存中移动时更新内部指针,以便在移动后它仍然有效,比较类似于
  C/C++当中对链表的删除/插入操作。但是,这种做法需要对Rust进行大量的修改,并且会
  造成较大的性能损失,其原因就在于需要跟踪所有结构字段的类型,并检查每个移动操作
  是否需要进行指针更新。

  \item 存储偏移地址而非绝对地址

  即不进行内部指针的更新,内部指针存储的不再是绝对的地址,而是从结构体首地址开始的
  偏移量。由于整体移动结构时,结构内部的地址排列和偏移并不会发生改变,因此,无需
  进行内部指针的更新。但是,这种实现方式要求编译器检查所有的自引用,因为引用的值
  可能来自于用户的输入,因此,这种检查方式几乎是不可能的。

  \item 禁止结构体移动

  如同上面所说,只有在内存当中发生了自引用结构体的移动才会出现悬空指针,当完全
  禁用了自引用结构上的移动操作,自然就避免了悬空指针的出现。这种方式可以在类型
  系统上直接实现。
\end{itemize}

由于第三种方式提供了零成本的抽象,没有增加额外的运行成本,因此被Rust采用,这就是
所谓的\colorblock{Pining(固定/锚点)}。

禁止内存地址移动,或者说固定内存地址,通常可以采取的一种方式是将其放在内存的堆区(heap),
因为堆区分配的值在大多数情况下已经有一个固定的内存地址,在调用allocate时被创建,
直到调用deacllocate将其释放,在此之间,指针指向的堆值将保持在相同的内存地址,
因此可以利用该特性在堆区上创建自引用数据结构:
\begin{code-block}{rust}
fn main() {
    let mut heap_value = Box::new(SelfReferential {
        // 使用0x00当作初始的地址
        self_ptr: 0 as *const _,
    });

    // *heap_value表示从Box指针当中取出真正的元素
    // &*heap_value表示对Box当中的真正元素进行引用(取地址)操作
    // as *const 表示将其转换成对应类型的指针
    let ptr = &*heap_value as *const SelfReferential;

    // 将自身的地址赋给自己的字段
    heap_value.self_ptr = ptr;
    println!("heap value at: {:p}", heap_value);
    println!("internal reference: {:p}", heap_value.self_ptr);
}
struct SelfReferential {
    self_ptr: *const Self,
}
\end{code-block}

结构体SelfReferential包含了一个字段self\_ptr,该字段指向实例本身的地址,构成了
一个自己对自己的引用,如果执行上述代码,可以发现,结构体实例(self)和结构体字段
(self\_ptr)的地址实际上是一样的:
\begin{figure}[H]
  \centering
  \includegraphics[width=\linewidth]{self_ref_on_heap.png}
  \caption{堆区实现自引用结构}
  \label{fig:self_ref_on_heap}
\end{figure}

这是一个有效的自引用结构,并且,由于heap\_value只是一个指针,对它的移动操作(
比如赋值给其他变量,或者作为参数传递)并不会改变结构本身的地址,所以,self\_ptr
会一直保持有效。但是,这种“固定效果”很容易被打破或者破坏,比如将Box包裹的数据解引用
或者直接替换:
\begin{code-block}{rust}
use std::mem;

fn main() {
    let mut heap_value = Box::new(SelfReferential {
        self_ptr: 0 as *const _,
    });
    let ptr = &*heap_value as *const SelfReferential;
    heap_value.self_ptr = ptr;
    println!("heap value at: {:p}", heap_value);
    println!("internal reference: {:p}", heap_value.self_ptr);

    // break it
    let stack_value = mem::replace(&mut *heap_value, SelfReferential {
        self_ptr: 0 as *const _,
    });

    println!("value at: {:p}", &stack_value);
    println!("internal reference: {:p}", stack_value.self_ptr);
    let hp = *heap_value;
    println!("heap value at: {:p}", &hp);
    println!("internal reference: {:p}", &(hp.self_ptr));
}
struct SelfReferential { self_ptr: *const Self, }
\end{code-block}
则可以观察到,对应的结构体的内存地址全部发生了变化:
\begin{figure}[H]
  \centering
  \includegraphics[width=\linewidth]{self_ref_on_heap_break.png}
  \caption{自引用结构的内存破坏}
  \label{fig:self_ref_on_heap_break}
\end{figure}

\begin{warn}
这种操作是比较危险的,利用mem::replace函数,将原本由堆分配的值替换成新的结构体实例,
即将原始的heap\_value移动到了栈区,此时,self\_ptr变成了悬空指针,仍然指向了原始的
堆区地址,因此,使用堆区分配的方式,并不足以保障自引用结构的安全。
\end{warn}

造成上述问题的根本原因在于Box<T>允许开发者获得一个针对在堆上分配的T的\&mut T的引用,
这使得利用诸如mem::replace/mem::swap这样的函数通过\&mut T实现对堆区数据进行修改
成为可能,从而使得在堆区分配的值失效。为此,针对自引用结构,禁止创建\&mut T这样
的引用是必须的。

Rust的Pinging API通过使用Pin Wrapper类型以及Unpin这个标记型Trait来提供了一种可靠
的解决方案。这种解决方案背后的思路是,搜集所有Pin或者需要进行固定的结构体/类型的
特定方法——这些特定的方法可以通过获取可变的引用(\&mut)来获取被Pin Wrapper包括的
实际类型的值——并将这些方法在Unpin Trait上进行实现;Unpin Trait是一个自动的Trait,
除了那些需要进行显式的选择退出类型,其他所有的类型都实现了它;通过标记自引用结构是
实现的显式选择退出Unpin,这将导致没有任何一个安全(safe)的方法可以从Pin<Box<T>>
类型当中获取一个合法有效的可变引用(\&mut T),因此,这种方式实现的自引用结构,
其内部可以保证内部的自引用一直有效。

利用这种思路,我们可以将之前编写的自引用结构进行改写:
\begin{code-block}{rust}
use core::marker::PhantomPinned;
struct SelfReferential {
    self_ptr: *const Self,
    _pin: PhantomPinned,
}
\end{code-block}

通过在结构体当中插入一个类型为PhantomPinned的字段,我们实现了结构体的选择退出。
PhantomPinned类型是一个零字节大小(不占据内存空间)的标记类型,其作用在于标记对应
的数据类型不实现Unpin Trait。由于Auto Trait的工作机制,该类型(PhantomPinned,
即没有实现Unpin)的标记字段足以影响整个结构体,使之不受Unpin Trait的影响。另外
需要改动的地方,则是将上述的Box<SelfReferential>修改为Pin<Box<SelfReferential>>
类型,实现这个改动,最简单的方式就是用Box::pin,如下:
\begin{code-block}{rust}
let mut heap_value = Box::pin(SelfReferential {
    self_ptr: 0 as *const _,
    _pin: PhantomPinned,
});
\end{code-block}
由于PhantomPinned是零字节大小的类型,因此只需要使用类型名初始化对应的字段即可。
再来看看修正之后的完整代码:
\begin{code-block}{rust}
use std::mem;
use std::marker::PhantomPinned;

fn main() {
    let mut heap_value = Box::pin(SelfReferential {
        self_ptr: 0 as *const _,
        _pin: PhantomPinned,
    });
    let ptr = &*heap_value as *const SelfReferential;
    heap_value.self_ptr = ptr;
    println!("heap value at: {:p}", heap_value);
    println!("internal reference: {:p}", heap_value.self_ptr);
    // break it
    let stack_value = mem::replace(&mut *heap_value, SelfReferential {
        self_ptr: 0 as *const _,
        _pin: PhantomPinned,
    });
    println!("value at: {:p}", &stack_value);
    println!("internal reference: {:p}", stack_value.self_ptr);
}
struct SelfReferential {
    self_ptr: *const Self,
    _pin: PhantomPinned,
}
\end{code-block}
如果对上述代码进行编译,编译器将提示下列的错误:
\begin{figure}[H]
  \centering
  \includegraphics[width=\linewidth]{self_ref_pinned.png}
  \caption{Pinned的自引用结构}
  \label{fig:self_ref_pinned}
\end{figure}

这些错误实际上都是由于Pin<Box<SelfReferential>>这种数据类型包含了PhantomPinned这种
特殊的标记数据类型,从而使得整个类型不再实现DerefMut这个Trait,进而导致了错误的发生。
但这实际上正是我们想要的结果:因为DerefMut Trait本身就会返回一个\&mut T的可变引用,
现实的需求要求这种情况的发生。

但是,这也带来了一个额外的负面作用:无法对结构体当中的self\_ptr进行有效的初始化。
这是因为编译器无法区分\&mut引用的有效和无效使用。为了让初始化操作正常进行,使用不安
全的get\_unchecked\_mut方法便成了不多的选择之一:
\begin{code-block}{rust}
unsafe {
    let mut_ref = Pin::as_mut(&mut heap_value);
    // mut_ref.get_unchecked_mut().self_ptr = ptr; 效果和下面相同
    Pin::get_unchecked_mut(mut_ref).self_ptr = ptr;
}
\end{code-block}

get\_unchecked\_mut函数是Pin<\&mut T>而不是Pin<Box<T>>上的函数,因此需要进行一次
转换(as\_mut),并且该函数返回一个\&mut T,因此,可以利用其返回结果进行修改。
到目前为止,还剩下的错误则是replace操作,但该操作试图将在堆区上分配的值移动到栈区,
这会直接破坏存储在self\_ptr当中的自引用。使用Pin以及退出Unpin则可以防止这个移动操作,
从而安全的使用自引用结构。不过,也应当看到,在目前的Rust版本当中,由于编译器还不能
证明或者解决自引用的创建/初始化是安全的,因此,还需要上述的unsafe操作块。

虽然上述的方式可以很好的解决我们所面临的问题,不过由于所有的操作都是在堆上进行,
而堆区内存的分配显而易见的带来了性能损耗。所幸,Rust的Pinning API也允许创建Pin<\&mut T>
这种在内存栈上分配的实例。和Pin<Box<T>>拥有T类型的所有权不同,Pin<\&mut T>只是
针对T的一个临时引用,这使得在内存栈上的Pin变得更加复杂。更重要的是,一个Pin<\&mut T>
必须保证被引用的T在整个生命周期当中都保持固定,但是,这对于基于栈内存的变量是难以
验证的,因此,通常情况下,\colorblock{并不推荐在栈内存上使用Pinning API进行变量固定}。

有了以上的了解之后,利用Pin对之前的示例进行改写:
\begin{code-block}{rust}
use std::mem;
use std::marker::PhantomPinned;
use std::pin::Pin;
use std::future::{self,Future};
use std::task::{Context, Poll};
struct StringLen<F> {
    inner_future: F,
    _pin: PhantomPinned,
}
impl<F> Future for StringLen<F> where F: Future<Output = String> {
    type Output = usize;
    fn poll(mut self: Pin<&mut Self>, cx: &mut Context<'_>) -> Poll<Self::Output> {
        match self.as_mut().poll(cx) {
            Poll::Ready(s) => Poll::Ready(s),
            Poll::Pending => Poll::Pending,
        }
    }
}
fn string_len(string: impl Future<Output = String>)
    -> impl Future<Output = usize>
{
    StringLen {
        inner_future: string,
        _pin: PhantomPinned,
    }
}
fn file_len() -> impl Future<Output = usize> {
    let file_content_future = async_read_file("foo.txt");
    string_len(file_content_future)
}
fn async_read_file(name: &str) -> impl Future<Output = String> {
    future::ready(String::from(name))
}
\end{code-block}
虽然这只是一个对Future的简单模拟,但总算是可以编译通过了。正如在上面看到的一样,
async/await创建的实例通常是自引用结构,通过将实例(self)封装在Pin当中,使之退出
Unpin,可以确保对应的Future在轮询调用当中不会在内存当中发生移动,从而确保了所有
的内部引用仍然有效。

不过需要注意的是,在第一次调用轮询poll之前对future进行移动是可以的。这是因为Future
属于懒加载的一种模式,在第一次调用轮询前什么也不会做。因此,编译器生成的状态机
通常情况下只包含了函数参数,但是不包括内部引用。为了轮询调用poll,首先需要将Future
封装到Pin当中,来保证其在内存当中不会发生移动。另外,由于栈区的pinning更加难以操作,
通常情况下,使用Box::Pin配合Pin::as\_mut是一个最优选择。

\subsection{执行器(Executors)与唤醒器(Wakers)}
使用async/await,可以以完全异步的方式处理Future。不过,正如上面所介绍的,Future
在被轮询之前什么也不做。这意味着我们必须在某些时候对它们调用轮询(即唤醒),否则
异步代码永远也不会执行。

对于单个Future,我们总是可以使用上面描述的循环手动等待每个future。然而,这种方法
效率非常低,并且对于包含了大量Future的程序来说也不实用。这个问题最常见的解决方案
是定义一个全局执行器,该执行器负责轮询系统中的所有后续操作,直到它们完成为止。

执行器的目的在于允许Future成为一个独立的任务,然后对这些Future进行轮询,直到这些
Future(任务)全部完成为止。因此,全局的执行器最大的优点在于:当Future返回Poll::Pending
时,执行器可以在不同的Future之间进行切换。由于异步操作总是以并行/并发的方式执行的,
因此CPU通常一直处于忙碌状态,提高了资源的利用率。

由于Rust的标准库并没有提供Future的标准执行器,因此,执行器绝大部分是第三方的类库
进行实现的。这些类库在实现执行器时,通常利用了现代的多核技术,一般是创建一个线程
池来充分利用所有的CPU核。为了避免一次次轮询Future带来的开销,执行器通常和Future
支持的唤醒器(Waker)一起工作。

唤醒器的思路是,一个特殊的waker类型当作参数封装在Context类型当中传递给poll调用,
并且这个类型由执行器创建,异步任务可以使用这个waker表示任务结束。因此执行器可以
不再针对Poll::Pending的Future调用poll操作,而是等待相应的waker通知,比如之前的
简单例子:
\begin{code-block}{rust}
async fn write_file() {
    async_write_file("foo.txt", "Hello").await;
}
\end{code-block}
函数async\_write\_file异步地将字符串“Hello”写入foo.txt文件,由于硬盘写入需要一些
时间,因此针对Future的第一个poll调用可能会返回poll::Pending。不过,硬盘驱动程序
可以在内部存储传递给poll调用的Waker,并在文件被写入磁盘时使用它通知执行程序(执行者,Executor)。
通过这种方式,执行器在收到唤醒器通知之前就不需要浪费任何时间再次对Future进行poll调用。

虽然Rust并没有自己提供的标准执行器和唤醒器,但是可以利用Rust提供的标准库,自己
实现一个简单的执行器和唤醒器!在实现自己的执行器和唤醒器之前,我们需要了解一下多任务。
在操作系统当中,并行/并发都涉及到了多任务处理。而多任务处理主要有2种方式:抢占式
多任务以及协作式多任务。抢占式多任务依赖于操作系统在运行的任务之间强制切换,而
协作式多任务则要求任务通过yield或者同类型的操作定期自动放弃对CPU的控制。协作方法
的最大优点是任务可以自己保存状态,从而实现更有效的上下文切换,并使任务之间共享
相同的调用堆栈成为可能。而Rust的Future和async/await是合作多任务模式,因此,在实
现自己的执行器和唤醒器时,需要注意下列的需求:
\begin{itemize}
  \item 每一个添加到执行器当中future都是协作式任务(task,有些时候会将future直接当作task看待)
  \item Future不是通过yield操作,而是通过返回Poll::Pending或Poll::Ready来放弃对CPU的控制权
  \begin{itemize}
    \item 没有任何操作可以强制future放弃CPU的控制权,除非future永远不从poll操作返回,比如在无限循环当中自旋(spinning lock)
    \item 每一个future都可以或者可能阻塞执行器当中的其他future的执行过程,需要保证每个future都不是“恶意的”
  \end{itemize}
  \item Future会在内部存储所有在下一次poll调用时会被使用到的状态/状态。通过async/await,编译器会自动检测所需要的所有变量,并将其存储在生成的状态机当中
  \begin{itemize}
    \item 通常只保存所需要的最少变量
    \item 由于poll操作返回时会放弃对应的堆栈区,因此可以使用相同的堆栈区去轮询其他的future
  \end{itemize}
\end{itemize}

在接下来的示例当中将使用标准库(std)来实现一个Future的完成运行流程,包括执行器和
唤醒器的实现。首先编写我们的异步函数:
\begin{code-block}{rust}
async fn async_number() -> u32 {
    42
}
async fn example_task() {
    let number = async_number().await;
    println!("async number: {}", number);
}
\end{code-block}

async\_number是一个异步函数,Rust编译器会将其在内部转换成状态机。由于该函数只是
返回一个数字,所以在第一次轮询操作(poll)结束时该函数最终会直接返回一个Poll::Ready(42)。
example\_task类似,只是该函数需要等待async\_number的返回结果。Future已经通过async/await
生成好了(example\_task返回的),如果想要运行这个future,则需要对它调用poll操作,
直到接收到Poll::Ready所代表的完成信号为止。因此,现在需要一个任务管理器:
\begin{code-block}{rust}
use std::future::Future;
use std::pin::Pin;
pub struct Task {
    future: Pin<Box<dyn Future<Output = ()>>>,
}
\end{code-block}
该任务(task)结构体是一个包裹类型,包裹了一个具有Pinned属性的、在堆上动态分配的动态Future
类型,并且,这个Future类型的输出类型(即关联类型Output)为空类型。该结构体有如下特点:
\begin{itemize}
  \item Future的关联类型返回(),表示任务(即future)不会返回任何结果,我们的目的仅仅只是使得这个future可以正常运行
  \item dyn关键字表示存储在Box当中的是一个Trait对象,意味着future的方法是动态分配的,这使得在Task类型当中存储不同类型的Future成为可能,也使得可以创建多个不同的任务(future)
  \item Pin保证包裹的对象不能在内存当中移动,防止状态机产生的自引用结构失效
\end{itemize}

接下来实现Task结构体的初始化(new)方法以及poll方法,由于需要使用Task结构体轮询
存储在内存当中的future,因此,poll方法是必需的:
\begin{code-block}{rust}
impl Task {
    pub fn new(future: impl Future<Output = ()> + 'static) -> Task {
        Task {
            future: Box::pin(future),
        }
    }
    fn poll(&mut self, context: &mut Context) -> Poll<()> {
        self.future.as_mut().poll(context)
    }
}
\end{code-block}
new方法接收一个关联类型为()的任意Future,并且通过Box::pin将其在内存当中固定,然后
封装在Task结构体当中进行返回。需要注意,这里需要一个static的生命周期,因为返回的
Task可以在任意时间存在,所以,Future也需要在任意时间内有效。

由于Future Trait的poll方法定义为\codeinline{rust}{fn poll(self: Pin<&mut Self>, cx: &mut Context<'_>) -> Poll<Self::Output>;},
因此,首先需要通过as\_mut方法将self.future从\codeinlinebg{rust}{Pin<Box<T>>}(Box::pin的返回)转换成
Pin<\&mut T>,然后对转换之后的future调用poll函数。由于Task::poll方法应该只由稍后
创建的执行器调用,因此需要将该函数设置为私有模式。

接下来构造一个简单的执行器:
\begin{code-block}{rust}
pub struct SimpleExecutor {
    task_queue: VecDeque<Task>,
}

impl SimpleExecutor {
    pub fn new() -> SimpleExecutor {
        SimpleExecutor {
            task_queue: VecDeque::new(),
        }
    }
    pub fn spawn(&mut self, task: Task) {
        self.task_queue.push_back(task)
    }
}
\end{code-block}
该执行器包含了一个双端队列,通过spawn方法在队列尾部插入新的任务,从队列头部弹出下一个
任务进行执行。

为了调用Future的poll方法,接下来需要创建一个Context类型的对象,这个对象封装了一个
唤醒器(Waker)来监听Future的完成状态。为此,首先创建一个RawWaker实例,它定义了
不同Waker方法的实现,然后使用Waker::from\_raw函数将其转换为Waker。
\begin{code-block}{rust}
fn dummy_waker() -> Waker {
    unsafe { Waker::from_raw(dummy_raw_waker()) }
}

fn dummy_raw_waker() -> RawWaker {
    fn no_op(_: *const ()) {}
    fn clone(_: *const ()) -> RawWaker {
        dummy_raw_waker()
    }
    let vtable = &RawWakerVTable::new(clone, no_op, no_op, no_op);
    RawWaker::new(0 as *const (), vtable)
}
\end{code-block}

RawWaker类型需要显式地定义一个函数虚表(vtable),这个虚表当中需要包含RawWaker被
复制(clone),唤醒(woken)以及销毁(dropped)时应该被调用的函数。这几个函数的
形式应当如下:
\begin{code-block}{rust}
clone: unsafe fn(*const ()) -> RawWaker,
wake: unsafe fn(*const ()),
wake_by_ref: unsafe fn(*const ()),
drop: unsafe fn(*const ()),
\end{code-block}
这个虚表的布局是由RawWakerVTable类型定义的,每个函数都接收一个*const()参数,它
基本上是一个类型擦除(不关心)的\&self指针,指向某个结构体,例如分配在堆上。使用*const()
指针而不是正确的引用的原因是RawWaker类型应该是非泛型的,但仍然应该支持任意类型。

一般情况下,RawWaker是为包装在Box或Arc类型中的某些在堆区分配的结构所创建的。
对于这类类型,可以使用Box::into\_raw之类的方法将Box<T>转换成*const T指针,而这种
类型的指针可以被转换为一个匿名*const()指针,并传递给RawWaker::new方法。由于每个
虚表函数都接收相同的*const()作为实参,因此函数可以安全地将指针强制转换回Box<T>
或者是一个\&T来操作它。但是这样的操作会涉及到指针的强制转换,因此整个
过程实际上是非常危险的,非常容易导致错误的行为。\colorblock{除非必要,不建议手动
创建RawWaker}。

在上述的代码当中,首先定义了2个内部函数no\_op和clone。这2个函数都接收一个*const()指针
作为参数,不同的是,no\_op不做任何处理和操作,而clone函数通过调用dummy\_raw\_waker
返回一个新的RawWaker。接下来使用这两个函数来创建一个最小的RawWakerVTable:clone函数
用于clone操作,no\_op函数用于所有其他操作。

虚表创建成功之后,再使用RawWaker::new函数来创建RawWaker。由于虚表函数并没有使用第一个
参数,因此,在创建RawWaker时,第一个参数使用了空指针作为代替(0 as *const())。

有了唤醒器,我们就可以在执行器上实现一个run方法来调用这个唤醒器。最简单的运行方法
就是在循环中反复轮询所有排队的任务,直到全部完成。注意,这并不是很有效,因为它没
有利用Waker类型的通知机制,不过这并不影响它的正常运行,如下代码所示:
\begin{code-block}{rust}
impl SimpleExecutor {
    pub fn run(&mut self) {
        while let Some(mut task) = self.task_queue.pop_front() {
            let waker = dummy_waker();
            let mut context = Context::from_waker(&waker);
            match task.poll(&mut context) {
                Poll::Ready(()) => {} // task done
                Poll::Pending => self.task_queue.push_back(task),
            }
        }
    }
}
\end{code-block}
该函数使用while-loop来处理task\_queue中的所有任务。对于每个任务,它首先通过
由dummy\_waker函数返回的Waker实例来创建一个Context类型的实例,然后对每个任务
调用poll进行轮询,如果poll方法返回poll::Ready,则任务结束,接着下一个任务;
如果任务仍然是Poll::Pending,我们将它再次添加到队列的后面,以便在后续循环迭代中
再次轮询它,直到该任务执行完毕。

接下来看一下完整的代码:
\begin{code-block}{rust}
use std::collections::VecDeque;
use std::future::Future;
use std::pin::Pin;
use std::task::{Context, Poll,Waker, RawWaker,RawWakerVTable};
pub struct Task {
    future: Pin<Box<dyn Future<Output = ()>>>,
}
impl Task {
    pub fn new(future: impl Future<Output = ()> + 'static) -> Task {
        Task { future: Box::pin(future), }
    }
    fn poll(&mut self, context: &mut Context) -> Poll<()> {
        self.future.as_mut().poll(context)
    }
}
pub struct SimpleExecutor { task_queue: VecDeque<Task>, }
impl SimpleExecutor {
    pub fn new() -> SimpleExecutor {
        SimpleExecutor { task_queue: VecDeque::new(), }
    }
    pub fn spawn(&mut self, task: Task) {
        self.task_queue.push_back(task)
    }
    pub fn run(&mut self) {
        while let Some(mut task) = self.task_queue.pop_front() {
            let waker = dummy_waker();
            let mut context = Context::from_waker(&waker);
            match task.poll(&mut context) {
                Poll::Ready(()) => {} // task done
                Poll::Pending => self.task_queue.push_back(task),
            }
        }
    }
}
fn dummy_waker() -> Waker {
    unsafe { Waker::from_raw(dummy_raw_waker()) }
}
fn dummy_raw_waker() -> RawWaker {
    fn no_op(_: *const ()) {}
    fn clone(_: *const ()) -> RawWaker { dummy_raw_waker() }
    let vtable = &RawWakerVTable::new(clone, no_op, no_op, no_op);
    RawWaker::new(0 as *const (), vtable)
}
async fn async_number() -> u32 { 42 }
async fn example_task() {
    let number = async_number().await;
    println!("async number: {}", number);
}
fn main() {
    let mut executor = SimpleExecutor::new();
    executor.spawn(Task::new(example_task()));
    executor.run();
}
\end{code-block}

到此为止,Future的整体运行机制以及执行器的实现思路已经明白无误的剖析在了面前,
并且上述代码全部使用Rust的标准库(std)实现的。从上面的例子可以看到,Rust的标准
库针对Future只保证下列的必需条件:
\begin{itemize}
  \item 标准的Future Trait
  \item 一个合乎常理的方法创建任务,并且可以通过async/await对Future进行暂停/恢复
  \item 标准的Waker Trait,可以唤醒暂停的Future
\end{itemize}
但是,\colorblock{并不包含异步任务的执行}。

\begin{note}
Future,Waker以及相关的Task,在标准库(std)当中存在,在核心库(core)当中也存在,
并且,std和core当中的定义和实现是一模一样的,因此,async/await以及Future等异步
操作不仅可以用于普通的Rust开发,也同样可以用于没有std模式的嵌入式开发,甚至是操作
系统开发。除此之外,其他常用的异步开发框架,比如futures,async-std等,其内部定义
的Future,Waker以及Task等Trait,和Rust std都是完全相同的。这些类库当中的相关Trait
完全等价,使用时无需进行转换。不过,确切的说,除了core当中的Future定义,
Future的定义只有2类:一种是std当中的,而另外一种则是future-rs提供的futures::future::Future,
其中future-rs当中的是最原始的定义,而为了给Rust提供异步支持,才将future-rs当中的
相关定义移入到了std当中,也就是说,std当中的Future实际上是future-rs的最小子集。
而同样的,async-std、tokio等类库当中的Future实现,实际上也是future-rs的重新导出。
\end{note}
\subsection{Future补遗}
\subsubsection{胖指针与虚表}
Rust当中也存在指针,但是存在一些特殊的指针,比如下面的代码:
\begin{code-block}{rust}
use std::mem::size_of;
trait SomeTrait {}
fn main() {
    info!("======== The size of different pointers in Rust: ========");
    info!("&dyn Trait:-----{}", size_of::<&dyn SomeTrait>());
    info!("&[&dyn Trait]:--{}", size_of::<&[&dyn SomeTrait]>());
    info!("Box<Trait>:-----{}", size_of::<Box<dyn SomeTrait>>());
    info!("&i32:-----------{}", size_of::<&i32>());
    info!("&[i32]:---------{}", size_of::<&[i32]>());
    info!("Box<i32>:-------{}", size_of::<Box<i32>>());
    info!("&Box<i32>:------{}", size_of::<&Box<i32>>());
    info!("[&dyn Trait;4]:-{}", size_of::<[&dyn SomeTrait; 4]>());
    info!("[i32;4]:--------{}", size_of::<[i32; 4]>());
}
\end{code-block}
如果执行上述代码,我们会发现每个指针的大小都不一样:
\begin{figure}[H]
  \centering
  \includegraphics[width=\linewidth]{fat_pointer.png}
  \caption{不同类型的指针}
  \label{fig:fat_pointer}
\end{figure}
有的是8字节大小(64位系统的默认指针大小),而有的则是16字节。这些16字节的指针,就是
所谓的胖指针(Fat Pointer)。胖指针通常携带了额外的信息:
\begin{itemize}
  \item \&[i32]以及[i32;4]
  \begin{itemize}
    \item 前8个字节是指向数组中第一个元素的实际指针,或 slice 引用的数组的一部分
    \item 后8个字节,则表示的是切片以及数组的长度
  \end{itemize}
  \item \&dyn Trait
  \begin{itemize}
    \item 前8个字节指向Trait对象的数据段(data)
    \item 后8个字节指向Trait对象的虚表(vtable)
  \end{itemize}
\end{itemize}
正是由于虚表和胖指针的存在,才使得Rust的Trait如同Java的Interface、C++的虚函数、
C的函数指针一样,可以针对不同的类型进行操作。我们可以直接使用代码来模拟Trait是如何
使用胖指针和虚表来实现动态分发(针对不同的类型使用相同的方法定义)的:
\begin{code-block}{rust}
use std::mem::transmute;
trait SomeTrait {
    fn add(&self) -> i32;
    fn sub(&self) -> i32;
    fn mul(&self) -> i32;
}
// 结构体的内存布局和对齐方式采用兼容C语言的模式进行
#[repr(C)]
struct FatPointer<'a> {
    // 使用data指向Trait的数据段
    data: &'a mut Data,
    // 指向Trait的vtable段。由于需要传入长度和对齐等字面值,
    // 因此最容易的方法是使用无符号数的指针
    vtable: *const usize,
}
struct Data {
    a: i32,
    b: i32,
}
fn add(s: &Data) -> i32 {
    s.a + s.b
}
fn sub(s: &Data) -> i32 {
    s.a - s.b
}
fn mul(s: &Data) -> i32 {
    s.a * s.b
}
fn main() {
    let mut data = Data { a: 3, b: 4 };
    // 使用vec构造一个类似的虚表结构
    // 虚表(vtable)是具有固定格式的特殊用途的指针数组,
    // 每个元素都有特殊的含义
    let vtable = vec![
        0,  // 指向Drop的指针,这里的0表示NULL/None,即在这个虚表当中没有实现drop函数
        6,  // 虚表的长度
        8,  // 虚表当中元素的对其方式,此处表示按照8字节对齐

        // 从第4个元素开始,表示的是Trait当中定义的字段/方法
        // 这里的字段顺序必须和Trait当中定义的顺序相同
        add as usize, // 函数指针
        sub as usize,
        mul as usize
    ];
    let fat_pointer = FatPointer {
        data: &mut data,
        // 将vtable转换成指针
        vtable: vtable.as_ptr(),
    };
    // 使用transmute将fat_pointer转换成 dyn SomeTrait的对象
    // 等同于 let res: &dyn SomeTrait = unsafe { transmute(fat_pointer) };
    let res = unsafe { transmute::<FatPointer, &dyn SomeTrait>(fat_pointer) };
    info!("{}", res.add());
    info!("{}", res.sub());
    info!("{}", res.mul());
}
\end{code-block}

实际上,上述代码的模拟与Rust标准库的Trait非常类似,可以参考TraitOjbect的定义
\footnote{TraitObject:\url{https://doc.rust-lang.org/std/raw/struct.TraitObject.html}}。
虽然Rust的源代码实现有些区别,但Trait的实现方式,其工作机制和模式,可以通过上述
的代码全部展现出来。

\subsubsection{Future的基本使用方式}
虽然对Future的原理和实现方式有了大致的了解,但是Future的使用方式,还没有进行介绍。
Future基本分为2大类:async/await修饰的普通函数,以及实现Future Trait的结构体。这
2种的类型的使用方式类似,唯一有区别的是,结构体需要实现Future Trait的相关方法,
如下的代码所示:
\begin{code-block}{rust}
use std::task::{Context,Poll};
use std::pin::Pin;
use futures::future::Future;
use futures::executor::block_on;

pub struct Number {
    number: u8,
    polled: bool,
}

impl Number {
    pub fn new(number: u8) -> Self {
       Self {
            number: number,
            polled: false,
       }
    }
}

// 结构体实现Future Trait
impl Future for Number {
    type Output = u8;
    fn poll(
       self: Pin<&mut Self>,
       cx: &mut Context<'_>,
    ) -> Poll<Self::Output> {
        let mut this = self.get_mut();
        if this.polled {
            Poll::Ready(this.number)
        } else {
            // 调用唤醒器,否则调用poll时会直接进入死循环
            cx.waker().wake_by_ref();
            this.polled = true;
            Poll::Pending
        }
    }
}

async fn get_num() -> u8 {
    18
}

async fn outer() -> u8 {
    get_num().await
}

fn main(){
    let num = Number::new(98);
    // 将num对象当作一个future使用
    let res = block_on(num);
    println!("{}", res);
    // 同上
    let num = async { Number::new(100).await };
    let res = block_on(num);
    println!("{}", res);
    // 将异步函数当作future使用,2种不同的使用方式
    //let num = async {get_num().await};
    let num = get_num();
    let res = block_on(num);
    println!("{}", res);
    // 将异步函数当作future使用,2种不同的使用方式
    //let res = async {outer().await};
    let res = outer();
    let res = block_on(res);
    println!("{}", res);
}
\end{code-block}

\section{异步编程的实现方案-async-std}
Async-std是一种Future执行器的实现方案。其基本的使用于futures类似,都是使用block\_on
设置异步任务的阻塞运行。相比于futures,async-std提供了一些扩展功能,比如异步的
计时器等。

常见的使用方式都是相同的:
\begin{code-block}{rust}
use async_std::task::block_on;
pub async fn async_hello() {
    info!("This is the async hello");
}

fn main() {
    let _ = block_on(async_hello());
}

\end{code-block}

不过,async-std也还提供了一种简便的方式,这种方式不再要求使用block\_on进行显式的
阻塞执行,但是,使用这种方式,要求启用async-std的特性
\footnote{关于async-std的特性说明:\url{https://docs.rs/async-std/1.9.0/async_std/\#features}}:
\begin{code-block}{toml}
[dependencies]
async-std = {version = "1.9.0", features = ["attributes"]}
\end{code-block}
只用启用了上述的特性,才可以如下进行使用:
\begin{code-block}{rust}
pub async fn async_hello() {
    info!("This is the async hello");
}

#[async_std::main]
async fn main() {
    async_hello().await;
}
\end{code-block}
上述方式相当于将main函数也当作一个异步函数,而该函数的执行,则是由async-std这个
运行时自己进行相关任务的调度。同样的,如果是在测试函数或者测试用例当中,也可以使用
这种方式:
\begin{code-block}{rust}
pub async fn async_hello() {
    info!("This is the async hello");
}

// 注意,同样需要启用attributes属性,并且,main需要修改为test
#[async_std::test]
async fn test_units() {
    async_hello().await;
}
\end{code-block}

如果是多个异步函数的执行,同样可以使用block\_on或者直接的await方式进行操作,但是,
需要注意的是,如果不加控制,不同的异步操作,默认情况下是串行执行的,比如下面的代码:
\begin{code-block}{rust}
use std::time::Duration;
use async_std;
pub async fn connect_db_fake() -> String {
    async_std::task::sleep(Duration::from_secs(1)).await;
    info!("This is the async connect_db_fake function");
    "connect_db_fake".to_owned()
}

pub async fn open_file_fake() -> String {
    async_std::task::sleep(Duration::from_secs(2)).await;
    info!("This is the async open_file_fake function");
    "open_file".to_owned()
}

fn main() {
    let db_fake = async_std::task::block_on(connect_db_fake());
    let file_fake = async_std::task::block_on(open_file_fake());
    ...
}
\end{code-block}
上述代码当中db\_fake和file\_fake并没有什么直接的关系,二者是完全可以并行执行的,
但是,实际的执行结果是,两行代码总计耗时3秒左右,并没有并行起来。如果想让这些
并没有关联关系的操作完全并行起来,则需要使用futures-rs提供的功能:
\begin{code-block}{toml}
[dependencies]
futures = "0.3.15"
async-std = {version = "1.9.0", features = ["attributes"]}
\end{code-block}
并行的代码修改如下:
\begin{code-block}{rust}
use futures;
fn main() {
    let now = time::Instant::now();
    let (fake_db, fake_file) =
        block_on(futures::future::join(connect_db_fake(), open_file_fake()));
    let elapsed = now.elapsed();
    assert_eq!("open_file", fake_file);
    assert_eq!("connect_db_fake", fake_db);
    println!("All of async function used {:#?} ", elapsed);
    assert!(elapsed > Duration::from_secs(1));
    assert!(elapsed < Duration::from_secs(3));
}
\end{code-block}
并行之后的代码,总体的执行时间取决于耗时最长的任务。当然,如果需要并行多个任务,
可以使用join2,join3,join4和join5等函数。针对常见的函数式编程,比如map-reduce
操作,则可以使用join\_all等方法:
\begin{code-block}{rust}
use futures::future::join_all;
use std::time::{self, Duration};
use std::sync::{Arc, Mutex};
use async_std;
pub async fn get_cities() -> Vec<String> {
    let cities = vec![
        "shanghai".to_owned(),
        "beijing".to_owned(),
        "chongqing".to_owned(),
    ];
    let city_vec = Arc::new(Mutex::new(vec![]));
    let _ = join_all(
        cities
            .into_iter()
            .map(|city| build_city(city_vec.clone(), city)),
    )
    .await;
    return city_vec.lock().unwrap().clone();
}

async fn build_city(city_vec: Arc<Mutex<Vec<String>>>, city: String) {
    async_std::task::sleep(Duration::from_secs(1)).await;
    city_vec
        .lock()
        .unwrap()
        .push(format!("Super City {}", city))
}

fn main() {
    let now = time::Instant::now();
    let res = block_on(get_cities());
    assert_eq!(
        vec![
            "Super City shanghai".to_owned(),
            "Super City beijing".to_owned(),
            "Super City chongqing".to_owned()
        ],
        res
    );
    let used = now.elapsed();
    assert!(used < Duration::from_secs(2));
}
\end{code-block}
注意,join\_all也是并行执行的,其耗费的时间同样取决于耗时最长的任务。

而如果只是需要并行任务当中的某一个或者某几个完成,则需要使用select操作:
\begin{code-block}{rust}
pub async fn fake_select_two() -> u8 {
    let future1 = async {
        // 该future一直是pending状态,永远不会返回
        future::pending::<()>().await;
        1
    };
    let future2 = async { future::ready(2).await };
    // 针对select操作,所有的future都必须使用pin_mut
    pin_mut!(future1);
    pin_mut!(future2);
    match future::select(future1, future2).await {
        // select返回(future的Output,另一个future)
        Either::Left((value1, _ignore_future)) => value1,
        Either::Right((value2, _ignore_future)) => value2,
    }
}
\end{code-block}

\begin{note}
Join,join\_all以及join系列的函数和join!这个宏,以及select函数和select!宏,都
包含在futures-rs这个类库当中(tokio当中也有),但是,这些函数以及宏定义,是兼容
目前所有的异步类库的。因此,可以在async-std以及tokio当中使用。
\end{note}

需要注意的是,异步任务(包括async-std、futures以及tokio等)通常是并行运行,但是
这个“并行”的基础是\colorblock{通过共享一个执行线程实现的}。这就意味着阻塞一个操作系统(std)
线程的操作,比如std::thread::sleep将停止所有共享该线程的任务的执行,无法和
async-std的并行模型很好的配合。因此,下列代码在执行过程当中,实际上是串行阻塞式
的运行的:
\begin{code-block}{rust}
task::block_on(async {
    // this is std::fs, which blocks
    std::fs::read_to_string("test_file");
})
\end{code-block}

针对文件系统的异步io,async-std提供了与std::fs兼容的async\_std::fs进行异步化:
\begin{code-block}{rust}
use async_std::fs::File;
use async_std::{io::{self, prelude::WriteExt}};
pub async fn async_file_opts() -> io::Result<()> {
    let mut file = File::create("a.txt").await?;
    Ok(file.write_all(b"hello world").await?)
}
\end{code-block}

同样是由于Rust的异步实现都是通过共享线程实现的,因此,必然会带来这样的问题:
有的异步任务属于定时任务,需要定期的执行,如果通过普通的sleep操作进行控制权
让出,必然会阻塞当前线程的运行,使得异步操作失败。针对这种问题,async-std提供
了自己的定时器解决方案,只不过,这个定时器方案需要启用async-std的unstable特性:
\begin{code-block}{rust}
[dependencies]
async-std = {version = "1.9.0", features = ["attributes", "unstable"]}
\end{code-block}
而代码则需要进行如下的变化:
\begin{code-block}{rust}
use std::time::Duration;
use async_std::stream::{self, StreamExt};

fn main() {
    // 生成一个定时器stream,用于定时发送提示
    let mut intvl = stream::interval(Duration::from_secs(1));
    let mut count = 0;
    // while let Some(_) = intvl.next().await {
    while let Some(_) = block_on(intvl.next()) {
        count += 1;
        if count > 5 {
            break;
        }
        println!("{} seconds elapsed ", count);
    }
}
\end{code-block}
通过在async-std的运行时内部利用stream实现一个定时器,可以很好的和当前线程进行
合理的分离,使得异步的定时任务成为可能。而在上面的代码当中,出现了一个关键字,
或者说是新的概念:stream。Stream实际上对应的是Rust当中迭代器的概念。在目前通用的
Rust异步编程类库当中,Stream的定义和实现上都是类似的,以futures-rs为例,其定义
大致如下\footnote{来源:\url{https://docs.rs/futures/0.3.16/futures/stream/trait.Stream.html}}
\begin{code-block}{rust}
pub trait Stream {

    type Item;

    fn poll_next(
        self: Pin<&mut Self>,
        cx: &mut Context<'_>
    ) -> Poll<Option<Self::Item>>;

    fn size_hint(&self) -> (usize, Option<usize>) { ... }
}
\end{code-block}
而async-std当中的Stream定义则稍微有些不同,不过,核心部分是类似的\footnote{async:\url{https://docs.rs/async-std/1.9.0/async_std/stream/trait.Stream.html}}:
\begin{code-block}{rust}
pub fn poll_next(
    self: Pin<&mut Self>,
    cx: &mut Context<'_>
) -> Poll<Option<Self::Item>>
\end{code-block}
并且,async-std也特意说明了,async-std当中的Stream实际上是对futures-rs当中的Stream的重新
导出(re-export),因此,二者在本质上并没有什么区别。和普通future相比,Stream在实现
上实际上并没有太大的区别,在编译器内部也是通过状态机或者生成器实现的。比如,一个
简单的Stream实现如下:
\begin{code-block}{rust}
use std::pin::Pin;

use async_std::prelude::*;
use async_std::stream;
use async_std::task::{Context, Poll};

fn increment(
    s: impl Stream<Item = i32> + Unpin,
) -> impl Stream<Item = i32> + Unpin {

    struct Increment<S>(S);

    impl<S: Stream<Item = i32> + Unpin> Stream for Increment<S> {

        type Item = S::Item;

        fn poll_next(
            mut self: Pin<&mut Self>,
            cx: &mut Context<'_>,
        ) -> Poll<Option<Self::Item>> {

            match Pin::new(&mut self.0).poll_next(cx) {
                Poll::Pending => Poll::Pending,
                Poll::Ready(None) => Poll::Ready(None),
                Poll::Ready(Some(item)) => Poll::Ready(Some(item + 1)),
            }
        }
    }

    Increment(s)
}

let mut s = increment(stream::once(7));
\end{code-block}
只不过,在真实的生产环境当中,stream的使用并没有这么复杂:
\begin{code-block}{rust}
// 下列2行任选其一即可
// use futures::stream::{self, StreamExt};
use async_std::stream::{self, StreamExt};

fn test_stream_future() {
    // futures-rs的用法
    //let mut streamf = stream::iter(1..=10);

    // async-std的用法
    let mut streamf = stream::from_iter(1..=10);

    let res = block_on(streamf.next());
    assert_eq!(Some(1), res);

    // 迭代器消费了一个元素
    let all_item = block_on(streamf.collect::<Vec<u8>>());
    assert_eq!(vec![2, 3, 4, 5, 6, 7, 8, 9, 10], all_item);
}
\end{code-block}
可以看到,futures-rs/async-std的stream和普通的迭代器基本没有太大的差别。而上面
的定时器,可以看作是async-std实现的一个stream特例。
\begin{critical}
默认情况下,针对有限的stream(即数据有限的迭代器),标准迭代器的所有方法都可以
使用,但是,针对无限的stream(即数据无限的迭代器),那些需要遍历所有stream当中
的元素的方法或者函数,则应当禁止使用,否则可能造成用于无法返回,比如下面的例子:
\begin{code-block}{rust}
// 构造一个无限的流,流当中的每个元素都是1
let ones = async_std::stream::repeat(1);
// 无法结束,因为min函数需要迭代流当中的所有元素
// 这是无法完成的操作
let least = ones.min().await.unwrap();
println!("The smallest number one is {}.", least);
\end{code-block}
\end{critical}

针对Rust常用的多线程通信方案channel,async-std以及futures-rs同样提供了对应的异步方案:
\begin{code-block}{rust}
use async_std::channel::{unbounded, RecvError, TryRecvError};
async fn test_async_channel() {
    let (sender, recver) = unbounded(); // 创建无缓冲的channel,如果是有缓冲
                                        // let (sender, recver) = bounded(2);
    assert_eq!(sender.send(1).await, Ok(()));
    assert_eq!(recver.recv().await, Ok(1));
    assert_eq!(recver.try_recv(), Err(TryRecvError::Empty));
    // 不管是unbounded还是bounded的channel,如果sender
    // 没有继续发送消息,也没有进行关闭,但是,继续使用
    // recver.recv() 操作,会导致对应的任务进入死锁状态
    // 无法继续向后执行,因此,下面的代码需要注意
    // assert_eq!(recver.recv().await, Ok(_));
    drop(sender);
    assert_eq!(recver.try_recv(), Err(TryRecvError::Closed));
    assert_eq!(recver.recv().await, Err(RecvError));
}
\end{code-block}
这些channel的使用方式和标准库当中的channel基本类似,唯一的区别可能就是他们是异步
方式。

\begin{note}
在async-std以及其他的异步执行器出来之前,使用最多的是future-rs。Future-rs提供了
Rust异步编程的所有完整实现,包括贡献了async/await关键字。不过,由于其他的异步执行器
发展比较快,生态比较完善,因此,相比较而言,future-rs使用并不是特别多。尤其是,
针对执行器,迭代器以及channel等的实现上,async-std以及tokio等都非常灵活和完善,
并且和future-rs的使用方式相兼容,因此,直接使用future-rs当作异步执行器并不是特别
常见,而是使用future-rs的其他功能。而async-std的目的是提供一个异步的Rust标准库(std)
的实现方案,在目前的版本当中,Rust的std方法,大部分都可以在\colorblock{async-std当中
找到对应的异步实现},因此可以考虑在大多数的异步环境当中,直接使用async-std替换
对应的标准库。
\end{note}

\section{异步编程的实现方案-tokio}
\section{异步编程的实现方案-mio}

\section{优秀的并发-Crossbeam}
默认情况下,Rust标准库的多线程并发是非常安全和方便的,但是,也存在一些特殊情况,
会导致标准库的多线程使用起来受到诸多的限制,比如,在递归函数当中使用多线程:
\begin{code-block}{rust}
use std::thread;
const THRESHOLD: usize = 4;
// 由于Rust的跨线程通信的限制,要求input参数必须是static的生命周期
pub fn find_max(input: &'static [i32]) -> Option<i32> {
    if input.len() <= THRESHOLD {
        return input.iter().cloned().max();
    }
    let middle = input.len() / 2;
    let (left, right) = input.split_at(middle);
    // 由于thread限制,必须使用move关键字
    let thread_left = thread::spawn(move || find_max(left));
    let thread_right = thread::spawn(move || find_max(right));
    let max_left = thread_left.join().unwrap().unwrap();
    let max_right = thread_right.join().unwrap().unwrap();
    Some(max_left.max(max_right))
}
fn main() {
    static ARRAY_REF: &[i32] = &[12, 3, 45, 98, 100, 23, 878, 8765, 123, -897, 866666, 1241];
    let res = find_max(ARRAY_REF);
    info!("The res is {:?}", res);
}
\end{code-block}
由于诸多的限制,上述代码当中,如果需要对多个数组进行排序,则这些数组必须使用static
关键字进行标识,无法处理普通的数组,并且最终会导致生成的二进制文件比较大。

除此之外,比如Rust的通道,只存在多生产者单消费者这一种模式,这也并不符合现实生活
当中的多生产者多消费者的模型。为了改进Rust的并行/并发,目前大多数的开发者使用
Crossbeam\footnote{\url{https://github.com/crossbeam-rs/crossbeam}}替代标准库的thread,
比如,上述的递归函数当中使用多线程,就可以修改为如下的模式:
\begin{code-block}{rust}
extern crate crossbeam;
pub fn find_max_crossbeam(input: &[i32]) -> Option<i32> {
    if input.len() <= THRESHOLD {
        return input.iter().cloned().max();
    }
    let middle = input.len() / 2;
    let (left, right) = input.split_at(middle);
    crossbeam::scope(|s| {
        let thread_left = s.spawn(|_| find_max_crossbeam(left));
        let thread_right = s.spawn(|_| find_max_crossbeam(right));
        let max_left = thread_left.join().unwrap().unwrap();
        let max_right = thread_right.join().unwrap().unwrap();
        Some(max_left.max(max_right))
    })
    .unwrap()
}
fn main() {
    static ARRAY_REF: &[i32] = &[12, 3, 45, 98, 100, 23, 878, 8765, 123, -897, 866666, 1241];
    let res = short_lived::find_max_crossbeam(ARRAY_REF);
    info!("The res is {:?}", res);
    let array = [
        12, 3, 45, 98, 100, 23, 878, 8765, 123, -897, 866666, 12411234,
    ];
    let res = short_lived::find_max_crossbeam(&array);
    info!("The res is {:?}", res);
}
\end{code-block}
通过这样修改的函数,不管是针对static生命周期的还是普通生命周期的数据,都能够自如的处理。

同样的,也可以对Rust标准库的通道(Channel)进行优化,此时,则需要配合使用
\href{https://github.com/crossbeam-rs/crossbeam}{Crossbeam-Channel}。比如下面的例子:
启动2个并行的通道,一个通道负责消息的生产发送,一个通道负责消息的接收和处理。

\input{rust_part_8}



