\chapter{Golang}

\section{代理设置}
由于Golang是google的项目,因此,有的公用类库是依赖于google的域名解析的,导致在
一些情况下,无法更新或者下载相关的类库代码。解决方式就是设置代理。
Golang下载代码主要是通过go和其他一些版本控制工具进行下载的,通常的,版本控制工具
选择的都是git。因此,设置代理的时候,需要针对go和git设置。以windows为例。
\begin{code-block}{bash}
set http_proxy=http://10.1.1.10:8123
git config --global http.proxy http://10.1.1.10:8123
# 如果是使用sock5代理,则可以使用下面的方式
git config --global http.proxy socks5://localhost:8588
\end{code-block}

设置完成之后,即可进行go get更新和下载。Linux环境类似。

\section{安装Golang的开发工具}
只有设置好代理之后,才能正常的安装开发golang所需要使用的开发工具。
\begin{code-block}{bash}
go get -u -v github.com/nsf/gocode
go get -u -v github.com/rogpeppe/godef
go get -u -v github.com/zmb3/gogetdoc
go get -u -v github.com/golang/lint/golint
go get -u -v github.com/lukehoban/go-outline
go get -u -v sourcegraph.com/sqs/goreturns
go get -u -v golang.org/x/tools/cmd/gorename
go get -u -v github.com/tpng/gopkgs
go get -u -v github.com/newhook/go-symbols
go get -u -v golang.org/x/tools/cmd/guru
go get -u -v github.com/cweill/gotests/
go get -u -v github.com/alecthomas/gometalinter
gometalinter --install -u
\end{code-block}

Golang代码补齐依赖于gocode,而gocode不是一个常驻的服务,也不是一个类似于
python或者c/c++一样的编译型的解释器。Gocode更类似于一个实时的代码分析服务器,
需要进行补齐时,访问gocode服务器,获取返回进行代码补齐。因此,最好是把gocode做成
一个常驻性的服务一直在后台运行,这需要对gocode的代码做部分的修改。
\begin{code-block}{bash}
cd go/src/github.com/nsf/gocode
git checkout -b backend
git revert e11212347fbcdc8a33e9955b141f250f4eb14e94
git commit
go build .
cp gocode.exe go/bin/
\end{code-block}

在windows下,后台程序一般是以服务的形式运行,所以,针对windows平台,我们可以通过
添加服务的方式添加gocode的常驻进程。
\begin{code-block}{bash}
sc create gocode binPath="c:\go\bin\gocode.exe"
\end{code-block}
然后在windows服务中,启动gocode即可。
