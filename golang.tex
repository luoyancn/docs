\chapter{Golang}

\section{代理设置}
由于Golang是google的项目,因此,有的公用类库是依赖于google的域名解析的,导致在
一些情况下,无法更新或者下载相关的类库代码。解决方式就是设置代理。
Golang下载代码主要是通过go和其他一些版本控制工具进行下载的,通常的,版本控制工具
选择的都是git。因此,设置代理的时候,需要针对go和git设置。以windows为例。
\begin{code-block}{bash}
set http_proxy=http://10.1.1.10:8123
git config --global http.proxy http://10.1.1.10:8123
# 如果是使用sock5代理,则可以使用下面的方式
git config --global http.proxy socks5://localhost:8588
\end{code-block}

设置完成之后,即可进行go get更新和下载。Linux环境类似。

\section{安装Golang的开发工具}
只有设置好代理之后,才能正常的安装开发golang所需要使用的开发工具。
\begin{code-block}{bash}
go get -u -v github.com/nsf/gocode
go get -u -v github.com/rogpeppe/godef
go get -u -v github.com/zmb3/gogetdoc
go get -u -v github.com/golang/lint/golint
go get -u -v github.com/lukehoban/go-outline
go get -u -v sourcegraph.com/sqs/goreturns
go get -u -v golang.org/x/tools/cmd/gorename
go get -u -v github.com/tpng/gopkgs
go get -u -v github.com/newhook/go-symbols
go get -u -v golang.org/x/tools/cmd/guru
go get -u -v github.com/cweill/gotests/
go get -u -v github.com/alecthomas/gometalinter
gometalinter --install -u
go get -u golang.org/x/crypto
go get -u golang.org/x/image
go get -u golang.org/x/mobile
go get -u golang.org/x/net
go get -u golang.org/x/oauth2
go get -u golang.org/x/sys
go get -u golang.org/x/text
\end{code-block}

Golang代码补齐依赖于gocode,而gocode不是一个常驻的服务,也不是一个类似于
python或者c/c++一样的编译型的解释器。Gocode更类似于一个实时的代码分析服务器,
需要进行补齐时,访问gocode服务器,获取返回进行代码补齐。因此,最好是把gocode做成
一个常驻性的服务一直在后台运行,这需要对gocode的代码做部分的修改。
\begin{code-block}{bash}
cd go/src/github.com/nsf/gocode
git checkout -b backend
git revert e11212347fbcdc8a33e9955b141f250f4eb14e94
git commit
go build .
cp gocode.exe go/bin/
\end{code-block}

在windows下,后台程序一般是以服务的形式运行,所以,针对windows平台,我们可以通过
添加服务的方式添加gocode的常驻进程。
\begin{code-block}{bash}
sc create gocode binPath="c:\go\bin\gocode.exe set propose-builtins true autobuild true close-timeout 43200"
\end{code-block}
然后在windows服务中,启动gocode即可。

\section{模块初始化}
每个golang的模块都有一个隐藏的方法init,用来进行模块的初始化。当然,我们还可以进行
初始化的定制。具体就像下面一样
\begin{code-block}{go}
func init() {
    fmt.Printf("OS: %s, Arch: %s", runtime.GOOS, runtime.GOARCH)
}
\end{code-block}

\section{range的使用}
range通常用来进行迭代列表或者字典,通常的使用规则如下表\nameref{tab:usage_of_range}
\begin{center}
  \rowcolors{2}{green!80!yellow!50}{green!70!yellow!40}
  \begin{tabularx}{\textwidth}{|X|X|X|}
  \hline
  表达式类型& 第一返回值& 第二返回值\\ \hline
  [n]Ele& 数组索引值& 数组元素 \\
  string& 字符串索引& 字符数组对应的值\\
  map[k]v& map的键 & map键对应的值\\
  chan E & chan的元素 & - \\ \hline
  \end{tabularx}
  \label{tab:usage_of_range}
\end{center}

具体的使用如下
\begin{code-block}{go}
ints := []int{1, 2, 3, 4, 5, 6}
for index, value := range ints {
    fmt.Printf("%d: %d\n", index, value)
}
for index := range ints {
    fmt.Printf("%d: %d\n", index, ints[index])
}
dict := map[string]int{"lucifer": 18, "titans": 24, "garuda": 36}
for key, value := range dict {
    fmt.Printf("key is %s and value is %d\n", key, value)
}
\end{code-block}

\section{goto的使用}
goto的用法和c/c++当中的一样,也可以用来实现for循环,具体如下。
\begin{code-block}{go}
func goto_loop() {
    index := 0
loop:
    if index < 10 {
        fmt.Printf("index: \t%d\n", index)
        index++
        goto loop
    }
}
\end{code-block}

\section{Golang的继承}
Golang和c一样,并没有类的概念,因此没有继承。但是,由于golang有结构体的存在,因此,
可以使用组合的方式来实现继承。
\begin{code-block}{go}
type User struct{
    name string
    age uint
    address string
}

func (this *User) GetName() string{
    return this.name
}

type Student struct{
    User
    class string
    score uint
    order uint
}
\end{code-block}

在上边的例子中,Student结构内嵌了一个User结构,其结果就是Student结构也存在name,age
address等等属性,并且,GetName方法同样对Student结构是适用的。

\section{Golang的命令行参数}
和Python一样,golang提供了命令行处理的类库。比较常用的就是flag。
例子如下:
\begin{code-block}{go}
var (
    flag_name = flag.String("name", "demo", "The name of user. String value")
    flag_age  = flag.Int("age", 18, "The age of user. Int value")
)

func show_usage() {
    fmt.Fprintf(os.Stderr, "Usage: %s [-name] [-age]\n\n", os.Args[0])
    fmt.Fprintf(os.Stderr, "Flags:\n")
    flag.PrintDefaults()
}

func main(){
    flag.Usage = show_usage
    flag.Parse()
    fmt.Println(*flag_name)
    fmt.Println(*flag_age)
}
\end{code-block}
