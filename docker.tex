\chapter{Docker}

\section{通过代理拉取docker镜像}
有的时候,docker的image repo是被墙掉的。因此,需要通过代理的方式拉取。
一般的,代理通常有socket5和http代理,但是docker,wget之类的一般只支持http代理。
因此,需要转换一下。

\subsection{设置socket5代理}
Socket5代理一般需要shadowsocks的支持。首先设置socket5代理,并且将socket5转换为
http代理
\begin{code-block}{bash}
dnf install python-shadowsocks polipo -y
cat >/opt/server.json<<EOF
{
    "server":"107.191.52.9",
    "server_port":8964,
    "local_address": "127.0.0.1",
    "local_port":1080,
    "password":"laozhang",
    "method":"aes-256-cfb"
}
EOF
sslocal -c /root/server.json

cat > /etc/polipo/config<<EOF
logSyslog = true
daemonise = false
pidFile = /var/run/polipo/polipo.pid
logFile = /var/log/polipo/polipo.log
proxyAddress = "0.0.0.0"
allowedClients = "0.0.0.0/0"
socksParentProxy = "localhost:1080"
socksProxyType = socks5
EOF

polipo -c /etc/polipo/config
\end{code-block}

通过以上的方式,就可以将socket5的代理转换为http代理。

\subsection{设置docker使用代理}
\begin{code-block}{bash}
vi /usr/lib/systemd/system/docker.service
[Unit]
Description=Docker Application Container Engine
Documentation=https://docs.docker.com
After=network.target
[Service]
Type=notify
Environment="http_proxy=http://127.0.0.1:8123"
ExecStart=/usr/bin/dockerd
ExecReload=/bin/kill -s HUP $MAINPID
LimitNOFILE=infinity
LimitNPROC=infinity
LimitCORE=infinity
TimeoutStartSec=0
Delegate=yes
KillMode=process
[Install]
WantedBy=multi-user.target

systemctl daemon-reload
systemctl restart docker
\end{code-block}

通过以上的步骤,就可以实现使用代理拉取docker镜像了。
