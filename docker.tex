\chapter{Docker}

\section{Docker的repo源}
Docker目前分为社区版和企业版(>1.13),而生产环境中特别是centos,还是建议使用以前
的老版本,这样不会损失太多的功能。
\begin{code-block}{bash}
# cenos的repo源
# 需要注意的是,在centos上安装docker时,一定要指定版本号
# yum install docker-engine-1.13.1-1.el7.centos docker-engine-selinux-1.13.1-1.el7.centos -y
[docker]
name=Docker
baseurl=https://yum.dockerproject.org/repo/main/centos/$releasever
enabled=1
gpgcheck=1
gpgkey=https://yum.dockerproject.org/gpg

# fedora的repo源
[docker]
name=Docker
baseurl=https://yum.dockerproject.org/repo/main/fedora/$releasever
enabled=1
gpgcheck=1
gpgkey=https://yum.dockerproject.org/gpg
\end{code-block}

\section{通过代理拉取docker镜像}
有的时候,docker的image repo是被墙掉的。因此,需要通过代理的方式拉取。
一般的,代理通常有socket5和http代理,但是docker,wget之类的一般只支持http代理。
因此,需要转换一下。

\subsection{设置docker使用代理}
\begin{code-block}{bash}
vi /usr/lib/systemd/system/docker.service
[Unit]
Description=Docker Application Container Engine
Documentation=https://docs.docker.com
After=network.target
[Service]
Type=notify
Environment="http_proxy=http://127.0.0.1:8123"
ExecStart=/usr/bin/dockerd
ExecReload=/bin/kill -s HUP $MAINPID
LimitNOFILE=infinity
LimitNPROC=infinity
LimitCORE=infinity
TimeoutStartSec=0
Delegate=yes
KillMode=process
[Install]
WantedBy=multi-user.target

systemctl daemon-reload
systemctl restart docker
\end{code-block}

通过以上的步骤,就可以实现使用代理拉取docker镜像了。

\section{自定义镜像}
Docker的repo中已经提供了比较多的可用镜像,但是,总有一些镜像是需要我们自己进行定制的。
如何从零开始定制呢?主要有几种方式:
\begin{itemize}
  \item 通过kickstart创建docker镜像
  \item 从虚拟机制作镜像。
\end{itemize}

\subsection{从虚拟机制作镜像}
从虚拟机制作镜像适用于任何linux操作系统,但是,制作出来的镜像由于包含kernel,man
手册等相关于docker无关的文件,因此,文件体积较大。好处在于非常通用。
\begin{outline}[enumerate]
  \1 安装一个minimal vm
  \1 修改操作系统的部分设置和属性
\begin{code-block}{bash}
dnf erase NetworkManager NetworkManager-glib NetworkManager-config-server -y
cat >>/etc/sysconfig/network-scripts/ifcfg-eth0<<-EOF
TYPE=Ethernet
BOOTPROTO=dhcp
NAME=eth0
ONBOOT=yes
DEVICE=eth0
EOF
sed -i 's/SELINUX=enforcing/SELINUX=disabled/g' /etc/selinux/config
systemctl stop firewalld
dnf clean all
\end{code-block}

  \1 打包相关文件
\begin{code-block}{bash}
tar --numeric-owner --exclude=/proc --exclude=/sys --exclude=/mnt \
    --exclude=/var/cache --exclude=/usr/share/{foomatic,backgrounds,perl5,\
    fonts,cups,qt4,groff,kde4,icons,pixmaps,emacs,gnome-background-properties,\
    sounds,gnome,games,desktop-directories}  \
    --exclude=/var/log -zcvf /mnt/rhel7.tar.gz.tar.gz /
\end{code-block}

  \1 导入docker repo
\begin{code-block}{bash}
cat rhel7.tar.gz | docker import - rhel7
\end{code-block}

\end{outline}

\subsection{通过kickstart创建docker镜像}
Kickstart创建的docker镜像文件体积小,启动迅速,比较适合。但是,他有一个比较糟糕的缺点,
就是redhat的宿主机只能制作redhat的docker镜像,无法制作ubuntu的镜像;如果需要制作ubuntu的
docker镜像,则需要切换到ubuntu的宿主机上。下面的例子以fedora宿主机为例。
\begin{outline}[enumerate]
  \1 安装制作docker镜像的依赖
\begin{code-block}{bash}
dnf install libguestfs-tools-c appliance-tools libguestfs-tools-c -y
\end{code-block}

  \1 编写一个kickstart文件如附件
\textattachfile{init.ks}{\textcolor{blue}{init.ks}}
%\par{\parshape0 \linewidth\textwidth
%\begin{mdframed}[topline=true, bottomline=true, leftline=true,
%                 rightline=true, backgroundcolor=lbcolor,
%                 userdefinedwidth=\textwidth]
%\inputminted[fontsize=\scriptsize,linenos=false,
%             breaklines=true]{bash}{init.ks}
%\end{mdframed}
%\par}

  \1 创建docker的image文件
\begin{code-block}{bash}
# 特别注意的是,需要在英文环境下进行操作,否则可能不成功
export LANG=en_US.utf8
appliance-creator -c init.ks -d -v -t /tmp -o /tmp/Fedora24 \
      --name "Fedora24" --release 24 --format=qcow2
\end{code-block}

  \1 导入镜像到docker repo当中
\begin{code-block}{bash}
virt-tar-out -a /tmp/Fedora24/Fedora24/Fedora24-sda.qcow2 / - | docker import - fedora24
\end{code-block}
\end{outline}

虽然kickstart文件不能完全跨平台,但是在fedora上,我们可以通过kickstart制作redhat的docker镜像。
具体过程如上,不再赘述。而redhat的kickstart文件如附件:
\textattachfile{cern.ks}{\textcolor{blue}{cern.ks}}

另外,Fedora官方提供了一些标准的最小化docker镜像的kickstart文件,具体可以查看
\url{https://pagure.io/fedora-kickstarts.git}。
%\begin{mdframed}[topline=true, bottomline=true, leftline=true,
%                 rightline=true, backgroundcolor=lbcolor,
%                 userdefinedwidth=\textwidth]
%  \inputminted[fontsize=\scriptsize,linenos=false,
%               breaklines=true]{bash}{cern.ks}
%\end{mdframed}

\subsection{通过yum/dnf创建docker镜像}
由于kickstart需要依赖于qemu-img及其它虚拟化的支持,有的时候,可能创建失败。因此,
我们可以采取更加通用的方式,来创建docker的镜像。以在centos7上制作cern的镜像为例。
\begin{code-block}{bash}
export cern=/tmp/cern
mkdir -p ${cern}/var/lib/rpm
rpm --root $cern --initdb

# 不能使用7.5以上的,否则会增加其他的依赖关系,增加image的大小
rpm --root ${cern} -ivh \
    http://linuxsoft.cern.ch/cern/centos/7.4.1708/os/x86_64/Packages/centos-release-7-4.1708.el7.centos.x86_64.rpm

# 如果在centos上进行build,可以追加cern的repo
# cp /etc/yum.repos.d/cern.repo ${cern}/etc/yum.repos.d/

sed -i '/enabled/d' ${cern}/etc/yum.repos.d/CentOS-Base.repo \
    ${cern}/etc/yum.repos.d/CentOS-CR.repo
sed -i '/gpgkey/aenabled=1' ${cern}/etc/yum.repos.d/CentOS-Base.repo  \
    ${cern}/etc/yum.repos.d/CentOS-CR.repo
yum -y --installroot=${cern} install yum rpm systemd openssh-server sudo epel-release --nogpgcheck
ssh-keygen -t dsa -f ${cern}/etc/ssh/ssh_host_dsa_key -N ""
ssh-keygen -t rsa -f ${cern}/etc/ssh/ssh_host_rsa_key -N ""
ssh-keygen -t ecdsa -f ${cern}/etc/ssh/ssh_host_ecdsa_key -N ""
ssh-keygen -t ed25519 -f ${cern}/etc/ssh/ssh_host_ed25519_key -N ""
cp ~/.bash_profile ${cern}/root/
cp ~/.bashrc ${cern}/root/
# 添加term环境变量,让clear命令可以正常使用
sed -i '/Source/iexport TERM=xterm-256color' ${cern}/root/.bashrc
yum -y --installroot=${cern} update --nogpgcheck
yum  --installroot=${cern} clean all
rm -rf ${cern}/var/cache/*
rm -rf ${cern}/usr/lib/locale/*
rm -rf ${cern}/usr/share/locale/*
rm -rf ${cern}/var/lib/yum/*
rm -rf ${cern}/tmp/*
rm -rf ${cern}/var/tmp/*

#如果是fedora,可以做部分的修改
export fedora=/tmp/fedora
mkdir -p ${fedora}/var/lib/rpm
rpm --root $fedora --initdb

#dnf -y --installroot=${fedora} install --releasever 28\
rpm --root ${fedora} -ivh \
https://dl.fedoraproject.org/pub/fedora/linux/releases/28/Everything/x86_64/os/Packages/f/fedora-release-28-1.noarch.rpm\
https://dl.fedoraproject.org/pub/fedora/linux/releases/28/Everything/x86_64/os/Packages/f/fedora-repos-28-1.noarch.rpm \
https://dl.fedoraproject.org/pub/fedora/linux/releases/28/Everything/x86_64/os/Packages/f/fedora-gpg-keys-28-1.noarch.rpm

dnf -y --installroot=${fedora} install dnf rpm systemd openssh-server
ssh-keygen -t dsa -f ${fedora}/etc/ssh/ssh_host_dsa_key -N ""
ssh-keygen -t rsa -f ${fedora}/etc/ssh/ssh_host_rsa_key -N ""
ssh-keygen -t ecdsa -f ${fedora}/etc/ssh/ssh_host_ecdsa_key -N ""
ssh-keygen -t ed25519 -f ${fedora}/etc/ssh/ssh_host_ed25519_key -N ""
cp ~/.bash_profile ${fedora}/root/
cp ~/.bashrc ${fedora}/root/
# 添加term环境变量,让clear命令可以正常使用
sed -i '/Source/iexport TERM=xterm-256color' ${fedora}/root/.bashrc
dnf -y --installroot=${fedora} erase xkeyboard-config xkeyboard-config acl shared-mime-info
dnf -y --installroot=${fedora} update
dnf --installroot=${fedora} clean all
rm -rf ${fedora}/var/cache/* ${fedora}/usr/lib/locale/* \
    ${fedora}/usr/share/locale/* ${fedora}/var/lib/yum/* \
    ${fedora}/tmp/* ${fedora}/var/tmp/* ${fedora}/usr/share/man/* \
    ${fedora}/usr/share/licenses/* ${fedora}/usr/share/doc/*

# 导入docker
tar -C cern -c . | docker import - cern

# 可以制作成tar包,提供给其他操作系统使用
# tar -cf cern.tar cern
# 以下操作在新节点执行
# tar -xvf cern.tar
# tar -C cern -c . | docker import - cern
\end{code-block}

这种方式不会出现问题,只是占用的空间稍大(1-5\%);但是,这种方式要求比较严格,
只能制作和当前操作系统一致的镜像,即不能在redhat上创建fedora的image;ubunt和redhat
系列的也无法相互制作。

\subsection{添加daemon支持}
通常的,docker服务都是单独的运行一个程序,无法在docker内部执行systemctl等命令,更没办法
登录的docker容器之后,通过init或者systemd的方式启动一个服务的守护进程。但有的时候,我们
需要将docker当作一个虚拟机使用,这就要求docker容器内部支持init或者systemcd。我们可以使用
自定义的镜像达成我们的目的。但是,我们需要先行build一个支持daemon的docker image。
Docker File 如下:
\begin{code-block}{bash}
FROM fedora

LABEL maintainer="zhangjl@awcloud.com" software="yum rpm systemd openssh-server sudo epel-release" \
distro="CentOS/Cern/RedHat" version="7.5.1804.5"

RUN echo "root:luoyan" | chpasswd &&\
(cd /lib/systemd/system/sysinit.target.wants/; \
for i in *; do [ $i == systemd-tmpfiles-setup.service ] || rm -f $i; done); \
rm -f /lib/systemd/system/multi-user.target.wants/*;\
rm -f /etc/systemd/system/*.wants/*;\
rm -f /lib/systemd/system/local-fs.target.wants/*; \
rm -f /lib/systemd/system/sockets.target.wants/*udev*; \
rm -f /lib/systemd/system/sockets.target.wants/*initctl*; \
rm -f /lib/systemd/system/basic.target.wants/*;\
rm -f /lib/systemd/system/anaconda.target.wants/*;\
systemctl enable sshd.service

EXPOSE 22 80 3306 5000 5672 6379 8000 8080 8773 8774 8775 8776 8888 9000 9292 9696 9999 11211 15672 35357 55672

CMD ["/usr/sbin/init"]
\end{code-block}

使用上述的docker file创建一个支持daemon的镜像
\begin{code-block}{bash}
docker build . -t cern -f Dockerfile
\end{code-block}

然后像使用虚拟机一样的使用docker容器。
\begin{code-block}{bash}
docker run -tdi --privileged -v /opt/shared/.ssh:/root/.ssh \
    -p 60022:22 --name cern --hostname cern cern
\end{code-block}

\section{搭建Docker的私有源}
目前,搭建docker的私有源,一般使用vmware的harbor进行。本例亦是如此。
\begin{code-block}{bash}
wget https://github.com/vmware/harbor/releases/download/0.5.0/harbor-offline-installer-0.5.0.tgz \
    -O /opt/harbor-offline-installer-0.5.0.tgz
cd /opt/
tar -zxvf harbor-offline-installer-0.5.0.tgz

cd /opt/harbor
vi harbor.cfg
hostname = 10.1.1.16
ui_url_protocol = http
email_server = smtp.exmail.qq.com
email_server_port = 465
email_username = notify@awcloud.com
email_password = r00tawcloud.
email_from = admin notify@awcloud.com
email_ssl = true
harbor_admin_password = luoyan
db_password = luoyan
sed -i -e 's/80:80/5000:80/g' -e 's/443:443/9999:443/g' docker-compose.yml
sed -i 's/$ui_url/$ui_url:5000/g' common/templates/registry/config.yml
./install.sh
\end{code-block}

然后使用浏览器,登录http://10.1.1.16:5000即可。

将网络上的公开的repo作为自己私有repo的镜像时,需要做如下的操作:
\begin{code-block}{bash}
docker pull rhel7/pod-infrastructure:latest
docker login -u admin -p luoyan http://10.1.1.16:5000
docker tag rhel7/pod-infrastructure:latest 10.1.1.16:5000/rhel7/pod-infrastructure:latest
# push 之前,需要保证10.1.1.16上有rhel7这个project存在
docker push 10.1.1.16:5000/rhel7/pod-infrastructure:latest
\end{code-block}

\section{查看远端Docker源的镜像列表}
查看远端的docker镜像有一个前提条件,就是远端也运行了docker,并且将2375端口开放给所有的网段进行访问。
\begin{code-block}{bash}
# 192.168.138.250 表示远端的docker地址
docker --host 192.168.138.250:2375 images
\end{code-block}

如果是查询registry的镜像,则需要使用api进行查询。
\begin{code-block}{bash}
# 查看registry中的仓库名
curl http://192.168.138.225:5000/v2/_catalog |python -m json.tool

查看mysql这个仓库的tags列表
curl http://192.168.138.225:5000/v2/mysql/tags/list |python -m json.tool
\end{code-block}

\section{更换docker的runtime}
由于Docker是所有容器共享内核,安全隔离性存在一些问题,因此虚拟机容器是一个比较好的选择。目前,可以提供虚拟机容器的平台
主要是hyper,clearcontainers以及2者的结合产物kata container。接下来以clearcontainers作为说明,以修改docker的
runtime。
\begin{code-block}{bash}
cat >/etc/yum.repos.d<<EOF
[clearcontainers]
name=Clear Containers 3.X
type=rpm-md
baseurl=http://download.opensuse.org/repositories/home:/clearcontainers:/clear-containers-3/Fedora_27/
gpgcheck=1
gpgkey=http://download.opensuse.org/repositories/home:/clearcontainers:/clear-containers-3/Fedora_27/repodata/repomd.xml.key
enabled=1
EOF
dnf -y install cc-runtime cc-proxy cc-shim

cat >/etc/docker/daemon.json<<EOF
{
    "default-runtime": "cc-runtime",
    "runtimes": {
        "cc-runtime": {
            "path": "/usr/bin/cc-runtime"
        }
    }
}
EOF

systemctl daemon-reload
systemctl restart docker
\end{code-block}
完成以上操作之后,利用docker进行新的容器创建,则会直接使用clear container作为runtime,不再使用docker本身的runtime了。
