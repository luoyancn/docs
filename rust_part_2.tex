\section{生命周期}
在之前的所有权一节,有这么一个函数示例:
\begin{code-block}{rust}
fn first_word(s: &str) -> &str {
    return &s[..];
}

fn copy_ref(s: &str) -> &str {
    // 也可以是&s,为啥?
    return s;
}
\end{code-block}
上述的函数都运行正常。对函数进行改造,改造成下列的样式:
\begin{code-block}{rust}
fn longest(x: &str, y: &str) -> &str {
    if x.len() > y.len() {
        x
    } else {
        y
    }
}
\end{code-block}
即,返回2个字符串当中最长的。如果对这样的代码进行编译,则会出现错误:
\begin{figure}[H]
  \centering
  \includegraphics[scale=0.2]{rust_strref_err.png}
  \caption{试图返回多个引用当中的某一个}
  \label{fig:rust_strref_err}
\end{figure}
错误表示,函数应该返回一个有生命周期的命名变量。错误的原因是,Rust编译器无法知道
函数返回的到底是x还是y的引用,无法确定对应的变量的生命周期。

Rust当中,针对引用和借用,有一个特殊的机制:借用检查器,其作用比较作用域来确保所
有的借用都是有效的。
\begin{code-block}{rust}
{
    let r;                      // ---------+-- 'a
    {                            //          |
        let x = 5;             // -+-- 'b  |
        r = &x;                 //  |       |
    }                            // -+       |
    println!("r: {}", r); // ---------+
}
\end{code-block}

其中'a表示变量r原本的作用域(生命周期),'b则表示变量x的有效作用域。进入'b作用域
之后,r变量引用了一个作用域为'b的变量x,当退出'b之后,x失去作用,导致作为x的引用
的r也失去作用,被回收,因此,上述代码无法进行编译:'b的作用范围比'a要小。

为了解决这类的问题,Rust引入了生命周期的操作。生命周期的定义通常使用'+名称的方式
进行定义,表示一个变量或者函数的有效范围,如下:
\begin{code-block}{rust}
&i32        // 引用
&'a i32     // 带有显式生命周期的引用
&'a mut i32 // 带有显式生命周期的可变引用
\end{code-block}
生命周期不仅可以用于变量,同样可以作用与函数和方法上:
\begin{code-block}{rust}
fn main() {
    let string1 = String::from("abcd");
    let string2 = "xyz";

    let res = longest(&string1, string2);
    println!("The result is {}", res);
    println!("The result is {}", res);
}

fn longest<'a>(x: &'a str, y: &'a str) -> &'a str {
    if x.len() > y.len() {
        x
    } else {
        y
    }
}
\end{code-block}
上述代码表示,参数列表当中的所有引用都必须拥有相同的生命周期'a,通过生命周期的限定,
上述代码可以正常编译,并且正常执行。需要注意,如果在参数上使用生命周期,则函数/方法
的前面,则必须加上生命周期,否则会提示参数列表当中的生命周期没有定义。

生命周期同样可以应用于结构体字段定义当中,如下:
\begin{code-block}{rust}
struct ImportantExcerpt<'a> {
    part: &'a str,
}
\end{code-block}

上述结构体的初始化,则可以直接使用字符串的引用进行实现:
\begin{code-block}{rust}
let i = ImportantExcerpt { part: "zhangjl" };
println!("{}", i.part);
\end{code-block}

对于带有生命周期的结构体,在使用的时候,尤其是函数定义和方法定义时,有一些必须
注意的细节:
\begin{outline}[enumerate]
\1 传入外部引用数据模式

使用这种模式,通常情况下,不需要对函数添加生命周期,和普通函数相同。不过,也可以
使用添加生命周期的完整形式:
\begin{code-in-enumerate}{rust}
fn init_struct(source: &str) -> ImportantExcerpt {
    return ImportantExcerpt { part: source };
}

// 使用生命周期的完整形式,实际上是上述函数的完整签名形式
// fn init_struct<'a>(source: &'a str) -> ImportantExcerpt<'a> {
//     return ImportantExcerpt { part: source };
// }

...

// 调用函数
let b = init_struct("luoyan");
\end{code-in-enumerate}
由于上述代码当中,结构体的变量的有效生命周期和外部引用的相同,因此,可以简化生命
周期的使用。

\1 使用函数局部变量

在这种方式下,由于局部引用变量的作用域有限,返回函数之后就不存在了,因此,必须使用
显式的生命周期,而显式的生命周期使用同样有2种形式:
\begin{code-in-enumerate}{rust}
fn init_struct<'a>() -> ImportantExcerpt<'a> {
    return ImportantExcerpt { part: "luoyan"};
}

// 使用静态生命周期,'static表示静态生命周期,为固定关键字
// fn init_struct() -> ImportantExcerpt<'static> {
//     return ImportantExcerpt { part: "luoyan"};
// }
\end{code-in-enumerate}

\1 实现Trait

包含有引用数据类型的结构体,也可以实现各种标准库的Trait。在实现Trait的时候,也
必须使用生命周期:
\begin{code-in-enumerate}{rust}
// 可替换成下面的代码
// impl<'a> fmt::Display for ImportantExcerpt<'a> {
// static可以替换为_
impl fmt::Display for ImportantExcerpt<'static> {
    fn fmt(&self, f: &mut fmt::Formatter) -> fmt::Result {
        write!(f, "{}", self.part)
    }
}
\end{code-in-enumerate}

\1 添加结构体方法

结构体存在引用数据类型,同样要求结构体的方法在实现时需要进行额外的处理,添加生命
周期的使用,同样的,结构体的方法可以使用命名生命周期,也可以使用固定生命周期:
\begin{code-in-enumerate}{rust}
// 使用命名生命周期的结构体方法声明
impl<'a> ImportantExcerpt<'a> {
    fn show(&self) {
        println!("{}", self.part);
    }

    fn reset(&mut self, other: &'a str) {
        self.part = other;
    }

    fn get(&self) -> &str {
        return self.part;
    }
}

// 使用固定生命周期的结构体方法声明
impl ImportantExcerpt<'static> {
    fn show(&self) {
        println!("{}", self.part);
    }

    fn reset(&mut self, other: &'static str) {
        self.part = other;
    }

    fn get(&self) -> &str {
        return self.part;
    }
}
\end{code-in-enumerate}

\end{outline}

在上述的代码当中,很多地方都使用了'static静态生命周期。这是一种特殊的生命周期,
能够存活于整个程序期间,所有的字符串字面值都拥有'static生命周期。但是,并不是
任何情况都建议使用static生命周期。

由于生命周期和泛型以及Trait都非常类似,不可避免的,有可能会遇到几者合用的的情况,
在使用的时候,需要将生命周期与泛型使用,分割开,并且,生命周期应当放在首位。
\begin{code-block}{rust}
fn longest_with_an_announcement<'a, T>(x: &'a str, y: &'a str, ann: T) -> &'a str
    where T: Display
{
    println!("Announcement! {}", ann);
    if x.len() > y.len() {
        x
    } else {
        y
    }
}
\end{code-block}

\section{测试}
Rust的测试与其他语言相同,分为单元测试和集成测试。但不管是单元测试,还是集成测试,
在测试当中,都需要遵循相同的测试规则。在默认的lib类型的crate当中,默认情况下,自动
生成的lib.rs会生成如下的代码:
\begin{code-block}{rust}
#[cfg(test)]
mod tests {
    #[test]
    fn it_works() {
        assert_eq!(2 + 2, 4);
    }
}
\end{code-block}
其中,\#[cfg(test)]表示这是一个测试模块,而\#[test]则表示接下来的函数或者方法是测试
函数,it\_works表示测试的函数/方法名,可以变更为其他的名称。其中,assert!、assert\_eq!
和assert\_ne!这3个宏定义,用于检测运行结果、是否相等/是否不等,比如检测返回值当中
是否包含特定的字符串:
\begin{code-block}{rust}
pub fn greeting(name: &str) -> String {
    format!("Hello {}!", name)
}

#[cfg(test)]
mod tests {
    // 引用暴露的模块代码
    use super::*;

    #[test]
    fn greeting_contains_name() {
        let result = greeting("Carol");
        assert!(result.contains("Carol"));
    }
}
\end{code-block}

如果需要测试panic的代码,则可以使用should\_panic宏进行,该宏表示期望对应的函数在
运行的时候出现panic:
\begin{code-block}{rust}
pub struct Guess {
    value: i32,
}

impl Guess {
    pub fn new(value: i32) -> Guess {
        if value < 1 || value > 100 {
            panic!("Guess value must be between 1 and 100, got {}.", value);
        }

        Guess {
            value
        }
    }
}

#[cfg(test)]
mod tests {
    use super::*;

    #[test]
    #[should_panic]
    fn greater_than_100() {
        Guess::new(200);
    }
}
\end{code-block}
如果测试失败,想在测试结果当中,提示出具体的测试错误信息,则可以添加should\_panic
属性中的expected参数:
\begin{code-block}{rust}
#[cfg(test)]
mod tests {
    use super::*;

    #[test]
    #[should_panic(expected = "Guess value must be between 1 and 100")]
    fn greater_than_100() {
        Guess::new(200);
    }
}
\end{code-block}

运行测试用例时,只需要简单的输入如下的指令即可:
\begin{code-block}{bash}
// 默认并行的方式运行所有的测试用例
cargo test

// 串行的方式运行所有的测试用例
cargo test -- --test-threads=1

// 运行指定的测试用例,可匹配以add开头的所有测试用例
cargo test add
\end{code-block}

需要单独说明的是Rust的集成测试。集成测试通常针对lib型的crate。其测试过程大致如下:
\begin{outline}[enumerate]
\1 创建一个lib,并编写代码

\begin{code-in-enumerate}{bash}
cargo new --lib shared
\end{code-in-enumerate}

\1 在shared的src同级目录下,创建集成测试用例目录:
\begin{code-in-enumerate}{bash}
# 文件夹名称固定为tests
mkdir tests
\end{code-in-enumerate}

\1 在tests下创建集成测试用例
\begin{code-in-enumerate}{bash}
echo > tests/units.rs<<EOF
// 导入的lib名称必须是当前crate的名称
use shared;

#[test]
fn it_adds_two() {
    assert_eq!(4, adder::add_two(2));
}
EOF
\end{code-in-enumerate}
然后执行测试即可。
\end{outline}

\section{Rust的函数式编程}
Rust同样支持函数式编程。相比于其他语言,Rust的函数式编程性能和效率更高。Rust常见的
函数式编程模式包括闭包和迭代器2大类。

\subsection{闭包}
Rust的闭包和Python当中的非常类似,都可以直接读取外部的变量。其定义的形式基本如下:
\begin{code-block}{rust}
let expensive_closure = |num| {
    println!("calculating slowly...");
    num * 10
};

let res = expensive_closure(10);
\end{code-block}
其中两个||表示定义一个闭包,中间的num表示闭包的参数。如果闭包需要处理多个参数,则
应该改写为:
\begin{code-block}{rust}
let expensive_closure = |num1, num2| {
    num1 * num2
};
\end{code-block}

从实际的使用当中可以看到,Rust的闭包实际上就是一个匿名函数,在Rust当中,函数都有
参数类型/返回值的声明,但是,在上述的代码当中,却没有看到相关的定义和声明。这是
因为Rust的闭包通常很短,并只关联于小范围的上下文而非任意情境。在这些有限制的上下
文中,编译器有能力可靠的推断参数和返回值的类型,如同能够推断大部分变量的类型一样。
不过,不注明参数/返回类型,有可能出现一种迷惑性的使用:即无法传入正确的数据类型,
如下:
\begin{code-block}{rust}
let example_closure = |x| x;

let s = example_closure(String::from("hello"));
let n = example_closure(5);
\end{code-block}
按照上述代码的定义,example\_closure只是将输入参数原封不动的返回给调用者,第1次
调用时,编译器会将该闭包推断为输入/输出为字符串类型,然后这些类型信息会被锁定到
该闭包当中。后续再传入数值,由于闭包的类型已经锁定,要求传入字符串,但实际传入的
是数值,结果就会导致上述代码出现错误:
\begin{figure}[H]
  \centering
  \includegraphics[width=\linewidth]{rust_closure_diffrent_type.png}
  \caption{试图处理不同数据类型的闭包}
  \label{fig:rust_closure_diffrent}
\end{figure}

闭包的完整定义(包括类型)则如下:
\begin{code-block}{rust}
let live_closure = |num: i32| -> (i32, i32) {
    println!("calculating slowly...");
    thread::sleep(Duration::from_secs(2));
    (num * 10, num * 20)
    // 或者修改为return语句
    // return (num*10, num*20);
};

// 如果不需要返回值,则闭包的写法需要注意一下:
let other = |x| {
    println!("{}", x);
};
\end{code-block}

\subsection{特殊的闭包}
默认的情况下,包括Python和Golang,闭包都只是匿名函数。不过,在Rust当中,闭包可以
用在结构体当中,其主要用途就是memoization或lazy evaluation(惰性求值),即懒加载。
当结构体当中存放闭包时,则必须注明闭包的类型。而在结构体/枚举当中使用闭包,则需要
使用trait和泛型:Fn、FnMut和FnOnce。这3者的区别如下:
\begin{enumerate}
  \item FnOnce:闭包内对外部变量存在转移操作,导致外部变量不可用,所以只能call一次
  \item FnMut:闭包内对外部变量直接使用,并进行修改
  \item Fn:闭包内对外部变量直接使用,不进行修改
\end{enumerate}

使用这些trait的时候,则必须注明闭包的参数/返回值的类型。比如,闭包接收一个u32的
参数,返回一个u32,则对应的Fn trait bound则如下:
\begin{code-block}{rust}
Fn(u32) -> u32
\end{code-block}

一个包含闭包的结构体示例如下:
\begin{code-block}{rust}
struct Cacher<T>
where
    T: Fn(u32) -> u32,
{
    calculation: T,
    value: Option<u32>,
}
\end{code-block}
对该结构体的解读如下:结构体Cacher包含一个泛型calculation,而这个泛型则是一个使用
了Fn的闭包,这个闭包接收一个u32的参数,并最终返回一个u32。Value则是用于存放calculation
的计算结果,便于第二次调用时,直接返回而无需计算。根据上述需求,整个结构体的方法
实现如下:
\begin{code-block}{rust}
impl<T> Cacher<T>
where
    T: Fn(u32) -> u32,
{
    pub fn new(calculation: T) -> Cacher<T> {
        Cacher {
            calculation: calculation,
            value: None,
        }
    }

    pub fn value(&mut self, arg: u32) -> u32 {
        match self.value {
            Some(v) => v,
            None => {
                let v = (self.calculation)(arg);
                self.value = Some(v);
                v
            }
        }
    }
}
\end{code-block}
注意,在上述的结构体以及结构体方法当中,首次出现了trait bound和where的使用。需要
特别说明事实,trait bound几乎可以用于Rust的任何场景。New方法接收一个泛型作为初始化
参数,这个泛型就是一个Fn的闭包;而value方法则是根据根据当前结构体的数据,直接进行
数据的返回,或者计算,再返回。该结构体的使用方式如下:
\begin{code-block}{rust}
let mut cacher = Cacher::new(|x: u32| -> u32 { x * 10 });
let mut val = cacher.value(32);
println!("The val of cacher is {}", val);

val = cacher.value(45);
println!("The val of cacher second time is {}", val);
\end{code-block}
只是稍微可惜的是,这个表示缓存的结构体还存在bug,2次传入不同的数据,却得到了相同的
结果。问题在于字段value的定义。可以考虑使用Hashmap或者其他数据类型来替换value。一种
可能的解决方法如下:
\begin{code-block}{rust}
struct Cacher<T>
where
    T: Fn(u32) -> u32,
{
    calculation: T,
    value: BTreeMap<u32, Option<u32>>,
}

impl<T> Cacher<T>
where
    T: Fn(u32) -> u32,
{
    pub fn new(calculation: T) -> Cacher<T> {
        Cacher {
            calculation: calculation,
            value: BTreeMap::new(),
        }
    }

    pub fn value(&mut self, arg: u32) -> u32 {
        // 从现有的结果记录当中查询是否存在arg对应的计算结果
        match self.value.get(&arg) {
            // 找到则直接返回
            Some(Some(x)) => *x,
            // 没有找到,则计算一次,并放入当前的结果集合
            Some(None) | None => {
                let v = (self.calculation)(arg);
                self.value.insert(arg, Some(v));
                v
            }
        }
    }
}
\end{code-block}

闭包同样可以捕获运行环境的上下文,即在闭包内部直接使用外部的所有变量:
\begin{code-block}{rust}
fn main() {
    let x = 4;
    let equal_to_x = |z| z == x;
    let y = 4;
    assert!(equal_to_x(y));
}
\end{code-block}
X在闭包出现之前已经存在,定义闭包equal\_to\_x的时候,可以直接使用外部的x,而无需
重新声明。

\subsection{迭代器}
迭代器是Rust函数式编程的另外一个利器,负责遍历序列中的每一项和决定序列何时结束的
逻辑,我们在使用的时候,就无需判断开始条件和结束条件。在Rust当中,迭代器是惰性的,
只有使用到了,才会在内存当中进行展开。Rust的迭代器必须实现一个Iterator的triat,
这个trait的定义类似如下的结构:
\begin{code-block}{rust}
pub trait Iterator {
    type Item;
    fn next(&mut self) -> Option<Self::Item>;
    ...
}
\end{code-block}
其中的type Item和Self::Item定义了trait的关联数据类型,即该trait要求同时定义一个
Item类型,该类型被用作next方法的返回值类型。Next方法是Iterator被要求实现的唯一
方法,其一次返回一个项,最后返回一个None。

Rust的next方法得到的是迭代器的不可变引用,iter方法生成一个不可变引用的迭代器。
如果我们需要一个获取所有权并返回拥有所有权的迭代器,则可以调用into\_iter而不是iter。
类似的,如果我们希望迭代可变引用,则可以调用iter\_mut而不是iter;如果一旦调用了
into\_iter,则迭代完成之后,迭代器不再有效,比如下方代码:
\begin{code-block}{rust}
let v = vec![1, 2, 3];
let v3: Vec<_> = v.into_iter().map(|x| x * 12).collect();
println!("{:?}", v3);
println!("{:?}", v);
\end{code-block}
一旦进行编译,则会提示如下的错误:
\begin{figure}[H]
  \centering
  \includegraphics[width=\linewidth]{rust_iter_move.png}
  \caption{迭代器的所有权转移}
  \label{fig:rust_iter_move}
\end{figure}

实际上,上述的操作相当于对一个迭代器进行了消费。一般说来,调用next方法的方法被称为
消费适配器(consuming adaptors),因为调用他们会消耗迭代器。一个消费适配器的例子
是sum方法。这个方法获取迭代器的所有权并反复调用next来遍历迭代器,因而会消费迭代器。
当其遍历每一个项时,它将每一个项加总到一个总和并在迭代完成时返回总和。在这个过程
完成之后,原有的迭代器将无法再继续使用,因为其所有权已经进行了转移。
\begin{code-block}{rust}
let v = vec![1, 2, 3];
let v_item = v.iter();
let total1: u32 = v_item.sum();
// 迭代器v_item不再有效
println!("{:?}", v_item);
\end{code-block}

Iterator trait中定义了另一类方法,被称为迭代器适配器(iterator adaptors),允许
我们将当前迭代器变为不同类型的迭代器,并且可以链式调用多个迭代器适配器。不过因为
所有的迭代器都是惰性的,必须调用一个消费适配器方法以便获取迭代器适配器调用的结果。
比较常见的,就是使用map函数(迭代适配器,遍历迭代器的所有元素)来生成新的迭代器。
与之相对应的,collect方法则是消费迭代器并将结果收集到一个数据结构中。同样需要注意
的是,任何的迭代消费器,都不能进行类型的自动推导,需要手动的指定对应的数据类型。
比如,sum的结果通常是数值类型,而collect的结果则通常是vec类型。

迭代器和闭包通常结合使用,因为闭包可以捕获环境,比如常用的filter迭代器适配器:
\begin{code-block}{rust}
let v = vec![1, 2, 3];
// 使用的是iter,即引用数据类型,但是filter使用的本身是引用,因此,需要进行
// 2次的解引用操作
let res: Vec<_> = v.iter().filter(|s| *(*s) == 2).collect();
println!("{:?}", res);
// 原始的v仍然可用,没有发生所有权转移
println!("{:?}", v);

// 发生了所有权转移,变量v在后续的操作当中,无法被继续使用
let res1: Vec<_> = v.into_iter().filter(|s| *s == 2).collect();
println!("{:?}", res1);
\end{code-block}

Filter和迭代器使用的时候,需要特别注意所有权以及引用数据类型,特别是复合数据类型。
不同的操作会导致复合数据类型的所有权的变更。
\begin{code-block}{rust}
struct Shoe {
    size: u32,
    style: String,
}

fn main() {
    let shoes = vec![
        Shoe {
            size: 10,
            style: String::from("sneaker"),
        },
        Shoe {
            size: 13,
            style: String::from("sandal"),
        },
        Shoe {
            size: 10,
            style: String::from("boot"),
        },
    ];

    // 正确,返回的结果r实际上是shoes的部分数据的引用
    let r: Vec<_> = shoes.iter().filter(|x| x.size == 10).collect();

    // 错误,无法编译,由于collect返回的是引用,无法直接转换成引用原本的数据类型
    let r1: Vec<Shoes> = shoes.iter().filter(|x| x.size == 10).collect();

    // 正确,使用into_iter获取了相关的所有权,不再是引用,而是原始数据类型
    let r2: Vec<Shoes> = shoes.into_iter().filter(|x| x.size == 10).collect();
    // 在此之后,shoes变量无法再使用,所有权已经发生了变更

    // 错误,shoes的所有权已经发生了变更,此处已经无效
    shoes_in_my_size(shoes, 10);
}

// 调用者发生了所有权转移,调用该函数之后,参数shoes无法再被使用
fn shoes_in_my_size(shoes: Vec<Shoe>, shoe_size: u32) -> Vec<Shoe> {
    shoes.into_iter().filter(|s| s.size == shoe_size).collect()
}
\end{code-block}

\subsection{自定义迭代器}
可以通过在vector上调用iter、into\_iter或iter\_mut来创建一个迭代器,也可以用标准库
中其他的集合类型创建迭代器,比如哈希map。另外,可以实现Iterator trait来创建任何
我们希望的迭代器,如下:
\begin{code-block}{rust}
impl Counter {
    fn new(max: u32) -> Counter {
        return Counter {
            current: 0,
            max: max,
        };
    }
}

impl Iterator for Counter {
    type Item = u32;
    fn next(&mut self) -> Option<Self::Item> {
        self.current += 1;

        if self.current <= self.max {
            Some(self.current)
        } else {
            None
        }
    }
}
\end{code-block}
然后,即可像普通的集合数据类型Vec一样,使用for和next进行操作:
\begin{code-block}{rust}
let c = Counter::new(10);

// 忽略开头的n个数据
// for item in c.skip(1) {
// 像迭代器一样的使用类型
for item in c {
   println!("{}", item);
}

// 需要注意,c的所有权已经被转移,在此之后,无法再使用变量c

let c1 = Counter::new(10);
let c2 = Counter::new(20);

let sum: u32 = c1
    .zip(c2.skip(10))
    .map(|(a, b)| a * b)
    .filter(|x| x % 3 == 0)
    .sum();
println!("{}", sum);
\end{code-block}
上述的自定义迭代器并不完整,比如,默认情况下转移了变量的所有权,无法使用变量的引用
进行迭代等等。这些问题可以在后续进行进一步的改进。

\section{智能指针}
Rust当中同样存在指针,最常用的指针就包括引用数据类型。除了引用数据之外,引用类型
没有其他任何特殊的操作,也不存在其他额外的开销。除此之外,Rust还拥有智能指针,
这是一种数据结构,其表现类似于真正的指针,但是,拥有额外的元数据和功能。普通的引用
只是借用数据,而智能指针则是拥有指向的数据。

常见的智能指针包括String以及Vec<T>,通常使用结构体实现。和普通结构体明显区别的是,
智能指针实现了Deref和Drop这2个trait。Deref trait允许智能指针结构体实例表现的像引
用一样,这样就可以编写既用于引用、又用于智能指针的代码;Drop trait允许我们自定义
当智能指针离开作用域时运行的代码。在标准库当中最常用的智能指针主要包含下列3种:
\begin{enumerate}
  \item Box<T>:用于在堆上分配值
  \item Rc<T>:引用计数类型,其数据可以有多个所有者
  \item Ref<T> 和 RefMut<T>:通过RefCell<T>访问,RefCell<T>是一个在运行时而不是在编译时执行借用规则的类型
\end{enumerate}

\subsection{使用Box指向内存堆上的数据}
最简单直接的智能指针是box,其类型是 Box<T>。Box允许你将一个值放在堆上而不是栈上,
留在栈上的则是指向堆数据的指针。相比于普通的变量,Box的数据存放在内存堆上,但是
没有任何的性能损失,通常用于下列的场景当中:
\begin{itemize}
\item 当有一个在编译时未知大小的类型,而又想要在需要确切大小的上下文中使用这个类型值的时候
\item 当有大量数据并希望在确保数据不被拷贝的情况下转移所有权的时候
\item 当希望拥有一个值并只关心它的类型是否实现了特定trait而不是其具体类型的时候
\end{itemize}

第一种情况通常用于递归数据类型;第二种情况,转移大量数据的所有权会消耗大量的时间,
通过box将数据放在内存堆上,只有少量的指针数据在栈上被拷贝,减小了时间消耗;第三种
情况则通常称之为trait对象。

使用Box分配和使用堆上的数据示例如下:
\begin{code-block}{rust}
let v = Box::new(5);
let s = 10;

// Box类型无法直接和其他数据类型进行计算,必须进行转换
// 在本例当中,v的数据类型为Box<{integer}>
let b = v.as_ref() + s;
// 或者修改为如下的方式
// let b = *v + s;

println!("{}", b);
println!("{}", v);
\end{code-block}

Rust编译器要求在编译期间就能够确定对应的类型所占用的存储空间,但是,Rust当中也
存在无法在编译期间明确大小的数据类型,即递归数据类型。这种特殊类型的值,可以是
相同类型的另一个值,并且,这种嵌套关系可以无限进行下去,因此,Rust是不知道递归
数据类型的存储空间的。但是,可以通过在递归类型当中插入Box,以此为基准进行递归
数据类型的创建。

递归数据类型是一种特殊的数据类型,来源于Lisp,常见的开发语言当中没有与之相对的,
一个简单的递归数据类型的定义如下:
\begin{code-block}{rust}
enum List {
    Cons(i32, List),
    Nil,
}
\end{code-block}
Cons由2部分组成:自己包含的数据i32和另外一个List对象,其最后一项值包含一个叫做Nil
的值且没有下一项,代表递归的终止条件就是Nil。可以明显的看到,该数据类型理论上可
以无限的递归循环下去,我们无法在编译阶段就明确其存储空间占据的大小,因此上述代码
目前还无法编译通过。为了使得上述代码成功编译,可以利用Box特性:
\begin{code-block}{rust}
#[derive(Debug)]
enum List {
    Cons(i32, Box<List>),
    Nil,
}

use crate::List::{Cons, Nil};
fn main() {
    let list = Cons(1, Box::new(Cons(2, Box::new(Cons(3, Box::new(Nil))))));
    println!("{:?}", list);
}
\end{code-block}

通过这样的改变,Cons的大小就确定了:需要存放一个i32大小的值,以及一个Box指针大小
的数据(usize),从内存结构上看,其分布大致如下:
\begin{figure}[H]
  \centering
  \includegraphics[scale=0.4]{rust_box.png}
  \caption{递归数据类型的内存示意}
  \label{fig:rust_box}
\end{figure}

\subsection{Deref Trait:将智能指针当作常规引用}
Deref Trait允许重载解引用操作符(*),将智能指针当作常规引用。在此之前,先看看
引用和原始数据之间的联系:
\begin{code-block}{rust}
let x = 5;
let y = &x;

// 正确,引用被转换成原始数据类型
let r = y + 9;
println!("{}, {}", y, r );

// 提示错误,代码无法进行编译和运行
println!("{}", y == 5);

// 提示错误,代码无法进行编译和运行
if y > 2 {
    ...
}
// 提示错误,代码无法进行编译和运行
assert_eq!(5, y);

// 正确,通过解引用,将引用变更为对应的类型
assert_eq!(5, y);

let mut yy = &x;
// 错误,无法编译
yy = yy + 10;
\end{code-block}
上述代码看起来是没有错误的,但是在编译的时候,会提示一个错误信息:
\begin{figure}[H]
  \centering
  \includegraphics[width=\linewidth]{rust_pointer_error.png}
  \caption{尝试直接进行引用类型和其他类型的比较}
  \label{fig:rust_pointer_error}
\end{figure}
其根本原因在于,y虽然在使用上大多数情况和x没有什么区别,但是,实质上,y是一个引用
数据类型(指针),数值和x是具体的数值类型,引用和数值类型之间无法进行相互的比较。
如果需要进行对比,则需要对y进行解引用操作,或者直接使用原始的x。同样的,引用数据
类型和数值类型进行计算,得到的结果是数值类型,而并非引用数据类型。如果使用Box来替换
引用数据类型,其结果相同:
\begin{code-block}{rust}
let x = 5;
let y = Box::new(x);
// 提示错误,同样是由于数据类型不匹配
assert_eq!(5, y);
// 正确,通过解引用,将引用变更为对应的类型
assert_eq!(5, *y);
\end{code-block}

同样的,可以用Deref Trait实现自定义的智能指针。从本质上讲,Box<T>实际上是一个被
定义为包含一个元素的元组结构体(元组当中只包含一个元素),可以根据这个思路自定义
类似Box的智能指针:
\begin{code-block}{rust}
struct Pointer<T>(T);

impl<T> Pointer<T> {
    fn new(x: T) -> Pointer<T> {
        Pointer(x)
    }
}
\end{code-block}
上述代码使用泛型T作为元组的参数,使得该结构体可以嵌入/使用任何数据类型。接着实现
该结构的Deref Trait:
\begin{code-block}{rust}
impl<T> Deref for Pointer<T> {
    type Target = T; // 定义关联类型
    fn deref(&self) -> &T {
        &self.0
    }
}
\end{code-block}
也就是说,针对智能指针Pointer,已经可以实现对泛型数据的封装,并且可以正常的进行
解引用操作:
\begin{code-block}{rust}
// 代码可以正常的运行
let p = Pointer::new(5);
assert_eq!(5, *p);

let s = Pointer::new("lucifer");
assert_eq!("lucifer", *s);
\end{code-block}

在Rust当中,实现了Deref Trait的数据类型,在使用时,会将其引用转换为原始数据类型,
这种转换通常称之为解引用强制多态。当这种特定类型的引用作为实参传递给和形参类型
不同的函数或方法时,解引用强制多态将自动发生:
\begin{code-block}{rust}
fn hello(name: &str) {
    println!("Hello, {}!", name);
}

fn main() {
    let m = Pointer::new(String::from("Rust"));
    hello(&m);
}
\end{code-block}
上述代码当中,使用\&m调用hello函数,其为Pointer<String>值的引用,因为在Pointer<T>
上实现了Deref trait,Rust可以通过deref调用将Pointer<String>变为\&String,同时标准
库中提供了String上的Deref实现,其会返回字符串slice,Rust再次调用deref将\&String
变为\&str,这就符合hello 函数的定义了。

如果没有解引用强制多态的特性,则函数的调用则必须变更为如下的样式:
\begin{code-block}{rust}
hello(&(*m)[..]);
\end{code-block}
即(*m)将Pointer<String>解引用为String。接着\&和[..]获取了整个String的字符串slice
来匹配hello的签名,这无疑是一种低效的使用方式。

Deref重载的是不可变引用的*运算符,DerefMut则用于重载可变引用的*运算符。当发现如下
的几种情况时,Rust会进行解引用的强制多态:
\begin{itemize}
  \item 当T: Deref<Target=U> 时从\&T到\&U
  \item 当T: DerefMut<Target=U> 时从\&mut T到\&mut U
  \item 当T: Deref<Target=U> 时从\&mut T到\&U
\end{itemize}
第一种情况表明如果有一个\&T,而T实现了返回U类型的Deref,则可以直接得到\&U;第二种
情况表明对于可变引用也有着相同的行为;第3种情况,将可变引用强转为不可变引用,但
反过来是不行的,即不可变引用永远也不能强转为可变引用。

\subsection{使用Drop trait清理}
对于智能指针模式来说第二个重要的trait是Drop,其允许我们在值要离开作用域时执行一
些代码。默认可以为任何类型提供Drop trait的实现,同时所指定的代码被用于释放类似
于文件或网络连接的资源。

在其他一些语言中,我们不得不记住在每次使用完智能指针实例后调用清理内存或资源的代码。
如果忘记的话,运行代码的系统可能会因为负荷过重而崩溃。在Rust中,可以指定每当值离
开作用域时被执行的代码,编译器会自动插入这些代码,不需要在程序中到处编写在实例结
束时清理这些变量的代码,而且还不会出现内存泄漏。

Drop trait比较类似于C++的析构函数,用于指定对应的对象在离开作用域时需要进行的清理
代码,要求实现一个drop函数。简单的示例如下:
\begin{code-block}{rust}
struct SmartPointer {
    data: String,
}

impl Drop for SmartPointer {
    fn drop(&mut self) {
        println!("Drop the smart pointer {}", self.data);
    }
}

fn main() {
    let p = SmartPointer::new("zhangjl");
}
\end{code-block}
虽然在main函数当中,没有执行任何的输出操作,但是当代码结束运行时,还是会打印出
drop函数的执行结果,就如同析构函数一般。但和析构函数不同的是,析构函数可以通过
del操作调用,而Rust的drop函数是无法调用或者禁用的。如果需要提前释放变量,则需要
使用std::mem::drop进行替换:
\begin{code-block}{rust}
fn main() {
    let p = SmartPointer::new("zhangjl");
    drop(p);
    println!("Hello World");
}
\end{code-block}

\subsection{引用计数智能指针}
在Rust当中,大部分情况下,变量的所有权是明确且单一的。但是,有的场景下,要求单个
值有多个所有者。例如,在图数据结构中,多个边可能指向相同的节点,而这个节点从概念
上讲为所有指向它的边所拥有,节点直到没有任何边指向它之前都不应该被清理。为了解决
类似的问题,Rust使用Rc<T>进行引用计数的表达。引用计数意味着记录一个值引用的数量
来知晓这个值是否仍在被使用,如果某个值有零个引用,就代表没有任何有效引用并可以
被清理,反之则必须保留。特别注意的是,Rc<T>只能用于单线程/单进程环境。

简单的Rc使用示例如下:
\begin{code-block}{rust}
use std::rc::Rc;
struct User {
    name: String,
}

fn main() {
    let u = User {
        name: "zhangjl".to_string(),
    };

    let u_ref = Rc::new(u);
    println!("{}", u_ref.as_ref().name);
    let u_ref2 = Rc::clone(&u_ref);
}
\end{code-block}
Rc::clone的实现并不像类型的clone方法实现的是深拷贝一样,该函数只会增加对象的引用
计数,因此类型的clone方法可能会消耗大量的时间,而Rc::clone并不会耗费额外的时间。
Rc允许一个值有多个所有者,不过,它只允许以只读的方式进行共享数据,如果需要对数据
进行更改,则必须利用内部可变性模式以及RefCell类型,以此解决Rc的只读限制。

内部可变性是一种Rust的设计模式,这种模式允许即使在有不可变引用的时候,也可以进行
数据的改变,其重点是在数据结构当中使用unsafe来模糊Rust通常的可变性和借用规则。Unsafe
代码通常被封装进入安全的代码(API)当中,但是,外部类型仍然是不可变的。在不使用
unsafe代码的前提下,通常则是使用RefCell来进行数据的改变。

与Rc拥有多个所有者不同,RefCell代表的是数据的唯一所有权。在Rust当中,所有权和借用
规则是非常重要的:
\begin{enumerate}
  \item 在任意时刻,同一个变量只能拥有一个可变引用,或者任意数量的不可变引用
  \item 引用总是必须有效的
\end{enumerate}
如下代码:
\begin{code-block}{rust}
fn main() {
    let mut a = 24;
    add(&mut a);
    println!("{}", a);

    let mut b = &mut a;
    // 错误
    println!("{}, {}", b, a);
}
fn add(b: &mut u32) {
    *b = *b + 10
}
\end{code-block}

对于引用和Box,借用规则的不可变性表现在编译时期;对于RefCell,借用规则的不可变性
则体现在运行时期,因此,一旦RefCell违反了借用规则,则运行时将出现panic。同样的,
RefCell也只能使用于单线程/单进程场景。因此,在实质上,不管是RefCell还是其他,都
还是必须要遵循借用规则的。借用规则存在一个推论:当有一个不可变变量时,不能通过
可变引用借用他,如下:
\begin{code-block}{rust}
let x = 5;
let y = &mut x;
\end{code-block}
上述代码就会出现错误。

然而,在特定情况下,令一个值在其方法内部能够修改自身,而在其他代码中仍视为不可变,
是很有用的。RefCell<T>是一个获得内部可变性的方法,他并没有完全绕开借用规则,编译
器中的借用检查器允许内部可变性并相应地在运行时检查借用规则。如果违反了这些规则,
会出现panic而不是编译错误。如下的特殊情况,模拟Mock对象进行测试:
\begin{code-block}{rust}
pub trait Messager {
    fn send(&self, msg: &str);
}

pub struct LimitTracker<'a, T: Messager> {
    messager: &'a T,
    value: usize,
    max: usize,
}

impl<'a, T> LimitTracker<'a, T>
where
    T: Messager,
{
    pub fn new(messeger: &T, max: usize) -> LimitTracker<T> {
        LimitTracker {
            messager: messeger,
            value: 0,
            max: max,
        }
    }
    pub fn set_value(&mut self, value: usize) {
        self.value = value;
        self.messager
            .send(&format!("The value of tracker is {}", self.value));
    }
}
\end{code-block}

然后实现一个Messager,用于测试消息的发送,其实现如下:
\begin{code-block}{rust}
struct MockMessager {
    sent_messages: Vec<String>,
}

impl MockMessager {
    fn new() -> MockMessager {
        MockMessager {
            sent_messages: Vec::new(),
        }
    }
}

impl Messager for MockMessager {
    fn send(&self, message: &str) {
        self.sent_messages.push(String::from(message));
    }
}
\end{code-block}

按照我们的设想,测试的main函数应当如下:
\begin{code-block}{rust}
fn main() {
    let mock_messenger = MockMessager::new();
    let mut limit_tracker = LimitTracker::new(&mock_messenger, 100);
    limit_tracker.set_value(80);
    assert_eq!(mock_messenger.sent_messages.len(), 1);
}
\end{code-block}

但是,上述代码在进行编译时,则会出现错误:
\begin{figure}[H]
  \centering
  \includegraphics[width=\linewidth]{rust_ref_error.png}
  \caption{尝试修改不可变的引用}
  \label{fig:rust_ref_error}
\end{figure}

我们尝试去修改不可变的self引用的值,而这种情况,则正好可以利用RefCell进行实现,
其实现的代码如下:
\begin{code-block}{rust}
struct MockMessager {
    sent_messages: RefCell<Vec<String>>,
}

impl MockMessager {
    fn new() -> MockMessager {
        MockMessager {
            sent_messages: RefCell::new(Vec::new()),
        }
    }
}

impl Messager for MockMessager {
    fn send(&self, message: &str) {
        self.sent_messages.borrow_mut().push(String::from(message));
    }
}

fn main() {
    let mock_messenger = MockMessager::new();
    let mut limit_tracker = LimitTracker::new(&mock_messenger, 100);
    limit_tracker.set_value(80);
    assert_eq!(mock_messenger.sent_messages.borrow().len(), 1);
}
\end{code-block}
对于send方法的实现,第一个参数仍为self的不可变借用,这是符合方法定义的,调用
self.sent\_messages中RefCell的borrow\_mut方法来获取RefCell中值的可变引用,随之
就可以进行self当中的变量的修改了。

当创建不可变和可变引用时,我们分别使用\&和\&mut语法;对于RefCell<T>来说,则是
borrow和borrow\_mut 方法,这是属于RefCell<T>安全API的一部分。Borrow方法返回
Ref<T>类型的智能指针,borrow\_mut方法返回RefMut类型的智能指针。这两个类型都实现
了Deref,所以可以当作常规引用对待。但是,RefCell同样遵循借用规则,在任何时候只
允许有多个不可变借用或一个可变借用。

在实际使用当中RefCell常常和Rc结合使用:如果有一个储存了RefCell<T>的 Rc<T> 的话,
就可以得到有多个所有者并且可以修改的值了,比如下列的示例:
\begin{code-block}{rust}
use std::cell::RefCell;
use std::rc::Rc;

#[derive(Debug)]
enum List {
    Cons(Rc<RefCell<i32>>, Rc<List>),
    Nil,
}

use crate::List::{Cons, Nil};

fn main() {
    let value = Rc::new(RefCell::new(5));
    let a = Rc::new(Cons(Rc::clone(&value), Rc::new(Nil)));
    let b = Cons(Rc::new(RefCell::new(6)), Rc::clone(&a));
    let c = Cons(Rc::new(RefCell::new(10)), Rc::clone(&a));
    *value.borrow_mut() += 10;
    println!("a after = {:?}", a);
    println!("b after = {:?}", b);
    println!("c after = {:?}", c);
}
\end{code-block}
执行之后,其结果如下:
\begin{figure}[H]
  \centering
  \includegraphics[width=\linewidth]{rust_ref_cell.png}
  \caption{Rc和RefCell修改多个引用的值}
  \label{fig:rust_ref_cell}
\end{figure}

\subsection{引用循环和内存泄漏}
Rust的内存安全性保证使其难以意外地制造永远也不会被清理的内存,即内存泄漏,但Rust
并不保证完全避免内存泄漏,但是,和其他语言不同的是,Rust的内存泄漏是安全的。在使用
智能指针时,特别是Rc和RefCell时,需要注意有可能出现的内存泄漏,如下代码所示:
\begin{code-block}{rust}
use crate::List::{Cons, Nil};
use std::cell::RefCell;
use std::rc::Rc;

#[derive(Debug)]
enum List {
    Cons(i32, RefCell<Rc<List>>),
    Nil,
}

impl List {
    fn tail(&self) -> Option<&RefCell<Rc<List>>> {
        match self {
            Cons(_, item) => Some(item),
            Nil => None,
        }
    }
}
\end{code-block}
现在Cons成员的第二个元素是RefCell<Rc<List>>,这意味着能够修改Cons成员所指向的List。
这里还增加了一个tail方法,允许其在有Cons成员的时候访问其第二项。在使用时候如下
进行操作:
\begin{code-block}{rust}
fn main() {
    let a = Rc::new(Cons(5, RefCell::new(Rc::new(Nil))));

    println!("a initial rc count = {}", Rc::strong_count(&a));
    println!("a next item = {:?}", a.tail());

    let b = Rc::new(Cons(10, RefCell::new(Rc::clone(&a))));

    println!("a rc count after b creation = {}", Rc::strong_count(&a));
    println!("b initial rc count = {}", Rc::strong_count(&b));
    println!("b next item = {:?}", b.tail());

    if let Some(link) = a.tail() {
        *link.borrow_mut() = Rc::clone(&b);
    }

    println!("b rc count after changing a = {}", Rc::strong_count(&b));
    println!("a rc count after changing a = {}", Rc::strong_count(&a));

    println!("a next item = {:?}", a.tail());
}
\end{code-block}
变量a中创建了一个Rc<List>实例来存放初值,变量b中创建了指向列表a的List的另一个Rc<List>
实例,也即是说,b是a的头部,a是b的尾部,a的尾部没有内容。但是,紧接着,将a的尾部
指向了b,就创建了一个循环,所以,在最后进行a的尾部输出的时候,就会导致整个程序栈溢出。

引用循环无法被Rust编译器所检查或者捕获到,由此带来的内存泄漏,Rust编译器同样无法
被检测到。解决这样的问题,一种是及时的使用测试工具以及单元测试进行排查,另外,则是
使用Weak替换Rc。

Rc::clone会增加Rc的强引用计数(strong\_count),只有当强引用计数归为0,对应的Rc才会
被清理。而Rc::downgrade则是创建Rc的弱引用,得到Weak类型的智能指针,但是,并不会增加
强引用计数,即不会改变strong\_count的数值,但是会增加弱引用计数(weak\_count)的
数值,不过,在使用当中,无需使得weak\_count为0,就可以回收Rc变量。总结的来说:
\begin{itemize}
  \item 强引用(strong\_count)表示共享Rc实例的所有权
  \item 弱引用(weak\_count)并没有所有权关系
  \item 任何弱引用的循环,都会在强引用计数为0时被打断
\end{itemize}

因为Weak<T>引用的值可能已经被丢弃了,为了使用Weak<T>所指向的值,我们必须确保其值
仍然有效,为此可以调用Weak<T>实例的upgrade方法,这会返回Option<Rc<T>>。如果Rc<T>
值还未被丢弃,则结果是Some;如果Rc<T>已被丢弃,则结果是 None,所以它不会返回非法
指针,进而导致程序崩溃。

关于弱引用,最简单的例子便是树形结构:父节点需要知道子节点,子节点也需要知道父节点,
但是,删除子节点,并不意味着父节点就失效了,父节点仍然是有效的;而删除父节点则不一样,
父节点被删除之后,其对应的子节点便已经无效了,对于这样的例子,其示例代码如下:
\begin{code-block}{rust}
#[derive(Debug)]
struct Node {
    value: i32,
    parent: RefCell<Weak<Node>>,
    children: RefCell<Vec<Rc<Node>>>,
}
\end{code-block}
即一个节点就能够引用其父节点,但不拥有其父节点。其使用示例如下:
\begin{code-block}{rust}
let leaf = Rc::new(Node {
    value: 3,
    parent: RefCell::new(Weak::new()),
    children: RefCell::new(vec![]),
});
println!("leaf parent = {:?}", leaf.parent.borrow().upgrade());

let branch = Rc::new(Node {
    value: 5,
    parent: RefCell::new(Weak::new()),
    children: RefCell::new(vec![Rc::clone(&leaf)]),
});

*leaf.parent.borrow_mut() = Rc::downgrade(&branch);
println!("leaf parent = {:?}", leaf.parent.borrow().upgrade());
\end{code-block}
\section{并行与并发}
Rust也同样支持常见的并行和并发操作,也同样分为进程,线程以及消息通信等等。

\subsection{线程}
Rust的线程操作必须使用闭包完成。在之前看到的闭包当中,通常采用的都是有参的闭包,
而在Rust的线程操作当中,则经常会遇到无参数的闭包;Rust的线程使用thread::spawn函数
进行实现:
\begin{code-block}{rust}
use std::thread;
use std::time::Duration;

fn main() {
    thread::spawn(|| {
        for i in 1..10 {
            println!("hi number {} from the spawned thread!", i);
            thread::sleep(Duration::from_millis(1));
        }
    });

    for i in 1..5 {
        println!("hi number {} from the main thread!", i);
        thread::sleep(Duration::from_millis(1));
    }
}
\end{code-block}
和其他语言的线程概念一样,当主线程结束时,所有的线程都会被终止。因此上述代码当中,
子线程(spawn)无法将所有的循环执行完成。为了达成所有进/线程执行完成之后才退出主
进/线程的目的,和其他的开发语言相同,需要在主进程当中调用join函数:
\begin{code-block}{rust}
fn main() {
    let handle = thread::spawn(|| {
        for i in 1..10 {
            println!("hi number {} from the spawned thread!", i);
            thread::sleep(Duration::from_millis(1));
        }
    });

    for i in 1..5 {
        println!("hi number {} from the main thread!", i);
        thread::sleep(Duration::from_millis(1));
    }
    handle.join().unwrap();
}
\end{code-block}
Thread::spawn的返回值是JoinHandle,是一个拥有所有权的值,当对其调用join方法时,
它会等待对应线程结束;而join的返回值是一个Result,可以按照之前介绍的方式进行处理。
同时,Join函数是一个阻塞式函数,只有当该函数运行结束之后,才会继续进行后续的操作。

多数情况下,Rust的线程不可能只会在内部运行,而和外部没有数据交互。但是,如果我们
直接使用外部数据,则会出现错误,比如下方的代码:
\begin{code-block}{rust}
fn main() {
    let v = vec![1, 2, 3];
    let handle = thread::spawn(|| {
        println!("Here's a vector: {:?}", v);
    });
    handle.join().unwrap();
}
\end{code-block}
\begin{figure}[H]
  \centering
  \includegraphics[width=\linewidth]{rust_thread_out_params.png}
  \caption{试图访问线程外部资源}
  \label{fig:rust_thread_out_params}
\end{figure}
线程使用的是闭包,从闭包的定义来说,是可以捕获并使用外部变量和数据的;但是,Rust
不知道这个线程到底会运行多长时间,因此无法知道对外部变量的引用是否一直有效,比如
下方的代码:
\begin{code-block}{rust}
fn main() {
    let v = vec![1, 2, 3];
    let handle = thread::spawn(|| {
        println!("Here's a vector: {:?}", v);
    });
    drop(v);
    handle.join().unwrap();
}
\end{code-block}
启动线程的同时,立即将v进行丢弃,线程内部无法知道v在运行阶段是否继续有效,就会
出现错误,因此,如果在线程当中使用默认的闭包模式,则无法对应的闭包是无法捕获以及
使用外部的变量和数据的。此时,则需要使用move闭包进行替换,即强制闭包获取外部变量
的所有权,而不是由Rust进行借用推断。但是需要注意,一旦使用move之后,在线程之外,
变量将无法再进行使用:
\begin{code-block}{rust}
fn main() {
    let v = vec![1, 2, 3];
    let handle = thread::spawn(move || {
        println!("Here's a vector: {:?}", v);
    });
    // 下方代码无法再进行执行
    // println!("{:?}", v);
    handle.join().unwrap();
}
\end{code-block}

\subsection{消息通信和消息传递}
每个线程做自己的事情,但是,不管什么编程语言,都需要考虑线程之间的数据交互问题。
Rust向Golang进行了学习,使用通信替换共享内存,来进行线程之间的数据传输。同样的,
Rust当中用于消息传递并发的主要工具是通道,该概念和Golang的通道概念相同。Rust的通道
分为2个角色:发送者和接收者,发送者发送消息,接收者接收消息,当发送者或者接收者任一
被丢弃时,则对应的通道被视为关闭。

Rust的通道采用mpsc::channel函数实现,mpsc表示多个生产者,单个消费者,因此,Rust
标准库实现通道的方式意味着一个通道可以有多个产生值的发送(sending)端,但只能有
一个消费这些值的接收(receiving)端。通道的实现示例如下:
\begin{code-block}{rust}
use std::sync::mpsc;
fn main() {
    let (sender, recevier) = mpsc::channel();
}
\end{code-block}
其中,函数的第一个返回值为发送者,第二个参数为接收者。使用通道发送数据通信的示例
如下:
\begin{code-block}{rust}
use std::sync::mpsc;
use std::thread;

fn main() {
    let (sender, recevier) = mpsc::channel();

    thread::spawn(move || {
        let val = "lucifer".to_string();
        match sender.send(val) {
            Ok(_) => println!("Send success"),
            Err(error) => println!("Send failed :{:?}", error),
        }
    });

    let res = match recevier.recv() {
        Ok(s) => s,
        Err(error) => {
            println!("Cannot recevie anything from sender: {:?}", error);
            "".to_string()
        }
    };
    println!("The result of channel is {}", res);
}
\end{code-block}
接收者接收消息有2种模式:默认的recv是阻塞式,返回一个Result<T, E>,当通道关闭时,
将返回Result当中的Error;而try\_recv是非阻塞式,同样是返回一个Result<T, E>,但是,
Result当中的Error表示没有接收到任何消息,可以使用for循环进行反复的尝试读取操作。
另外需要注意的是,Send函数会改变变量的所有权,当该函数执行之后,被发送的消息
(变量)将无法再使用。

但是,通道可以反复使用,而且和Golang的类似,Rust的通道也是可以进行迭代的,特别
是在接收消息时,通常采用for循环进行操作,减少了错误处理的代码,使得代码更具可读性:
\begin{code-block}{rust}
use std::sync::mpsc;
use std::thread;

fn main() {
    let (sender, recevier) = mpsc::channel();

    let handler = thread::spawn(move || {
        let vals = vec!["lucifer", "titans", "garuda"];
        for val in vals {
            match sender.send(val) {
                Ok(_) => println!("Send success"),
                Err(error) => println!("Send failed :{:?}", error),
            }
        }
    });

    for msg in recevier {
        println!("The msg is {}", msg);
    }

    match handler.join() {
        Err(error) => println!("Error{:?}", error),
        _ => (),
    }
}
\end{code-block}

同样的,由于Rust的通道默认是多生产者/单消费者,因此,可以通过多个发送端向单个接
收端发送消息。实际使用当中的多个发送端,则通常是某个发送端的克隆对象,如下:
\begin{code-block}{rust}
use std::sync::mpsc;
use std::thread;

fn main() {
    let (sender, recevier) = mpsc::channel();
    let sender_copy = sender.clone();

    let handler = thread::spawn(move || {
        let vals = vec!["lucifer", "titans", "garuda"];
        for val in vals {
            match sender.send(val) {
                Ok(_) => println!("Send success"),
                Err(error) => println!("Send failed :{:?}", error),
            }
        }
    });

    let handler_copy = thread::spawn(move || {
        let vals = vec!["zhangjl", "luoyan", "zhangzz"];
        for val in vals {
            match sender_copy.send(val) {
                Err(error) => println!("Send failed :{:?}", error),
                _ => (),
            }
        }
    });

    for msg in recevier {
        println!("The msg is {}", msg);
    }

    match handler_copy.join() {
        Err(error) => println!("Error{:?}", error),
        _ => (),
    }

    match handler.join() {
        Err(error) => println!("Error{:?}", error),
        _ => (),
    }
}
\end{code-block}

\subsection{共享状态}
在其他语言当中,有些特殊的场景,还是必须使用原有的线程并发概念——锁——来进行资源的
访问/读写控制。Rust当中同样存在锁,比较常见的就是互斥锁(互斥器,Mutex)以及原子
计数器(Arc)。在基本的操作上,互斥锁的使用和其他语言当中没有太大的区别:
\begin{code-block}{rust}
use std::sync::Mutex;
fn main() {
    let m = Mutex::new(5);
    {
        let mut num = m.lock().unwrap();
        *num = 6;
    }
    println!("m = {:?}", m);
}
\end{code-block}
注意,上述代码如果将内部大括号去除,则运行结束之后,m的状态还是锁定状态;但是,
有大括号,则表示大括号内部的段是一个有效的生命周期,当该生命周期结束之后,互斥
锁将自动释放。一旦获取了锁,就可以将返回值(在这里是num)视为一个其内部数据的
\underline{\color{red} \textbf{可变引用}}。类型系统确保了我们在使用m中的值之前
获取锁:Mutex<i32>并不是一个i32,所以必须获取锁才能使用这个i32值。

实质上,Mutex是一个智能指针,lock调用返回一个叫做MutexGuard的智能指针。这个智能
指针实现了Deref来指向其内部数据;同时也提供了一个Drop实现,使得MutexGuard离开作
用域时自动释放锁,即锁的释放是自动发生的。

但是默认情况下,Mutex是无法用于进行线程间的数据共享,如下:
\begin{code-block}{rust}
use std::rc::Rc;
use std::sync::Mutex;
use std::thread;

fn main() {
    let counter = Rc::new(Mutex::new(0));
    let mut handles = vec![];

    for _ in 0..10 {
        let counter = Rc::clone(&counter);
        let handle = thread::spawn(move || {
            let mut num = counter.lock().unwrap();

            *num += 1;
        });
        handles.push(handle);
    }

    for handle in handles {
        handle.join().unwrap();
    }

    println!("Result: {}", *counter.lock().unwrap());
}
\end{code-block}
上述代码会出现下面的类似错误:
\begin{figure}[H]
  \centering
  \includegraphics[scale=0.215]{rust_mutex_share_error.png}
  \caption{试图通过Rc共享Mutex的数据}
  \label{fig:rust_mutex_share_error}
\end{figure}
即之前提到的,Rc类型只能用于单线程/单进程环境。

而共享引用计数则需要使用Arc,它是可以安全的用于并发环境的类型,即原子引用计数,
可以在线程间进行共享所有权。Arc和Rc有相同的API,基本使用方法上类似。所有,可以直
接对上述代码进行修改:
\begin{code-block}{rust}
use std::sync::{Arc, Mutex};
use std::thread;
fn main() {
    let counter = Arc::new(Mutex::new(0));
    let mut handles = vec![];
    for _ in 0..10 {
        let counter = Arc::clone(&counter);
        let handle = thread::spawn(move || {
            let mut num = counter.lock().unwrap();
            *num += 1;
        });
        handles.push(handle);
    }
    for handle in handles {
        handle.join().unwrap();
    }
    println!("Result: {}", *counter.lock().unwrap());
}
\end{code-block}
通过这样简单的修改,成功实现了10个进程当中对同一个数值进行加法操作的功能。

\section{Match与模式匹配}
Match是Rust常用的语法糖,其用法不局限于之前所讲的范围。关于match的用法,还有很多,
并且,多数和模式匹配有关,接下来可以看一些常见的match和模式匹配的使用方式。
\begin{outline}[enumerate]
\1 多种匹配模式

在match表达式当中,可以用|匹配多个模式,表示或运算:
\begin{code-in-enumerate}{rust}
let x = 1;
match x {
    1 | 2 => println!("one or two"),
    3 => println!("three"),
    _ => println!("anything"),
}
\end{code-in-enumerate}

\1 使用..=匹配范围

..=语法允许匹配一个数值范围内的任意数据,常用于数值和字符:
\begin{code-in-enumerate}{rust}
let x = 5;
match x {
    1..=5 => println!("one through five"),
    _ => println!("something else"),
}

let y = 'c';
match y {
    'a'..='j' => println!("early ASCII letter"),
    'k'..='z' => println!("late ASCII letter"),
    _ => println!("something else"),
}
\end{code-in-enumerate}

\1 解构结构体

Let模式可以将结构体当中的字段/元素进行解构,单独或者批量赋予其他元素:
\begin{code-in-enumerate}{rust}
struct Point {
    x: i32,
    y: i32,
}
fn main() {
    let p = Point { x: 0, y: 7 };
    // 将p的x字段的值赋予a,y字段的值赋予b,a和b是整数类型,不是引用
    let Point { x: a, y: b } = p;
    // let Point {x: ref a, y: ref b} = p; 和上面类似,但是a和b是整数类型的引用
    // let Point {x: a, y: _} = p; 表示只需要将x的值赋予a,但不需要对y进行解构
    assert_eq!(0, a);
    assert_eq!(7, b);
    // let Point {x, y} = p; 将p的x字段的值赋予变量x,y字段的值赋予变量y
}
\end{code-in-enumerate}

\1 解构枚举类型

Match本身就是应枚举而生的,因此天然的可以使用它对枚举进行解构:
\begin{code-in-enumerate}{rust}
enum Message {
    Quit,
    Move { x: i32, y: i32 },
    Write(String),
    ChangeColor(i32, i32, i32),
}

fn main() {
    let msg = Message::ChangeColor(0, 160, 255);

    match msg {
        Message::Quit => {
            println!("The Quit variant has no data to destructure.")
        }
        Message::Move { x, y } => {
            println!("Move in the x direction {} and in the y direction {}", x, y);
        }
        Message::Write(text) => println!("Text message: {}", text),
        Message::ChangeColor(r, g, b) => {
            println!("Change the color to red {}, green {}, and blue {}", r, g, b)
        }
    }
}
\end{code-in-enumerate}

同样的,如果枚举当中嵌套了枚举,仍然可以使用match进行解构:
\begin{code-in-enumerate}{rust}
enum Color {
    Rgb(i32, i32, i32),
    Hsv(i32, i32, i32),
}

enum Message {
    Quit,
    Move { x: i32, y: i32 },
    Write(String),
    ChangeColor(Color),
}

fn main() {
    let msg = Message::ChangeColor(Color::Hsv(0, 160, 255));

    match msg {
        Message::ChangeColor(Color::Rgb(r, g, b)) => {
            println!("Change the color to red {}, green {}, and blue {}", r, g, b)
        }
        Message::ChangeColor(Color::Hsv(h, s, v)) => {
            println!(
                "Change the color to hue {}, saturation {}, and value {}",
                h, s, v
            )
        }
        _ => (),
    }
}
\end{code-in-enumerate}

\1 解构复合数据

用复杂的方式来混合、匹配和嵌套解构模式,解析出我们感兴趣的数据:
\begin{code-in-enumerate}{rust}
let ((feet, inches), Point {x, y}) = ((3, 10), Point { x: 3, y: -10 });
\end{code-in-enumerate}

\1 忽略不需要的元素

在Rust的当中,默认可以使用\_对不必要的变量进行忽略,通常用在match的最后分支,但是,
实际上也可以用去其他任意的模式,甚至是函数参数:
\begin{code-in-enumerate}{rust}
// 需要传入2个参数,但是忽略第一个参数
fn foo(_: i32, y: i32) {
    println!("This code only uses the y parameter: {}", y);
}

fn main() {
    foo(3, 4);
}
\end{code-in-enumerate}

除了使用\_进行忽略之外,还可以使用..语法糖进行忽略,但是针对结构体和元组存在区别:
结构体当中,忽略的是没有被列出的字段;而元组忽略的则是范围:
\begin{code-in-enumerate}{rust}
struct Point {
    x: i32,
    y: i32,
    z: i32,
}

fn main() {
    let origin = Point { x: 0, y: 0, z: 0 };
    // 将point的y进行忽略
    match origin {
        Point { x,z, .. } => println!("x is {}, z is {}", x, z),
    }

    let numbers = (2, 4, 8, 16, 32);
    match numbers {
        // 忽略元组当中除第1、2和最后一项的所有元素
        (first, second, .., last) => {
            println!("Some numbers: {}, {}, {}, ", first, second, last);
        }
    }
}
\end{code-in-enumerate}

同样的,忽略操作也可以用于闭包当中:
\begin{code-in-enumerate}{rust}
let player_scores = [("Jack", 20), ("Jane", 23), ("Jill", 18), ("John", 19)];
// 对player_scores进行迭代,忽略其中第二个元素,_可以被替换为_score
let players: Vec<_> = player_scores.iter().map(|&(player, _)| player).collect();
// 输出的结果当中将只会有字符串数据
println!("{:?}", players);
\end{code-in-enumerate}


\1 @绑定

运算符@允许我们在创建一个存放值的变量的同时测试其值是否匹配模式,比如测试字段是
否位于指定范围内,同时也希望能将其值绑定到另外的变量中以便此分支相关联的代码可以
使用它:
\begin{code-in-enumerate}{rust}
enum Message {
    Hello { id: i32 },
}

let msg = Message::Hello { id: 5 };

match msg {
    // 将变量id保存到另一个变量ip_variable当中
    Message::Hello { id: id_variable @ 3..=7 } => {
        println!("Found an id in range: {}", id_variable)
    },
    Message::Hello { id: 10..=12 } => {
        println!("Found an id in another range")
    },
    Message::Hello { id } => {
        println!("Found some other id: {}", id)
    },
}
\end{code-in-enumerate}
\end{outline}

\section{高级特征}
Rust设计不仅仅是为了开发应用程序,其设计之初,就是为了解决内存安全的问题,并且可以
广泛用于各种场景,包括C语言的专属领域:操作系统设计。在编写操作系统的过程当中,
C语言使用了很多的高级宏定义以及一些精妙的设计,而Rust同样如此。为了和硬件打交道,
Rust被设计为可以拥有直接操作硬件的能力,这些都是其高级特性的一部分。Rust的高级
特性主要包含下列内容:
\begin{enumerate}
  \item 不安全 Rust
  \item 高级 Trait
  \item 高级函数和闭包
  \item 宏
\end{enumerate}

\subsection{Unsafe}
Rust屏蔽了一系列的不安全操作来换取应用程序的稳定性和可靠性,但是,可以通过关键字
unsafe,切换到不安全的运行环境当中,并且在unsafe的代码块当中运行。常见的不安全操作
如下:
\begin{enumerate}
  \item 解引用裸指针
  \item 使用不安全的方法/函数
  \item 访问/修改可变的静态变量
  \item 实现不安全的Trait
  \item 访问union的字段
\end{enumerate}
在使用的时候,原则需要明确:保持unsafe块尽可能小,将不安全代码封装进一个安全的
抽象并提供安全API是一种常见的安全操作和手段。

所谓的裸指针,和普通的指针和智能指针相比,存在如下的区别:
\begin{enumerate}
  \item 允许忽略借用规则,可以同时拥有不可变和可变的指针,或多个指向相同位置的可变指针
  \item 不保证指向有效的内存
  \item 允许为空
  \item 不能实现任何自动清理功能
\end{enumerate}
Rust当中存在2个裸指针:分别写作*const T(不可变)和*mut T(可变),其基本的定义方式
如下:
\begin{code-block}{rust}
let mut num = 5;
let r1 = &num as *const i32; // 不可变的裸指针
let r2 = &mut num as *mut i32; // 可变的裸指针
\end{code-block}

裸指针的定义是安全的,但是,它的使用是不安全的,因此裸指针的使用必须在unsafe块
当中:
\begin{code-block}{rust}
fn main() {
    let mut num = 5;

    let r1 = &num as *const i32;
    let r2 = &mut num as *mut i32;

    unsafe {
        *r2 = 10;
        // r1,r2和num都会变更为10
        println!("{},{}", *r1, *r2);
    }
}
\end{code-block}
同样的,unsafe也可以用于定义函数/方法,不过也需要在unsafe块当中使用;但是,unsafe
的方法可以作为安全方法进行导出,在使用时,则不需要使用unsafe进行标记:
\begin{code-block}{rust}
fn main() {
    let mut num = 5;
    // 定义裸指针
    let r1 = &num as *const i32;
    let r2 = &mut num as *mut i32;

    // 使用不安全的函数/方法
    unsafe {
        unsafe_change(r1, r2);
    }
    println!("{}", num);

    safe_change(r1, r2);
    println!("{}", num);
}

// 定义不安全的函数/方法
unsafe fn unsafe_change(r1: *const i32, r2: *mut i32) {
    *r2 = 10;
    println!("{},{}", *r1, *r2);
}

// 将不安全的函数/方法封装进安全的方法当中
fn safe_change(r1: *const i32, r2: *mut i32) {
    unsafe {
        *r2 = 100;
    }
}
\end{code-block}

作为不安全的一部分,某些时候直接在Rust当中调用C语言的类库可以获得更好的性能,此时,
则同样需要在unsafe块当中使用,比如在Rust当中调用标准C的abs(绝对值)函数:
\begin{code-block}{rust}
extern "C" {
    fn abs(input: i32) -> i32;
}

fn main() {
    unsafe {
        println!("The unsafe from C: {}", abs(-200));
    }
}
\end{code-block}
上述代码出现的extern关键字,有助于创建和使用外部函数接口(Foreign Function
Interface,FFI)。外部函数接口是一个编程语言用以定义函数的方式,其允许不同(外部)
编程语言调用这些函数。Extern块中声明的函数在Rust代码中总是不安全的,

特别需要注意的是,Rust当中的可变全局变量(static)同样是不安全的,需要在unsafe
代码块当中使用;而不可变的全局常量(const和static)则不需要在unsafe块当中;另外,
全局变量同样可以是任意数据类型的:
\begin{code-block}{rust}
use std::fmt;

struct Version {
    major: u8,
    minor: u8,
}

impl fmt::Display for Version {
    fn fmt(&self, f: &mut fmt::Formatter) -> fmt::Result {
        write!(
            f,
            "The version of this bin is {}.{}",
            self.major, self.minor
        )
    }
}

// 不可变的全局常量
const __CONST_NUM__: Version = Version { major: 1, minor: 4 };
const __VERSION__: &str = "v1.4.0";
static __NAME__: &str = "lucifer";
// 可变的全局变量
static mut __COUNTER__: u8 = 1;

fn main() {
    println!("{}", __CONST_NUM__);
    println!("{}", __NAME__);
    unsafe {
        println!("{}", __COUNTER__);
    }
}
\end{code-block}

\subsection{高级Trait}
Trait的语法当中,使用了如下的代码形式:
\begin{code-block}{rust}
impl Iterator for Counter {
    type Item = u32;
    fn next(&mut self) -> Option<Self::Item> {
        ...
    }
}
\end{code-block}
其中,type Item表示关联数据类型,Item表示占位类型,next方法定义表明它返回
Option<Self::Item>类型的值。这个trait的实现者会指定Item的具体类型。

在Trait当中,除了默认方法,方法覆写之外,还存在着运算符重载的功能。但是,和C++
不同,Rust并不允许创建自定义的运算符,或者重载任意运算符,只有std::ops当中所列出
的运算符和相应的trait可以通过实现运算符相关的trait来实现重载,比如下面,实现Add
trait来实现对+的运算符重载:
\begin{code-block}{rust}
use std::fmt;
use std::ops::{Add, AddAssign};

struct Point {
    x: u8,
    y: u8,
}

// 实现结构体的 c = a + b
impl Add for Point {
    type Output = Point;
    fn add(self, other: Point) -> Point {
        Point {
            x: self.x + other.x,
            y: self.y + other.y,
        }
    }
}

// 实现结构体的 a = a + b;
impl AddAssign for Point {
    fn add_assign(&mut self, other: Point) {
        self.x = self.x + other.x;
        self.y = self.y + other.y;
    }
}

impl fmt::Display for Point {
    fn fmt(&self, f: &mut fmt::Formatter) -> fmt::Result {
        write!(f, "x:{}, y:{}", self.x, self.y)
    }
}

fn main() {
    let a = Point { x: 3, y: 4 };
    let b = Point { x: 5, y: 6 };
    let c = a + b;

    let mut d = Point { x: 100, y: 101 };
    d = d + c;
    println!("{}", d);
}
\end{code-block}

以Add Trait为例,其内部的实现如下:
\begin{code-block}{rust}
#[lang = "add"]
pub trait Add<Rhs = Self> {
    type Output;
    #[must_use]
    fn add(self, rhs: Rhs) -> Self::Output;
}
\end{code-block}
其中,RHS=Self这个语法叫做默认类型参数。RHS是一个泛型类型参数,它用于定义add方法
中的rhs参数。如果实现Add trait时不指定RHS的具体类型,RHS的类型将是默认的Self类型
也就是在实现Add Trait的类型。在上述例子当中,RHS就是Point这个类型。但是,也可以使用
不同的数据类型,比如下面的例子:
\begin{code-block}{rust}
struct Meters(i32);
struct Millimeters(i32);

impl Add<Meters> for Millimeters {
    type Output = Millimeters;

    fn add(self, other: Meters) -> Millimeters {
        Millimeters(self.0 + (other.0 * 1000))
    }
}
\end{code-block}
定义一个结构体米,和结构体毫米,然后定义毫米与米的加法操作,当结构体毫米与结构体
米进行相加时(注意顺序),将结果转换成毫米结果:
\begin{code-block}{rust}
let meters = Meters(1);
let mill_meters = Millimeters(10);
let mill_meters_other = mill_meters + meters;
\end{code-block}
但是,如果将上述代码的顺序更换如下:
\begin{code-block}{rust}
let mill_meters_other = meters + mill_meters;
\end{code-block}
则会出现如下的错误:
\begin{figure}[H]
  \centering
  \includegraphics[width=\linewidth]{rust_override_error.png}
  \caption{尝试进行不同类型的加法重载操作}
  \label{fig:rust_override_error}
\end{figure}
修复上述的错误也很简单,增加结构体米的加法操作重载运算符即可:
\begin{code-block}{rust}
struct Meters(i32);
struct Millimeters(i32);

impl Add<Meters> for Millimeters {
    type Output = Millimeters;

    fn add(self, other: Meters) -> Millimeters {
        Millimeters(self.0 + (other.0 * 1000))
    }
}

impl Add<Millimeters> for Meters {
    type Output = Millimeters;

    fn add(self, other: Millimeters) -> Millimeters {
        Millimeters(self.0 * 1000 + other.0)
    }
}
\end{code-block}
这样,在进行结构体米和结构体毫米之间的加法操作是,无需考虑操作数的顺序。

在之前,也提到了Deref Trait的用法,通常用于进行智能指针的解引用操作,使得智能指针
可以直接当作指定的类型使用。不过Deref Trait不仅仅可以针对智能指针,也可以对自定义
的数据类型添加其他各种操作。比较典型的例子,现在有一个Vec,其中包含的数据类型是
String,如果需要打印这个Vec<String>,则必须使用debug这个宏定义;如果不想使用这个
debug,则必须在Vec<String>上实现Display Trait,但是Display Trait是无法直接作用在
Vec<String>上的,因此我们可以采用一种方式,在一个结构体当中包含匿名的Vec<String>,
然后在这个结构体上实现Display Trait,如下:
\begin{code-block}{rust}
use std::fmt;
use std::ops::{Deref, DerefMut};

struct VecWrapper(Vec<String>);

impl Deref for VecWrapper {
    type Target = Vec<String>;
    fn deref(&self) -> &Vec<String> {
        &self.0
    }
}

impl DerefMut for VecWrapper {
    fn deref_mut(&mut self) -> &mut Vec<String> {
        &mut self.0
    }
}

impl fmt::Display for VecWrapper {
    fn fmt(&self, f: &mut fmt::Formatter) -> fmt::Result {
        write!(f, "[")?;
        for item in &self.0 {
            write!(f, "{}, ", item)?;
        }
        write!(f, "]")
    }
}

fn main() {
    let mut v = VecWrapper(vec![String::from("hello"), String::from("world")]);
    v.push("zhangjl".to_string());
    for item in &v.0 {
        println!("{}", item);
    }

    println!("{}", v);
}
\end{code-block}
在上述代码当中,使用VecWrapper将Vec<String>进行简单的封装,然后使用Deref Trait
实现对VecWrapper的解引用(包括可变和不可变),将对VecWrapper的解引用操作重定向
到直接访问Vec<String>,这样带来的好处如下:
\begin{enumerate}
  \item 无需针对VecWrapper进行额外的其他操作,即可使用所有Vec<String>的所有方法
  \item 可以如同Vec一样的进行任意的操作
  \item 可以实现VecWrapper的自定义函数/方法,但又不影响原本的Vec操作
\end{enumerate}

另外需要注意,在上述代码当中,我们再次使用了?操作符,由于write本身返回的是一个
Result类型,但是,如果write操作后面添加分号符,表示目前只考虑了正确的模式,忽略
了错误的处理,因此编译器会提示警告。为了消除这个警告,则可以使用?代替Result类型,
同时继续正确和错误分支的处理。

同样的,上述的代码VecWrapper可以做成泛型,如下:
\begin{code-block}{rust}
use std::ops::{Deref, DerefMut};
struct VecWrapper<T>(Vec<T>);
impl<T> Deref for VecWrapper<T> {
    type Target = Vec<T>;
    fn deref(&self) -> &Vec<T> {
        &self.0
    }
}

impl<T> DerefMut for VecWrapper<T> {
    fn deref_mut(&mut self) -> &mut Vec<T> {
        &mut self.0
    }
}
\end{code-block}
上述的代码只是将Vec做了一次封装,可以使用任何的Vec方法,但是,我们将无法对这个
类型实现Display Trait,因为泛型T本身是无法实现Display Trait的。

Trait的另外一个非常重要的用途就是实现继承。涉及到继承实现,不可避免的会遭遇到
函数/方法的重载/覆写,尤其是多重继承的时候。在Rust当中,同样允许不同的Trait有相
同的函数/方法定义,也同样允许一个类型实现多个Trait,比如下面的代码:
\begin{code-block}{rust}
trait Pilot {
    fn fly(&self);
    fn name();
}

trait Wizard {
    fn fly(&self);
    fn name();
}

struct Empty;

impl Pilot for Empty {
    fn fly(&self) {
        println!("This is the implement of Pilot fly method");
    }
    fn name() {
        println!("This is the name method of Pilot implement");
    }
}

impl Wizard for Empty {
    fn fly(&self) {
        println!("This is the implement of Wizard fly method");
    }
    fn name() {
        println!("This is the name method of Wizard implement");
    }
}
\end{code-block}
Trait Pilot和Wizard定义了一个同名的方法,以及一个同名的关联函数(没有self做参数),
然后结构体Empty实现了这2个Trait,但是本身没有任何的方法/函数。但是,在使用的时候,
则必须注意,一定要进行Trait的指定或者转换,否则由于存在同名函数/方法,可能导致
代码的二义性出现,从而导致错误:
\begin{code-block}{rust}
fn main() {
    let empty = Empty {};

    empty.fly();
    Empty::name();
}
\end{code-block}
\begin{figure}[H]
  \centering
  \includegraphics[width=\linewidth]{rust_same_name.png}
  \caption{实现包含同名函数/方法的多个Trait}
  \label{fig:rust_same_name}
\end{figure}
解决这种问题的方法主要有2种思路:1是增加Empty结构体自身的同名函数/方法的实现,
但这种思路相当于完全没有利用Trait的任何功能;2是对Empty进行Trait的指定,如下:
\begin{code-block}{rust}
fn main() {
    let empty = Empty {};

    Wizard::fly(&empty);
    Pilot::fly(&empty);

    <Empty as Wizard>::name();
    <Empty as Pilot>::name();
}
\end{code-block}
同样的,如果不同的Trait包含了同名的方法/函数,但是参数和返回值定义不同,在使用的
时候,也需要进行明确的指定:
\begin{code-block}{rust}
trait Pilot {
    fn fly(&self);
    fn name();
}

trait Wizard {
    fn fly(&self, name: &str);
    fn name(age: u8) -> u8;
}

...

fn main() {
    let empty = Empty {};
    Wizard::fly(&empty, "lucifer");
    <Empty as Wizard>::name(64);
}
\end{code-block}

\subsection{高级函数和闭包}
Rust的函数和闭包都有很多类似的地方,和C/C++的函数也类似,确切的说,是非常类似于
C/C++当中的函数指针,因此,Rust的函数和闭包,也可以作为函数的参数以及返回值。但是,
函数作为参数和返回值与闭包有些区别,先看使用函数作为参数与返回值,如下:
\begin{code-block}{rust}
fn newmethod() -> fn(u32) -> u32 {
    calc
}

fn fn_as_params(age: u32, f: fn(u32) -> u32) {
    println!("In the fn_as_params: {}", f(age));
}

fn calc(age: u32) -> u32 {
    age * 100
}

fn main() {
    let b = newmethod();
    println!("{}", b(32));

    fn_as_params(32, calc);
}
\end{code-block}
可以看到,函数作为参数和返回值,基本用法和C/C++当中的方式是一致的。但是,使用闭包
的情形有些区别:闭包缺少具体的大小(size)描述,如果直接传递闭包,则会因为编译器
无法知道当前参数的大小而报错,因此,当使用闭包作为参数时,需要如下进行处理:
\begin{code-block}{rust}
// 直接使用闭包
fn hello(age: u32, func: &dyn Fn(u32) -> u32) {
    println!("{}", func(age));
}

// 使用Box智能指针
fn recv(age: u32, func: Box<dyn Fn(u32) -> u32>) {
    println!("{}", func(age));
}

// 使用Fn Trait
fn asparams(age: u32, func: impl Fn(u32) -> u32) -> u32{
    func(age)
}

fn main() {
    let c = |x| x * 10;
    recv(10, Box::new(c));
    hello(8, &c);
    // 2种方式都可以
    asparams(8, &c);
    asparams(8, c);
}
\end{code-block}

同样的,使用闭包作为函数的返回值,也是需要进行额外的特殊处理:
\begin{code-block}{rust}
// 使用Box智能指针
fn newclosur() -> Box<dyn Fn(u32) -> u32> {
    Box::new(|x| x * 100)
}

// 使用Fn Trait
fn return_closur() -> impl Fn(u32) -> u32 {
    |x| x * 120
}

fn main() {
    let a = newclosur();
    println!("{}", a(12));

    let a = return_closur();
    println!("{}", a(18));
}
\end{code-block}

