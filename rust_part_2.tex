\section{生命周期}
在之前的所有权一节,有这么一个函数示例:
\begin{code-block}{rust}
fn first_word(s: &str) -> &str {
    return &s[..];
}

fn copy_ref(s: &str) -> &str {
    // 也可以是&s,为啥?
    return s;
}
\end{code-block}
上述的函数都运行正常。对函数进行改造,改造成下列的样式:
\begin{code-block}{rust}
fn longest(x: &str, y: &str) -> &str {
    if x.len() > y.len() {
        x
    } else {
        y
    }
}
\end{code-block}
即,返回2个字符串当中最长的。如果对这样的代码进行编译,则会出现错误:
\begin{figure}[H]
  \centering
  \includegraphics[scale=0.2]{rust_strref_err.png}
  \caption{试图返回多个引用当中的某一个}
  \label{fig:rust_strref_err}
\end{figure}
错误表示,函数应该返回一个有生命周期的命名变量。错误的原因是,Rust编译器无法知道
函数返回的到底是x还是y的引用,无法确定对应的变量的生命周期。

Rust当中,针对引用和借用,有一个特殊的机制:借用检查器,其作用比较作用域来确保所
有的借用都是有效的。
\begin{code-block}{rust}
{
    let r;                      // ---------+-- 'a
    {                            //          |
        let x = 5;             // -+-- 'b  |
        r = &x;                 //  |       |
    }                            // -+       |
    println!("r: {}", r); // ---------+
}
\end{code-block}

其中'a表示变量r原本的作用域(生命周期),'b则表示变量x的有效作用域。进入'b作用域
之后,r变量引用了一个作用域为'b的变量x,当退出'b之后,x失去作用,导致作为x的引用
的r也失去作用,被回收,因此,上述代码无法进行编译:'b的作用范围比'a要小。

为了解决这类的问题,Rust引入了生命周期的操作。生命周期的定义通常使用'+名称的方式
进行定义,表示一个变量或者函数的有效范围,如下:
\begin{code-block}{rust}
&i32        // 引用
&'a i32     // 带有显式生命周期的引用
&'a mut i32 // 带有显式生命周期的可变引用
\end{code-block}
生命周期不仅可以用于变量,同样可以作用与函数和方法上:
\begin{code-block}{rust}
fn main() {
    let string1 = String::from("abcd");
    let string2 = "xyz";

    let res = longest(&string1, string2);
    println!("The result is {}", res);
    println!("The result is {}", res);
}

fn longest<'a>(x: &'a str, y: &'a str) -> &'a str {
    if x.len() > y.len() {
        x
    } else {
        y
    }
}
\end{code-block}
上述代码表示,参数列表当中的所有引用都必须拥有相同的生命周期'a,通过生命周期的限定,
上述代码可以正常编译,并且正常执行。需要注意,如果在参数上使用生命周期,则函数/方法
的前面,则必须加上生命周期,否则会提示参数列表当中的生命周期没有定义。

生命周期同样可以应用于结构体字段定义当中,如下:
\begin{code-block}{rust}
struct ImportantExcerpt<'a> {
    part: &'a str,
}
\end{code-block}

上述结构体的初始化,则可以直接使用字符串的引用进行实现:
\begin{code-block}{rust}
let i = ImportantExcerpt { part: "zhangjl" };
println!("{}", i.part);
\end{code-block}

对于带有生命周期的结构体,在使用的时候,尤其是函数定义和方法定义时,有一些必须
注意的细节:
\begin{outline}[enumerate]
\1 传入外部引用数据模式

使用这种模式,通常情况下,不需要对函数添加生命周期,和普通函数相同。不过,也可以
使用添加生命周期的完整形式:
\begin{code-in-enumerate}{rust}
fn init_struct(source: &str) -> ImportantExcerpt {
    return ImportantExcerpt { part: source };
}

// 使用生命周期的完整形式,实际上是上述函数的完整签名形式
// fn init_struct<'a>(source: &'a str) -> ImportantExcerpt<'a> {
//     return ImportantExcerpt { part: source };
// }

...

// 调用函数
let b = init_struct("luoyan");
\end{code-in-enumerate}
由于上述代码当中,结构体的变量的有效生命周期和外部引用的相同,因此,可以简化生命
周期的使用。

\1 使用函数局部变量

在这种方式下,由于局部引用变量的作用域有限,返回函数之后就不存在了,因此,必须使用
显式的生命周期,而显式的生命周期使用同样有2种形式:
\begin{code-in-enumerate}{rust}
fn init_struct<'a>() -> ImportantExcerpt<'a> {
    return ImportantExcerpt { part: "luoyan"};
}

// 使用静态生命周期,'static表示静态生命周期,为固定关键字
// fn init_struct() -> ImportantExcerpt<'static> {
//     return ImportantExcerpt { part: "luoyan"};
// }
\end{code-in-enumerate}

\1 实现Trait

包含有引用数据类型的结构体,也可以实现各种标准库的Trait。在实现Trait的时候,也
必须使用生命周期:
\begin{code-in-enumerate}{rust}
// 可替换成下面的代码
// impl<'a> fmt::Display for ImportantExcerpt<'a> {
// static可以替换为_
impl fmt::Display for ImportantExcerpt<'static> {
    fn fmt(&self, f: &mut fmt::Formatter) -> fmt::Result {
        write!(f, "{}", self.part)
    }
}
\end{code-in-enumerate}

\1 添加结构体方法

结构体存在引用数据类型,同样要求结构体的方法在实现时需要进行额外的处理,添加生命
周期的使用,同样的,结构体的方法可以使用命名生命周期,也可以使用固定生命周期:
\begin{code-in-enumerate}{rust}
// 使用命名生命周期的结构体方法声明
impl<'a> ImportantExcerpt<'a> {
    fn show(&self) {
        println!("{}", self.part);
    }

    fn reset(&mut self, other: &'a str) {
        self.part = other;
    }

    fn get(&self) -> &str {
        return self.part;
    }
}

// 使用固定生命周期的结构体方法声明
impl ImportantExcerpt<'static> {
    fn show(&self) {
        println!("{}", self.part);
    }

    fn reset(&mut self, other: &'static str) {
        self.part = other;
    }

    fn get(&self) -> &str {
        return self.part;
    }
}
\end{code-in-enumerate}

\end{outline}

在上述的代码当中,很多地方都使用了'static静态生命周期。这是一种特殊的生命周期,
能够存活于整个程序期间,所有的字符串字面值都拥有'static生命周期。但是,并不是
任何情况都建议使用static生命周期。

由于生命周期和泛型以及Trait都非常类似,不可避免的,有可能会遇到几者合用的的情况,
在使用的时候,需要将生命周期与泛型使用,分割开,并且,生命周期应当放在首位。
\begin{code-block}{rust}
fn longest_with_an_announcement<'a, T>(x: &'a str, y: &'a str, ann: T) -> &'a str
    where T: Display
{
    println!("Announcement! {}", ann);
    if x.len() > y.len() {
        x
    } else {
        y
    }
}
\end{code-block}

\section{测试}
Rust的测试与其他语言相同,分为单元测试和集成测试。但不管是单元测试,还是集成测试,
在测试当中,都需要遵循相同的测试规则。在默认的lib类型的crate当中,默认情况下,自动
生成的lib.rs会生成如下的代码:
\begin{code-block}{rust}
#[cfg(test)]
mod tests {
    #[test]
    fn it_works() {
        assert_eq!(2 + 2, 4);
    }
}
\end{code-block}
其中,\#[cfg(test)]表示这是一个测试模块,而\#[test]则表示接下来的函数或者方法是测试
函数,it\_works表示测试的函数/方法名,可以变更为其他的名称。其中,assert!、assert\_eq!
和assert\_ne!这3个宏定义,用于检测运行结果、是否相等/是否不等,比如检测返回值当中
是否包含特定的字符串:
\begin{code-block}{rust}
pub fn greeting(name: &str) -> String {
    format!("Hello {}!", name)
}

#[cfg(test)]
mod tests {
    // 引用暴露的模块代码
    use super::*;

    #[test]
    fn greeting_contains_name() {
        let result = greeting("Carol");
        assert!(result.contains("Carol"));
    }
}
\end{code-block}

如果需要测试panic的代码,则可以使用should\_panic宏进行,该宏表示期望对应的函数在
运行的时候出现panic:
\begin{code-block}{rust}
pub struct Guess {
    value: i32,
}

impl Guess {
    pub fn new(value: i32) -> Guess {
        if value < 1 || value > 100 {
            panic!("Guess value must be between 1 and 100, got {}.", value);
        }

        Guess {
            value
        }
    }
}

#[cfg(test)]
mod tests {
    use super::*;

    #[test]
    #[should_panic]
    fn greater_than_100() {
        Guess::new(200);
    }
}
\end{code-block}
如果测试失败,想在测试结果当中,提示出具体的测试错误信息,则可以添加should\_panic
属性中的expected参数:
\begin{code-block}{rust}
#[cfg(test)]
mod tests {
    use super::*;

    #[test]
    #[should_panic(expected = "Guess value must be between 1 and 100")]
    fn greater_than_100() {
        Guess::new(200);
    }
}
\end{code-block}

运行测试用例时,只需要简单的输入如下的指令即可:
\begin{code-block}{bash}
// 默认并行的方式运行所有的测试用例
cargo test

// 串行的方式运行所有的测试用例
cargo test -- --test-threads=1

// 运行指定的测试用例,可匹配以add开头的所有测试用例
cargo test add
\end{code-block}

需要单独说明的是Rust的集成测试。集成测试通常针对lib型的crate。其测试过程大致如下:
\begin{outline}[enumerate]
\1 创建一个lib,并编写代码

\begin{code-in-enumerate}{bash}
cargo new --lib shared
\end{code-in-enumerate}

\1 在shared的src同级目录下,创建集成测试用例目录:
\begin{code-in-enumerate}{bash}
# 文件夹名称固定为tests
mkdir tests
\end{code-in-enumerate}

\1 在tests下创建集成测试用例
\begin{code-in-enumerate}{bash}
echo > tests/units.rs<<EOF
// 导入的lib名称必须是当前crate的名称
use shared;

#[test]
fn it_adds_two() {
    assert_eq!(4, adder::add_two(2));
}
EOF
\end{code-in-enumerate}
然后执行测试即可。
\end{outline}

\section{Rust的函数式编程}
Rust同样支持函数式编程。相比于其他语言,Rust的函数式编程性能和效率更高。Rust常见的
函数式编程模式包括闭包和迭代器2大类。

\subsection{闭包}
Rust的闭包和Python当中的非常类似,都可以直接读取外部的变量。其定义的形式基本如下:
\begin{code-block}{rust}
let expensive_closure = |num| {
    println!("calculating slowly...");
    num * 10
};

let res = expensive_closure(10);
\end{code-block}
其中两个||表示定义一个闭包,中间的num表示闭包的参数。如果闭包需要处理多个参数,则
应该改写为:
\begin{code-block}{rust}
let expensive_closure = |num1, num2| {
    num1 * num2
};
\end{code-block}

从实际的使用当中可以看到,Rust的闭包实际上就是一个匿名函数,在Rust当中,函数都有
参数类型/返回值的声明,但是,在上述的代码当中,却没有看到相关的定义和声明。这是
因为Rust的闭包通常很短,并只关联于小范围的上下文而非任意情境。在这些有限制的上下
文中,编译器有能力可靠的推断参数和返回值的类型,如同能够推断大部分变量的类型一样。
不过,不注明参数/返回类型,有可能出现一种迷惑性的使用:即无法传入正确的数据类型,
如下:
\begin{code-block}{rust}
let example_closure = |x| x;

let s = example_closure(String::from("hello"));
let n = example_closure(5);
\end{code-block}
按照上述代码的定义,example\_closure只是将输入参数原封不动的返回给调用者,第1次
调用时,编译器会将该闭包推断为输入/输出为字符串类型,然后这些类型信息会被锁定到
该闭包当中。后续再传入数值,由于闭包的类型已经锁定,要求传入字符串,但实际传入的
是数值,结果就会导致上述代码出现错误:
\begin{figure}[H]
  \centering
  \includegraphics[width=\linewidth]{rust_closure_diffrent_type.png}
  \caption{试图处理不同数据类型的闭包}
  \label{fig:rust_closure_diffrent}
\end{figure}

闭包的完整定义(包括类型)则如下:
\begin{code-block}{rust}
let live_closure = |num: i32| -> (i32, i32) {
    println!("calculating slowly...");
    thread::sleep(Duration::from_secs(2));
    (num * 10, num * 20)
    // 或者修改为return语句
    // return (num*10, num*20);
};

// 如果不需要返回值,则闭包的写法需要注意一下:
let other = |x| {
    println!("{}", x);
};
\end{code-block}

\subsection{特殊的闭包}
默认的情况下,包括Python和Golang,闭包都只是匿名函数。不过,在Rust当中,闭包可以
用在结构体当中,其主要用途就是memoization或lazy evaluation(惰性求值),即懒加载。
当结构体当中存放闭包时,则必须注明闭包的类型。而在结构体/枚举当中使用闭包,则需要
使用trait和泛型:Fn、FnMut和FnOnce。这3者的区别如下:
\begin{enumerate}
  \item FnOnce:闭包内对外部变量存在转移操作,导致外部变量不可用,所以只能call一次
  \item FnMut:闭包内对外部变量直接使用,并进行修改
  \item Fn:闭包内对外部变量直接使用,不进行修改
\end{enumerate}

使用这些trait的时候,则必须注明闭包的参数/返回值的类型。比如,闭包接收一个u32的
参数,返回一个u32,则对应的Fn trait bound则如下:
\begin{code-block}{rust}
Fn(u32) -> u32
\end{code-block}

一个包含闭包的结构体示例如下:
\begin{code-block}{rust}
struct Cacher<T>
where
    T: Fn(u32) -> u32,
{
    calculation: T,
    value: Option<u32>,
}
\end{code-block}
对该结构体的解读如下:结构体Cacher包含一个泛型calculation,而这个泛型则是一个使用
了Fn的闭包,这个闭包接收一个u32的参数,并最终返回一个u32。Value则是用于存放calculation
的计算结果,便于第二次调用时,直接返回而无需计算。根据上述需求,整个结构体的方法
实现如下:
\begin{code-block}{rust}
impl<T> Cacher<T>
where
    T: Fn(u32) -> u32,
{
    pub fn new(calculation: T) -> Cacher<T> {
        Cacher {
            calculation: calculation,
            value: None,
        }
    }

    pub fn value(&mut self, arg: u32) -> u32 {
        match self.value {
            Some(v) => v,
            None => {
                let v = (self.calculation)(arg);
                self.value = Some(v);
                v
            }
        }
    }
}
\end{code-block}
注意,在上述的结构体以及结构体方法当中,首次出现了trait bound和where的使用。需要
特别说明事实,trait bound几乎可以用于Rust的任何场景。New方法接收一个泛型作为初始化
参数,这个泛型就是一个Fn的闭包;而value方法则是根据根据当前结构体的数据,直接进行
数据的返回,或者计算,再返回。该结构体的使用方式如下:
\begin{code-block}{rust}
let mut cacher = Cacher::new(|x: u32| -> u32 { x * 10 });
let mut val = cacher.value(32);
println!("The val of cacher is {}", val);

val = cacher.value(45);
println!("The val of cacher second time is {}", val);
\end{code-block}
只是稍微可惜的是,这个表示缓存的结构体还存在bug,2次传入不同的数据,却得到了相同的
结果。问题在于字段value的定义。可以考虑使用Hashmap或者其他数据类型来替换value。一种
可能的解决方法如下:
\begin{code-block}{rust}
struct Cacher<T>
where
    T: Fn(u32) -> u32,
{
    calculation: T,
    value: BTreeMap<u32, Option<u32>>,
}

impl<T> Cacher<T>
where
    T: Fn(u32) -> u32,
{
    pub fn new(calculation: T) -> Cacher<T> {
        Cacher {
            calculation: calculation,
            value: BTreeMap::new(),
        }
    }

    pub fn value(&mut self, arg: u32) -> u32 {
        // 从现有的结果记录当中查询是否存在arg对应的计算结果
        match self.value.get(&arg) {
            // 找到则直接返回
            Some(Some(x)) => *x,
            // 没有找到,则计算一次,并放入当前的结果集合
            Some(None) | None => {
                let v = (self.calculation)(arg);
                self.value.insert(arg, Some(v));
                v
            }
        }
    }
}
\end{code-block}

闭包同样可以捕获运行环境的上下文,即在闭包内部直接使用外部的所有变量:
\begin{code-block}{rust}
fn main() {
    let x = 4;
    let equal_to_x = |z| z == x;
    let y = 4;
    assert!(equal_to_x(y));
}
\end{code-block}
X在闭包出现之前已经存在,定义闭包equal\_to\_x的时候,可以直接使用外部的x,而无需
重新声明。

\subsection{迭代器}
迭代器是Rust函数式编程的另外一个利器,负责遍历序列中的每一项和决定序列何时结束的
逻辑,我们在使用的时候,就无需判断开始条件和结束条件。在Rust当中,迭代器是惰性的,
只有使用到了,才会在内存当中进行展开。Rust的迭代器必须实现一个Iterator的triat,
这个trait的定义类似如下的结构:
\begin{code-block}{rust}
pub trait Iterator {
    type Item;
    fn next(&mut self) -> Option<Self::Item>;
    ...
}
\end{code-block}
其中的type Item和Self::Item定义了trait的关联数据类型,即该trait要求同时定义一个
Item类型,该类型被用作next方法的返回值类型。Next方法是Iterator被要求实现的唯一
方法,其一次返回一个项,最后返回一个None。

Rust的next方法得到的是迭代器的不可变引用,iter方法生成一个不可变引用的迭代器。
如果我们需要一个获取所有权并返回拥有所有权的迭代器,则可以调用into\_iter而不是iter。
类似的,如果我们希望迭代可变引用,则可以调用iter\_mut而不是iter;如果一旦调用了
into\_iter,则迭代完成之后,迭代器不再有效,比如下方代码:
\begin{code-block}{rust}
let v = vec![1, 2, 3];
let v3: Vec<_> = v.into_iter().map(|x| x * 12).collect();
println!("{:?}", v3);
println!("{:?}", v);
\end{code-block}
一旦进行编译,则会提示如下的错误:
\begin{figure}[H]
  \centering
  \includegraphics[width=\linewidth]{rust_iter_move.png}
  \caption{迭代器的所有权转移}
  \label{fig:rust_iter_move}
\end{figure}

实际上,上述的操作相当于对一个迭代器进行了消费。一般说来,调用next方法的方法被称为
消费适配器(consuming adaptors),因为调用他们会消耗迭代器。一个消费适配器的例子
是sum方法。这个方法获取迭代器的所有权并反复调用next来遍历迭代器,因而会消费迭代器。
当其遍历每一个项时,它将每一个项加总到一个总和并在迭代完成时返回总和。在这个过程
完成之后,原有的迭代器将无法再继续使用,因为其所有权已经进行了转移。
\begin{code-block}{rust}
let v = vec![1, 2, 3];
let v_item = v.iter();
let total1: u32 = v_item.sum();
// 迭代器v_item不再有效
println!("{:?}", v_item);
\end{code-block}

Iterator trait中定义了另一类方法,被称为迭代器适配器(iterator adaptors),允许
我们将当前迭代器变为不同类型的迭代器,并且可以链式调用多个迭代器适配器。不过因为
所有的迭代器都是惰性的,必须调用一个消费适配器方法以便获取迭代器适配器调用的结果。
比较常见的,就是使用map函数(迭代适配器,遍历迭代器的所有元素)来生成新的迭代器。
与之相对应的,collect方法则是消费迭代器并将结果收集到一个数据结构中。同样需要注意
的是,任何的迭代消费器,都不能进行类型的自动推导,需要手动的指定对应的数据类型。
比如,sum的结果通常是数值类型,而collect的结果则通常是vec类型。

迭代器和闭包通常结合使用,因为闭包可以捕获环境,比如常用的filter迭代器适配器:
\begin{code-block}{rust}
let v = vec![1, 2, 3];
// 使用的是iter,即引用数据类型,但是filter使用的本身是引用,因此,需要进行
// 2次的解引用操作
let res: Vec<_> = v.iter().filter(|s| *(*s) == 2).collect();
println!("{:?}", res);
// 原始的v仍然可用,没有发生所有权转移
println!("{:?}", v);

// 发生了所有权转移,变量v在后续的操作当中,无法被继续使用
let res1: Vec<_> = v.into_iter().filter(|s| *s == 2).collect();
println!("{:?}", res1);
\end{code-block}

Filter和迭代器使用的时候,需要特别注意所有权以及引用数据类型,特别是复合数据类型。
不同的操作会导致复合数据类型的所有权的变更。
\begin{code-block}{rust}
struct Shoe {
    size: u32,
    style: String,
}

fn main() {
    let shoes = vec![
        Shoe {
            size: 10,
            style: String::from("sneaker"),
        },
        Shoe {
            size: 13,
            style: String::from("sandal"),
        },
        Shoe {
            size: 10,
            style: String::from("boot"),
        },
    ];

    // 正确,返回的结果r实际上是shoes的部分数据的引用
    let r: Vec<_> = shoes.iter().filter(|x| x.size == 10).collect();

    // 错误,无法编译,由于collect返回的是引用,无法直接转换成引用原本的数据类型
    let r1: Vec<Shoes> = shoes.iter().filter(|x| x.size == 10).collect();

    // 正确,使用into_iter获取了相关的所有权,不再是引用,而是原始数据类型
    let r2: Vec<Shoes> = shoes.into_iter().filter(|x| x.size == 10).collect();
    // 在此之后,shoes变量无法再使用,所有权已经发生了变更

    // 错误,shoes的所有权已经发生了变更,此处已经无效
    shoes_in_my_size(shoes, 10);
}

// 调用者发生了所有权转移,调用该函数之后,参数shoes无法再被使用
fn shoes_in_my_size(shoes: Vec<Shoe>, shoe_size: u32) -> Vec<Shoe> {
    shoes.into_iter().filter(|s| s.size == shoe_size).collect()
}
\end{code-block}

\subsection{自定义迭代器}
可以通过在vector上调用iter、into\_iter或iter\_mut来创建一个迭代器,也可以用标准库
中其他的集合类型创建迭代器,比如哈希map。另外,可以实现Iterator trait来创建任何
我们希望的迭代器,如下:
\begin{code-block}{rust}
impl Counter {
    fn new(max: u32) -> Counter {
        return Counter {
            current: 0,
            max: max,
        };
    }
}

impl Iterator for Counter {
    type Item = u32;
    fn next(&mut self) -> Option<Self::Item> {
        self.current += 1;

        if self.current <= self.max {
            Some(self.current)
        } else {
            None
        }
    }
}
\end{code-block}
然后,即可像普通的集合数据类型Vec一样,使用for和next进行操作:
\begin{code-block}{rust}
let c = Counter::new(10);

// 忽略开头的n个数据
// for item in c.skip(1) {
// 像迭代器一样的使用类型
for item in c {
   println!("{}", item);
}

// 需要注意,c的所有权已经被转移,在此之后,无法再使用变量c

let c1 = Counter::new(10);
let c2 = Counter::new(20);

let sum: u32 = c1
    .zip(c2.skip(10))
    .map(|(a, b)| a * b)
    .filter(|x| x % 3 == 0)
    .sum();
println!("{}", sum);
\end{code-block}
上述的自定义迭代器并不完整,比如,默认情况下转移了变量的所有权,无法使用变量的引用
进行迭代等等。这些问题可以在后续进行进一步的改进。

\section{智能指针}
