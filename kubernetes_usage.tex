\section{kubernetes基本组成}
kubernetes主要分为2种角色:master和node。其中,master包含了kube-apiserver,kube-scheduler和
kube-controller-manager;而node则包含了kubelet和kube-proxy。当然,可以把master和node节点混合
在一起。每个服务的角色和用途不一样。
\begin{itemize}
    \item kube-apiserver:提供api访问请求控制,访问etcd和转发etcd的访问控制请求。用户自己做Active-Active模式。
    \item kube-scheduler:负责pod的调度。默认Active-Backend模式
    \item kube-controller-manager:包含Node,route,service和volume等控制器。默认Active-Backend模式
    \item kubelet:控制pod(容器)。只负责当前节点,单点。
    \item kube-proxy:负责为Service提供cluster内部的服务发现和负载均衡。只负责当前节点,单点。
\end{itemize}

创建一组pod时,其内部大致流程如图 \nameref{fig:create_pod}所示
\begin{figure}[H]
  \centering
  \includegraphics[scale=0.5]{create_pod.png}
  \caption{新建pod}
  \label{fig:create_pod}
\end{figure}

\begin{enumerate}
    \item 指令传到APIserver,API server将pod的创建信息固化到etcd上
    \item scheduler监控APIserver的watch端口,查看到etcd中有创建pod的消息,下面就为pod选择合适的node节点,并进行绑定,绑定成功后,scheduler会调用APIServer的API的增加接口在etcd中创建一个boundpod对象,描述在一个工作节点上绑定运行的所有pod信息
    \item kubelet监控APIserver的watch端口监听pod信息,发现有新的pod绑定在该节点上的时候,则根据etcd中的boundpod信息进行pod创建
    \item docker会从image仓库中查看docker的信息,并下载docker image最终进行container的创建
    \item controller-manager会监听API server的端口,对node、pod副本、资源等进行管理
\end{enumerate}

而通过外部访问pod时,其内部流程又存在一些区别,其大致如图\nameref{fig:visit_pod}所示
\begin{figure}[H]
    \centering
    \includegraphics[scale=0.4]{visit_pod.png}
    \caption{访问pod}
    \label{fig:visit_pod}
\end{figure}
\begin{enumerate}
    \item controller-manager会监控API server的端口,然后管理service和endpoint的创建,其中endpoint主要是提供了server对应pod 的副本的访问地址
    \item proxy是service的主要实现者,他通过监听API server的端口,发现service,为service创建一个代理接口socket server用来接收来自server的访问请求,并创建Iptables,利用其规则使service的请求重定向到socket server。
    \item 在收到service请求之后,proxy将请求转发到后端的pod上,实现请求并实现负载均衡
\end{enumerate}

\section{kubernetes基本操作}

\subsection{NameSpace}
\begin{code-block}{bash}
# 查看所有namespace
kubectl get namespace
# 查看所有namespace,并包含lable信息
kubectl get namespace --show-labels
# 查看namespace的详细信息
kubectl describe namespace kube-system
# 从命令行新建namespace
kubectl create namespace zhangjl

# 从文件创建namespace
cat >zhangjl.yml<<EOF
apiVersion: v1
kind: Namespace
metadata:
  name: zhangjl
EOF
kubectl create -f zhangjl.yml

# 新建带有lable的namespace
cat >zhangjl.yml<<EOF
apiVersion: v1
kind: Namespace
metadata:
  name: zhangjl
  labels:
    namespace-name: zhangjl
    owner: zhangjl
EOF
kubectl create -f zhangjl.yml
\end{code-block}

\subsection{Contenxt}
\begin{code-block}{bash}
# 获取所有上下文环境
kubectl config get-contexts

# 查看context的具体信息
kubectl config view

# 查看当前使用的context
kubectl config current-context

# 新增context
kubectl config set-context zhangjl --namespace=zhangjl --cluster=kubernetes --user=admin

# 切换至新的context
kubectl config use-context zhangjl

\end{code-block}

\subsection{Label}
\begin{code-block}{bash}
# 查看对象的label
kubectl get namespace --show-labels
kubectl get pod -o wide --show-labels

# 为对象添加label.label的添加可以多次进行。overwrite表示,如果对应的key存在,就进行更新,
# 否则进行创建
kubectl label namespace zhangjl ns-name=zhangjl name=zhangjl --overwrite
kubectl label pod zhangjl ns-name=zhangjl user=zhangjl --overwrite

# 删除label
kubectl label namespace zhangjl ns-name- name-
kubectl label pod zhangjl ns-name- name-

# 根据label进行搜索
kubectl get pod --all-namespaces  -l k8s-app==dashboard
kubectl get pod --all-namespaces  -l k8s-app!=dashboard

# selector 搜索
kubectl get pod --all-namespaces  -l 'k8s-app in (dashboard,default-http-backend)'
kubectl get pod --all-namespaces --selector='k8s-app==calico-kube-controllers'

# 多label搜索
kubectl get pod --all-namespaces  -l k8s-app!=dashboard,k8s-app==default-http-backend

# 混用selector和等式搜索
kubectl get pod --all-namespaces  -l 'k8s-app notin (dashboard,kube-dns), k8s-app==default-http-backend'
kubectl get pod --all-namespaces  --selector='k8s-app notin (dashboard,kube-dns), k8s-app==default-http-backend'
\end{code-block}
特别需要注意的是,node的label会影响pod的调度,pod的label会影响service和deployment等资源的选择。
另外,kubernetes当中,内置了很多label,具体的如下:
\begin{itemize}
    \item kubernetes.io/hostname
    \item failure-domain.beta.kubernetes.io/zone
    \item failure-domain.beta.kubernetes.io/region
    \item beta.kubernetes.io/instance-type
    \item beta.kubernetes.io/os
    \item beta.kubernetes.io/arch
\end{itemize}

\subsection{Pod}
\begin{code-block}{bash}
# 获取所有pod
kubectl get --all-namespace pod

# 展示pod的详情
kubectl describe --namespace=kube-system pod zhangjl

# 显示pod的日志
kubectl logs --namespace=kube-system zhangjl

# 指定namespace,并且指定node的label进行pod创建
cat >pod.yml<<EOF
apiVersion: v1
kind: Pod
metadata:
  name: nginx
  labels:
    env: test
    k8sapp: nginx
  namespace: zhangjl # 限定pod所在的namespace
spec:
  containers:
  - name: nginx
    image: nginx
    imagePullPolicy: IfNotPresent
  nodeSelector: # nodeSelector调度指令将在之后的版本进行废弃,使用affinity/anti-affinity代替
    disk_type: ssd # 指定pod所使用的磁盘类型
EOF
kubectl create -f pod.yml

# 使用anti-affinity/affinity进行pod的节点指定
# 根据node的亲和性与反亲和性进行调度
cat>pod.yml<<EOF
apiVersion: v1
kind: Pod
metadata:
  name: nginx
  labels:
    env: test
    k8sapp: nginx
    rules: node-affinity
  namespace: zhangjl
spec:
  containers:
  - name: nginx
    image: nginx
    imagePullPolicy: IfNotPresent
  affinity:
    nodeAffinity:
      requiredDuringSchedulingIgnoredDuringExecution:
        nodeSelectorTerms:
        - matchExpressions:
          - key: kubernetes.io/hostname
            operator: In
            values:
            - k8s1
            - k8s3
      preferredDuringSchedulingIgnoredDuringExecution:
      - weight: 1
        preference:
          matchExpressions:
          - key: disk_type
            operator: In
            values:
            - nvme
EOF

kubectl create -f pod.yml

# 根据pod的亲和性与反亲和性进行调度
# 在pod层面进行调度,比较的对象是节点上已经存在的
# pod。如果节点上没有pod,则该节点不在调度的考虑范围内。
cat>pod-affinity-pod.yml<<EOF
apiVersion: v1
kind: Pod
metadata:
  name: with-pod-affinity
  labels:
    affinity-rule: on_pod_level
  namespace: zhangjl
spec:
  affinity:
    podAffinity:
      # 要求调度之后的节点上,已经运行的pod当中,所有pod
      # 的label的k8sapp的值需要在nginx当中。强制性规则
      requiredDuringSchedulingIgnoredDuringExecution:
      - labelSelector:
          matchExpressions:
          - key: k8sapp
            operator: In
            values:
            - nginx
        topologyKey: kubernetes.io/hostname
    podAntiAffinity:
      # 希望调度之后的节点上,已经运行的pod当中,其label的env
      # 不在test当中。倾向性规则。
      preferredDuringSchedulingIgnoredDuringExecution:
      - weight: 100
        podAffinityTerm:
          labelSelector:
            matchExpressions:
            - key: env
              operator: NotIn
              values:
              - test
          topologyKey: kubernetes.io/hostname
  containers:
  - name: with-pod-affinity
    image: nginx
EOF

kubectl create -f pod-affinity-pod.yml
\end{code-block}
亲和性规则只针对pod和node。但是两者的规则是一致的。同样的,这些调度规则可以用于pod,replicate和deployment当中。
\begin{itemize}
    \item requiredDuringSchedulingIgnoredDuringExecution   必须满足,没有满足条件的node,pod会创建失败
    \item preferredDuringSchedulingIgnoredDuringExecution  尽力满足,没有满足条件的node,pod也会创建成功
    \item IgnoredDuringExecution的意思是,上面两条规则只在pod创建时起作用,如果pod已经运行,后来又改了node的lable,node不满足pod运行条件,但已经运行的pod不受影响
    \item requiredDuringSchedulingRequiredDuringExecution  现在不支持,后面版本会支持,作用基本等同requiredDuringSchedulingIgnoredDuringExecution,不同之处是pod改变label,对已经运行在其上的pod也会有影响。
\end{itemize}

\subsection{Replication}

\subsection{Deployment}