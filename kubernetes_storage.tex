\subsection{Voume}
\begin{code-block}{bash}
# 使用git作为volume。但前提是要求kubernetes的node上安装有git
cat >git-volume.yml<<EOF
# kubernetes 1.8
apiVersion: extensions/v1beta1
# kubernetes 1.10
#apiVersion: apps/v1
kind: Deployment
metadata:
  name: nginx-cluster
  namespace: zhangjl
spec:
  selector:
    matchLabels:
      run: nginx-cluster
      schedule: pod-affinity
  replicas: 3
  template:
    metadata:
      labels:
        run: nginx-cluster
        schedule: pod-affinity
        app: nginx-web-server
    spec:
      containers:
      - name: nginx-cluster
        image: nginx
        ports:
        - containerPort: 80
        volumeMounts:
          - mountPath: /opt
            name: git-volume
      volumes:
      - name: git-volume
        gitRepo:
          repository: "https://github.com/luoyancn/docs.git"
          revision: "2bfe235cc6a913ff0ce496dd99318b5f1a19c9ac"
EOF

# 使用emptydir作为volume
cat >empty-dir.yml<<EOF
# kubernetes 1.8
apiVersion: extensions/v1beta1
# kubernetes 1.10
#apiVersion: apps/v1
kind: Deployment
metadata:
  name: nginx-cluster
  namespace: zhangjl
spec:
  selector:
    matchLabels:
      run: nginx-cluster
      schedule: pod-affinity
  replicas: 3
  template:
    metadata:
      labels:
        run: nginx-cluster
        schedule: pod-affinity
        app: nginx-web-server
    spec:
      containers:
      - name: nginx-cluster
        image: nginx
        ports:
        - containerPort: 80
        volumeMounts:
          - mountPath: /opt
            name: emptydir
      volumes:
      - name: emptydir
        emptyDir: {}
EOF
\end{code-block}
EmptyDir和gitrepo模式的volume,其真正的数据都放在了node的物理存储上,具体的位置为:
/var/lib/kubelet/pods/{podid}/volumes/。如果在node上对这些路径进行文件写入,实际上
会反映到对应的pod当中。

Kubernetes不仅支持emptyDir和gitrepo,还支持其他各种存储,包括ceph,glusterfs,当然也包括了
物理机存储(hostpath)。在性能上,hostpath是最好的;但是,hostpath也是最不灵活的:与主机息息相关
无法迁移。
\begin{code-block}{bash}
cat >host-path-volume.yml<<EOF
# kubernetes 1.8
apiVersion: extensions/v1beta1
# kubernetes 1.10
#apiVersion: apps/v1
kind: Deployment
metadata:
  name: nginx-cluster
  namespace: zhangjl
spec:
  selector:
    matchLabels:
      run: nginx-cluster
      schedule: pod-affinity
  replicas: 3
  template:
    metadata:
      labels:
        run: nginx-cluster
        schedule: pod-affinity
        app: nginx-web-server
    spec:
      containers:
      - name: nginx-cluster
        image: nginx
        ports:
        - containerPort: 80
        volumeMounts:
          - mountPath: /opt
            name: host-volume
      volumes:
      - name: host-volume
        hostPath:
          path: /mnt/host-volume
          type: DirectoryOrCreate
EOF
\end{code-block}
HostPath模式下,volume的使用类型主要有以下几种:
\begin{itemize}
  \item DirectoryOrCreate:以目录的方式挂载。如果主机上对应的目录不存在则新建
  \item Directory:目录方式。要求对应路径一定存在
  \item FileOrCreate:以文件方式挂载。如果主机对应路径不存在则新建
  \item File:文件方式,要求对应路径一定存在
  \item Socket:以socket方式
  \item CharDevice:以字符设备方式
  \item BlockDevice:以块存储方式
\end{itemize}