\section{Gerrit安装}
Gerrit可以通过两种方式进行安装:docker和普通方式。Docker方式简单,普通方式
则相对比较复杂——需要安装ldap等相关依赖。

\subsection{Docker方式}

\begin{enumerate}
  \item 拉取必要镜像

  \begin{code-block}{bash}
  docker pull gerritcodereview/gerrit osixia/phpldapadmin osixia/openldap
  \end{code-block}

  \item 创建必要路径

  \begin{code-block}{bash}
  cd /gerrit
  mkdir cache db etc git index ldap/etc ldap/var -p
  \end{code-block}

  \item 修改gerrit配置文件

  \begin{code-block}{bash}
  cat >/gerrit/etc/gerrit.config<<EOF
  [gerrit]
    basePath = git

  [index]
    type = LUCENE

  [auth]
    type = ldap
    gitBasicAuth = true

  [ldap]
    server = ldap://ldap
    username=cn=admin,dc=example,dc=org
    accountBase = dc=example,dc=org
    accountPattern = (&(objectClass=person)(uid=${username}))
    accountFullName = displayName
    accountEmailAddress = mail

  [sendemail]
    smtpServer = localhost
    # 如果需要邮件通知,则如下配置,以qq邮箱为例
    # enable = true
    # smtpServer = smtp.exmail.qq.com
    # smtpServerPort = 465
    # smtpEncryption = SSL
    # sslVerify = true
    # smtpUser = *****
    # from = *****
    # smtpPass = *****

  [sshd]
    listenAddress = *:29418

  [httpd]
    listenUrl = http://*:8080/

  [cache]
    directory = cache

  [container]
    user = root
  EOF
  \end{code-block}

  \item 修改gerrit的ldap配置文件,并修改路径的权限

  \begin{code-block}{bash}
  cat /gerrit/etc/secure.config<<EOF
  [ldap]
    password = secret
  EOF

  # 目前必须为777,否则无法启动
  chmod -R 777 /gerrit
  \end{code-block}

  \item 编写docker-compose文件

  \begin{code-block}{yml}
  version: '3'

  services:
    gerrit:
      image: gerritcodereview/gerrit
      ports:
        - "29418:29418"
        - "80:8080"
      depends_on:
        - ldap
      volumes:
        - /gerrit/etc:/var/gerrit/etc
        - /gerrit/git:/var/gerrit/git
        - /gerrit/db:/var/gerrit/db
        - /gerrit/index:/var/gerrit/index
        - /gerrit/cache:/var/gerrit/cache
      environment:
        # 修改成有效的ip地址,使用localhost/127.0.0.1将造成其他ip无法访问
        - CANONICAL_WEB_URL=http://<ip address >
      # 初始化操作时,需要打开下面的注释
      # command: init

    ldap:
      image: osixia/openldap
      ports:
        - "389:389"
        - "636:636"
      environment:
        - LDAP_ADMIN_PASSWORD=secret
      volumes:
        - /gerrit/ldap/var:/var/lib/ldap
        - /gerrit/ldap/etc:/etc/ldap/slapd.d

    ldap-admin:
      image: osixia/phpldapadmin
      ports:
        - "6443:443"
      environment:
        - PHPLDAPADMIN_LDAP_HOSTS=ldap
  \end{code-block}

  \item 初始化环境

  修改上述的docker-compose.yml文件,打开init,然后执行
  \begin{code-block}{bash}
  docker-compose up gerrit
  \end{code-block}
  当上述指令成功退出(exit\_code为0),则说明初始化成功。此时,再次将上述的
  docker-compose.yml文件修改,注释掉init,然后启动:
  \begin{code-block}{bash}
  docker-compose up -d
  \end{code-block}
  当gerrit容器的日志显示Ready之后,说明整体环境运行正常,可以使用了。

  \item ldap用户管理

  登陆\url{https://serverip:6443},用户名为“cn=admin,dc=example,dc=org”,
  密码为secret,登陆成功之后,按照ldap的规则创建用户即可。ldap的用户
  就是gerrit的用户。注意,第一个登陆的用户会被当成表成gerrit的管理员。

\end{enumerate}

\subsection{普通方式}

\subsubsection{安装openldap服务}
\begin{enumerate}

  \item 安装ldap相关软件

  \begin{code-block}{bash}
  yum install openldap-servers openldap-clients -y
  \end{code-block}

  \item 启动ldap服务

  \begin{code-block}{bash}
  cp /usr/share/openldap-servers/DB_CONFIG.example /var/lib/ldap/DB_CONFIG
  chown -R ldap:ldap  /var/lib/ldap/
  systemctl start slapd
  \end{code-block}

  \item 修改ldap密码

  \begin{code-block}{bash}
  # 生成ldap的新密码,默认会生成类似{SSHA}unxPSArS17ggMRVDQElbH243CK5Gy5bc的输出
  slappasswd -s password

  # 编写修改的ldap脚本,注意,olcRootPW的值,即是上面的输出
  cat >chrootpw.ldif<<EOF
  dn: olcDatabase={0}config,cn=config
  changetype: modify
  # 如果出现错误,可尝试使用replace替换下面的add
  add: olcRootPW
  olcRootPW: {SSHA}b/fmgl5zYUIag/NKjeYj/Sm3/ymCtLeb
  EOF

  # 执行修改命令
  ldapadd -Y EXTERNAL -H ldapi:/// -f chrootpw.ldif
  \end{code-block}

  \item 导入ldap的基本模式

  \begin{code-block}{bash}
  ldapadd -Y EXTERNAL -H ldapi:/// -f /etc/openldap/schema/cosine.ldif
  ldapadd -Y EXTERNAL -H ldapi:/// -f /etc/openldap/schema/nis.ldif
  ldapadd -Y EXTERNAL -H ldapi:/// -f /etc/openldap/schema/inetorgperson.ldif
  \end{code-block}


  \item 配置ldap的domain信息

  \begin{code-block}{bash}
  cat > chdomain.ldif<<EOF
  dn: olcDatabase={1}monitor,cn=config
  changetype: modify
  replace: olcAccess
  olcAccess: {0}to * by dn.base="gidNumber=0+uidNumber=0,cn=peercred,cn=external,cn=auth" read by dn.base="cn=admin,dc=polaris,dc=com" read by * none

  dn: olcDatabase={2}hdb,cn=config
  changetype: modify
  replace: olcSuffix
  olcSuffix: dc=polaris,dc=com

  dn: olcDatabase={2}hdb,cn=config
  changetype: modify
  replace: olcRootDN
  olcRootDN: cn=admin,dc=polaris,dc=com

  dn: olcDatabase={2}hdb,cn=config
  changetype: modify
  add: olcRootPW
  # 注意,密码同上
  olcRootPW: {SSHA}b/fmgl5zYUIag/NKjeYj/Sm3/ymCtLeb

  dn: olcDatabase={2}hdb,cn=config
  changetype: modify
  add: olcAccess
  olcAccess: {0}to attrs=userPassword,shadowLastChange by dn="cn=admin,dc=polaris,dc=com" write by anonymous auth by self write by * none
  olcAccess: {1}to dn.base="" by * read
  olcAccess: {2}to * by dn="cn=admin,dc=polaris,dc=com" write by * read
  EOF

  ldapmodify -Y EXTERNAL -H ldapi:/// -f chdomain.ldif
  \end{code-block}

  \item 设置ldap的基本信息

  \begin{code-block}{bash}
  cat > basedomain.ldif<<EOF
  dn: dc=polaris,dc=com
  objectClass: top
  objectClass: dcObject
  objectclass: organization
  o: polaris
  dc: polaris
  EOF

  # -w passwd即上面的密码
  ldapadd -x -D cn=admin,dc=polaris,dc=com -W -f basedomain.ldif -w passwd
  \end{code-block}

  \item 安装ldap的web管理
  \begin{code-block}{bash}
  yum install -y httpd php php-mbstring php-pear phpldapadmin

  # 修改http配置/etc/httpd/conf/httpd.conf
  # www.example.com行下新增如下内容
  ServerName www.polaris.com

  # 151行处修改为如下
  AllowOverride All

  # 164行处修改如下
  DirectoryIndex index.html index.php index.cgi

  # 文件末尾增加如下内容
  ServerTokens Prod
  KeepAlive On

  # 修改phpldapadmin /etc/phpldapadmin/config.php
  # 397行取消注释,398行注释掉,如下
  # $servers->setValue('login','attr','dn');
  # $servers->setValue('login','attr','uid');

  # 修改/etc/httpd/conf.d/phpldapadmin.conf,如下
  # <IfModule mod_authz_core.c>
  #   Require local
  #   Require ip 172.16.0.0/16
  # 允许所有地址访问
  #   Require all granted
  # </IfModule>

  # 启动httpd
  systemctl start httpd
  systemctl enable httpd

  # 使用浏览器访问http://<ip>/ldapadmin,用户名/密码cn=admin,dc=polaris,dc=com/passwd
  \end{code-block}

  \item 使用命令行添加用户
  \begin{code-block}{bash}
  cat > newuser.ldif<<EOF
  dn: cn=sys,cn=developer,ou=gerrit,dc=polaris,dc=com
  cn: sys
  gidNumber: 501
  givenName: sys
  homeDirectory: /home/users/sys
  loginShell: /bin/bash
  mail: sys@awcloud.com
  objectClass: inetOrgPerson
  objectClass: posixAccount
  objectClass: top
  sn: sys
  uid: sys
  uidNumber: 1005
  userPassword: {MD5}js1AV54zKRLnQEV9kc3Qiw==
  EOF

  ldapadd -x -D cn=admin,dc=polaris,dc=com -W -f newuser.ldif
  \end{code-block}


\end{enumerate}

\subsubsection{安装gerrit服务}
安装gerrit需要java>=11.0.16.1,并且需要使用gerrit用户,因此需要新建用户,并切换成gerrit用户:
\begin{code-block}{bash}
useradd -d /home/gerrit  -p gerrit  gerrit
su - gerrit
\end{code-block}

然后使用gerrit用户进行操作
\begin{code-block}{bash}
java -jar gerrit-3.6.2.war init -d gerrit-review
# 注意,参数最好保持一致
# Authentication method          [openid/?]: ldap
# Git/HTTP authentication        [http/?]: ldap
# LDAP server                    [ldap://localhost]:
# LDAP username                  : cn=admin,dc=polaris,dc=com
# Account BaseDN                 : dc=polaris,dc=com
# Group BaseDN                   [dc=polaris,dc=com]:
# Install Verified label         [y/N]?
# Canonical URL                  [http://localhost:8080/]: http://10.1.1.105:8080
# Install plugin hooks version v3.6.2 [y/N]? y
# Install plugin webhooks version v3.6.2 [y/N]? y
# Install plugin replication version v3.6.2 [y/N]? y
\end{code-block}

安装成功之后,即可使用ldap当中的用户登陆gerrit(\url{ http://10.1.1.105:8080/#/admin/projects/})进行操作了。
启动gerrit服务如下:
\begin{code-block}{bash}
gerrit-review/bin/gerrit.sh start
\end{code-block}

随后,ldap的用户即可使用ssh协议拉取代码了:
\begin{code-block}{bash}
git clone "ssh://admin@10.1.1.42:29418/demo" && scp -p -P 29418 admin@10.1.1.42:hooks/commit-msg "demo/.git/hooks/"
# 新版本的gerrit与sftp协议存在部分兼容的问题,上述指令可能出错,则换成下面的指令:
git clone "ssh://admin@10.1.1.42:29418/demo" && scp -O -P 29418 admin@10.1.1.42:hooks/commit-msg "demo/.git/hooks/"
\end{code-block}

如果需要开启gerrit的管理功能,则需要修改gerrit的配置:
\begin{code-block}{bash}
cat >>gerrit-review/etc/gerrit.config<<EOF
[plugins]
        allowRemoteAdmin = true
EOF
\end{code-block}
并修改gerrit.sh脚本:
\begin{code-block}{bash}
vi gerrit-review/bin/gerrit.sh +37
# 加入如下内容
# export GERRIT_SITE=/home/gerrit/gerrit-review
# 对所有用户开放权限
ln -s /home/gerrit/gerrit-review/bin/gerrit.sh /usr/bin/gerrit
\end{code-block}

此时,即可以对gerrit进行命令行的管理了:
\begin{code-block}{bash}
# 查看gerrit的插件
ssh -p 29418 gerrit@localhost gerrit plugin ls
# 重载插件
ssh -p 29418 gerrit@localhost gerrit plugin reload replication
\end{code-block}

\subsection{集成GitLab}
Gerrit本身可以存放git代码,也可以集成GitLab,将代码放置在GitLab当中。比较简单的
做法是Gerrit和GitLab都放置一份,只是Gerrit定期向GitLab同步即可。下面就主要讲解
这种方式的做法。

\begin{enumerate}
  \item 生成gerrit用户的ssh-key(gerrit服务器),并上传到gerrit页面上,以及GitLab页面上
  \item GitLab创建项目(mirror),并初始化
  \item 在Gerrit上创建同名的项目(mirror),然后进入gerrit服务器

  \begin{code-block}{bash}
  rm -rf gerrit-review/git/mirror.git
  # 之后直接从GitLab克隆同名的项目
  git clone --bare git@gitlab.corp.awcloud.com:awcloudhardware/mirror.git
  \end{code-block}

  \item 创建gerrit的同步配置文件

  \begin{code-block}{bash}
  cat >gerrit-review/etc/replication.config<<EOF
  [remote "mirror"]
  url =  git@gitlab.corp.awcloud.com:awcloudhardware/mirror.git
  push = +refs/heads/*:refs/heads/*
  push = +refs/tags/*:refs/tags/*
  push = +refs/changes/*:refs/changes/*
  threads = 3
  timeout = 30
  projects = "mirror"
  EOF
  \end{code-block}

  注意,remote当中的名称需要和projects当中的保持一致。如果需要同步多个项目,
 则直接添加多个remote配置即可,也可以如下进行配置:
  \begin{code-block}{bash}
  cat >gerrit-review/etc/replication.config<<EOF
  [remote "mirror"]
  url =  git@gitlab.corp.awcloud.com:awcloudhardware/mirror.git
  projects = "mirror"

  [remote "abc"]
  url =  git@gitlab.corp.awcloud.com:awcloudhardware/abc.git
  projects = "abc"

  [remote "gitlab"]
  url =  git@gitlab.corp.awcloud.com:awcloudhardware/${name}.git
  push = +refs/heads/*:refs/heads/*
  push = +refs/tags/*:refs/tags/*
  push = +refs/changes/*:refs/changes/*
  threads = 3
  timeout = 30
  EOF
  \end{code-block}
  \item 重启Gerrit服务即可,或者重载replication插件

  \begin{code-block}{bash}
  ssh -p 29418 gerrit@localhost gerrit plugin reload replication
  \end{code-block}
\end{enumerate}

\subsection{参考资料}
\url{https://cloud.tencent.com/developer/article/1026081}

\url{https://www.cnblogs.com/yinzhengjie/p/11057383.html}

\url{https://summerandwinter.github.io/mac-gerrit-gitlab.html}

\url{https://notes.lzwang.ltd/Docker/deploy/docker_deploy_gerrit/#docker-composeyaml}
